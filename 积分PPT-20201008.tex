%!TEX program = 
%!TEX TS-program = 
%!TEX encoding = UTF-8 Unicode
%
%

% file name: Calculus-1-2-main.tex
% Usage: 1. ConTeXt, WinEdt 10.2
%        2. ConTeXt (LuaTeX), TeXworks 0.6.5
% Effect: 幻灯片
% Version: 2.12
% Date: 2020/10/1 13:28

\usemodule[pre-split] % 加载幻灯片模板
%\usemodule[pre-split-claude] % 加载幻灯片模板
%\usemodule[pre-colorful] % 加载幻灯片模板

%%%%%%%%%%%%%%%%%%%%%%%%%%%%%%%%%% 中文设置开始
\mainlanguage[cn]
\language[cn]
\enableregime[utf]
\setscript[hanzi] % hyphenation

% Enable fonts
\usetypescriptfile[mscore]
\setupbodyfont [mschinese,20.7pt]
\setupbodyfont [mschinese,12pt] % 12pt, 14pt
\usebodyfont   [mschinese-light,12pt] % 12pt, 14pt
\definebodyfontenvironment[24pt]
%\definebodyfontenvironment[18pt]

%\setuppagenumber [numberconversion=cn]
%\definestructureconversionset[chinese][numbers][cn]
%\setupheads [sectionconversionset=chinese]

%%%%%%%%%%%%%% 定义标题字体 加粗 两倍
\definefont [sectionFont] [name:comicneueangularbold sa 2]
\setuphead[section][style=\sectionFont]
%%%%%%%%%%%%%%%%%%%%%%%%%%%%%%%%%% 中文设置结束

%---------------------------
%---------------------------
%---------------------------
%添加页码,左上角
% 用法: \myProgressBar
% 效果:页面底部动态显示当前页码和总页码矩形条
\define\myProgressBar{
%------------------------------
%%%-----------------------------------颜色,半透明
\definecolor [transparentred]  [r=1,t=.5,a=1]
\definecolor [transparentblue] [b=1,t=.5,a=1]
\definecolor [transparentgreen] [g=0.1,t=.5,a=3]
\definecolor [solidyellow]  [y=1,t=1,a=1]
%------------------------------

\definelayer[mybg]  % name of the layer 页面左上角添加页码
    [x=0mm, y=153mm,  % from upper left corner of paper
     width=0.1\paperwidth, height=0.1\paperheight] % let the layer cover the full paper

\setlayer[mybg]	% name of the layer
    [hoffset=0cm, voffset=0cm]  % placement (from upper left corner of the layer)
    {\framed[background=color,backgroundcolor=transparentblue,
foregroundcolor=white,
frame=on, width=\dimexpr \textwidth*\pagenumber/\lastuserpagenumber\relax, height=0.5cm]{{} \hfill \green \bf \userpagenumber}
     \framed[background=color,
backgroundcolor=gray-1,
width=\dimexpr \textwidth - \textwidth*\pagenumber/\lastuserpagenumber\relax, 
height=0.5cm]{}
     \framed[background=color,backgroundcolor=transparentblue,height=0.5cm]{\green \bf \lastuserpagenumber}}  % the actual contents of the layer

%------------------------------
\setupbackgrounds[page][background=mybg] %添加页码,左上角
}

%---------------------------
%---------------------------
%--------------------------
%%%%%%%%%%%%%%%%%%%%%%%%%%%%%%%%%%%%%%

%-------------------------------------
%%%%%%%%%%%%%%%%%%%%%%%%%%%%%%%%%%%%%%
%% defining an enumeration environment replacing Plain-TeX's \proclaim
\defineenumeration[proclaim]
	[text=,
	style=bf, % slanted
	title=yes,
	titleleft=,
	titleright=,
	location=serried,
	width=fit,
	right={. }]

%% this can be changed according to one's wishes
\setupnumber[proclaim][way=bysection, numbersection=yes] 

%% end definition \proclaim
%%%%%%%%%%%%%%%%%%%%%%%%%%%%%%%%%%%%%%
%-------------------------------------
%%%%%%%%%%%%%%%%%%%%%%%%%%%%%%%%%%%%%%
%-------------------------------------
% https://wiki.contextgarden.net/SlideWithSteps
%%%%%%%%%%%% begin test-step-ok.tex
%%K test-step-ok.tex

%%K Here we use the plain \TeX\ command \phantom{} in order to have
%%K some material appear step by step.
%%K The structure is quite simple, but since I am not very
%%K familiar with ConTeXt, I cannot write elegant code as does
%%K Hans Hagen...
%%K
%%K An advantage is that the code can be used also in
%%K plain TeX, in LaTeX and other macro-packages.
%%K

\newif\ifSteppingSlide
\SteppingSlidetrue   %%K this is when you want a step by step presentation
%\SteppingSlidefalse %%K this is when you want to print the slides
%
\newcount\StepsCounter
\StepsCounter=0
%
\newcount\NumberOfSteps
\NumberOfSteps=10
%
\newcount\BeforeStepNumber
\BeforeStepNumber=0
%
%%K StepBetween[number1,number2]{material} will make "material"
%%K appear between steps "number1" and "number2"
\def\StepBetween[#1,#2]#3{%
\ifSteppingSlide
 \ifnum#1>\StepsCounter \phantom{#3}
  \else
    \ifnum#2<\StepsCounter \phantom{#3}
      \else \relax #3
     \fi
  \fi
  \else {#3}
\fi}
%
%%K Step{number1}{material} will make "material"
%%K appear beginning with step "number1" until "NumberOfSteps"
\def\Step#1#2{\StepBetween[#1,\NumberOfSteps]{#2}}
%
%%K OnlyStep{number1}{material} will make "material"
%%K appear only on step "number1"
\def\OnlyStep#1#2{\StepBetween[#1,#1]{#2}}
%
%%K StepBefore{number1}{material} will make "material"
%%K appear only on all steps before "number1"
\def\StepBefore#1#2{
\global\BeforeStepNumber=#1
\StepBetween[0,\BeforeStepNumber]{#2}}
%
\long\def\SlideWithSteps#1#2{
\ifSteppingSlide
\global\StepsCounter=0
\global\NumberOfSteps=#1
\MakeSteps{#2}
\else #2
\fi}
%
\long\def\MakeSteps#1{\loop #1
\ifnum\StepsCounter<\NumberOfSteps
\global\advance\StepsCounter by 1\vfill\page
\repeat
\vfill\page}
%


%%K This is the end of the macros
%%%%%%%%%%%%%%%%%%%%%%%%%%%%%%%%%%%%%%%%%%%%%

%%----------------------------------------------------
%% 英文短横线
%% usage:   \EnglishRule
\setupMPvariables
  [EnglishRule]
  [height=1ex,width=\localhsize,color=darkgray]

\startuniqueMPgraphic{EnglishRule}{height,width,color}
  numeric height ; height = \MPvar{height} ;
  x1 = 0 ; x3 = \MPvar{width} ; x2 = x4 = .5x3 ;
  y1 = y3 = 0 ; y2 := -y4 = height/2 ;
  fill z1..z2..z3 & z3..z4..z1 & cycle withcolor \MPvar{color} ;
\stopuniqueMPgraphic

\defineblank
  [EnglishRule]
  [medium]

\def\EnglishRule%
  {\startlinecorrection[EnglishRule]
   \setlocalhsize \noindent \reuseMPgraphic{EnglishRule}
   \stoplinecorrection}

%% 英文短横线
%%--------------------------------------------------

%%%%%%%%%%%%%%%%%%%%%%%%%%%%%%%%%%%%%%
%-------------------------------------

%%%%-------------------输入代码,开始
\defineframedtext
  [framedcode]
  [strut=yes,
   offset=2mm,
   width=18cm,
   align=right]

\definetyping[code][numbering=line,
                    bodyfont=,
                    before={\startframedcode},
                    after={\stopframedcode}]


%%%%-------------------输入代码,结束

%%------------------------------------


%%------------------------------------



%%%%%%%%%%%%%%%%%%%%%%%%%%%%%%%%%%%%%%
%-------------------------------------


\starttext % 正文部分开始

\TitlePage {  
             考研数学(二)   
%\EnglishRule
\crlf \bigskip
   \date[][weekday,{,~},month,day,{,~},year]
} % 首页标题


%%%%%%%%%%%%%%%%%%%%%%%%
%\switchtobodyfont[64pt] % 定义字体大小
%
%\startstandardmakeup[align=middle] % 水平居中,垂直居中
%  \bf 考研数学(二)\index{00 首页标题}
%\stopstandardmakeup
%%%%%%%%%%%%%%%%%%%%%%%%
\switchtobodyfont[14pt] % 定义字体大小 24pt

\Topics {考研数学(二)} % 目录
\Topic{01 考研大纲的发布时间}
%\Subject{考研大纲的发布时间}
\index{01 考研大纲的发布时间}
\bf

1. 2021考研: 2020年9月9日发布

1. 2020考研: 2019年7月8日发布

2. 2019考研: 2018年9月15日发布

3. 2018考研: 2017年9月15日发布

4. 2017考研: 2016年8月26日发布

5. 2016考研: 2015年9月18日发布

6. 2015考研: 2014年9月13日发布

\medskip
\blackrule[color=black,width=\textwidth,height=.01cm,depth=0cm] % 分隔线

从上述分析可知,

1. 除了 2020 考研以外, 历年考研大纲发布时间都是集中在7月初-9月中旬。

%2. 2010年-2021年数学大纲的变动很小。

%3. 2021年大纲公布时间应该和往年一样的,也是在放暑假之前发布。

%4. 现在可以参考:2020年全国各院校专业课考研大纲汇总。


\Topic{02 数学考研大纲}
\index{02 2021年数学(二)大纲}
%数一:
%
%新: 分值: 高数 xx\%(xx分), 线性代数 xx\%(xx分),  概率论与数理统计 xx\%(xx分), 满分150分。
%
%旧: 2000年分值: 高数 57.3\%(86分), 线性代数 20\%(30分),  概率论与数理统计 22.7\%(34分), 满分150分。

\blackrule[color=black,width=\textwidth,height=.01cm,depth=0cm]


数二:

{\darkgreen 新}: 分值: 高数80\%(120分), 线性代数20\%(30分), 满分150分。

\bigskip
{\darkgreen 旧}: 分值: 高数78\%(116分), 线性代数22\%(34分), 满分150分。

\page

\blackrule[color=red,width=\textwidth,height=.1cm,depth=0cm]

题量变化:

新: {\darkgreen 22}个题目

旧: 23个题目

\medskip


题型变化:

\medskip

单项选择题:(概念、性质)

新: {\darkgreen 10}小题, 共50分。

旧: 8小题, 共32分。

\medskip


填空题:(运算、方法、准确率)

新: {\darkgreen 6}小题, 共30分。

旧: 6小题, 共24分。

\medskip

解答题:(包括证明题、 题目变少综合性强)

新: {\darkgreen 6}小题, 共70分。

旧: 9小题, 共94分。

\blackrule[color=black,width=\textwidth,height=.01cm,depth=0cm]


\page 

%数三:

%{\darkgreen 新}: 分值: 

%高数80\%(90分), 

%线性代数20\%(30分),  

%概率论与数理统计\%(30分), 

%满分150分。

%\bigskip

%{\darkgreen 旧}: 分值: 

%高数78\%(82分), 

%线性代数22.7\%(34分),  

%概率论与数理统计22.7\%(34分), 

%满分150分。

%\page

1. 2021年考研数学(二)大纲(积分部分)考试内容和考试要求 

{\bf 一元积分学{ } 考试内容}

\medskip
1. 原函数和不定积分的概念。

2. 不定积分的基本性质。

3. 基本积分公式。

4. 定积分的概念和基本性质。


5. 定积分中值定理。

\blackrule[color=black,width=\textwidth,height=.01cm,depth=0cm]

6. 积分上限的函数及其导数。

7. 牛顿-莱布尼茨(Newton-Leibniz)公式。

8. 不定积分和定积分的换元积分法与分部积分法。

9. 有理函数、  三角函数的有理式和简单无理函数的积分。\reference[三角函数]{三角函数}

10. 反常(广义)积分、  定积分的应用。

\page

{\bf 一元积分学{ } 考试要求}
 
1. 理解原函数的概念, 理解不定积分和定积分的概念。
 
2. 掌握不定积分的基本公式、 
掌握不定积分和定积分的性质及定积分中值定理、 \textreference[积分中值定理]{积分中值定理} 
掌握换元积分法与分部积分法。
\index{积分中值定理}
 
3. 会求有理函数、 三角函数有理式和简单无理函数的积分。
 
4. 理解积分上限的函数, 会求它的导数, 掌握牛顿-莱布尼茨(Newton-Leibniz)公式。

5. ({\darkgreen 旧}) 了解反常积分的概念, 会计算反常积分。

5. ({\darkgreen 新})理解反常积分的概念, 了解{\darkgreen 反常积分收敛的比较判别法}, 会计算反常积分。

\blackrule[color=black,width=\textwidth,height=.01cm,depth=0cm]
 
6. 掌握用定积分表达和计算一些几何量与物理量 
(平面图形的面积、 平面曲线的弧长、 旋转体的体积及侧面积、
平行截面面积为已知的立体体积、 功、 引力、 压力、
质心、形心等)及函数的平均值。
\medskip
\blackrule[color=black,width=\textwidth,height=.01cm,depth=0cm] % 分隔线

\page

%%-------------------------------------------------------------------

%%-------------------------------------------------------------------


\page

{\bf 多元函数微积分学{ } 考试内容}
 
1. 多元函数的概念。

2. 二元函数的几何意义。

3. 二元函数的极限与连续的概念。

4. 有界闭区域上二元连续函数的性质。

5. 多元函数的偏导数和全微分。

\blackrule[color=black,width=\textwidth,height=.01cm,depth=0cm]

6. 多元复合函数。

7. 隐函数的求导法。

8. 二阶偏导数。

9. 多元函数的极值和条件极值。

10. 最大值和最小值。

\blackrule[color=black,width=\textwidth,height=.01cm,depth=0cm]

11. 二重积分的概念、基本性质和计算。

\page

({\darkgreen 仅数一}) 1. 二重积分与三重积分的概念、性质、计算和应用。

({\darkgreen 仅数一}) 2. 两类曲线积分的概念、性质及计算。

({\darkgreen 仅数一}) 3. 两类曲线积分的关系。

({\darkgreen 仅数一}) 4. 格林(Green)公式。

({\darkgreen 仅数一}) 5. 平面曲线积分与路径无关的条件。

({\darkgreen 仅数一}) 6. 二元函数全微分的原函数。

({\darkgreen 仅数一}) 7. 两类曲面积分的概念、性质及计算。

({\darkgreen 仅数一}) 8. 两类曲面积分的关系。

({\darkgreen 仅数一}) 9. 高斯(Gauss)公式。

({\darkgreen 仅数一}) 10. 斯托克斯(Stokes)公式。

({\darkgreen 仅数一})  11. 散度、旋度的概念及计算。

({\darkgreen 仅数一})  12. 曲线积分和曲面积分的应用。


\page

{\bf 多元函数微积分学{ } 考试要求}
 
1. 了解多元函数的概念, 了解二元函数的几何意义。
 
2. 了解二元函数的极限与连续的概念, 了解有界闭区域上二元连续函数的性质。
 
3. 了解多元函数偏导数与全微分的概念, 会求多元复合函数一阶、 二阶偏导数, 会求全微分, 了解隐函数存在定理, 会求多元隐函数的偏导数。
 
4. 了解多元函数极值和条件极值的概念, 掌握多元函数极值存在的必要条件, 了解二元函数极值存在的充分条件, 会求二元函数的极值, 会用拉格朗日乘数法求条件极值, 会求简单多元函数的最大值和最小值, 并会解决一些简单的应用问题。

5. 了解二重积分的概念与基本性质, 掌握二重积分的计算方法 (直角坐标、 极坐标)。

\page

({\darkgreen 仅数一}) 1. 理解二重积分、 三重积分的概念, 了解重积分的性质, 了解二重积分的中值定理。

({\darkgreen 仅数一}) 2. 掌握二重积分的计算方法 (直角坐标、 极坐标), 会计算三重积分 (直角坐标、 柱面坐标、 球面坐标)。

({\darkgreen 仅数一}) 3. 理解两类曲线积分的概念, 了解两类曲线积分的性质及两类曲线积分的关系。

({\darkgreen 仅数一}) 4. 掌握两类曲线积分的计算方法。

({\darkgreen 仅数一}) 5. 掌握格林公式并会运用平面曲线积分与路径无关的条件, 会求二元函数全微分的原函数。

({\darkgreen 仅数一}) 6. 了解两类曲面积分的概念、 性质及两类曲面积分的关系, 掌握计算两类曲面积分的方法, 掌握用高斯公式计算曲面积分的方法, 并会用斯托克斯公式计算曲线积分。

({\darkgreen 仅数一}) 7. 了解散度与旋度的概念, 并会计算。

({\darkgreen 仅数一}) 8. 会用重积分、 曲线积分及曲面积分求一些几何量与物理量 (平面图形的面积、 体积、 曲面面积、 弧长、 质量、 质心、 形心、 转动惯量、 引力、 功及流量等)。

%%%%%%%-----------------
\page

2021年新大纲的变化

(一)反常积分比较判别法 (数一、 二、 三都要求)

(二)无穷级数积分判别法 (数一、 三要求)

(1)旧大纲: 了解反常积分的概念, 会计算反常积分。

新大纲: 理解反常积分的概念, 了解反常积分收敛的比较判别法, 会计算反常积分。

(2)旧大纲: 掌握正项级数收敛的比较判别法和比值判别法, 会用根值判别法。

新大纲: 掌握正项级数收敛的比较判别法、 比值判别法、 根值判別法, 会用积分判別法。
%%-------------------------------------------------------------------


%%-------------------------------------------------------------------




%%----------------------------------%%----------------------------------
%%----------------------------------%%----------------------------------
\startuseMPgraphic{FunnyFrame}
  picture p ; numeric o ; path a, b ; pair c ;
  p := textext.rt(\MPstring{FunnyFrame}) ;
  a := unitsquare xyscaled(1.0OverlayWidth,OverlayHeight) ; %指定宽度1.15倍
  o := BodyFontSize ;
  p := p shifted (2o,OverlayHeight-ypart center p) ; %标题右移的距离
  drawoptions (withpen pencircle scaled 1pt withcolor .625red) ;
  b := a randomized (o/2) ;
  fill b withcolor .98white ; draw b ; %方框内部的填充色
  b := (boundingbox p) randomized (o/8) ;
  fill b withcolor .95white ; 
  draw b ;draw p withcolor black;
  setbounds currentpicture to a ;
\stopuseMPgraphic

\def\StartFrame{\startFunnyText}
\def\StopFrame {\stopFunnyText }
\defineoverlay[FunnyFrame][\useMPgraphic{FunnyFrame}]

\defineframedtext[FunnyText][frame=off,background=FunnyFrame,width=fit] %fit, \textwidth

\def\FrameTitle#1%
  {\setMPtext{FunnyFrame}{\hbox spread 1em{\hss\strut#1\hss}}}








%%----------------------------------
%%----------------------------------


%%------------------------




%%------------------------
\startcomment[accents][color=darkgreen]  % 单独一页,讲解,注释用
标签
\stopcomment


%%------------------------


\Topic{03 考点分布}
\index{03 考点分布}

%%------------------------
\usemodule[chart]
\setupFLOWcharts[
height=2.5\lineheight,
width=6\bodyfontsize,
dx=1\bodyfontsize,
dy=0.2\bodyfontsize,
]

\setupFLOWshapes
[framecolor=pragmacolor,
background=color,
backgroundcolor=white,
]

\setupFLOWlines[framecolor=pragmacolor]

\startFLOWchart[example]

\startFLOWcell 
  \name {01}
  \location {0,4}
  \text {一元函数积分学 {\darkgreen 1}}
  \connection [rl] {11}
  \connection [rl] {12}
  \connection [rl] {14}
  \connection [rl] {15}
\stopFLOWcell

\startFLOWcell
  \name {11}
  \location{2,1}
  \text {奇偶性和周期性}
  \connection [rl] {31}
\stopFLOWcell

\startFLOWcell
  \name {12}
  \location{2,3}
  \text {比较大小}
  \connection [rl] {32}
  \connection [rl] {33}
\stopFLOWcell

\startFLOWcell
  \name {14}
  \location{2,5}
  \text {定积分定义}
  \connection [rl] {34}
  \connection [rl] {35}
  \connection [rl] {36}
\stopFLOWcell

\startFLOWcell
  \name {15}
  \location{2,7}
  \text {反常积分}
  \connection [rl] {37}
\stopFLOWcell

\startFLOWcell
  \name {31}
  \location{3,1}
  \text {七种情况}
\stopFLOWcell

\startFLOWcell
  \name {32}
  \location{3,3}
  \text {几何意义}
  \connection [rl] {41}
\stopFLOWcell

\startFLOWcell
  \name {33}
  \location{3,2}
  \text {保号性}
  \connection [rl] {42}
  \connection [rl] {43}
\stopFLOWcell

\startFLOWcell
  \name {34}
  \location{3,4}
  \text {基本形}
  \connection [rl] {47}
\stopFLOWcell

\startFLOWcell
  \name {35}
  \location{3,5}
  \text {放缩形}
  \connection [rl] {45}
  \connection [rl] {46}
\stopFLOWcell

\startFLOWcell
  \name {36}
  \location{3,6}
  \text {变量形}
\stopFLOWcell

\startFLOWcell
  \name {37}
  \location{3,7}
  \text {概念及审敛法}
\stopFLOWcell

\startFLOWcell
  \name {41}
  \location{4,3}
  \text {面积大小}
\stopFLOWcell

\startFLOWcell
  \name {42}
  \location{4,1}
  \text {正负号}
\stopFLOWcell

\startFLOWcell
  \name {43}
  \location{4,2}
  \text {差值法}
\stopFLOWcell

\startFLOWcell
  \name {45}
  \location{4,5}
  \text {夹逼准则}
\stopFLOWcell

\startFLOWcell
  \name {46}
  \location{4,6}
  \text {先放缩再找$\frac{i}{n}$}
\stopFLOWcell

\startFLOWcell
  \name {47}
  \location{4,4}
  \text {包含$\frac{i}{n}$}
\stopFLOWcell
\stopFLOWchart
\FLOWchart[example]

%%------------------------
\page


%%------------------------

\usemodule[chart]
\setupFLOWcharts[
height=2.5\lineheight,
width=6\bodyfontsize,
dx=1\bodyfontsize,
dy=0.2\bodyfontsize,
]

\setupFLOWshapes
[framecolor=pragmacolor,
background=color,
backgroundcolor=white,
]

\setupFLOWlines[framecolor=pragmacolor]
\startFLOWchart[example]

\startFLOWcell 
  \name {01}
  \location {0,4}
  \text {一元函数积分学 {\darkgreen 2}}
  \connection [rl] {11}
  \connection [rl] {12}
  \connection [rl] {14}
  \connection [rl] {15}
  \connection [rl] {16}
\stopFLOWcell

\startFLOWcell
  \name {11}
  \location{2,1}
  \text {基本积分公式}
\stopFLOWcell

\startFLOWcell
  \name {12}
  \location{2,3}
  \text {不定积分的计算}
  \connection [rl] {31}
  \connection [rl] {32}
  \connection [rl] {33}
  \connection [rl] {34}
\stopFLOWcell

\startFLOWcell
  \name {14}
  \location{2,5}
  \text {定积分的计算}
  \connection [rl] {35}
  \connection [rl] {36}
  \connection [rl] {37}
\stopFLOWcell

\startFLOWcell
  \name {15}
  \location{2,8}
  \text {变限积分}
  \connection [rl] {38}
\stopFLOWcell

\startFLOWcell
  \name {16}
  \location{2,7}
  \text {反常积分的计算}
\stopFLOWcell

\startFLOWcell
  \name {31}
  \location{3,1}
  \text {凑微分法}
\stopFLOWcell

\startFLOWcell
  \name {32}
  \location{3,3}
  \text {换元法}
  \connection [rl] {41}
  \connection [rl] {42}
  \connection [rl] {43}
\stopFLOWcell

\startFLOWcell
  \name {33}
  \location{3,2}
  \text {分部积分法}
\stopFLOWcell

\startFLOWcell
  \name {34}
  \location{3,4}
  \text {有理函数的积分}
\stopFLOWcell

\startFLOWcell
  \name {35}
  \location{3,5}
  \text {Wallis公式}
\stopFLOWcell

\startFLOWcell
  \name {36}
  \location{3,6}
  \text {对称性}
\stopFLOWcell

\startFLOWcell
  \name {37}
  \location{3,7}
  \text {分段函数}
\stopFLOWcell

\startFLOWcell
  \name {38}
  \location{3,8}
  \text {计算}
  \connection [rl] {45}
  \connection [rl] {46}
  \connection [rl] {47}
  \connection [rl] {48}
\stopFLOWcell

\startFLOWcell
  \name {41}
  \location{4,3}
  \text {根式代换}
\stopFLOWcell

\startFLOWcell
  \name {42}
  \location{4,1}
  \text {三角代换}
\stopFLOWcell

\startFLOWcell
  \name {43}
  \location{4,2}
  \text {倒代换}
\stopFLOWcell

\startFLOWcell
  \name {45}
  \location{4,6}
  \text {换序求导型}
\stopFLOWcell

\startFLOWcell
  \name {46}
  \location{4,7}
  \text {拆分求导型}
\stopFLOWcell

\startFLOWcell
  \name {47}
  \location{4,8}
  \text {换元求导型}
\stopFLOWcell

\startFLOWcell
  \name {48}
  \location{4,5}
  \text {直接求导型}
\stopFLOWcell
\stopFLOWchart
\FLOWchart[example]

%%------------------------
%%------------------------
%% 

\usemodule[chart]
\setupFLOWcharts[
height=2.5\lineheight,
width=6\bodyfontsize,
dx=1\bodyfontsize,
dy=0.2\bodyfontsize,
]

\setupFLOWshapes
[framecolor=pragmacolor,
background=color,
backgroundcolor=white,
]

\setupFLOWlines[framecolor=pragmacolor]
\startFLOWchart[example]

\startFLOWcell 
  \name {01}
  \location {0,4}
  \text {一元函数积分学 {\darkgreen 3}}
  \connection [rl] {11}
  \connection [rl] {12}
\stopFLOWcell

\startFLOWcell
  \name {11}
  \location{2,1}
  \text {研究对象}
  \connection [rl] {31}
  \connection [rl] {32}
  \connection [rl] {33}
  \connection [rl] {34}
  \connection [rl] {35}
  \connection [rl] {36}
\stopFLOWcell

\startFLOWcell
  \name {12}
  \location{2,7}
  \text {研究内容}
  \connection [rl] {38}
\stopFLOWcell

\startFLOWcell
  \name {31}
  \location{3,1}
  \text {函数}
\stopFLOWcell

\startFLOWcell
  \name {32}
  \location{3,3}
  \text {函数列}
\stopFLOWcell

\startFLOWcell
  \name {33}
  \location{3,2}
  \text {参数方程}
\stopFLOWcell

\startFLOWcell
  \name {34}
  \location{3,4}
  \text {偏导函数}
\stopFLOWcell

\startFLOWcell
  \name {35}
  \location{3,5}
  \text {变限积分函数}
\stopFLOWcell

\startFLOWcell
  \name {36}
  \location{3,6}
  \text {微分方程的解函数}
\stopFLOWcell

\startFLOWcell
  \name {38}
  \location{3,7}
  \text {几何应用}
  \connection [rl] {41}
  \connection [rl] {42}
  \connection [rl] {43}
  \connection [rl] {45}
  \connection [rl] {46}
  \connection [rl] {47}
  \connection [rl] {48}
\stopFLOWcell

\startFLOWcell
  \name {41}
  \location{4,3}
  \text {面积}
\stopFLOWcell

\startFLOWcell
  \name {42}
  \location{4,1}
  \text {旋转体的体积}
\stopFLOWcell

\startFLOWcell
  \name {43}
  \location{4,2}
  \text {平均值}
\stopFLOWcell

\startFLOWcell
  \name {45}
  \location{4,6}
  \text {平面曲线的弧长}
\stopFLOWcell

\startFLOWcell
  \name {46}
  \location{4,7}
  \text {旋转曲面的侧面积}
\stopFLOWcell

\startFLOWcell
  \name {47}
  \location{4,4}
  \text {形心坐标}
\stopFLOWcell

\startFLOWcell
  \name {48}
  \location{4,5}
  \text {截面面积已知的V}
\stopFLOWcell
\stopFLOWchart
\FLOWchart[example]

%%------------------------
%%------------------------
%% 

\usemodule[chart]
\setupFLOWcharts[
height=2.5\lineheight,
width=6\bodyfontsize,
dx=1\bodyfontsize,
dy=0.2\bodyfontsize,
]

\setupFLOWshapes
[framecolor=pragmacolor,
background=color,
backgroundcolor=white,
]

\setupFLOWlines[framecolor=pragmacolor]
\startFLOWchart[example]

\startFLOWcell 
  \name {01}
  \location {0,4}
  \text {一元函数积分学 {\darkgreen 4}}
  \connection [rl] {11}
  \connection [rl] {12}
\stopFLOWcell

\startFLOWcell
  \name {11}
  \location{2,1}
  \text {积分等式}
  \connection [rl] {31}
  \connection [rl] {33}
\stopFLOWcell

\startFLOWcell
  \name {12}
  \location{2,6}
  \text {积分不等式}
  \connection [rl] {38}
  \connection [rl] {32}
  \connection [rl] {34}
  \connection [rl] {35}
  \connection [rl] {36}
\stopFLOWcell

\startFLOWcell
  \name {31}
  \location{3,1}
  \text {常用积分等式}
\stopFLOWcell

\startFLOWcell
  \name {32}
  \location{3,3}
  \text {函数单调性}
\stopFLOWcell

\startFLOWcell
  \name {33}
  \location{3,2}
  \text {积分形式中值定理}
\stopFLOWcell

\startFLOWcell
  \name {34}
  \location{3,4}
  \text {被积函数}
  \connection [rl] {41}
  \connection [rl] {42}
  \connection [rl] {43}
  \connection [rl] {45}
  \connection [rl] {46}
  \connection [rl] {47}
  \connection [rl] {48}
\stopFLOWcell

\startFLOWcell
  \name {35}
  \location{3,5}
  \text {夹逼准则极限问题}
\stopFLOWcell

\startFLOWcell
  \name {36}
  \location{3,6}
  \text {曲边梯形的面积}
\stopFLOWcell

\startFLOWcell
  \name {42}
  \location{4,1}
  \text {积分的保号性}
\stopFLOWcell

\startFLOWcell
  \name {43}
  \location{4,2}
  \text {拉格朗日中值定理}
\stopFLOWcell

\startFLOWcell
  \name {45}
  \location{4,6}
  \text {泰勒公式}
\stopFLOWcell

\startFLOWcell
  \name {46}
  \location{4,3}
  \text {放缩法}
\stopFLOWcell

\startFLOWcell
  \name {47}
  \location{4,4}
  \text {分部积分法}
\stopFLOWcell

\startFLOWcell
  \name {48}
  \location{4,5}
  \text {换元法}
\stopFLOWcell
\stopFLOWchart
\FLOWchart[example]

%%------------------------
%%------------------------
%% 

\usemodule[chart]
\setupFLOWcharts[
height=2.5\lineheight,
width=9\bodyfontsize,
dx=1\bodyfontsize,
dy=0.2\bodyfontsize,
]

\setupFLOWshapes
[framecolor=pragmacolor,
background=color,
backgroundcolor=white,
]

\setupFLOWlines[framecolor=pragmacolor]
\startFLOWchart[example]

\startFLOWcell 
  \name {01}
  \location {0,4}
  \text {一元函数积分学 {\darkgreen 5}}
  \connection [rl] {11}
  \connection [rl] {12}
\stopFLOWcell

\startFLOWcell
  \name {11}
  \location{2,1}
  \text {物理应用 (微元法)}
  \connection [rl] {31}
  \connection [rl] {33}
  \connection [rl] {38}
  \connection [rl] {32}
  \connection [rl] {34}
  \connection [rl] {35}
  \connection [rl] {36}
\stopFLOWcell

\startFLOWcell
  \name {31}
  \location{3,1}
  \text {总路程}
\stopFLOWcell

\startFLOWcell
  \name {32}
  \location{3,3}
  \text {变力沿直线做功}
\stopFLOWcell

\startFLOWcell
  \name {33}
  \location{3,2}
  \text {提取物体做功}
\stopFLOWcell

\startFLOWcell
  \name {34}
  \location{3,4}
  \text {静水压力}
\stopFLOWcell

\startFLOWcell
  \name {35}
  \location{3,5}
  \text {细杆质心}
\stopFLOWcell

\startFLOWcell
  \name {36}
  \location{3,6}
  \text {曲边梯形的面积}
\stopFLOWcell
\stopFLOWchart
\FLOWchart[example]

%%------------------------


%%----------------------------------------------------------------
%% 逐步显示

\usemodule[tikz]         


\starttikzpicture[      scale=3,line cap=round
                        axes/.style=,         
                        important line/.style={very thick},
                        information text/.style={rounded corners,fill=red!10,inner sep=1ex} ]
        \draw[xshift=0.0cm]
                node[right,text width=17cm,information text]
                {\bf
%%-----------------------



和差角公式。(牢记)
\index{和差角公式}

 
$\sin (\alpha \pm \beta) =\sin \alpha \sin \beta \pm \cos \alpha\sin\beta$\crlf\medskip
$\cos (\alpha \pm \beta) =\cos \alpha \cos \beta \mp \sin \alpha\sin\beta$\crlf\medskip
$\tan(\alpha\pm \beta) = \displaystyle \frac{\tan\alpha\pm \tan\beta}{1\mp \tan \alpha\tan\beta}$
%%-----------------------
};

\stoptikzpicture

%% 逐步显示
%%----------------------------------------------------------------


%%----------------------------------------------------------------
%% 逐步显示

\usemodule[tikz]         


\starttikzpicture[      scale=3,line cap=round
                        axes/.style=,         
                        important line/.style={very thick},
                        information text/.style={rounded corners,fill=red!10,inner sep=1ex} ]
        \draw[xshift=0.0cm]
                node[right,text width=17cm,information text]
                {\bf
%%-----------------------



倍角公式。(牢记)
\index{倍角公式}

 
$\sin 2\alpha = 2\sin \alpha \cos \alpha$\crlf\medskip
$\cos 2\alpha =\cos^2 \alpha - \sin^2 \alpha 
 = 1-2\sin^2\alpha = 2\cos^2\alpha-1$\crlf\medskip
$ \sin ^2\alpha = \displaystyle \frac{1-\cos 2\alpha}{2}$\crlf\medskip
$ \cos ^2\alpha = \displaystyle \frac{1+\cos 2\alpha}{2}$
%%-----------------------
};

\stoptikzpicture

%% 逐步显示
%%----------------------------------------------------------------



%%----------------------------------------------------------------
%% 逐步显示

\usemodule[tikz]         


\starttikzpicture[      scale=3,line cap=round
                        axes/.style=,         
                        important line/.style={very thick},
                        information text/.style={rounded corners,fill=red!10,inner sep=1ex} ]
        \draw[xshift=0.0cm]
                node[right,text width=17cm,information text]
                {\bf
%%-----------------------



和差与积的关系。(牢记)
\index{和差与积的关系}

 
$ 2\sin\alpha\cos \beta = \sin(\alpha+\beta)+\sin(\alpha-\beta)$\crlf\medskip
$ 2\cos\alpha\sin \beta = \sin(\alpha+\beta) - \sin(\alpha-\beta)$\crlf\medskip

$ 2\cos\alpha\cos \beta = \cos(\alpha+\beta)+\cos(\alpha-\beta)$\crlf\medskip
$ -2\sin\alpha\sin \beta = \cos(\alpha+\beta) - \cos(\alpha-\beta)$\crlf\medskip
 
%%-----------------------
};

\stoptikzpicture

%% 逐步显示
%%----------------------------------------------------------------

%%----------------------------------------------------------------
%% 逐步显示

\usemodule[tikz]         


\starttikzpicture[      scale=3,line cap=round
                        axes/.style=,         
                        important line/.style={very thick},
                        information text/.style={rounded corners,fill=red!10,inner sep=1ex} ]
        \draw[xshift=0.0cm]
                node[right,text width=17cm,information text]
                {\bf
%%-----------------------



和差与积的关系。(牢记)
\index{和差与积的关系}

  

$ \sin \alpha + \sin \beta = 2\sin \displaystyle \frac{\alpha + \beta}{2}\cos \frac{\alpha - \beta}{2}$\crlf\medskip

$ \sin \alpha - \sin \beta = 2\cos \displaystyle \frac{\alpha + \beta}{2}\sin \frac{\alpha - \beta}{2}$\crlf\medskip

$ \cos \alpha + \cos \beta = 2\cos \displaystyle \frac{\alpha + \beta}{2}\cos \frac{\alpha - \beta}{2}$\crlf\medskip

$ \cos \alpha - \cos \beta = -2\sin \displaystyle \frac{\alpha + \beta}{2}\sin \frac{\alpha - \beta}{2}$\crlf\medskip

%%-----------------------
};

\stoptikzpicture

%% 逐步显示
%%----------------------------------------------------------------



\medskip
\FrameTitle{函数极限的计算} % 解答
\index{函数极限}
\index{泰勒公式}


\StartFrame\bf
泰勒公式。(牢记)
\index{牢记}
\index{泰勒公式}

\startitemize[n] 
  \item $e^x
        = \displaystyle  1+x+\frac{x^2}{2!}+\cdots +\frac{x^n}{n!}
        = \sum_{n=0}^{\infty}\frac{x^n}{n!}$
  \item $\displaystyle \sin x = x - \frac{1}{3!}x^3 + \cdots + (-1)^n \frac{1}{(2n+1)!} x^{2n+1} + \cdots$\crlf 
$= \displaystyle \sum_{n=0}^{\infty} (-1)^n \frac{x^{2n+1}}{(2n+1)!}$
  \item $\displaystyle \cos x = 1 - \frac{1}{2!}x^2 + \cdots + (-1)^n \frac{1}{(2n)!} x^{2n} + \cdots$\crlf 
$= \displaystyle \sum_{n=0}^{\infty} (-1)^n \frac{x^{2n}}{(2n)!}$
  \item $\displaystyle \ln (1+x)= x-\frac{1}{2}x^2+\cdots+(-1)^{n-1}\frac{x^n}{n}+\cdots$\crlf
$=\displaystyle \sum_{n=1}^{\infty}(-1)^{n-1}\frac{x^n}{n}$, $-1<x\leq 1$
\stopitemize



\StopFrame

%%----------------------------------
%%----------------------------------



%%----------------------------------------------------------------
%% 逐步显示

\usemodule[tikz]         


\starttikzpicture[      scale=3,line cap=round
                        axes/.style=,         
                        important line/.style={very thick},
                        information text/.style={rounded corners,fill=red!10,inner sep=1ex} ]
        \draw[xshift=0.0cm]
                node[right,text width=17cm,information text]
                {\bf



泰勒公式。(牢记)

 
    $\displaystyle {\red  \frac{1}{1-x}} =1+x+x^2+x^3+\cdots+x^n+\cdots$ 
$\displaystyle = \sum_{n=0}^{\infty}x^n$, $|x|<1$\crlf

    $\displaystyle {\darkgreen \frac{1}{1+x}} =1-x+x^2-\cdots+(-1)^nx^n+\cdots$  
$\displaystyle = \sum_{n=0}^{\infty}(-1)^nx^n$, $|x|<1$

};

\stoptikzpicture

%% 逐步显示
%%----------------------------------------------------------------

\medskip
\FrameTitle{函数极限的计算} % 解答
\index{函数极限}
\index{泰勒公式}


\StartFrame\bf
泰勒公式。(牢记)

\startitemize[n] 
  \item $\displaystyle (1+x)^{\alpha}=1+\alpha x+\frac{\alpha(\alpha -1)}{2}x^2+o(x^2), (x\rightarrow 0)$
  \item $\displaystyle \tan x =x + \frac{1}{3}x^3+o(x^3)$, \quad $(x\rightarrow 0)$
  \item $\displaystyle \arcsin x=x+\frac{1}{6}x^3+o(x^3)$, \quad $(x\rightarrow 0)$
  \item $\displaystyle \arctan x = x -\frac{1}{3}x^3+o(x^3)$, \quad $(x\rightarrow 0)$
\stopitemize



\StopFrame

%%----------------------------------
%%----------------------------------

%%----------------------------------
% 例题模板
%%----------------------------------
\FrameTitle{\bf 奇偶性} % 数学题目


\StartFrame\bf


\startitemize[n]
  \item   设 $f(x)$ 为奇函数, 则 $f'(x)$ 为偶函数。
  \item   设 $f(x)$ 为偶函数, 则 $f'(x)$ 为奇函数。
  \item 设 $f(x)$ 是以 $T$ 为周期的周期函数, 则 $f'(x)$ 是以 $T$ 为周期的周期函数.
  \item 设 $f(x)$ 为奇函数且 $a\neq 0$, 则 $\displaystyle \int _a^x f(t)dt$ 为偶函数.
  \item 设 $f(x)$ 为奇函数且 $a= 0$, 则 $\displaystyle \int _a^x f(t)dt$ 为偶函数.
  \item 设 $f(x)$ 为偶函数且 $a\neq 0$, 则 $\displaystyle \int _a^x f(t)dt$ 的奇偶性不确定.
  \item 设 $f(x)$ 为偶函数且 $a= 0$, 则 $\displaystyle \int _a^x f(t)dt$ 为奇函数.
\stopitemize


\StopFrame

%%----------------------------------
%%----------------------------------





%%----------------------------------
% 例题模板
%%----------------------------------
\FrameTitle{\bf 奇偶性} % 数学题目


\StartFrame\bf


\startitemize[n]
  \item 设 $f(x)$ 是以 $T$ 为周期的周期函数且 $a=0$, 则 \crlf
$\displaystyle \int _a^x f(t)dt$ 是以 T 为周期的周期函数.
  \item 设 $f(x)$ 是以 $T$ 为周期的周期函数且 $\displaystyle \int_0^T f(x)dx=0$, $a\neq 0$, 则 \crlf
 $\displaystyle \int _a^x f(t)dt$ 是以 $T$ 为周期的周期函数.
  \item 设 $f(x)$ 是以 $T$ 为周期的周期函数, 则对任意常数 $a$ 都有 \crlf
 $\displaystyle \int_0^T f(x)dx=\int_a^{a+T} f(x)dx$
\stopitemize


\StopFrame

%%----------------------------------
%%----------------------------------





\page

\FrameTitle{例题} % 题目说明


\StartFrame\bf
设$f(x)$连续,则在下列变上限积分中,必为偶函数的是( \kern2em  )
\definehspace[half][0.5em]

\dontleavehmode% %% blackrules break hmode, usually.
\blackrule[width=0.5em]\hspace[half]%
\blackrule[width=0.5em]\hspace[half]%
\blackrule[width=0.5em]\hspace[half]%
\blackrule[width=0.5em]
\startitemize[n]
  \item (A) $\displaystyle \int_0^x t\left[f(t)+f(-t)dt\right]$
  \item (B) $\displaystyle \int_0^x t\left[f(t)-f(-t)dt\right]$ 
  \item (C) $\displaystyle \int_0^x \left[f(t^2)dt\right]$ 
  \item (D) $\displaystyle \int_0^x \left[f^2(t)dt\right]$
\stopitemize
\StopFrame


%%----------------------------------
%%----------------------------------



%%----------------------------------------------------------------
%% 逐步显示, 开始

\usemodule[tikz]         


\starttikzpicture[      scale=3,line cap=round
                        axes/.style=,         
                        important line/.style={very thick},
                        information text/.style={rounded corners,fill=red!10,inner sep=1ex} ]
        \draw[xshift=0.0cm]
                node[right,text width=17cm,information text]
                {\bf
%%---------------------


解: 奇函数的原函数是偶函数,(A)中被积函数为奇函数,故选(A)。\crlf\medskip

(B)、(C)中被积分函数都是偶函数, 偶函数 $f(x)$ 的原函数只有 $\displaystyle\int_0^x f(t)dt$ 为奇函数, 因为其他原函数与此原函数相差一个常数, 而奇函数加上一个非零常数后就不再是奇函数了。 故(B)、(C)错误。\crlf\medskip

(D)中被积分函数仅能定为非负函数, 故变上限积分不一定是偶函数。
 
 
%%---------------------
};

\stoptikzpicture

%% 逐步显示, 结束
%%----------------------------------------------------------------

\page
\FrameTitle{例题} % 解答


\StartFrame\bf
设 $f(x)$ 是以 $T$ 为周期的可微函数, 则下列函数中以 $T$ 为周期的函数是 \hbox{( \kern2em  )}

(A) $\displaystyle \int_a^xf(t)dt$  \, (B) $\displaystyle \int_a^xf(t^2)dt$ \medskip

 (C) $\displaystyle \int_a^xf'(t^2)dt$  \, (D) $\displaystyle \int_a^xf(t)f'(t)dt$

\StopFrame


%%----------------------------------%%----------------------------------
%%----------------------------------%%----------------------------------





%%----------------------------------------------------------------
%% 逐步显示, 开始

\usemodule[tikz]         


\starttikzpicture[      scale=3,line cap=round
                        axes/.style=,         
                        important line/.style={very thick},
                        information text/.style={rounded corners,fill=red!10,inner sep=1ex} ]
        \draw[xshift=0.0cm]
                node[right,text width=17cm,information text]
                {\bf
%%---------------------


解: (A) 若 $F(x)$ 以 $T$ 为周期, 且 $\displaystyle \int_0^T F(x)dx=0$, 则 $\displaystyle \int_a^x F(t)dt$ 以 $T$ 为周期, 所以 $(A)$  $\displaystyle \int_a^x f(t)dt$ 不一定。\crlf\medskip

(D) $\displaystyle \int_a^x f(t)f'(t)dt=\frac{1}{2}[f(t)]^2|_0^T = 0 $, 则 $(D$ 必以 $T$ 为周期。\crlf\medskip
 


%%---------------------
};

\stoptikzpicture

%% 逐步显示, 结束
%%----------------------------------------------------------------






%%----------------------------------------------------------------
%% 逐步显示, 开始

\usemodule[tikz]         


\starttikzpicture[      scale=3,line cap=round
                        axes/.style=,         
                        important line/.style={very thick},
                        information text/.style={rounded corners,fill=red!10,inner sep=1ex} ]
        \draw[xshift=0.0cm]
                node[right,text width=17cm,information text]
                {\bf
%%---------------------

 
(B) 中的 $f$, $(C)$ 中的 $f'$ 均是以 $T$ 为周期, 但 $f(t^2)$, $f'(t^2)$ 均为复合函数, 未必是周期函数。

(1)如 $f(x)=\sin x$ 为周期函数, 但 $f(x^2)=\sin x^2$ 不是周期函数。 因为 $\sin x^2$ 的零点为 $x^2=k\pi$, 即 $x=\pm \sqrt{k\pi}$, 其相邻两零点间的距离

$\displaystyle d=\sqrt{(k+1)\pi} - \sqrt{k\pi}=\frac{\sqrt{\pi}}{\sqrt{k+1}+\sqrt{k}}$

$k$ 越大, $d$ 越小, 没有周期性。 

(2)同理, 如 $\displaystyle \sin \frac{1}{x}$ 也不是周期函数。

所以(B)、(C)错误。
应选(D)。
 
%%---------------------
};

\stoptikzpicture

%% 逐步显示, 结束
%%----------------------------------------------------------------





\page
% 画图开始的
%%%%-------------------输入代码,开始
\defineframedtext
  [framedcode]
  [strut=yes,
   offset=2mm,
   width=18cm,
   align=right]

\definetyping[code][numbering=line,
                    bodyfont=,
                    before={\startframedcode},
                    after={\stopframedcode}]

\startcode
(* Wolfram Mathematica 11.3.0.0 *)
In[1]:= Plot[Sin[x^2], {x, -2 Pi, 2 Pi}, PlotStyle -> {Thick, Green}] Shift+Enter
\stopcode

\index{Wolfram Mathematica}


\placefigure[force]{}{\externalfigure[2020-09-04-192823.png][height=8cm]}

%%%%-------------------输入代码,结束



%%%%-------------------输入代码,开始
\defineframedtext
  [framedcode]
  [strut=yes,
   offset=2mm,
   width=18cm,
   align=right]

\definetyping[code][numbering=line,
                    bodyfont=,
                    before={\startframedcode},
                    after={\stopframedcode}]

\startcode
(* Wolfram Mathematica 11.3.0.0 *)
In[1]:= Plot[Sin[1/x], {x, -2 Pi, 2 Pi}, PlotStyle -> {Thick, Green}] Shift+Enter
\stopcode

\index{Wolfram Mathematica}


\placefigure[force]{}{\externalfigure[2020-09-04-193308.png][height=8cm]}

%%%%-------------------输入代码,结束



%%%%-------------------输入代码,开始
\defineframedtext
  [framedcode]
  [strut=yes,
   offset=2mm,
   width=18cm,
   align=right]

\definetyping[code][numbering=line,
                    bodyfont=,
                    before={\startframedcode},
                    after={\stopframedcode}]

\startcode
(* Wolfram Mathematica 11.3.0.0 *)
In[1]:= Plot[Sin[1/x], {x, -0.1 Pi, 0.1 Pi}, PlotStyle -> {Thick, Green}] Shift+Enter
\stopcode

\index{Wolfram Mathematica}


\placefigure[force]{}{\externalfigure[2020-09-04-193546.png][height=8cm]}

%%%%-------------------输入代码,结束

%%----------------------------------%%----------------------------------
%%----------------------------------%%----------------------------------

%%----------------------------------
%%----------------------------------

\page
\FrameTitle{例题} % 解答


\StartFrame\bf
计算
\startformula
I=\int_0^{n\pi} \sqrt{1-\sin 2x }\ dx
\stopformula

\StopFrame


%%----------------------------------




%%----------------------------------------------------------------
%% 逐步显示, 开始

\usemodule[tikz]         


\starttikzpicture[      scale=3,line cap=round
                        axes/.style=,         
                        important line/.style={very thick},
                        information text/.style={rounded corners,fill=red!10,inner sep=1ex} ]
        \draw[xshift=0.0cm]
                node[right,text width=17cm,information text]
                {\bf
%%---------------------


解:  被积函数 $\displaystyle \sqrt{1-\sin 2x}=|\cos x-\sin x|$ 以 $\pi$ 为周期, 所以有
\startformula
\startalign
  \NC I \NC = \sum_{k=0}^{n-1} \int_{k\pi}^{(k+1)\pi} |\cos x -\sin x| dx \NR
  \NC  \NC  = n\int_0^{\pi} |\cos x -\sin x|dx \NR
  \NC  \NC = n\int_0^{\frac{\pi}{4}} (\cos x -\sin x) dx + n \int_{\frac{\pi}{4}}^{\pi} (\sin x - \cos x) dx \NR 
  \NC  \NC = 2\sqrt{2}n \NR
\stopalign
\stopformula 
%%---------------------
};

\stoptikzpicture

%% 逐步显示, 结束
%%----------------------------------------------------------------



%%%%-------------------输入代码,开始
\defineframedtext
  [framedcode]
  [strut=yes,
   offset=2mm,
   width=18cm,
   align=right]

\definetyping[code][numbering=line,
                    bodyfont=,
                    before={\startframedcode},
                    after={\stopframedcode}]

\startcode
(* Wolfram Mathematica 11.3.0.0 *)
In[1]:= Plot[Sqrt[1 - Sin[2 x]], {x, -2 Pi, 2 Pi}, 
 PlotStyle -> {Thick, Green}] Shift+Enter
\stopcode

\index{Wolfram Mathematica}


\placefigure[force]{}{\externalfigure[2020-09-04-194736.png][height=8cm]}

%%%%-------------------输入代码,结束

%%----------------------------------


\page
\FrameTitle{积分的大小比较} % 解答

\StartFrame\bf
几何意义
\startitemize[n]
  \item $\displaystyle \int_a^b f(x)dx=F(b)-F(a)$
  \item $\displaystyle \int_{x_0}^x f'(t)dx=f(x)-f(x_0)$
  \item $\displaystyle \int_{-a}^a f(x)dx=2\int_0^a f(x)dx$, \quad 当$f(x)=f(-x)$时
  \item $\displaystyle \int_{-a}^a f(x)dx=0$, \quad 当$f(x)=-f(-x)$时
  \item $\displaystyle \int_a^b f(x)dx=F(b)-F(a)$
\stopitemize

 


\StopFrame


%%----------------------------------



%%----------------------------------


\medskip
\FrameTitle{积分的大小比较} % 解答

\StartFrame\bf
 
保号性
\startitemize[n]
  \item 可以直接看出正负号。 如 $|x|\geq 0$, 当 $x\in [\pi,2\pi]$ 时, 有 $\sin x\leq 0$
  \item 作差 $I_1-I_2$, 再换元。 常令 $\displaystyle x=\pi\pm t$, $\displaystyle x=\frac{\pi}{2}\pm t$
\stopitemize


\StopFrame


%%----------------------------------
%%----------------------------------


\page
\FrameTitle{例题: 2018数学二选择题第5题4分} % 解答

\StartFrame\bf
\index{积分比大小}
\index{2018数学二选择题第5题4分}

比较下列积分的大小: 设
\startformula
M = \int_{-\frac{\pi}{2}}^{\frac{\pi}{2}}\frac{(1+x)^2}{1+x^2}\cdot dx,\quad
N = \int_{-\frac{\pi}{2}}^{\frac{\pi}{2}}\frac{1+x}{e^x}\cdot dx,\quad
K = \int_{-\frac{\pi}{2}}^{\frac{\pi}{2}}\left(1+\sqrt{\cos x}\,\right) \cdot dx
\stopformula 

则 $M$, $N$, $K$ 的大小关系为 (\kern2em) \crlf

$(A)$ $M>N>K$ $(B)$ $M>K>N$ $(C)$ $K>M>N$ $(D)$ $K>N>M$
\StopFrame

%%----------------------------------



%%----------------------------------------------------------------
%% 逐步显示, 开始

\usemodule[tikz]         


\starttikzpicture[      scale=3,line cap=round
                        axes/.style=,         
                        important line/.style={very thick},
                        information text/.style={rounded corners,fill=red!10,inner sep=1ex} ]
        \draw[xshift=0.0cm]
                node[right,text width=17cm,information text]
                {\bf
%%---------------------


解:  这是在同一区间 $\left[-\displaystyle\frac{\pi}{2},\frac{\pi}{2}\right]$ 上比较三个定积分, 其被积函数均连续, 只需比较被积函数的大小。

先利用奇偶函数在对称区间上定积分性质, 化简

$\displaystyle  M= \int_{-\frac{\pi}{2}}^{\frac{\pi}{2}}\frac{(1+x)^2}{1+x^2}\cdot dx 
= \int_{-\frac{\pi}{2}}^{\frac{\pi}{2}} 1 dx + \int_{-\frac{\pi}{2}}^{\frac{\pi}{2}} \frac{2x}{1+x^2}dx
= \int_{-\frac{\pi}{2}}^{\frac{\pi}{2}} 1 dx
$



%%---------------------
};

\stoptikzpicture

%% 逐步显示, 结束
%%----------------------------------------------------------------



%%----------------------------------------------------------------
%% 逐步显示, 开始

\usemodule[tikz]         


\starttikzpicture[      scale=3,line cap=round
                        axes/.style=,         
                        important line/.style={very thick},
                        information text/.style={rounded corners,fill=red!10,inner sep=1ex} ]
        \draw[xshift=0.0cm]
                node[right,text width=17cm,information text]
                {\bf
%%---------------------




现只需在 $\displaystyle \left[-\frac{\pi}{2},\frac{\pi}{2}\right]$ 上比较以下三个函数 $1$, $1+\sqrt{\cos x}$, $\displaystyle \frac{1+x}{e^x}$, 

因为当 $x \in \left(-\displaystyle \frac{\pi}{2},\frac{\pi}{2}\right)$ 时, 有 $1< 1+\sqrt{\cos x}$, 所以

$\displaystyle M = \int_{-\frac{\pi}{2}}^{\frac{\pi}{2}} 1 dx 
< \int_{-\frac{\pi}{2}}^{\frac{\pi}{2}} \left(1+ \sqrt{\cos x} \, \right ) dx = K $


%%---------------------
};

\stoptikzpicture

%% 逐步显示, 结束
%%----------------------------------------------------------------



%%----------------------------------------------------------------
%% 逐步显示, 开始

\usemodule[tikz]         


\starttikzpicture[      scale=3,line cap=round
                        axes/.style=,         
                        important line/.style={very thick},
                        information text/.style={rounded corners,fill=red!10,inner sep=1ex} ]
        \draw[xshift=0.0cm]
                node[right,text width=17cm,information text]
                {\bf
%%---------------------




下面证明: 当 $x \in \left[-\displaystyle \frac{\pi}{2},\frac{\pi}{2} \right] $, $x\neq 0$ 时, 

$\displaystyle\frac{1+x}{e^x}<1 \Longleftrightarrow e^x > 1 + x \Longleftrightarrow f(x)=e^x-x-1>0$

法一: 辅助函数法, 令 $f(x)=e^x-x-1$

法二: 令 $f(x)=e^x-x-1$, 当 $x\neq 0$ 时, 用泰勒公式

$\displaystyle f(x)= f(0) +f'(0)\cdot x + \frac{1}{2}\cdot f''(\xi)\cdot x^2 = \frac{1}{2}\cdot e^{\xi} \cdot x^2 > 0$, $\xi \in (0,x)$

由以上证明可知, $\displaystyle \frac{1+x}{e^x}<1$, 故 $N<M$, 综上 $K>M>N$, 所以选(C) 
%%---------------------
};

\stoptikzpicture

%% 逐步显示, 结束
%%----------------------------------------------------------------



%%%%-------------------输入代码,开始
\page
\defineframedtext
  [framedcode]
  [strut=yes,
   offset=2mm,
   width=18cm,
   align=right]

\definetyping[code][numbering=line,
                    bodyfont=,
                    before={\startframedcode},
                    after={\stopframedcode}]

\startcode
(* Wolfram Mathematica 11.3.0.0 *)
In[1]:=  Plot[{(1 + x)^2/(1 + x^2), (1 + x)/(E^x), 
  1 + Sqrt[Cos[x]]}, {x, -Pi/2, Pi/2}, 
 PlotStyle -> {{Thick, Green, Thickness[0.01]}, {Dashed, Red, 
    Thickness[0.01]}, {Dotted, Blue, Thickness[0.01]}}]  Shift+Enter
\stopcode

\index{Wolfram Mathematica}


\placefigure[force]{}{\externalfigure[2020-09-04-200518.png][height=8cm]}

%%%%-------------------输入代码,结束
%%----------------------------------


\page
\FrameTitle{例题} % 解答

\StartFrame\bf
比较 $I_1$, $I_2$, $I_3$ 的大小, 其中
\startformula
I_k=\int_0^{k\pi}e^{x^2}\cdot\sin x \cdot dx,\quad (k=1,2,3)
\stopformula 
则有 (\kern2em)

$(A)$ $I_1<I_2<I_3$ \quad $(B)$ $I_3<I_2<_1$\quad $(C)$ $I_2 <I_3<I_1$ \quad $(D)$ $I_2<I_1<I_3$
\StopFrame

%%----------------------------------





%%----------------------------------------------------------------
%% 逐步显示, 开始

\usemodule[tikz]         


\starttikzpicture[      scale=3,line cap=round
                        axes/.style=,         
                        important line/.style={very thick},
                        information text/.style={rounded corners,fill=red!10,inner sep=1ex} ]
        \draw[xshift=0.0cm]
                node[right,text width=17cm,information text]
                {\bf
%%---------------------




解: 首先, 由 $I_2 = I_1 + \int_{\pi} ^{2\pi} e^{x^2} \sin x dx$ 及 
$\displaystyle \int_{\pi} ^{2\pi} e^{x^2} \sin x dx <0$, 可得 $I_2 < I_1$。


%%---------------------
};

\stoptikzpicture

%% 逐步显示, 结束
%%----------------------------------------------------------------







%%----------------------------------------------------------------
%% 逐步显示, 开始

\usemodule[tikz]         


\starttikzpicture[      scale=3,line cap=round
                        axes/.style=,         
                        important line/.style={very thick},
                        information text/.style={rounded corners,fill=red!10,inner sep=1ex} ]
        \draw[xshift=0.0cm]
                node[right,text width=17cm,information text]
                {\bf
%%---------------------



其次, $I_3 = I_1 + \displaystyle \int_{\pi} ^{3\pi} e^{x^2} \sin x dx$, 其中

$\displaystyle \int_{\pi} ^{3\pi} e^{x^2} \sin x dx= \int_{\pi} ^{2\pi} e^{x^2} \sin x dx + \int_{2\pi} ^{3\pi} e^{x^2} \sin x dx$

$= \displaystyle \int_{\pi} ^{2\pi} e^{x^2} \sin x dx + \int_{\pi} ^{2\pi} e^{(y+\pi)^2} \sin (y+\pi) dx$

$= \displaystyle \int_{\pi} ^{2\pi} \left [e^{x^2} - e^{(x+\pi)^2} \right ] \sin x dx >0$
 
所以, $I_3 > I_1$, 从而 $I_2 <I_1 <I_3$, 故选 $(D)$。
%%---------------------
};

\stoptikzpicture

%% 逐步显示, 结束
%%----------------------------------------------------------------




%%----------------------------------


\page
\FrameTitle{和式极限,定积分定义} % 解答

\StartFrame\bf

基本形
\startitemize[n]
  \item $n+i$ 或 $an+bi$, $ab\neq 0$ 或 $\displaystyle n+i=n\left(1+\frac{i}{n}\right)$
  \item $n^2+i^2$ 或 $\displaystyle n^2+i^2=n^2\left [1+\left(\frac{i}{n}\right)^2\right ]$
  \item $n^2+ni$ 或 $\displaystyle n^2+ni=n^2\left(1+\frac{i}{n}\right)$
  \item $\displaystyle \frac{i}{n}$
\stopitemize
定积分定义

\startitemize[n]
  \item $\displaystyle \lim_{n\rightarrow \infty}\sum^n_{i=1}f\left(0+\frac{1-0}{n}i\right)\cdot \frac{1-0}{n}=\int_0^1f(x)dx$
  \item $\displaystyle \lim_{n\rightarrow \infty}\sum^{n-1}_{i=0}f\left(0+\frac{1-0}{n}i\right)\cdot \frac{1-0}{n}=\int_0^1f(x)dx$
\stopitemize
\StopFrame

%%----------------------------------
%%----------------------------------


\page
\FrameTitle{和式极限,定积分定义} % 解答

\StartFrame\bf
变形
\startitemize[n]
  \item $n^2+i$,  找通项, 放大缩小, 夹逼准则。
  \item $\displaystyle \frac{i^2+1}{n^2}$, 则 $\displaystyle \left(\frac{i}{n}\right)^2<\frac{i^2+1}{n^2}<\left(\frac{i+1}{n}\right)^2$
  \item 通项中含有 $\displaystyle \frac{x}{n}\cdot i$, 则\crlf
\startformula
 \lim_{n\rightarrow\infty}\sum_{i=1}^n f\left(0+\frac{x-0}{n}i\right)\cdot\frac{x-0}{n}=\int_0^xf(t)dt
\stopformula
\stopitemize
\StopFrame

%%----------------------------------
%%----------------------------------


\page
\FrameTitle{例题} % 解答

\StartFrame\bf
\index{定积分定义}
计算
\startformula
I =  \lim_{n\rightarrow\infty}\frac{1}{n^2}\sum_{i=1}^n i\cdot\sin \frac{i}{n}
\stopformula

考点: 定积分定义。
\StopFrame

%%----------------------------------



%%----------------------------------------------------------------
%% 逐步显示, 开始

\usemodule[tikz]         


\starttikzpicture[      scale=3,line cap=round
                        axes/.style=,         
                        important line/.style={very thick},
                        information text/.style={rounded corners,fill=red!10,inner sep=1ex} ]
        \draw[xshift=0.0cm]
                node[right,text width=17cm,information text]
                {\bf
%%---------------------




解: $\displaystyle I = \lim_{n\rightarrow\infty}\sum_{i=1}^n \frac{1}{n}\cdot \frac{i}{n}\cdot\sin \frac{i}{n}$

$\displaystyle  = \int_0^1 x \sin x dx$

$\displaystyle  = -x\cos x |_0^1 +\int_0^1 \cos x dx$

$= \sin 1 -\cos 1$

%%---------------------
};

\stoptikzpicture

%% 逐步显示, 结束
%%----------------------------------------------------------------


%%----------------------------------


\page
\FrameTitle{例题} % 解答

\StartFrame\bf
\index{等价无穷小}
\index{定积分定义}

计算
\startformula
I= \lim_{n\rightarrow\infty}\sum_{i=1}^n  \frac{\sin\frac{\pi}{n}}{2+\cos \frac{i\pi}{n}}
\stopformula

考点: 等价无穷小。 定积分定义。

\StopFrame

%%----------------------------------



%%----------------------------------------------------------------
%% 逐步显示, 开始

\usemodule[tikz]         


\starttikzpicture[      scale=3,line cap=round
                        axes/.style=,         
                        important line/.style={very thick},
                        information text/.style={rounded corners,fill=red!10,inner sep=1ex} ]
        \draw[xshift=0.0cm]
                node[right,text width=17cm,information text]
                {\bf
%%---------------------




解: 
\startformula
\startalign
  \NC  I  \NC  = \lim_{n\rightarrow\infty}\sum_{i=1}^n  \frac{\frac{\pi}{n}}{2+\cos \frac{i\pi}{n}} 
               = \pi\lim_{n\rightarrow\infty}\sum_{i=1}^n  \frac{1}{2+\cos \frac{i\pi}{n}}\cdot \frac{1}{n} \NR
  \NC \NC  = \pi \int_0^1 \frac{dx}{2+\cos \pi x} 
           = \pi \int_0^1 \frac{dx}{1+2\cos ^2 \frac{\pi x}{2}} \NR
  \NC  \NC = 2\int_0^1 \frac{d (\tan \frac{\pi x}{2})}{3+\tan ^2\frac{\pi x}{2}} 
           = \frac{2}{\sqrt{3}}\left.\arctan \frac{\tan \frac{\pi x}{2}}{\sqrt{3}} \right|_0^1 
           = \frac{\pi }{\sqrt{3}}
\stopalign
\stopformula



%%---------------------
};

\stoptikzpicture

%% 逐步显示, 结束
%%----------------------------------------------------------------


%%----------------------------------


\page
\FrameTitle{反常积分} % 解答

\StartFrame\bf
\index{反常积分}
\index{瑕点}
\index{广义积分}

概念
\startitemize[n]
  \item $\displaystyle \int_a^{+\infty} f(x)dx$ 称为无穷区间上的反常积分。
  \item $\displaystyle \int_a^bf(x)dx$, 其中 $\displaystyle \lim_{x\rightarrow a^+} f(x)=\infty$, 称 $a$ 为瑕点, 该积分为无界函数的反常积分。
\stopitemize

判別
\startitemize[n]
  \item 要求每个积分有且仅有一个瑕点。
\stopitemize

\StopFrame

%%----------------------------------
%%----------------------------------

\medskip

\FrameTitle{反常积分} % 解答

\StartFrame\bf
\index{反常积分}
\index{瑕点}
\index{广义积分}

比较对象,牢记。\index{牢记}
\startitemize[n]
  \item 当 $0<p<1$ 时, $\displaystyle \int_0^1 \frac{1}{x^p}dx$ 收敛。 $x\rightarrow 0^+$时,  $x^p$ 趋于 $0$ 的速度不快, 其倒数 $\displaystyle \frac{1}{x^p}$ 趋于 $+\infty$ 的速度亦不快, 积分收敛。
  \item 当 $p\geq 1$ 时, $\displaystyle \int_0^1 \frac{1}{x^p}dx$ 发散。 $x\rightarrow 0^+$ 时, $x^p$ 趋于 $0$ 的速度过快, 其倒数 $\displaystyle \frac{1}{x^p}$ 趋于 $+\infty$ 的速度亦过快, 积分发散。
\stopitemize

\StopFrame

%%----------------------------------
%%----------------------------------

\medskip

\FrameTitle{反常积分} % 解答

\StartFrame\bf
\index{反常积分}
\index{瑕点}
\index{广义积分}

比较对象,牢记。被积函数变形。\index{牢记}
\startitemize[n]
  \item 当 $0<p<1$ 时, $\displaystyle \int_0^1 \frac{1}{\sin^px}dx$ 收敛。 $x\rightarrow 0^+$ 时,  $\sin^px$ 趋于 $0$ 的速度不快, 其倒数 $\displaystyle \frac{1}{\sin^px}$ 趋于 $+\infty$ 的速度亦不快, 积分收敛。
  \item 当 $p\geq 1$ 时, $\displaystyle \int_0^1 \frac{1}{\sin^px}dx$ 发散。 $x\rightarrow 0^+$ 时, $\sin^px$ 趋于 $0$ 的速度过快, 其倒数 $\displaystyle \frac{1}{\sin^px}$ 趋于 $+\infty$ 的速度亦过快, 积分发散。
\stopitemize

\StopFrame

%%----------------------------------
%%----------------------------------

\medskip

\FrameTitle{反常积分} % 解答

\StartFrame\bf
\index{反常积分}
\index{瑕点}
\index{广义积分}

比较对象,牢记。 积分区间变形。\index{牢记}
\startitemize[n]
  \item 当 $0<p<1$ 时, $\displaystyle \int_0^{\frac{\pi}{2}} \frac{1}{\sin^px}dx$ 收敛。 $x\rightarrow 0^+$ 时, $\sin^px$ 趋于 $0$ 的速度不快, 其倒数 $\displaystyle \frac{1}{\sin^px}$ 趋于 $+\infty$ 的速度亦不快, 积分收敛。
  \item 当 $p\geq 1$ 时, $\displaystyle \int_0^{\frac{\pi}{2}} \frac{1}{\sin^px}dx$ 发散。  $x\rightarrow 0^+$ 时, $\sin^px$ 趋于 $0$ 的速度过快, 其倒数 $\displaystyle \frac{1}{\sin^px}$ 趋于 $+\infty$ 的速度亦过快, 积分发散。
\stopitemize

\StopFrame

%%----------------------------------
%%----------------------------------

\medskip

\FrameTitle{反常积分} % 解答

\StartFrame\bf
\index{反常积分}
\index{瑕点}
\index{广义积分}

比较对象,牢记。\index{牢记}
\startitemize[n]
  \item 当 $p\leq 1$ 时,$\displaystyle \int_1^{+\infty} \frac{1}{x^p}dx$ 发散。 $x\rightarrow +\infty$ 时, $x^p$ 趋于 $+\infty$ 的速度不够快, 其倒数 $\displaystyle \frac{1}{x^p}$ 趋于 $0$ 的速度亦不够快, 积分发散。
  \item 当 $p> 1$ 时, $\displaystyle \int_1^{+\infty} \frac{1}{x^p}dx$ 收敛。 $x\rightarrow +\infty$ 时, $x^p$ 趋于 $+\infty$ 的速度够快, 其倒数 $\displaystyle \frac{1}{x^p}$ 趋于 $0$ 的速度亦够快, 积分收敛。
\stopitemize

\StopFrame

%%----------------------------------
%%----------------------------------

\medskip

\FrameTitle{反常积分} % 解答

\StartFrame\bf
\index{反常积分}
\index{瑕点}
\index{广义积分}
\index{被积函数变形}

比较对象,牢记。被积函数变形。\index{牢记}
\startitemize[n]
  \item 当 $ax+b\geq k >0$, $p\leq 1$ 时, $\displaystyle \int_1^{+\infty} \frac{1}{(ax+b)^p}dx$ 发散。 $x\rightarrow +\infty$ 时,  $(ax+b)^p$ 趋于 $+\infty$ 的速度不够快, 其倒数 $\displaystyle \frac{1}{(ax+b)^p}$  趋于 $0$ 的速度亦不够快, 积分发散。
  \item 当 $ax+b\geq k >0$, $p> 1$ 时, $\displaystyle \int_1^{+\infty} \frac{1}{(ax+b)^p}dx$ 收敛。  $x\rightarrow +\infty$ 时,  $(ax+b)^p$ 趋于 $+\infty$ 的速度够快, 其倒数 $\displaystyle \frac{1}{(ax+b)^p}$ 趋于 $0$ 的速度亦够快, 积分收敛。
\stopitemize

\StopFrame

%%----------------------------------
%%----------------------------------


\page
\FrameTitle{例题} % 解答


\StartFrame\bf\index{反常积分}
设 $a>0$, $b>0$, 反常积分 $ \displaystyle \int_0^{+\infty}\frac{1}{x^a(2021+x)^b}dx$ 收敛, 则 (\kern2em)
\medskip

$(A)$ $a<1$ 且 $b>1$ \qquad\qquad $(B)$ $a>1$ 且 $b>1$ 

$(C)$ $a<1$ 且 $a+b>1$ \qquad $(D)$ $a>1$ 且 $a+b >1$
 
\StopFrame

%%----------------------------------







%%----------------------------------------------------------------
%% 逐步显示, 开始

\usemodule[tikz]         


\starttikzpicture[      scale=3,line cap=round
                        axes/.style=,         
                        important line/.style={very thick},
                        information text/.style={rounded corners,fill=red!10,inner sep=1ex} ]
        \draw[xshift=0.0cm]
                node[right,text width=17cm,information text]
                {\bf
%%---------------------
 

解:  $\displaystyle \int_0^{+\infty}\frac{1}{x^a(2021+x)^b}dx 
= \int_0^1 +\int_1^{+\infty}\frac{1}{x^a(2021+x)^b}dx
= I_1 + I_2$



%%---------------------
};

\stoptikzpicture

%% 逐步显示, 结束
%%----------------------------------------------------------------







%%----------------------------------------------------------------
%% 逐步显示, 开始

\usemodule[tikz]         


\starttikzpicture[      scale=3,line cap=round
                        axes/.style=,         
                        important line/.style={very thick},
                        information text/.style={rounded corners,fill=red!10,inner sep=1ex} ]
        \draw[xshift=0.0cm]
                node[right,text width=17cm,information text]
                {\bf
%%---------------------
 

\medskip 对于 $I_1$, 当 $x\rightarrow 0^+$ 时, $2021+x \rightarrow 2021^+$, 不是无穷小量, 故只要看 $x^a$ 即可, 即 $I_1$ 与 $\displaystyle \int_0^1 \frac{1}{x^a}dx$ 同敛散, 于是 $a<1$。




%%---------------------
};

\stoptikzpicture

%% 逐步显示, 结束
%%----------------------------------------------------------------







%%----------------------------------------------------------------
%% 逐步显示, 开始

\usemodule[tikz]         


\starttikzpicture[      scale=3,line cap=round
                        axes/.style=,         
                        important line/.style={very thick},
                        information text/.style={rounded corners,fill=red!10,inner sep=1ex} ]
        \draw[xshift=0.0cm]
                node[right,text width=17cm,information text]
                {\bf
%%---------------------
 
\medskip 对于 $I_2$, 当 $x\rightarrow +\infty$, $2021+x \rightarrow +\infty$, 且与 $x\rightarrow+\infty$ 的速度一样, 所以 $I_2$ 与  $\displaystyle \int_1^{+\infty}\frac{1}{x^ax^b}dx = \int_1^{+\infty}\frac{1}{x^{a+b}}dx$ 同敛散, 于是 $a+b>1$。

%%---------------------
};

\stoptikzpicture

%% 逐步显示, 结束
%%----------------------------------------------------------------






%%----------------------------------------------------------------
%% 逐步显示, 开始

\usemodule[tikz]         


\starttikzpicture[      scale=3,line cap=round
                        axes/.style=,         
                        important line/.style={very thick},
                        information text/.style={rounded corners,fill=red!10,inner sep=1ex} ]
        \draw[xshift=0.0cm]
                node[right,text width=17cm,information text]
                {\bf
%%---------------------
 
\medskip 综上, $a<1$ 且 $a+b>1$, 故选 $(C)$。


%%---------------------
};

\stoptikzpicture

%% 逐步显示, 结束
%%----------------------------------------------------------------






%%%%-------------------输入代码,开始
\page
\defineframedtext
  [framedcode]
  [strut=yes,
   offset=2mm,
   width=18cm,
   align=right]

\definetyping[code][numbering=line,
                    bodyfont=,
                    before={\startframedcode},
                    after={\stopframedcode}]

\startcode
(* Wolfram Mathematica 11.3.0.0 *)
In[1]:=  Plot[1/(x^{0.5} (2021 + x)^{1.5}), {x, 0, 2 Pi}, 
 PlotStyle -> {Thick, Green}]  Shift+Enter

In[2]:= Plot[1/(x^{0.4} (2021 + x)^{0.1}), {x, 0, 2 Pi}, 
 PlotStyle -> {Thick, Green}]  Shift+Enter

In[3]:= Plot[1/(x^2 (2021 + x)^{-3}), {x, 0, 2 Pi}, 
 PlotStyle -> {Thick, Green}]  Shift+Enter
\stopcode

\index{Wolfram Mathematica}

 

%%%%-------------------输入代码,结束



%%%%-------------------输入代码,开始
\medskip
\defineframedtext
  [framedcode]
  [strut=yes,
   offset=2mm,
   width=18cm,
   align=right]

\definetyping[code][numbering=line,
                    bodyfont=,
                    before={\startframedcode},
                    after={\stopframedcode}]

\startcode
(* Wolfram Mathematica 11.3.0.0 *)
In[1]:=  Plot[1/(x^{0.5} + x^{1.5}), {x, 0, 2 Pi}, 
 PlotStyle -> {Thick, Green}]  Shift+Enter

In[2]:= Plot[1/(x^{0.4} + x^{0.1}), {x, 0, 2 Pi}, 
 PlotStyle -> {Thick, Green}]  Shift+Enter

In[3]:= Plot[1/(x^2 + x^{-3}), {x, 0, 2 Pi}, 
 PlotStyle -> {Thick, Green}]  Shift+Enter
\stopcode

\index{Wolfram Mathematica}

 

%%%%-------------------输入代码,结束
%%----------------------------------



\page
\FrameTitle{例题} % 解答

\StartFrame\bf\index{反常积分}
设 $a>b>0$, 反常积分 $\displaystyle \int_0^{+\infty}\frac{1}{x^a+x^b}dx$ 收敛, 则(\kern2em)

$(A)$ $a>1$ 且 $b>1$  $(B)$ $a>1$ 且 $b<1$ 

$(C)$ $a<1$ 且 $a+b>1$  $(D)$ $a<1$ 且 $b<1$

\StopFrame

%%----------------------------------





%%----------------------------------------------------------------
%% 逐步显示, 开始

\usemodule[tikz]         


\starttikzpicture[      scale=3,line cap=round
                        axes/.style=,         
                        important line/.style={very thick},
                        information text/.style={rounded corners,fill=red!10,inner sep=1ex} ]
        \draw[xshift=0.0cm]
                node[right,text width=17cm,information text]
                {\bf
%%---------------------
 

解:  $\displaystyle \int_0^{+\infty}\frac{1}{x^a + x^b}dx 
= \int_0^1 +\int_1^{+\infty}\frac{1}{x^a + x^b}dx
= I_1 + I_2$




%%---------------------
};

\stoptikzpicture

%% 逐步显示, 结束
%%----------------------------------------------------------------





%%----------------------------------------------------------------
%% 逐步显示, 开始

\usemodule[tikz]         


\starttikzpicture[      scale=3,line cap=round
                        axes/.style=,         
                        important line/.style={very thick},
                        information text/.style={rounded corners,fill=red!10,inner sep=1ex} ]
        \draw[xshift=0.0cm]
                node[right,text width=17cm,information text]
                {\bf
%%---------------------
 

\medskip 对于 $I_1$, 当 $x\rightarrow 0^+$ 时, 由于 $a>b>0$, 故 $x^b$ 趋于 $0$ 的速度慢于 $x^a$ 趋于 $0$ 的速度, $x^a+x^b \sim x^b$, 于是 $I_1$ 与 $\displaystyle \int_0^1 \frac{1}{x^b}dx$ 同敛散, 则 $b<1$。


%%---------------------
};

\stoptikzpicture

%% 逐步显示, 结束
%%----------------------------------------------------------------





%%----------------------------------------------------------------
%% 逐步显示, 开始

\usemodule[tikz]         


\starttikzpicture[      scale=3,line cap=round
                        axes/.style=,         
                        important line/.style={very thick},
                        information text/.style={rounded corners,fill=red!10,inner sep=1ex} ]
        \draw[xshift=0.0cm]
                node[right,text width=17cm,information text]
                {\bf
%%---------------------
 


\medskip 对于 $I_2$, 当 $x\rightarrow +\infty$, 由于 $a>b>0$, 故 $x^a$ 趋于 $+\infty$ 的速度快于 $x^b$ 趋于 $+\infty$ 的速度, $x^a+x^b$ 与 $ x^a$ 为等价无穷大量, 于是 $I_2$ 与  $\displaystyle \int_1^{+\infty}\frac{1}{x^a}dx$ 同敛散, 则 $a>1$。


%%---------------------
};

\stoptikzpicture

%% 逐步显示, 结束
%%----------------------------------------------------------------






%%----------------------------------------------------------------
%% 逐步显示, 开始

\usemodule[tikz]         


\starttikzpicture[      scale=3,line cap=round
                        axes/.style=,         
                        important line/.style={very thick},
                        information text/.style={rounded corners,fill=red!10,inner sep=1ex} ]
        \draw[xshift=0.0cm]
                node[right,text width=17cm,information text]
                {\bf
%%---------------------
 


\medskip 综上, $a>1$ 且 $b<1$, 故选(B)。


%%---------------------
};

\stoptikzpicture

%% 逐步显示, 结束
%%----------------------------------------------------------------




%%----------------------------------



\page

\medskip
\FrameTitle{基本积分公式} % 解答

\StartFrame\bf\index{积分公式}

\startitemize[n]
  \item $\displaystyle \int \frac{dx}{ax+b}=\frac{1}{a}\ln|ax+b|+C$\medskip
  \item $\displaystyle \int (ax+b)^{\mu}dx=\frac{1}{a(\mu+1)}(ax+b)^{\mu +1}+C$,  ($\mu\neq -1$)\medskip
  \item $\displaystyle \int \frac{dx}{x(ax+b)}=-\frac{1}{b}\ln \left\|\frac{ax+b}{x}\right\| +C$\medskip
  \item $\displaystyle \int \frac{1}{x^2+a^2}dx=\frac{1}{a}\arctan \frac{x}{a}+C$\medskip
  \item $\displaystyle \int \frac{1}{x^2-a^2}dx=\frac{1}{2a}\ln \left|\frac{x-a}{x+a}\right|+C$\medskip
  \item $\displaystyle \int \frac{x}{ax^2+b}dx=\frac{1}{2a}\ln |ax^2+b|+C$\medskip
  \item $\displaystyle \int \frac{x^2}{ax^2+b}dx=\frac{x}{a}-\frac{b}{a}\int \frac{x^2}{ax^2+b}dx$\medskip
  \item $\displaystyle \int \frac{dx}{x(ax^2+b)}=\frac{1}{2b}\ln \left|\frac{x^2}{ax^2+b}\right|+C$
\stopitemize


\StopFrame

%%----------------------------------
%%----------------------------------

\medskip
\FrameTitle{基本积分公式} % 解答

\StartFrame\bf\index{积分公式}

\startitemize[n]
  \item $\displaystyle \int \frac{dx}{\sqrt{x^2+a^2}}=arsh\frac{x}{a}+C_1=\ln \left(x+\sqrt{x^2+a^2}\right)+C$\medskip
  \item $\displaystyle \int \frac{dx}{\sqrt{x^2-a^2}}=\frac{x}{|x|}arsh\frac{|x|}{a}+C_1=\ln \left|x+\sqrt{x^2+a^2}\right|+C$\medskip
  \item $\displaystyle \int \frac{dx}{\sqrt{a^2-x^2}}=\arcsin\frac{x}{a}+C$\medskip
  \item $\displaystyle \int \frac{dx}{\sqrt{ax^2+bx+c}}=\frac{1}{\sqrt{a}}\ln\left |2ax+b+2\sqrt{a}\sqrt{ax^2+bx+c}\right|+C$\medskip
  \item $\displaystyle \int\sin x dx=-\cos x +C $\medskip
  \item $\displaystyle \int\cos x dx=\sin x +C$\medskip
  \item $\displaystyle \int\tan x dx = -\ln |\cos x|+C$\medskip
  \item $\displaystyle \int \cot x dx = \ln |\sin x|+C$
\stopitemize


\StopFrame

%%----------------------------------
%%----------------------------------

\medskip
\FrameTitle{基本积分公式} % 解答

\StartFrame\bf\index{积分公式}

\startitemize[n]
  \item $\displaystyle \int\sec x dx= \ln \left|\tan \left( \frac{\pi}{4}+\frac{x}{2} \right) \right|+C=\ln|\sec x+\tan x|+C$\medskip
  \item $\displaystyle \int\csc x dx= \ln \left|\tan\frac{x}{2} \right|+C=\ln|\csc x -\cot x|+C$\medskip
  \item $\displaystyle \int \sec^2x dx = \tan x +C$\medskip
  \item $\displaystyle \int \csc^2 x dx = - \cot x+C$\medskip
  \item $\displaystyle \int \sin^2x dx = \frac{x}{2}-\frac{1}{4}\sin 2x +C$\medskip
  \item $\displaystyle \int \cos^2x dx = \frac{x}{2}+\frac{1}{4}\sin 2x +C$\medskip
  \item $\displaystyle \int \arcsin \frac{x}{a} dx = x\arcsin \frac{x}{a}+\sqrt{a^2-x^2}+C$\medskip
  \item $\displaystyle \int a^x dx=\frac{1}{\ln a}a^x+C$
\stopitemize


\StopFrame

%%----------------------------------
%%----------------------------------

\medskip
\FrameTitle{基本积分公式} % 解答

\StartFrame\bf\index{积分公式}

\startitemize[n]
  \item $\displaystyle \int e^{ax}dx=\frac{1}{a}e^{ax}+C$\medskip
  \item $\displaystyle \int \ln x dx= x\ln x -x+C$\medskip
  \item $\displaystyle \int \frac{dx}{x\ln x}=\ln |\ln x|+C$\medskip
  \item $\displaystyle \int _{-\pi}^{\pi}\cos nx dx = \int _{-\pi}^{\pi}\sin nx dx = 0$, $n\in \mathbb{Z}$\medskip
  \item $\displaystyle \int _{-\pi}^{\pi}\cos mx\sin nx dx = 0$, $m,n\in \mathbb{Z}$\medskip
  \item $\displaystyle \int _{-\pi}^{\pi}\cos mx\cos nx dx =\int _{-\pi}^{\pi}\sin mx\sin nx dx = 0$, $m \neq n$\medskip
  \item $\displaystyle \int _{-\pi}^{\pi}\cos mx\cos nx dx =\int _{-\pi}^{\pi}\sin mx\sin nx dx =  \pi$, $m = n$\medskip
  \item $I_n=\displaystyle \int_0^{\frac{\pi}{2}} \sin^n x dx=\int_0^{\frac{\pi}{2}} \cos^n x dx$, $I_n=\frac{n-1}{n}I_{n-2}$, $I_0=\frac{\pi}{2}$, $I_1=1$
\index{Wallis公式}
\stopitemize


\StopFrame

%%----------------------------------
%%----------------------------------

\medskip
\FrameTitle{基本积分公式} % 解答

\StartFrame\bf\index{积分公式}

\startitemize[n]
  \item $\displaystyle \int \frac{1}{x^2} dx=-\frac{1}{x}+C$\medskip
  \item $\displaystyle \int \frac{1}{\sqrt{x}} dx=2\sqrt{x}+C$\medskip
  \item $\displaystyle \int \frac{1}{x} dx=\ln |x|+C$\medskip
  \item $\displaystyle \int \tan^2 x dx= \tan x -x +C$\medskip
  \item $\displaystyle \int \cot^2 x dx=-\cot x -x +C$
\stopitemize


\StopFrame

%%----------------------------------
%%----------------------------------
\page
\medskip
\FrameTitle{凑微分法} % 解答

\StartFrame\bf\index{凑微分}
思想
\startitemize[n]
  \item $\displaystyle \int f[g(x)]g'(x)dx=\int f[g(x)]d[g(x)]=\int f(u)du$
  \item 当被积分函数比较复杂时, 拿出一部分放到$d$ 后面去, 若能凑成 $\int f(u) du$ 的形式, 则凑微分成功。
  \item 计算 $\displaystyle \int \frac{\ln ^5x}{x}dx$
  \item 解: $\displaystyle \int \frac{\ln ^5x}{x}dx
= \int \left(\ln  x\right)^5 \cdot \frac{1}{x}dx 
= \int \ln^5 xd(\ln x) 
=\frac{\ln ^6x}{6} +C$
\stopitemize


\StopFrame

%%----------------------------------
%%----------------------------------

\medskip
\FrameTitle{凑微分法} % 解答

\StartFrame\bf\index{凑微分}
方法
\startitemize[n]
  \item $\displaystyle dx=\frac{1}{a}d(ax+b),a\neq 0$
  \item $\displaystyle x^kdx=\frac{1}{k+1}d\left(x^{k+1}\right),k\neq -1$
\stopitemize


\StopFrame

%%----------------------------------
%%----------------------------------

\medskip
\FrameTitle{凑微分法} % 解答

\StartFrame\bf\index{凑微分}
\index{牢记}

牢记
\startitemize[n]
  \item $\displaystyle xdx=\frac{1}{2}d(x^2)$,
  \item $\displaystyle \sqrt{x}dx=\frac{2}{3}d\left(x^{\frac{3}{2}}\right)$
  \item $\displaystyle \frac{1}{\sqrt{x}}dx=2d\left(\sqrt{x}\right)$,
  \item $\displaystyle \frac{dx}{x^2}=d\left(-\frac{1}{x}\right)$,
  \item $\displaystyle \frac{1}{x}dx=d\left(\ln|x|\right)$
  \item $e^xdx=d(e^x)$,
  \item $\displaystyle a^xdx=\frac{1}{\ln a}d(a^x),a>0,a\neq 1$
\stopitemize


\StopFrame

%%----------------------------------
%%----------------------------------

\medskip
\FrameTitle{凑微分法} % 解答

\StartFrame\bf\index{凑微分}
\index{牢记}

牢记
\startitemize[n]
  \item $\sin x dx=d(-\cos x)$,
  \item $\cos xdx=d(\sin x)$
  \item $\displaystyle \frac{dx}{\cos ^2 x}=\sec^2x dx=d(\tan x)$,
  \item $\displaystyle \frac{dx}{\sin ^2 x}=\csc^2x dx=d(-\cot x)$
  \item $\displaystyle \frac{1}{1+x^2}dx = d(\arctan x)$,
  \item $\displaystyle \frac{1}{1-x^2}dx = d(\arcsin x)$ 
\stopitemize


\StopFrame

%%----------------------------------
%%----------------------------------

\medskip
\FrameTitle{凑微分法} % 解答

\StartFrame\bf
\index{牢记}

练习
\startitemize[n]
  \item 计算 $ I= \displaystyle \int \frac{\sqrt{x}}{\sqrt{4-x^3}}dx$
  \crlf\medskip 解: 

$ I= \displaystyle \frac{2}{3}\int \frac{d\left(x^{\frac{3}{2}}\right)}{\sqrt{4-\left(x^{\frac{3}{2}}\right)^2}}
                    = \frac{2}{3}\int \frac{d\left(\displaystyle \frac{1}{2}x^{\frac{3}{2}}\right)}{\sqrt{1-\left(\displaystyle \frac{1}{2}x^{\frac{3}{2}}\right)^2}}
                    = \frac{2}{3}\arcsin \left(\frac{1}{2}x^{\frac{3}{2}}\right)+C$
\stopitemize


\StopFrame

%%----------------------------------
%%----------------------------------

\medskip
\FrameTitle{凑微分法} % 解答

\StartFrame\bf
程序
\startitemize[n]
  \item 当被积函数可分为 $f(x)g(x)$ 或 $\displaystyle \frac{f(x)}{g(x)}$, 其中 $f(x)$ 较复杂, 如果对 $f(x)$ 或其主要部分求导数可以得到 $g(x)$ 的倍数, 常数倍或函数倍, 就用凑微分进行计算。\crlf
即若 $f'(x)=Ag(x)$, 则 $d[f(x)]=Ag(x)dx$, 于是\crlf
$I=\displaystyle \int f(x)g(x)dx=\frac{1}{A}\int f(x)Ag(x)dx=\frac{1}{A}\int f(x)\cdot d[g(x)]$
  \item 当对 $f(x)$ 求导得不到 $g(x)$ 的倍数时, 考虑被积函数的分子、 分母同乘以或同除以一个适当的因子, 恒等变形以达到凑微分的目的。\crlf
常用的因子有 $e^{\alpha x}$, $x^{\beta}$, $\sin x$, $\cos x$。
\stopitemize


\StopFrame

%%----------------------------------
%%----------------------------------


\medskip
\FrameTitle{凑微分法} % 解答

\StartFrame\bf
练习
\startitemize[n]
  \item 计算$\displaystyle \int \frac{\cos ^2x-\sin x}{\left(1+\cos x\cdot e^{\sin x}\right)\cos x}dx$\crlf\medskip
      解:  因为 $\displaystyle (\cos x \cdot e^{\sin x})'=(\cos ^2x-\sin x)e^{\sin x}$, 所以分子、分母同乘以因子 $e^{\sin x}$。\crlf\medskip
$\displaystyle  I= \int \frac{(\cos ^2x-\sin x)e^{\sin x}}{\left(1+\cos x\cdot e^{\sin x}\right)\cos x \cdot e^{\sin x}}dx$\medskip

$\displaystyle  = \int \frac{d(\cos x \cdot e^{\sin x})}{\left(1+\cos x\cdot e^{\sin x}\right)\cos x \cdot e^{\sin x}}$\medskip

$\displaystyle  = \int \left(\frac{1}{\cos x \cdot e^{\sin x}} - \frac{1}{1+\cos x \cdot e^{\sin x}} \right) d\left(\cos x \cdot e^{\sin x}\right)$\medskip

$\displaystyle  = \ln \left| \frac{\cos x \cdot e^{\sin x}}{1+\cos x \cdot e^{\sin x}} \right| +C
$
\stopitemize


\StopFrame

%%----------------------------------
%%----------------------------------


\medskip
\FrameTitle{换元法} % 解答

\StartFrame\bf
思想
\startitemize[n]
  \item 设 $x=g(u)$ 为单调可导函数,则有\crlf
$\displaystyle \int f(x)dx=\int f[g(u)]d[g(u)]=\int f[g(u)]g'(u)du$\crlf
当被积函数不容易积分时, 如含有根式、反三角函数, 可以通过换元的方法从 $d$ 后面拿出一部分放到前面来, 就成为 $\displaystyle \int f[g(u)]g'(u)du$ 的形式, 若 $\displaystyle \int f[g(u)]g'(u)du$ 容易积分, 则换元成功。 计算完后, 用反函数 $u=g^{-1}(x)$ 回代。
\stopitemize


\StopFrame

%%----------------------------------
%%----------------------------------



\medskip
\FrameTitle{换元法} % 解答

\StartFrame\bf
方法
\startitemize[n]
  \item 三角函数代换。 $\sqrt{a^2-x^2}$, 令 $\displaystyle x=a\sin t, |t|<\frac{\pi}{2}$,
  \item 三角函数代换。 $\sqrt{a^2+x^2}$, 令 $\displaystyle x=a\tan t, |t|<\frac{\pi}{2}$,
  \item 三角函数代换。 $\sqrt{x^2-a^2}$, 令 $\displaystyle x=a\sec t$, 若 $x>0$, 则 $\displaystyle 0<t<\frac{\pi}{2}$;
  \item 三角函数代换。 $\sqrt{x^2-a^2}$, 令 $x=a\sec t$, 若 $x<0$, 则 $\displaystyle \frac{\pi}{2}<t<\pi$;
\stopitemize


\StopFrame

%%----------------------------------
%%----------------------------------



\medskip
\FrameTitle{换元法} % 解答

\StartFrame\bf
恒等变形后作三角函数代换。
\startitemize[n]
  \item 当被积函数中含有根式 $\sqrt{ax^2+bx+c}$ 时, 配方, 再作三角函数代换。
\stopitemize


\StopFrame

%%----------------------------------
%%----------------------------------



\medskip
\FrameTitle{换元法} % 解答

\StartFrame\bf
根式代换。
\startitemize[n]
  \item 当被积函数中含有根式 $\sqrt[n]{ax+b}$, $\displaystyle \sqrt{\frac{ax+b}{cx+d}}$, $\sqrt{ae^{bx}+c}$ 时, 令整个根式为 $t$。
  \item 当被积函数中既含有 $\sqrt[n]{ax+b}$, 又含有 $\sqrt[m]{ax+b}$ 时, 一般取 $m$, $n$ 的最小公倍数 $L$, 令 $\sqrt[L]{ax+b}=t$。
\stopitemize


\StopFrame

%%----------------------------------
%%----------------------------------



\medskip
\FrameTitle{换元法} % 解答

\StartFrame\bf
倒代换。
\startitemize[n]
  \item 当被积函数分母的幂次比分子高两次及两次以上时, 作倒代换, 如令 $\displaystyle x=\frac{1}{t}$。
\stopitemize


\StopFrame

%%----------------------------------
%%----------------------------------



\medskip
\FrameTitle{换元法} % 解答

\StartFrame\bf
复杂函数的直接代换。
\startitemize[n]
  \item 当被积函数中含有 $a^x$, $e^x$, $\ln x \arcsin x,\arctan x$ 时, 可考虑直接令复杂函数等于 $t$。
  \item 当 $\ln x$, $\arcsin x$, $\arctan x$ 与 $n$ 次多项式 $P_n(x)$ 作乘积时, 优先考虑分部积分法。
\stopitemize


\StopFrame

%%----------------------------------
%%----------------------------------




\medskip
\FrameTitle{换元法} % 解答

\StartFrame\bf
练习
\startitemize[n]
  \item 计算 $\displaystyle \int \frac{1}{(2x+1)\sqrt{3+4x-4x^2}}dx$\crlf\medskip
 解: 因为 $\sqrt{3+4x-4x^2} = \sqrt{4-(2x-1)^2}$, 令 $2x-1=2\sin t$, 则
$\displaystyle  I = \int \frac{\cos t dt}{(2\sin t+2)2 \cos t}$\crlf\medskip
$\displaystyle  = \frac{1}{4}\int \frac{1}{1+\sin t}dt = \frac{1}{4}\int \frac{1-\sin t}{\cos ^2t}dt$
$\displaystyle  = \frac{1}{4}\left( \tan t - \frac{1}{\cos t}\right) + C$
\stopitemize


\StopFrame

%%----------------------------------
%%----------------------------------





\medskip
\FrameTitle{分部积分法} % 解答

\StartFrame\bf
思想
\startitemize[n]
  \item $\displaystyle \int udv=uv - \int vdu$\crlf
适用于求 $\displaystyle \int udv$ 比较困难, 而 $\displaystyle \int vdu$ 比较容易的情况。
\stopitemize


\StopFrame

%%----------------------------------
%%----------------------------------





\medskip
\FrameTitle{分部积分法} % 解答

\StartFrame\bf
方法
\startitemize[n]
  \item $u$, $v$ 的选取准则。\crlf

\blackrule[color=black,width=0.8\textwidth,height=.01cm,depth=0cm] 

法一:\quad 反\quad\quad $\Rightarrow$\quad 对\quad\quad $\Rightarrow$\quad 幂\quad\quad $\Rightarrow$\quad 指\quad\quad $\Rightarrow$\quad 三\crlf

\blackrule[color=black,width=0.8\textwidth,height=.01cm,depth=0cm] 

法二:\quad 反\quad\quad $\Rightarrow$\quad 对\quad\quad $\Rightarrow$\quad 幂\quad\quad $\Rightarrow$\quad 三\quad\quad $\Rightarrow$\quad 指\crlf

\blackrule[color=black,width=0.8\textwidth,height=.01cm,depth=0cm] 

相对位置在左边的宜选作 $u$, 用来求导; 相对位置在右边的宜选作 $v$, 用来积分。
\stopitemize


\StopFrame

%%----------------------------------
%%----------------------------------





\medskip
\FrameTitle{分部积分法} % 解答

\StartFrame\bf
练习: 被积函数为下列情形时, 如何选取分部积分中的 $u$, $v$:
\startitemize[n]
  \item $P_n(x)e^{kx}$, $P_n(x)\sin ax$, $P_n(x)\cos ax$, 一般来说, 选 $u = P_n(x)$。
  \item $e^{ax}\sin bx$, $e^{ax}\cos bx$, 可以取两因子中的任意一个。
  \item $P_n(x)\ln x$, $P_n(x)\arcsin x$, $P_n(x)\arctan x$, 一般来说, 选 $u = \ln x$, $u=\arcsin x$, $u=\arctan x$。
\stopitemize


\StopFrame

%%----------------------------------
%%----------------------------------





\medskip
\FrameTitle{分部积分法} % 解答

\StartFrame\bf
分部积分的推广公式:表格法。
\startitemize[n]
  \item 计算 $\displaystyle \int uv^{(n+1)}dx$

\blackrule[color=black,width=0.8\textwidth,height=.01cm,depth=0cm] 

$u$ 的各阶导数\quad\quad\quad\quad\quad $u$\quad $\Rightarrow$ $u'$ \quad\quad$\Rightarrow$ $u''$ \quad\quad$\Rightarrow$ \cdots $\Rightarrow$ $u^{(n+1)}$\crlf

\blackrule[color=black,width=0.8\textwidth,height=.01cm,depth=0cm] 

$v^{(n+1)}$的各阶原函数\quad $v^{(n)}$ $\Rightarrow$ $v^{(n-1)}$ $\Rightarrow$ $v^{(n-2)}$ $\Rightarrow$ \cdots $\Rightarrow$ $v$\crlf

\blackrule[color=black,width=0.8\textwidth,height=.01cm,depth=0cm] 

计算方法: 以 $u$ 作起点, 左上右下错位相乘, 各项符号正负相间, 最后一项为 $\displaystyle (-1)^{n+1}\int u^{(n+1)}vdx$
\stopitemize


\StopFrame

%%----------------------------------
%%----------------------------------





\medskip
\FrameTitle{分部积分法} % 解答

\StartFrame\bf
练习
\startitemize[n]
  \item $\displaystyle  \int \frac{x\ln x}{(x^2-1)^{\frac{3}{2}}}dx$ \crlf\medskip
  解: $\displaystyle  I= \int \ln x \cdot d\left(-\frac{1}{\sqrt{x^2-1}} \right)$\crlf\medskip
$\displaystyle  = -\frac{\ln x}{\sqrt{x^2-1}} + \int \frac{dx}{x \sqrt{x^2-1}}$
$\displaystyle  = -\frac{\ln x}{\sqrt{x^2-1}} + \int \frac{1}{\sqrt{1-\left(\frac{1}{x}\right)^2}} \cdot d \left(\frac{1}{x}\right)$\crlf\medskip
$\displaystyle  = -\frac{\ln x}{\sqrt{x^2-1}} - \arcsin \frac{1}{x}+C$ 
\stopitemize


\StopFrame

%%----------------------------------
%%----------------------------------





\medskip
\FrameTitle{分部积分法} % 解答

\StartFrame\bf\index{p146, 例9.2}
练习
\startitemize[n]

  \item 设 $\displaystyle  f(\sin ^2x)=\frac{x}{\sin x}$, 求 $\displaystyle  \int \frac{\sqrt{x}}{\sqrt{1-x}}\cdot f(x)dx$\crlf\medskip
解: 令 $u=\sin ^2x$, 则有 $\sin x=\pm \sqrt{u}$, \crlf\medskip
当 $\sin x =\sqrt{u}$ 时, $x=\arcsin \sqrt{u}$, \crlf\medskip
当 $\sin x = -\sqrt{u}$ 时, $\sin (-x)=\sqrt{u}$, $x=-\arcsin \sqrt{u}$, \crlf\medskip
因此, 有 $f(x)=\displaystyle \frac{\arcsin \sqrt{x}}{\sqrt{x}}$, 于是, 有\crlf\medskip
$\displaystyle I = \int \frac{\arcsin \sqrt{x}}{\sqrt{1-x}}dx $\crlf\medskip
$\displaystyle = -\int \frac{\arcsin \sqrt{x}}{\sqrt{1-x}}d(1-x) $
$\displaystyle = -2\int \arcsin \sqrt{x} d(\sqrt{1-x})$\crlf\medskip
$\displaystyle = -2\sqrt{1-x} \arcsin \sqrt{x} +2\int \sqrt{1-x} \frac{1}{\sqrt{1-x}}d(\sqrt{x})$\crlf\medskip
$\displaystyle = -2\sqrt{1-x} \arcsin \sqrt{x}+2\sqrt{x}+C$
\stopitemize


\StopFrame

%%----------------------------------
%%----------------------------------------------------------------
%% 逐步显示
\page
\usemodule[tikz]         


\starttikzpicture[      scale=3,line cap=round
                        axes/.style=,         
                        important line/.style={very thick},
                        information text/.style={rounded corners,fill=red!10,inner sep=1ex} ]
        \draw[xshift=0.0cm]
                node[right,text width=17cm,information text]
                {
                        \bf 注意: 当{\darkgreen 被积函数}中含有 $\arcsin u(x)$, $\arctan u(x)$ 时, \crlf\medskip
(1) 若可凑微分, 则\crlf\medskip
 $\int f(x)\cdot \arcsin u(x)dx =\int \arcsin u(x)d[F(x)]$, \crlf\medskip
接下来用分部积分法。\crlf\medskip
(2) 若不可凑微分, 则\crlf\medskip
令 $\arcsin u(x)=t$, 或 $u(x)=t$, 换元后再用分部积分法。

 
};

\stoptikzpicture

%%----------------------------------------------------------------
%%----------------------------------





\medskip
\page
\FrameTitle{分部积分法} % 解答

\StartFrame\bf\index{p146, 例9.3}
练习
\startitemize[n]

  \item 计算 $\displaystyle \int \frac{x+1}{\left(1+x^2\right)^2}$ 
\stopitemize


\StopFrame

%%----------------------------------

%%----------------------------------------------------------------
%% 逐步显示

\usemodule[tikz]         


\starttikzpicture[      scale=3,line cap=round
                        axes/.style=,         
                        important line/.style={very thick},
                        information text/.style={rounded corners,fill=red!10,inner sep=1ex} ]
        \draw[xshift=0.0cm]
                node[right,text width=17cm,information text]
                {
                        \bf 
解: 令 $x=\tan t$ (作直角三角形,便于还原变量), 则\crlf\medskip

$\displaystyle  I=\int \frac{\tan t +1}{\sec ^4 t}\cdot \sec^2 tdt$\crlf\medskip
$\displaystyle  = \int (\sin t \cos t+\cos^2t)dt$\crlf\medskip
$\displaystyle  = \frac{1}{2}\sin^2t +\frac{1}{2}t+\frac{1}{4}\sin 2t +C$\crlf\medskip
$\displaystyle  = \frac{x^2}{2(1+x^2)} + \frac{1}{2}\arctan x+\frac{x}{2(1+x^2)}+C$
 
};

\stoptikzpicture

%%----------------------------------------------------------------
%%----------------------------------





\medskip
\FrameTitle{分部积分法} % 解答

\StartFrame\bf\index{p147, 例9.4}
练习
\startitemize[n]
 
  \item $\displaystyle \int \frac{x^2dx}{(x\sin x+\cos x)^2}$
\stopitemize


\StopFrame

%%----------------------------------

%%----------------------------------------------------------------
%% 逐步显示

\usemodule[tikz]         


\starttikzpicture[      scale=3,line cap=round
                        axes/.style=,         
                        important line/.style={very thick},
                        information text/.style={rounded corners,fill=red!10,inner sep=1ex} ]
        \draw[xshift=0.0cm]
                node[right,text width=17cm,information text]
                {
                        \bf 
考点: $\displaystyle  d\left[\frac{1}{f(x)}\right] = -\frac{f'(x)}{f^2(x)}\cdot dx$ 
 
};

\stoptikzpicture

%%----------------------------------------------------------------



%%----------------------------------------------------------------
%% 逐步显示

\usemodule[tikz]         


\starttikzpicture[      scale=3,line cap=round
                        axes/.style=,         
                        important line/.style={very thick},
                        information text/.style={rounded corners,fill=red!10,inner sep=1ex} ]
        \draw[xshift=0.0cm]
                node[right,text width=17cm,information text]
                {
                        \bf 
分析: 若将被积函数的分子与分母同时乘以 $\cos x$, 并且令

$\displaystyle  u(x)=\frac{x}{\cos x}$, 
$\displaystyle  v'(x)=\frac{x\cos x}{(x\sin x+\cos x)^2}$ , \crlf\medskip
则利用分部积分公式\crlf\medskip

$\displaystyle \int u(x)v'(x)dx = u(x)v(x) - \int u'(x) v(x)dx$\crlf\medskip

即可使问题简化。
 
};

\stoptikzpicture

%%----------------------------------------------------------------







%%----------------------------------------------------------------
%% 逐步显示

\usemodule[tikz]         


\starttikzpicture[      scale=3,line cap=round
                        axes/.style=,         
                        important line/.style={very thick},
                        information text/.style={rounded corners,fill=red!10,inner sep=1ex} ]
        \draw[xshift=0.0cm]
                node[right,text width=17cm,information text]
                {
                        \bf  
解: 
$\displaystyle  I=\int \frac{x}{\cos x}\cdot \frac{x\cos x}{(x\sin x+\cos x)^2}dx$\crlf\medskip
$\displaystyle  = -\frac{x}{\cos x}\cdot \frac{1}{x\sin x+ \cos x}+\int \frac{1}{x\sin x + \cos x}\cdot \left(\frac{x}{\cos x}\right)'dx$\crlf\medskip 
$\displaystyle  = -\frac{x}{\cos x(x\sin x+ \cos x)}  + \int \frac{1}{\cos ^2x}dx$\crlf\medskip 
$\displaystyle  = -\frac{x}{\cos x(x\sin x+ \cos x)}  + \tan x +C$\crlf\medskip 
$\displaystyle  = \frac{\sin x  -x\cos x}{x\sin x+ \cos x)}  +C$
};

\stoptikzpicture

%%----------------------------------------------------------------
%%----------------------------------





\bigskip
\FrameTitle{有理函数的积分} % 解答

\StartFrame\bf
定义
\startitemize[n]
  \item 形如 $\displaystyle \int \frac{P_n(x)}{Q_m(x)}dx(n<m)$ 的积分称为有理函数的积分, 其中$P_n(x)$, $Q_m(x)$ 分别是 $x$ 的 $n$ 次多项式和 $m$ 次多项式。
\stopitemize


\StopFrame

%%----------------------------------
%%----------------------------------





\medskip
\FrameTitle{分部积分法} % 解答

\StartFrame\bf
思想
\startitemize[n]
  \item 先将 $Q_m(x)$ 因式分解, 再把 $\displaystyle \frac{P_n(x)}{Q_m(x)}$ 拆成若干最简有理分式之和。
\stopitemize


\StopFrame

%%----------------------------------
%%----------------------------------






\medskip
\FrameTitle{分部积分法} % 解答

\StartFrame\bf
方法
\startitemize[n]
  \item $Q_m(x)$ 的一次单因式 $(ax+b)$  
产生一项 $\displaystyle \frac{A}{ax+b}$ 
  \item $Q_m(x)$ 的k重因式 $(ax+b)^k$ 
产生 $k$ 项, 分别为 $\displaystyle \frac{A_1}{ax+b}$, $\displaystyle \frac{A_2}{(ax+b)^2}$, $\cdots$, $\displaystyle \frac{A_k}{(ax+b)^k}$ 
  \item $Q_m(x)$ 的二次单因式 $px^2+qx+r$  
产生一项 $\displaystyle \frac{Ax+B}{px^2+qx+r}$
  \item $Q_m(x)$ 的 $k$ 重二次因式 $(px^2+qx+r)^k$ 
产生 $k$ 项, 分别为 $\displaystyle \frac{A_1x+B_1}{px^2+qx+r}$, $\displaystyle \frac{A_2x+B_2}{(px^2+qx+r)^2}$, $\cdots$,  
$\displaystyle \frac{A_kx+B_k}{(px^2+qx+r)^k}$
\stopitemize


\StopFrame

%%----------------------------------
%%----------------------------------






\medskip
\FrameTitle{分部积分法} % 解答

\StartFrame\bf

\startitemize[n]
  \item 练习题: 计算$\displaystyle \int e^x\cdot\left(\frac{1-x}{1+x^2}\right)^2dx$
\stopitemize


\StopFrame

%%----------------------------------





%%----------------------------------------------------------------
%% 逐步显示

\usemodule[tikz]         


\starttikzpicture[      scale=3,line cap=round
                        axes/.style=,         
                        important line/.style={very thick},
                        information text/.style={rounded corners,fill=red!10,inner sep=1ex} ]
        \draw[xshift=0.0cm]
                node[right,text width=17cm,information text]
                {
                        \bf  
考点:分部积分, 可以出现{\darkgreen 积分再现}或{\darkgreen  积分抵消}的情形。



};

\stoptikzpicture

%%----------------------------------------------------------------


%%----------------------------------------------------------------
%% 逐步显示

\usemodule[tikz]         


\starttikzpicture[      scale=3,line cap=round
                        axes/.style=,         
                        important line/.style={very thick},
                        information text/.style={rounded corners,fill=red!10,inner sep=1ex} ]
        \draw[xshift=0.0cm]
                node[right,text width=17cm,information text]
                {
                        \bf  
解: 
$\displaystyle  I=\int e^x \cdot\frac{1+x^2-2x}{\left( 1+x^2 \right)^2} dx$\crlf\medskip
$\displaystyle  =\int e^x \cdot \frac{1}{1+x^2} dx - \int e^x\cdot \frac{2x}{(1+x^2)^2}dx$\crlf\medskip
$\displaystyle  =\int \frac{e^x}{1+x^2} dx + \int e^x\cdot d\left(\frac{1}{1+x^2}\right)$\crlf\medskip
$\displaystyle  =\int \frac{e^x}{1+x^2} dx + \frac{e^x}{1+x^2} - \int \frac{e^x}{1+x^2} dx$\crlf\medskip
$\displaystyle  = \frac{e^x}{1+x^2} + C $\crlf\medskip



};

\stoptikzpicture

%%----------------------------------------------------------------
%%----------------------------------









\medskip
\FrameTitle{分部积分法} % 解答

\StartFrame\bf

\startitemize[n]
  \item 练习题: 计算 $\displaystyle \int \frac{x^4+1}{x(x^2+1)^2}dx$
\stopitemize


\StopFrame

%%----------------------------------


%%----------------------------------------------------------------
%% 逐步显示

\usemodule[tikz]         


\starttikzpicture[      scale=3,line cap=round
                        axes/.style=,         
                        important line/.style={very thick},
                        information text/.style={rounded corners,fill=red!10,inner sep=1ex} ]
        \draw[xshift=0.0cm]
                node[right,text width=17cm,information text]
                {
                        \bf  
解: 令 $x^2=t$, 得
$\displaystyle  I=\frac{1}{2}\int \frac{t^2+1}{t(t+1)^2}dt$\crlf\medskip
设 $\displaystyle  \frac{t^2+1}{t(t+1)^2} = \frac{A}{t} + \frac{B}{t+1} + \frac{C}{(t+1)^2}$, 
通分, 去分母, 得 \crlf\medskip
$\displaystyle  t^2+1 = A(t+1)^2 +Bt(t+1) +Ct$, \crlf\medskip
依次令 $t=-1, 0$, 得 $2=-C, 1=A$,
再比较上式两边最高次幂系数, $1=A+B$, 
所以 $A=1$, $B=0$, $C=-2$,\crlf\medskip
于是
$\displaystyle  I= \frac{1}{2}\left[ \int \frac{1}{t} dt - \int \frac{2}{(t+1)^2}dt\right]$
$\displaystyle  = \frac{1}{2}\ln t+\frac{1}{t+1} +C$
$\displaystyle  = \ln |x| + \frac{1}{x^2+1} +C$

};

\stoptikzpicture

%%----------------------------------------------------------------






%%----------------------------------------------------------------
%% 逐步显示

\usemodule[tikz]         


\starttikzpicture[      scale=3,line cap=round
                        axes/.style=,         
                        important line/.style={very thick},
                        information text/.style={rounded corners,fill=red!10,inner sep=1ex} ]
        \draw[xshift=0.0cm]
                node[right,text width=17cm,information text]
                {
                        \bf  
考点:分部积分。变量替换。



};

\stoptikzpicture

%%----------------------------------------------------------------

%%----------------------------------






\medskip
\FrameTitle{2019数学二解答题第16题10分} % 解答

\StartFrame\index{2019数学二解答题第16题10分}
求下列不定积分
\startitemize[n]
  \item $\displaystyle \int\frac{3x+6}{(x-1)^2(x^2+x+1)}dx$
\stopitemize


\StopFrame

%%----------------------------------




%%----------------------------------------------------------------
%% 逐步显示

\usemodule[tikz]         


\starttikzpicture[      scale=3,line cap=round
                        axes/.style=,         
                        important line/.style={very thick},
                        information text/.style={rounded corners,fill=red!10,inner sep=1ex} ]
        \draw[xshift=0.0cm]
                node[right,text width=17cm,information text]
                {
                        \bf  
考点: 有理函数的积分。 积分公式 $\displaystyle  \int \frac{u'(x)}{u(x)}dx = \ln |u(x)| +C$
\index{积分公式}
\index{有理函数的积分}

};

\stoptikzpicture

%%----------------------------------------------------------------




%%----------------------------------------------------------------
%% 逐步显示

\usemodule[tikz]         


\starttikzpicture[      scale=3,line cap=round
                        axes/.style=,         
                        important line/.style={very thick},
                        information text/.style={rounded corners,fill=red!10,inner sep=1ex} ]
        \draw[xshift=0.0cm]
                node[right,text width=17cm,information text]
                {
                        \bf  
解:$\displaystyle  I= \int \frac{2x+1}{x^2+x+1}dx - \int \frac{2}{x-1}dx + \int \frac{3}{(x-1)^2}dx$\crlf\medskip
$\displaystyle  = \ln |x^2+x+1| - 2\ln |x-1| - \frac{3}{x-1} +C$

};

\stoptikzpicture

%%----------------------------------------------------------------
%%----------------------------------






\medskip
\FrameTitle{定积分的计算} % 解答

\StartFrame\bf
牢记重要公式
\startitemize[n]
  \item $\displaystyle  \int_a^bf(x)dx=\int_a^b f(a+b-x)dx$
  \item $\displaystyle  \int_a^b f(x)dx=\frac{1}{2}\int_a^b [f(x)+f(a+b-x)]dx$
  \item $\displaystyle  \int_a^b f(x)dx=\int_a^{\frac{a+b}{2}} [f(x)+f(a+b-x)]dx$
\stopitemize


\StopFrame

%%----------------------------------
%%----------------------------------






\medskip
\FrameTitle{定积分的计算} % 解答

\StartFrame\bf
牢记重要公式
\startitemize[n]
  \item $\displaystyle \int_0^{\frac{\pi}{2}}\sin ^n xdx=\int_0^{\frac{\pi}{2}}\cos ^n xdx$\crlf\medskip
$\displaystyle =\frac{n-1}{n}\cdot\frac{n-3}{n-2}\cdots\frac{2}{3}\cdot 1$, $n>1$ 奇数
  \item $\displaystyle \int_0^{\frac{\pi}{2}}\sin ^n xdx=\int_0^{\frac{\pi}{2}}\cos ^n xdx$\crlf\medskip
$\displaystyle =\frac{n-1}{n}\cdot\frac{n-3}{n-2}\cdots\frac{1}{2}\cdot \frac{\pi}{2}$, $n$ 正偶数
  \item $\displaystyle \int_0^{\pi}\sin ^n xdx=2\cdot\frac{n-1}{n}\cdot\frac{n-3}{n-2}\cdots\frac{2}{3}\cdot 1$, $n>1$ 奇数
  \item $\displaystyle \int_0^{\pi}\sin ^n xdx=2\cdot\frac{n-1}{n}\cdot\frac{n-3}{n-2}\cdots\frac{1}{2}\cdot \frac{\pi}{2}$, $n$ 正偶数
\stopitemize
\index{Wallis公式}

\StopFrame

%%----------------------------------









\medskip
\FrameTitle{定积分的计算} % 解答

\StartFrame\bf
牢记重要公式
\startitemize[n]

  \item $\displaystyle \int_0^{\pi}\cos ^n xdx=0$, $n$ 正奇数
  \item $\displaystyle \int_0^{\pi}\cos ^n xdx=2\cdot\frac{n-1}{n}\cdot\frac{n-3}{n-2}\cdots\frac{1}{2}\cdot \frac{\pi}{2}$, $n$ 正偶数
  \item $\displaystyle \int_0^{2\pi}\sin ^n xdx=\int_0^{2\pi}\cos ^n xdx=0$, $n$ 正奇数
  \item $\displaystyle \int_0^{2\pi}\sin ^n xdx=\int_0^{2\pi}\cos ^n xdx$\crlf\medskip
$\displaystyle  =4\cdot\frac{n-1}{n}\cdot\frac{n-3}{n-2}\cdots\frac{1}{2}\cdot \frac{\pi}{2}$, $n$ 正偶数
\stopitemize
\index{Wallis公式}

\StopFrame

%%----------------------------------
%%----------------------------------






\medskip
\FrameTitle{定积分的计算} % 解答

\StartFrame\bf
牢记重要公式
\startitemize[n]
  \item $\displaystyle \int_0^{\pi}xf(\sin x)dx=\frac{\pi}{2}\int_0^{\pi}f(\sin x)dx$
  \item $\displaystyle \int_0^{\pi}xf(\sin x)dx=\pi\int_0^{\frac{\pi}{2}}f(\sin x)dx$
  \item $\displaystyle \int_0^{\frac{\pi}{2}}f(\sin x)dx=\int_0^{\frac{\pi}{2}}f(\cos x)dx$
  \item $\displaystyle \int_0^{\frac{\pi}{2}}f(\sin x,\cos x)dx=\int_0^{\frac{\pi}{2}}f(\cos x,\sin x)dx$

\stopitemize


\StopFrame

%%----------------------------------









\medskip
\FrameTitle{定积分的计算} % 解答

\StartFrame\bf
牢记重要公式
\startitemize[n]
  \item $\displaystyle \int_a^b f(x)dx$\crlf\medskip
$\displaystyle = \int_{-\frac{\pi}{2}}^{\frac{\pi}{2}}f\left(\frac{a+b}{2}+\frac{b-a}{2}\sin t\right)\cdot \frac{b-a}{2}\cos tdt$
  \item $\displaystyle \int_a^b f(x)dx= \int_0^1 (b-a) \cdot f[a+(b-a)t]dt$
  \item $\displaystyle \int_{-a}^a f(x)dx= \int_0^a [f(x)+f(-x)]dx,a>0$
\stopitemize


\StopFrame

%%----------------------------------
%%----------------------------------






\medskip\page
\FrameTitle{定积分的计算} % 解答

\StartFrame\bf\index{p150, 例9.8}\index{区间再现公式}

\startitemize[n]
  \item 练习题: 设 $f(x)$ 为连续函数, 证明: 区间再现公式\crlf
$\displaystyle \int _a^b f(x)dx=\int_a^b f(a+b-x)dx$

\stopitemize


\StopFrame

%%----------------------------------

%%----------------------------------------------------------------
%% 逐步显示

\usemodule[tikz]         


\starttikzpicture[      scale=3,line cap=round
                        axes/.style=,         
                        important line/.style={very thick},
                        information text/.style={rounded corners,fill=red!10,inner sep=1ex} ]
        \draw[xshift=0.0cm]
                node[right,text width=17cm,information text]
                {
                        \bf  
解: 作变量代换, 令 $x=a+b-t$, 则\crlf 
$\displaystyle \int_a^b f(x)dx $ 
$\displaystyle  = \int_b^a f(a+b-t)(-dt) $ \crlf 
$\displaystyle  = \int_a^b f(a+b-t)dt$ 
$\displaystyle  = \int _a^b f(a+b-x)dx$

};

\stoptikzpicture

%%----------------------------------------------------------------



%%----------------------------------------------------------------
%% 逐步显示

\usemodule[tikz]         


\starttikzpicture[      scale=3,line cap=round
                        axes/.style=,         
                        important line/.style={very thick},
                        information text/.style={rounded corners,fill=red!10,inner sep=1ex} ]
        \draw[xshift=0.0cm]
                node[right,text width=17cm,information text]
                {
                        \bf  
{\darkgreen 变形1 }: 等式两边相加再除以 $2$, 有\crlf 
$\displaystyle  \int_a^b f(x)dx = \frac{1}{2} \int_a^b [f(x) + f(a+b-x)]dx$  

};

\stoptikzpicture

%%----------------------------------------------------------------




%%----------------------------------------------------------------
%% 逐步显示

\usemodule[tikz]         


\starttikzpicture[      scale=3,line cap=round
                        axes/.style=,         
                        important line/.style={very thick},
                        information text/.style={rounded corners,fill=red!10,inner sep=1ex} ]
        \draw[xshift=0.0cm]
                node[right,text width=17cm,information text]
                {
                        \bf   
{\darkgreen 变形2 }: 令 $F(x) = f(x)+ f(a+b-x)$, 则\crlf 
$F(a+b-x)=f(a+b-x)+f(x)=F(x)$,
故 $F(x)$ 以 $x=\frac{a+b}{2}$ 为对称轴, 故又有 
$\displaystyle  \int_a^b f(x)dx = \int_a^{\frac{a+b}{2} } [f(x) + f(a+b-x)]dx$

};

\stoptikzpicture

%%----------------------------------------------------------------




%%----------------------------------




\medskip
\FrameTitle{定积分的计算} % 解答

\StartFrame\bf\index{p150, 例9.9}

\startitemize[n]
  \item 练习题: 计算 $\displaystyle \int _0^1 \frac{\ln (1+x)}{1+x^2}dx$
\stopitemize


\StopFrame

%%----------------------------------




%%----------------------------------------------------------------
%% 逐步显示

\usemodule[tikz]         


\starttikzpicture[      scale=3,line cap=round
                        axes/.style=,         
                        important line/.style={very thick},
                        information text/.style={rounded corners,fill=red!10,inner sep=1ex} ]
        \draw[xshift=0.0cm]
                node[right,text width=17cm,information text]
                {
                        \bf   
%%%%----------------------------

解: 令 $x=\tan t$, 则\crlf
$\displaystyle  I= \int_0^{\frac{\pi}{4}} \ln (1+\tan t)dt$ 
$\displaystyle  = \int _0^{\frac{\pi}{4}} \ln (\sin x+ \cos x)dx - \int _0^{\frac{\pi}{4}} \ln \cos xdx$\crlf
$\displaystyle  = \int _0^{\frac{\pi}{4}}  \ln \left[\sqrt{2}\cos \left(\frac{\pi}{4}-x\right)\right]dx -\int _0^{\frac{\pi}{4}} \ln \cos xdx$\crlf
$\displaystyle  = \frac{\pi}{8} \ln 2+\int_0^{\frac{\pi}{4}} \ln \cos \left(\frac{\pi}{4} -x\right)dx - \int_0^{\frac{\pi}{4}}\ln \cos xdx$\crlf
$\displaystyle  = \frac{\pi}{8} \ln 2$ (作变量代换 $\darkgreen \displaystyle u=\frac{\pi}{4} - x$, 两个积分相互抵消)


%%%%----------------------------

};

\stoptikzpicture
%% 逐步显示
%%----------------------------------------------------------------





%%----------------------------------

\medskip
\FrameTitle{定积分的计算} % 解答

\StartFrame\bf\index{p151, 例9.10}
\index{变量替换}
\index{和差角公式}

\startitemize[n]
  \item 练习题: 计算 $\displaystyle \int _0^{\frac{\pi}{4}}\frac{xdx}{\cos\left( \frac{\pi}{4}-x\right)\cdot\cos x}$
\stopitemize


\StopFrame

%%----------------------------------



%%----------------------------------------------------------------
%% 逐步显示

\usemodule[tikz]         


\starttikzpicture[      scale=3,line cap=round
                        axes/.style=,         
                        important line/.style={very thick},
                        information text/.style={rounded corners,fill=red!10,inner sep=1ex} ]
        \draw[xshift=0.0cm]
                node[right,text width=17cm,information text]
                {
                        \bf   
%%%%----------------------------

解: 令 $\darkgreen \displaystyle x=\frac{\pi}{4} - t$, 即 $t= \frac{\pi}{4} -x$, 则\crlf
$\displaystyle  I= \int _0^{\frac{\pi}{4}}\frac{(\frac{\pi}{4} - t)dt}{\cos\left( \frac{\pi}{4}-t\right)\cdot\cos t}$
$\displaystyle  = \frac{\pi}{8}\int _0^{\frac{\pi}{4}}\frac{dt}{\cos\left( \frac{\pi}{4}-t\right)\cdot\cos t}$\crlf
$\displaystyle  = \frac{\sqrt{2}\pi}{8}\int _0^{\frac{\pi}{4}}\frac{dt}{(\sin t + \cos t)\cos t}$\crlf
$\displaystyle  = \frac{\sqrt{2}\pi}{8}\int _0^{\frac{\pi}{4}} \frac{d(\tan t)}{\tan t +1}$
$\displaystyle  = \left.\frac{\sqrt{2}\pi}{8} \ln (\tan t+1)\right|_0^{\frac{\pi}{4}} = \frac{\sqrt{2}\pi}{8}\ln 2$

%%%%----------------------------

};

\stoptikzpicture
%% 逐步显示
%%----------------------------------------------------------------




%%----------------------------------

\medskip
\FrameTitle{定积分的计算} % 解答

\StartFrame\bf\index{p151, 例9.11}\index{Wallis公式}
{\darkgreen Wallis公式}。 牢记。\index{牢记}
\startitemize[n]
  \item 设 $n$ 为非负整数, 证明\crlf
$\displaystyle \int_0^{\frac{\pi}{2}} \sin^nxdx = \int_0^{\frac{\pi}{2}} \cos^nxdx$, $n=0,1,2,\cdots$
\stopitemize


\StopFrame

%%----------------------------------




%%----------------------------------------------------------------
%% 逐步显示

\usemodule[tikz]         


\starttikzpicture[      scale=3,line cap=round
                        axes/.style=,         
                        important line/.style={very thick},
                        information text/.style={rounded corners,fill=red!10,inner sep=1ex} ]
        \draw[xshift=0.0cm]
                node[right,text width=17cm,information text]
                {
                        \bf   
%%%%----------------------------

解: 作变量代换, 令 $\displaystyle\darkgreen x=\frac{\pi}{2}-t$, 有\crlf
$\displaystyle  \int _0^{\frac{\pi}{2}} \sin^n x dx 
= \int_{\frac{\pi}{2}}^0 \sin^n \left(\frac{\pi}{2} - t\right)\cdot  \left(- dt\right)$
$\displaystyle  = \int _0^{\frac{\pi}{2}} \cos^n x dx, \quad n=0,1,2,\cdots $


%%%%----------------------------

};

\stoptikzpicture
%% 逐步显示
%%----------------------------------------------------------------




%%----------------------------------

\medskip
\FrameTitle{定积分的计算} % 解答

\StartFrame\bf\index{p151, 例9.11}\index{Wallis公式}
{\darkgreen Wallis公式}。 牢记。\index{牢记}
\index{Wallis公式证明过程}

\startitemize[n]
  \item 设 $n$ 为非负整数,  计算 $\displaystyle  I_n = \int_0^{\frac{\pi}{2}} \sin^nxdx$ 
\stopitemize


\StopFrame

%%----------------------------------




%%----------------------------------------------------------------
%% 逐步显示

\usemodule[tikz]         


\starttikzpicture[      scale=3,line cap=round
                        axes/.style=,         
                        important line/.style={very thick},
                        information text/.style={rounded corners,fill=red!10,inner sep=1ex} ]
        \draw[xshift=0.0cm]
                node[right,text width=17cm,information text]
                {
                        \bf   
%%%%----------------------------

解: 
$\displaystyle  I_0 = \int _0^{\frac{\pi}{2}} \sin^0 x dx = \int _0^{\frac{\pi}{2}} 1 dx = \frac{\pi}{2}$\crlf
$\displaystyle  I_1 = \int _0^{\frac{\pi}{2}} \sin^1 x dx = \int _0^{\frac{\pi}{2}} \sin x dx = -\cos x \big|_0^{\frac{\pi}{2}} = 1 $ \crlf
\bigskip

%%%%----------------------------

};

\stoptikzpicture
%% 逐步显示
%%----------------------------------------------------------------







%%----------------------------------------------------------------
%% 逐步显示

\usemodule[tikz]         


\starttikzpicture[      scale=3,line cap=round
                        axes/.style=,         
                        important line/.style={very thick},
                        information text/.style={rounded corners,fill=red!10,inner sep=1ex} ]
        \draw[xshift=0.0cm]
                node[right,text width=17cm,information text]
                {
                        \bf   
%%%%----------------------------

当 $n\geq 2$ 时, 有\crlf
$\displaystyle  I_n = \int _0^{\frac{\pi}{2}} \sin^n x dx$
$\displaystyle   = \int _0^{\frac{\pi}{2}} \sin^{n-1} x \sin x dx$
$\displaystyle   = \int _0^{\frac{\pi}{2}} \sin^{n-1} x d(\cos x)$\crlf
$\displaystyle  = -\sin ^{n-1}\cos x \big|_0^{\frac{\pi}{2}} +\int _0^{\frac{\pi}{2}} \cos x d(\sin ^{n-1}x)$\crlf
$\displaystyle  = \int _0^{\frac{\pi}{2}} \cos x \cdot (n-1)\sin ^{n-2}x \cdot \cos x dx$
$\displaystyle  = (n-1)\int _0^{\frac{\pi}{2}} \cos ^2x \sin ^{n-2}xdx$
\bigskip

%%%%----------------------------

};

\stoptikzpicture
%% 逐步显示
%%----------------------------------------------------------------




%%----------------------------------------------------------------
%% 逐步显示

\usemodule[tikz]         


\starttikzpicture[      scale=3,line cap=round
                        axes/.style=,         
                        important line/.style={very thick},
                        information text/.style={rounded corners,fill=red!10,inner sep=1ex} ]
        \draw[xshift=0.0cm]
                node[right,text width=17cm,information text]
                {
                        \bf   
%%%%----------------------------

$\displaystyle  I_n = (n-1)\int _0^{\frac{\pi}{2}} \cos ^2x \sin ^{n-2}xdx$
$\displaystyle  = (n-1)\int _0^{\frac{\pi}{2}} (1-\sin ^2x) \sin ^{n-2}xdx$\crlf
$\displaystyle  = (n-1)\int _0^{\frac{\pi}{2}} \sin^{n-2}x dx -(n-1)\int _0^{\frac{\pi}{2}} \sin^nxdx$
$\displaystyle  = (n-1)I_{n-2} + (n-1)I_n$\crlf\medskip
于是
$\displaystyle  nI_n = (n-1)I_{n-2}$, $\displaystyle  I_n =\frac{n-1}{n}I_{n-2}$, $\displaystyle  n=2, 3, \cdots$\crlf\medskip

%%%%----------------------------

};

\stoptikzpicture
%% 逐步显示
%%----------------------------------------------------------------





%%----------------------------------------------------------------
%% 逐步显示

\usemodule[tikz]         


\starttikzpicture[      scale=3,line cap=round
                        axes/.style=,         
                        important line/.style={very thick},
                        information text/.style={rounded corners,fill=red!10,inner sep=1ex} ]
        \draw[xshift=0.0cm]
                node[right,text width=17cm,information text]
                {
                        \bf   
%%%%----------------------------


若 $n>1$ 为奇数, 则\crlf
$\displaystyle  I_n = \int_0^{\frac{\pi}{2}} \sin^nxdx=\frac{n-1}{n}\cdot\frac{n-3}{n-2}\cdots\frac{2}{3}\cdot I_1
=\frac{n-1}{n}\cdot\frac{n-3}{n-2}\cdots\frac{2}{3}\cdot 1$\crlf\medskip
若 $n>1$ 为正偶数, 则\crlf
$\displaystyle  I_n = \int_0^{\frac{\pi}{2}} \sin^nxdx=\frac{n-1}{n}\cdot\frac{n-3}{n-2}\cdots\frac{2}{3}\cdot I_1
=\frac{n-1}{n}\cdot\frac{n-3}{n-2}\cdots\frac{1}{2}\cdot \frac{\pi}{2}$\crlf
称为 {\darkgreen Wallis公式}。
%%%%----------------------------

};

\stoptikzpicture
%% 逐步显示
%%----------------------------------------------------------------




%%----------------------------------

\medskip
\FrameTitle{定积分的计算} % 解答

\StartFrame\bf\index{p152, 例9.12}
\index{Wallis公式}


考点: Wallis公式。
\startitemize[n]
  \item 设 $n$ 为正整数,计算 $\displaystyle \int_0^{\pi} \sin^nxdx$ 
\stopitemize


\StopFrame

%%----------------------------------





%%----------------------------------------------------------------
%% 逐步显示

\usemodule[tikz]         


\starttikzpicture[      scale=3,line cap=round
                        axes/.style=,         
                        important line/.style={very thick},
                        information text/.style={rounded corners,fill=red!10,inner sep=1ex} ]
        \draw[xshift=0.0cm]
                node[right,text width=17cm,information text]
                {
                        \bf   
%%%%----------------------------

解: 
$\displaystyle  I= \int _0^{\pi} \sin^n x dx $
$\displaystyle  = \int _0^{\frac{\pi}{2}} \sin^n x dx + \int_{\frac{\pi}{2}}^{\pi} \sin^n x dx$\crlf
(第二个积分, 令 $\darkgreen \displaystyle  x=\pi - t$, 则有) \crlf
$\displaystyle  = \int _0^{\frac{\pi}{2}} \sin^n x dx + \int_{\frac{\pi}{2}}^0 \sin^n (\pi - t) d(\pi - t)$\crlf
$\displaystyle  = \int _0^{\frac{\pi}{2}} \sin^n x dx + \int _0^{\frac{\pi}{2}} \sin^n t dt  $
$\displaystyle  = 2\int _0^{\frac{\pi}{2}} \sin^n x dx$\crlf
 Wallis公式。

%%%%----------------------------

};

\stoptikzpicture
%% 逐步显示
%%----------------------------------------------------------------



%%----------------------------------

\medskip
\FrameTitle{定积分的计算} % 解答

\StartFrame\bf
\index{Wallis公式}

考点: Wallis公式。
\startitemize[n]
  \item 设 $n$ 为正整数, 计算 $\displaystyle  I= \int_0^{\pi} \cos^nxdx$ 
\stopitemize


\StopFrame

%%----------------------------------
%%----------------------------------




%%----------------------------------------------------------------
%% 逐步显示

\usemodule[tikz]         


\starttikzpicture[      scale=3,line cap=round
                        axes/.style=,         
                        important line/.style={very thick},
                        information text/.style={rounded corners,fill=red!10,inner sep=1ex} ]
        \draw[xshift=0.0cm]
                node[right,text width=17cm,information text]
                {
                        \bf   
%%%%----------------------------

解: 
$\displaystyle  I= \int _0^{\pi} \cos^n x dx = \int_0^{\frac{\pi}{2}}\cos^nxdx + \int^{\pi}_{\frac{\pi}{2}}\cos^nxdx $\crlf 
变量代换, 第二个积分令$x=\pi - t$, 有\crlf
$\displaystyle  = \int _0^{\pi} \cos^n x dx $\crlf
$\displaystyle  = \int_0^{\frac{\pi}{2}}\cos^nxdx + \int^0_{\frac{\pi}{2}}\cos^n(\pi -t)d(\pi -t) $\crlf
$\displaystyle  = \int_0^{\frac{\pi}{2}}\cos^nxdx + \int_0^{\frac{\pi}{2}}(- \cos t)^ndt$\crlf\medskip
当$n$为正奇数时, $\displaystyle  I=0$\crlf\medskip
当$n$为正偶数时, $\displaystyle  I=2\int_0^{\frac{\pi}{2}}\cos^nxdx$,  Wallis公式。 

%%%%----------------------------

};

\stoptikzpicture
%% 逐步显示
%%----------------------------------------------------------------





%%----------------------------------

\medskip
\FrameTitle{定积分的计算} % 解答

\StartFrame\bf
\index{Wallis公式}

考点: Wallis公式。
\startitemize[n] 
  \item 设 $n$ 为正整数,计算
\startformula
\displaystyle  I = \int_0^{2\pi} \sin^nxdx
\stopformula
\stopitemize


\StopFrame

%%----------------------------------
%%----------------------------------





%%----------------------------------------------------------------
%% 逐步显示

\usemodule[tikz]         


\starttikzpicture[      scale=3,line cap=round
                        axes/.style=,         
                        important line/.style={very thick},
                        information text/.style={rounded corners,fill=red!10,inner sep=1ex} ]
        \draw[xshift=0.0cm]
                node[right,text width=17cm,information text]
                {
                        \bf   
%%%%----------------------------

解: $\sin ^nx$ 是以 $2\pi$ 为周期的周期函数, 于是\crlf
\startformula
I = \int_{-\pi}^{\pi} \sin^nx dx
\stopformula
若 $n$ 为正奇数, 则 $\sin ^nx$ 为奇函数, 对称区间上奇函数的定积分为 $0$ , 故 $I=0$。\crlf
若 $n$ 为正偶数, 则 $\sin ^nx$ 为偶函数, 故 \crlf
\startformula
I=2\int_0^{\pi} \sin^nx dx = 4\int_0^{\frac{\pi}{2}} \sin^nx dx
\stopformula
Wallis公式。



%%%%----------------------------

};

\stoptikzpicture
%% 逐步显示
%%----------------------------------------------------------------







%%----------------------------------

\medskip
\FrameTitle{定积分的计算} % 解答

\StartFrame\bf
\index{Wallis公式}

考点: Wallis公式。
\startitemize[n]  
  \item 设 $n$ 为正整数, 计算
\startformula
\int_0^{2\pi} \cos^nx \cdot dx
\stopformula
\stopitemize


\StopFrame

%%----------------------------------
%%----------------------------------




%%----------------------------------------------------------------
%% 逐步显示

\usemodule[tikz]         


\starttikzpicture[      scale=3,line cap=round
                        axes/.style=,         
                        important line/.style={very thick},
                        information text/.style={rounded corners,fill=red!10,inner sep=1ex} ]
        \draw[xshift=0.0cm]
                node[right,text width=17cm,information text]
                {
                        \bf
%%%%----------------------------

解: $\cos ^nx$ 是以 $2\pi$ 为周期的周期函数, 于是\crlf
\startformula
\startalign
  \NC  I \NC  = \int_{-\pi}^{\pi} \cos^nx dx = 2 \int_0^{ \pi } \cos^nx dx  
       = 2 \int_0^{\frac{\pi}{2}} \cos^nx dx + 2\int_0^{\frac{\pi}{2}} (-\cos t)^n dt\NR
\stopalign
\stopformula




%%%%----------------------------

};

\stoptikzpicture
%% 逐步显示
%%----------------------------------------------------------------








%%----------------------------------------------------------------
%% 逐步显示

\usemodule[tikz]         


\starttikzpicture[      scale=3,line cap=round
                        axes/.style=,         
                        important line/.style={very thick},
                        information text/.style={rounded corners,fill=red!10,inner sep=1ex} ]
        \draw[xshift=0.0cm]
                node[right,text width=17cm,information text]
                {
                        \bf
%%%%----------------------------


若 $n$ 为正奇数, 则 $I=0$。 
若 $n$ 为正偶数, 则 
\startformula
I= 4\int_0^{\frac{\pi}{2}} \cos^nx dx
\stopformula




%%%%----------------------------

};

\stoptikzpicture
%% 逐步显示
%%----------------------------------------------------------------








%%----------------------------------

\medskip
\FrameTitle{定积分的计算} % 解答

\StartFrame\bf
练习
\startitemize[n]   
  \item 计算
\startformula
\int_{-\frac{\pi}{2}}^{\frac{\pi}{2}} \frac{\sin ^4x}{1+e^{-x}}dx
\stopformula
\stopitemize


\StopFrame

%%----------------------------------
%%----------------------------------




%%----------------------------------------------------------------
%% 逐步显示

\usemodule[tikz]         


\starttikzpicture[      scale=3,line cap=round
                        axes/.style=,         
                        important line/.style={very thick},
                        information text/.style={rounded corners,fill=red!10,inner sep=1ex} ]
        \draw[xshift=0.0cm]
                node[right,text width=17cm,information text]
                {
                        \bf
%%%%----------------------------

解: 积分区间关于原点对称, 所以根据对称区间上定积分的性质
\startformula \darkgreen 
\int_{-a}^a f(x)dx = \int_0^a \left[ \,f(x)+f(-x)\right]dx
\stopformula
 


%%%%----------------------------

};

\stoptikzpicture
%% 逐步显示
%%----------------------------------------------------------------







%%----------------------------------------------------------------
%% 逐步显示

\usemodule[tikz]         


\starttikzpicture[      scale=3,line cap=round
                        axes/.style=,         
                        important line/.style={very thick},
                        information text/.style={rounded corners,fill=red!10,inner sep=1ex} ]
        \draw[xshift=0.0cm]
                node[right,text width=17cm,information text]
                {
                        \bf
%%%%----------------------------

 
\startformula
\startalign
  \NC  I \NC  = \int_0^{\frac{\pi}{2}} \left[\frac{\sin ^4x}{1+e^{-x}} + \frac{\sin ^4(-x)}{1+e^{-(-x)}} \right]dx  
           = \int_0^{\frac{\pi}{2}} \sin^4x dx = \frac{3}{4}\cdot  \frac{1}{2}\cdot  \frac{\pi}{2} = \frac{3\pi }{16}\NR
\stopalign
\stopformula
 
Wallis公式。



%%%%----------------------------

};

\stoptikzpicture
%% 逐步显示
%%----------------------------------------------------------------








\medskip
\FrameTitle{定积分的计算} % 解答

\StartFrame\bf
诱导公式。 牢记。
\startitemize[n]
  \item $\sin (\pi\pm t) = \mp \sin t$
  \item $\cos (\pi\pm t) = - \cos t$
  \item $\sin \left(\frac{\pi}{2}\pm t \right) = \cos t$
  \item $\cos \left(\frac{\pi}{2}\pm t \right) = \mp \sin t$
\stopitemize


\StopFrame

%%----------------------------------
%%----------------------------------






\medskip
\FrameTitle{定积分的计算} % 解答

\StartFrame\bf\index{p153, 例9.17}
练习
\startitemize[n]
  \item 设 $f(x)$ 连续, 证明: 

\startformula
\int_0^{\pi} xf(\sin x)dx=\frac{\pi}{2}\int_0^{\pi}f(\sin x)dx\
\stopformula

\stopitemize


\StopFrame

%%----------------------------------
%%----------------------------------




%%----------------------------------------------------------------
%% 逐步显示

\usemodule[tikz]         


\starttikzpicture[      scale=3,line cap=round
                        axes/.style=,         
                        important line/.style={very thick},
                        information text/.style={rounded corners,fill=red!10,inner sep=1ex} ]
        \draw[xshift=0.0cm]
                node[right,text width=17cm,information text]
                {
                        \bf
%%%%----------------------------

解: 根据定积分的性质
\startformula \darkgreen 
\int_a^b f(x)dx = \frac{1}{2}\int_a^b \left[ \,f(x)+f(a+b-x)\right]dx
\stopformula
 


%%%%----------------------------

};

\stoptikzpicture
%% 逐步显示
%%----------------------------------------------------------------








%%----------------------------------------------------------------
%% 逐步显示

\usemodule[tikz]         


\starttikzpicture[      scale=3,line cap=round
                        axes/.style=,         
                        important line/.style={very thick},
                        information text/.style={rounded corners,fill=red!10,inner sep=1ex} ]
        \draw[xshift=0.0cm]
                node[right,text width=17cm,information text]
                {
                        \bf
%%%%----------------------------


\startformula
\startalign
  \NC  L \NC  = \int_0^{\pi} xf(\sin x)dx  
           = \frac{1}{2}\int_0^{\pi} \left\{xf(\sin x)+(\pi - x)f[\sin(\pi -x)]\right\}dx \NR
  \NC  \NC = \frac{1}{2}\int_0^{\pi} \left [xf(\sin x)+ \pi f(\sin x) - xf(\sin x)\right]dx  
           = \frac{\pi}{2}\int_0^{\pi} f(\sin x) dx = R \NR
\stopalign
\stopformula
 



%%%%----------------------------

};

\stoptikzpicture
%% 逐步显示
%%----------------------------------------------------------------








\medskip
\FrameTitle{定积分的计算} % 解答

\StartFrame\bf
练习
\startitemize[n]
  \item 设 $f(x)$ 连续, 证明: 
\startformula
\int_0^{\pi} xf(\sin x)dx=\pi\int_0^{\frac{\pi}{2}}f(\sin x)dx
\stopformula

\stopitemize


\StopFrame

%%----------------------------------
%%----------------------------------



%%----------------------------------------------------------------
%% 逐步显示

\usemodule[tikz]         


\starttikzpicture[      scale=3,line cap=round
                        axes/.style=,         
                        important line/.style={very thick},
                        information text/.style={rounded corners,fill=red!10,inner sep=1ex} ]
        \draw[xshift=0.0cm]
                node[right,text width=17cm,information text]
                {
                        \bf
%%%%----------------------------

解: 根据定积分的性质
\startformula \darkgreen 
\int_a^b f(x)dx = \int_a^{\frac{a+b}{2}} \left[ \,f(x)+f(a+b-x)\right]dx
\stopformula
 



%%%%----------------------------

};

\stoptikzpicture
%% 逐步显示
%%----------------------------------------------------------------







%%----------------------------------------------------------------
%% 逐步显示

\usemodule[tikz]         


\starttikzpicture[      scale=3,line cap=round
                        axes/.style=,         
                        important line/.style={very thick},
                        information text/.style={rounded corners,fill=red!10,inner sep=1ex} ]
        \draw[xshift=0.0cm]
                node[right,text width=17cm,information text]
                {
                        \bf
%%%%----------------------------

 
\startformula
\startalign
  \NC  L \NC  = \int_0^{\pi} xf(\sin x)dx 
           = \int_0^{\frac{\pi}{2}} \left\{xf(\sin x)+(\pi - x)f[\sin(\pi -x)]\right\}dx \NR
  \NC  \NC = \int_0^{\frac{\pi}{2}}  \left [xf(\sin x)+ \pi f(\sin x) - xf(\sin x)\right]dx 
          = \pi \int_0^{\frac{\pi}{2}}  f(\sin x) dx = R \NR
\stopalign
\stopformula
 



%%%%----------------------------

};

\stoptikzpicture
%% 逐步显示
%%----------------------------------------------------------------









\medskip
\FrameTitle{定积分的计算} % 解答

\StartFrame\bf
练习
\startitemize[n]
  \item 设 $f(x)$ 连续, 证明: $\displaystyle \int_0^{\frac{\pi}{2}} f(\sin x)dx=\int_0^{\frac{\pi}{2}}f(\cos x)dx$
\stopitemize


\StopFrame

%%----------------------------------
%%----------------------------------




%%----------------------------------------------------------------
%% 逐步显示

\usemodule[tikz]         


\starttikzpicture[      scale=3,line cap=round
                        axes/.style=,         
                        important line/.style={very thick},
                        information text/.style={rounded corners,fill=red!10,inner sep=1ex} ]
        \draw[xshift=0.0cm]
                node[right,text width=17cm,information text]
                {
                        \bf
%%%%----------------------------

解: 根据定积分的性质
\startformula \darkgreen 
\int_a^b f(x)dx = \int_a^b f(a+b-x) dx
\stopformula
 


%%%%----------------------------

};

\stoptikzpicture
%% 逐步显示
%%----------------------------------------------------------------







%%----------------------------------------------------------------
%% 逐步显示

\usemodule[tikz]         


\starttikzpicture[      scale=3,line cap=round
                        axes/.style=,         
                        important line/.style={very thick},
                        information text/.style={rounded corners,fill=red!10,inner sep=1ex} ]
        \draw[xshift=0.0cm]
                node[right,text width=17cm,information text]
                {
                        \bf
%%%%----------------------------
 
 
\startformula
\startalign
  \NC  L \NC  = \int_0^{\frac{\pi}{2}} f(\sin x)dx  
         = \int_0^{\frac{\pi}{2}} f\left[ \sin \left( \frac{\pi}{2} - x  \right) \right]dx  
         = \int_0^{\frac{\pi}{2}} f (\cos x) dx \NR
\stopalign
\stopformula
 



%%%%----------------------------

};

\stoptikzpicture
%% 逐步显示
%%----------------------------------------------------------------








\medskip
\FrameTitle{定积分的计算} % 解答

\StartFrame\bf\index{p154, 例9.20}
练习
\startitemize[n]
  \item 证明: $\displaystyle \int_0^{\frac{\pi}{2}} f(\sin x,\cos x)dx=\int_0^{\frac{\pi}{2}}f(\cos x,\sin x)dx$, 其中 $f(u,v)$ 连续。
\stopitemize


\StopFrame

%%----------------------------------
%%----------------------------------




%%----------------------------------------------------------------
%% 逐步显示

\usemodule[tikz]         


\starttikzpicture[      scale=3,line cap=round
                        axes/.style=,         
                        important line/.style={very thick},
                        information text/.style={rounded corners,fill=red!10,inner sep=1ex} ]
        \draw[xshift=0.0cm]
                node[right,text width=17cm,information text]
                {
                        \bf
%%%%----------------------------

解: 根据定积分的性质
\startformula \darkgreen 
\int_a^b f(x)dx = \int_a^b f(a+b-x) dx
\stopformula
  



%%%%----------------------------

};

\stoptikzpicture
%% 逐步显示
%%----------------------------------------------------------------









%%----------------------------------------------------------------
%% 逐步显示

\usemodule[tikz]         


\starttikzpicture[      scale=3,line cap=round
                        axes/.style=,         
                        important line/.style={very thick},
                        information text/.style={rounded corners,fill=red!10,inner sep=1ex} ]
        \draw[xshift=0.0cm]
                node[right,text width=17cm,information text]
                {
                        \bf
%%%%----------------------------
 
 
\startformula
\startalign
  \NC  L \NC  = \int_0^{\frac{\pi}{2}} f(\sin x,\cos x)dx  
        = \int_0^{\frac{\pi}{2}} f\left[ \sin \left( \frac{\pi}{2} - x  \right), \cos \left( \frac{\pi}{2} - x  \right) \right]dx \NR
  \NC  \NC  = \int_0^{\frac{\pi}{2}} f(\cos x,\sin x)dx = R \NR
\stopalign
\stopformula
 



%%%%----------------------------

};

\stoptikzpicture
%% 逐步显示
%%----------------------------------------------------------------








\medskip
\FrameTitle{定积分的计算} % 解答

\StartFrame\bf\index{p154, 例9.21}\index{定积分的性质}\index{立方和公式}\index{倍角公式}
考点: 立方和公式、倍角公式。
\startitemize[n]
  \item 计算
\startformula
\int_0^{\frac{\pi}{2}}\frac{\sin ^3x}{\sin x+\cos x}dx
\stopformula

\stopitemize


\StopFrame

%%----------------------------------
%%----------------------------------





%%----------------------------------------------------------------
%% 逐步显示

\usemodule[tikz]         


\starttikzpicture[      scale=3,line cap=round
                        axes/.style=,         
                        important line/.style={very thick},
                        information text/.style={rounded corners,fill=red!10,inner sep=1ex} ]
        \draw[xshift=0.0cm]
                node[right,text width=17cm,information text]
                {
                        \bf
%%%%----------------------------

解: 根据积分区间与被积分函数的特征, 可利用定积分的性质
\startformula \darkgreen 
\int_0^{\frac{\pi}{2}} f(\sin x,\cos x)dx=\int_0^{\frac{\pi}{2}}f(\cos x,\sin x)dx
\stopformula
  

%%%%----------------------------

};

\stoptikzpicture
%% 逐步显示
%%----------------------------------------------------------------







%%----------------------------------------------------------------
%% 逐步显示

\usemodule[tikz]         


\starttikzpicture[      scale=3,line cap=round
                        axes/.style=,         
                        important line/.style={very thick},
                        information text/.style={rounded corners,fill=red!10,inner sep=1ex} ]
        \draw[xshift=0.0cm]
                node[right,text width=17cm,information text]
                {
                        \bf
%%%%----------------------------
 
 
\startformula
\startalign
  \NC  L \NC  = \int_0^{\frac{\pi}{2}}\frac{\sin ^3x}{\sin x+\cos x}dx 
              = \int_0^{\frac{\pi}{2}}\frac{\cos ^3x}{\cos x+\sin x}dx  
          = \frac{1}{2}\int_0^{\frac{\pi}{2}}\frac{\sin ^3x + \cos ^3x}{\cos x+\sin x}dx \NR
  \NC    \NC  = \frac{1}{2}\int_0^{\frac{\pi}{2}} \left( \sin^2x-\sin x\cos x+\cos^2x \right)dx\NR
  \NC    \NC   = \frac{1}{2}\int_0^{\frac{\pi}{2}} \left( 1 -\sin x\cos x \right) dx 
              = \frac{\pi - 1}{4} \NR
\stopalign
\stopformula
 



%%%%----------------------------

};

\stoptikzpicture
%% 逐步显示
%%----------------------------------------------------------------








\medskip
\FrameTitle{区间简化公式} % 解答

\StartFrame\bf\index{区间简化公式}\index{p155, 区间简化公式}
经典的区间简化公式一。 
\startitemize[n]
  \item 令 $\displaystyle x - \frac{a+b}{2}= \frac{b-a}{2}\sin t $, 有

\startformula
\int_a^b f(x)dx = \int_{- \frac{\pi}{2}}^{ \frac{\pi}{2}} f\left(  \frac{a+b}{2} +  \frac{b-a}{2}\sin t \right)\cdot  \frac{b-a}{2}\cdot \cos tdt
\stopformula

\stopitemize


\StopFrame

%%----------------------------------
%%----------------------------------





\medskip
\FrameTitle{区间简化公式} % 解答

\StartFrame\bf\index{区间简化公式}\index{p155,例9.22}
练习。 牢记结论。
\startitemize[n]
  \item 计算

\startformula
I = \int_1^3 \frac{dx}{\sqrt{(3-x)(x-1)}} 
\stopformula

\stopitemize


\StopFrame

%%----------------------------------
%%----------------------------------









%%----------------------------------------------------------------
%% 逐步显示

\usemodule[tikz]         


\starttikzpicture[      scale=3,line cap=round
                        axes/.style=,         
                        important line/.style={very thick},
                        information text/.style={rounded corners,fill=red!10,inner sep=1ex} ]
        \draw[xshift=0.0cm]
                node[right,text width=17cm,information text]
                {
                        \bf
%%%%----------------------------

解: 根据定积分的性质
\startformula \darkgreen 
\int_a^b f(x)dx = \int_{-\frac{\pi}{2}}^{\frac{\pi}{2}} f\left( \frac{a+b}{2} + \frac{b-a}{2}\sin t \right)\frac{b-a}{2}\cdot \cos t dt 
\stopformula
 


%%%%----------------------------

};

\stoptikzpicture
%% 逐步显示
%%----------------------------------------------------------------










%%----------------------------------------------------------------
%% 逐步显示

\usemodule[tikz]         


\starttikzpicture[      scale=3,line cap=round
                        axes/.style=,         
                        important line/.style={very thick},
                        information text/.style={rounded corners,fill=red!10,inner sep=1ex} ]
        \draw[xshift=0.0cm]
                node[right,text width=17cm,information text]
                {
                        \bf
%%%%----------------------------

 
令 $x-2 =\sin t$, 有 $3-x=1-\sin t$, $x-1 = 1+\sin t$, 
\startformula
\startalign
  \NC  I \NC  = \int_{-\frac{\pi}{2}}^{\frac{\pi}{2}} \frac{1}{\sqrt{(1-\sin t)(1-\sin t)}}\cos t dt  \NR
  \NC  \NC  = \int_{-\frac{\pi}{2}}^{\frac{\pi}{2}} \frac{1}{\cos t}\cdot \cos tdt   
       = \int_{-\frac{\pi}{2}}^{\frac{\pi}{2}} dt = \pi \NR
\stopalign
\stopformula
 



%%%%----------------------------

};

\stoptikzpicture
%% 逐步显示
%%----------------------------------------------------------------




%%----------------------------------------------------------------
%% 逐步显示

\usemodule[tikz]         


\starttikzpicture[      scale=3,line cap=round
                        axes/.style=,         
                        important line/.style={very thick},
                        information text/.style={rounded corners,fill=red!10,inner sep=1ex} ]
        \draw[xshift=0.0cm]
                node[right,text width=17cm,information text]
                {
                        \bf
%%%%----------------------------

{\darkgreen \bf 推广}本题的结论。 {\darkgreen \bf 牢记}公式。
\bigskip
设 $a<b$, 令 $\displaystyle x-\frac{a+b}{2}=\frac{b-a}{2}\sin t$, 则有
\startformula
\startalign
  \NC  \NC \quad \int_a^b \frac{dx}{\sqrt{(b-x)(x-a)}}  \NR
  \NC  \NC  = \int_{-\frac{\pi}{2}}^{\frac{\pi}{2}} \frac{\frac{b-a}{2}\cos t}{\sqrt{\left(\frac{b-a}{2}+\frac{b-a}{2}\sin t\right)\left(\frac{b-a}{2}-\frac{b-a}{2}\sin t\right)}} dt  \NR
  \NC  \NC  = \int_{-\frac{\pi}{2}}^{\frac{\pi}{2}} \frac{\frac{b-a}{2}\cos t}{\frac{b-a}{2}\cos t} dt    
       = \pi \NR
\stopalign
\stopformula
 



%%%%----------------------------

};

\stoptikzpicture
%% 逐步显示
%%----------------------------------------------------------------





\medskip
\FrameTitle{区间简化公式} % 解答

\StartFrame\bf\index{区间简化公式}\index{p155, 区间简化公式}
经典的区间简化公式二。 
\startitemize[n]
  \item 令 $x - a = (b-a) t $, 有

\startformula
\int_a^b f(x)dx = \int_0^1 (b-a) f[a + (b-a)t]dt 
\stopformula

\stopitemize


\StopFrame

%%----------------------------------
%%----------------------------------






\medskip
\FrameTitle{区间简化公式} % 解答

\StartFrame\bf\index{区间简化公式}\index{p155,例9.23}\index{积分公式表}\index{Wallis公式}
考点: 积分公式表。 牢记结论。
\startitemize[n]
  \item 计算

\startformula
I = \int_1^3 \sqrt{(3-x)(x-1)} dx 
\stopformula

\stopitemize


\StopFrame

%%----------------------------------
%%----------------------------------





%%----------------------------------------------------------------
%% 逐步显示

\usemodule[tikz]         


\starttikzpicture[      scale=3,line cap=round
                        axes/.style=,         
                        important line/.style={very thick},
                        information text/.style={rounded corners,fill=red!10,inner sep=1ex} ]
        \draw[xshift=0.0cm]
                node[right,text width=17cm,information text]
                {
                        \bf
%%%%----------------------------


\startformula \darkgreen 
\int \sqrt{a^2-x^2}dx  = \frac{a^2}{2}\arcsin \frac{x}{a}+\frac{x}{2} \sqrt{a^2-x^2} +C, \,(a>|x|\geq 0)
\stopformula
 
 



%%%%----------------------------

};

\stoptikzpicture
%% 逐步显示
%%----------------------------------------------------------------







%%----------------------------------------------------------------
%% 逐步显示

\usemodule[tikz]         


\starttikzpicture[      scale=3,line cap=round
                        axes/.style=,         
                        important line/.style={very thick},
                        information text/.style={rounded corners,fill=red!10,inner sep=1ex} ]
        \draw[xshift=0.0cm]
                node[right,text width=17cm,information text]
                {
                        \bf
%%%%----------------------------

解法一: 根据定积分的性质
\startformula \darkgreen 
\int_a^b f(x)dx = \int_0^1 (b-a) \cdot f[a+(b-a)t] dt 
\stopformula
 


%%%%----------------------------

};

\stoptikzpicture
%% 逐步显示
%%----------------------------------------------------------------








%%----------------------------------------------------------------
%% 逐步显示

\usemodule[tikz]         


\starttikzpicture[      scale=3,line cap=round
                        axes/.style=,         
                        important line/.style={very thick},
                        information text/.style={rounded corners,fill=red!10,inner sep=1ex} ]
        \draw[xshift=0.0cm]
                node[right,text width=17cm,information text]
                {
                        \bf
%%%%----------------------------


 
令 $x-1 = 2 t$, 有 $3-x=2-2 t$, 
\startformula
\startalign
  \NC  I \NC  = \int_0^1 2 \cdot \sqrt{(2-2t)\cdot 2t}\cdot dt   
         = 4\int_0^1 \sqrt{t-t^2}dt \NR
  \NC  \NC  = 4\int_0^1 \sqrt{\left( \frac{1}{2} \right)^2 - \left( t- \frac{1}{2}\right)^2}dt  
       = \frac{\pi}{2} \NR
\stopalign
\stopformula
 



%%%%----------------------------

};

\stoptikzpicture
%% 逐步显示
%%----------------------------------------------------------------




%%----------------------------------------------------------------
%% 逐步显示

\usemodule[tikz]         


\starttikzpicture[      scale=3,line cap=round
                        axes/.style=,         
                        important line/.style={very thick},
                        information text/.style={rounded corners,fill=red!10,inner sep=1ex} ]
        \draw[xshift=0.0cm]
                node[right,text width=17cm,information text]
                {
                        \bf
%%%%----------------------------

解法二: 令 $x-2=\sin t$, 有

\startformula
\startalign
  \NC  I \NC  = \int_{-\frac{\pi}{2}}^{\frac{\pi}{2}}  \sqrt{(1-\sin t)(1+\sin t)}\cos t dt  \NR
  \NC  \NC  = \int_{-\frac{\pi}{2}}^{\frac{\pi}{2}} \cos^2t dt 
            = 2 \int_0^{\frac{\pi}{2}} \cos^2 tdt = \frac{\pi}{2} \NR
\stopalign
\stopformula
Wallis公式。



%%%%----------------------------

};

\stoptikzpicture
%% 逐步显示
%%----------------------------------------------------------------



%%----------------------------------------------------------------
%% 逐步显示

\usemodule[tikz]         


\starttikzpicture[      scale=3,line cap=round
                        axes/.style=,         
                        important line/.style={very thick},
                        information text/.style={rounded corners,fill=red!10,inner sep=1ex} ]
        \draw[xshift=0.0cm]
                node[right,text width=17cm,information text]
                {
                        \bf
%%%%----------------------------

{\darkgreen \bf 推广}本题的结论。 {\darkgreen \bf 牢记}公式。
\bigskip
设 $a<b$, 令 $\displaystyle x-\frac{a+b}{2}=\frac{b-a}{2}\sin t$, 则有
\startformula
\startalign
  \NC  \NC \quad \int_a^b \sqrt{(b-x)(x-a)} dx  
        = \int_{-\frac{\pi}{2}}^{\frac{\pi}{2}} \frac{b-a}{2}\cos t\cdot \frac{b-a}{2}\cos t dt  \NR
  \NC  \NC  = \left(\frac{b-a}{2}\right)^2\int_{-\frac{\pi}{2}}^{\frac{\pi}{2}} \cos^2t dt    
      = \frac{(b-a)^2}{8}\pi \NR
\stopalign
\stopformula
 



%%%%----------------------------

};

\stoptikzpicture
%% 逐步显示
%%----------------------------------------------------------------




\medskip
\FrameTitle{对称性下定积分的计算} % 解答

\StartFrame\bf
考点: 定积分的几何意义。 定积分的定义。 定积分的性质。 偶倍奇零及其变形。 奇偶性。 解题关键, 找对称点(对称中心), 作变量代换。 化成对称区间, 再判断被积函数的奇偶性。
\index{偶倍奇零}
\index{定积分的几何意义}
\index{定积分的定义}
\index{定积分的性质}
\index{奇偶性}

\startitemize[n]
  \item 计算
\startformula
\int_0^2 (x-1)dx
\stopformula

\stopitemize


\StopFrame

%%----------------------------------
%%----------------------------------




%%----------------------------------------------------------------
%% 逐步显示

\usemodule[tikz]         


\starttikzpicture[      scale=3,line cap=round
                        axes/.style=,         
                        important line/.style={very thick},
                        information text/.style={rounded corners,fill=red!10,inner sep=1ex} ]
        \draw[xshift=0.0cm]
                node[right,text width=17cm,information text]
                {
                        \bf
%%%%----------------------------

法一: 表示以 $y=x-1$ 为曲线边, $x$ 轴, $x=0$, $x=2$ 围成的曲边梯形的面积。 $0$ 到 $1$ 上的负面积与 $1$ 到 $2$ 上的正面积相互抵消, 所以面积为 $0$。 
 



%%%%----------------------------

};

\stoptikzpicture
%% 逐步显示
%%----------------------------------------------------------------




%%----------------------------------------------------------------
%% 逐步显示

\usemodule[tikz]         


\starttikzpicture[      scale=3,line cap=round
                        axes/.style=,         
                        important line/.style={very thick},
                        information text/.style={rounded corners,fill=red!10,inner sep=1ex} ]
        \draw[xshift=0.0cm]
                node[right,text width=17cm,information text]
                {
                        \bf
%%%%----------------------------

法二: 换元, 令$x-1=t$, 得
\startformula
I=\int_{-1}^1 t dt
\stopformula

$y=t$ 在区间 $[-1,1]$ 上关于点 $(0,0)$ 对称, 为奇函数。 偶倍奇零。

%%%%----------------------------

};

\stoptikzpicture
%% 逐步显示
%%----------------------------------------------------------------



%%----------------------------------------------------------------
%% 逐步显示

\usemodule[tikz]         


\starttikzpicture[      scale=3,line cap=round
                        axes/.style=,         
                        important line/.style={very thick},
                        information text/.style={rounded corners,fill=red!10,inner sep=1ex} ]
        \draw[xshift=0.0cm]
                node[right,text width=17cm,information text]
                {
                        \bf
%%%%----------------------------

{\darkgreen \bf 推广}本题的结论。 {\darkgreen \bf 牢记}公式。
\bigskip
设 $a<b$, 则有
\startformula
I=\int_a^b \left(x-\frac{b-a}{2}\right) dx =0
\stopformula

$y=t$ 在区间 $[-1,1]$ 上关于点 $(0,0)$ 对称, 为奇函数。 偶倍奇零。

%%%%----------------------------

};

\stoptikzpicture
%% 逐步显示
%%----------------------------------------------------------------






\medskip
\FrameTitle{对称性下定积分的计算} % 解答

\StartFrame\bf
练习
\startitemize[n]
  \item 计算
\startformula
\int_0^2 x\cdot (x-1)\cdot (x-2)\cdot dx
\stopformula
\stopitemize


\StopFrame

%%----------------------------------
%%----------------------------------




%%----------------------------------------------------------------
%% 逐步显示

\usemodule[tikz]         


\starttikzpicture[      scale=3,line cap=round
                        axes/.style=,         
                        important line/.style={very thick},
                        information text/.style={rounded corners,fill=red!10,inner sep=1ex} ]
        \draw[xshift=0.0cm]
                node[right,text width=17cm,information text]
                {
                        \bf
%%%%----------------------------

对称中心 $(1,0)$, 作变量代换 $x-1=t$, 

%%%%----------------------------

};

\stoptikzpicture
%% 逐步显示
%%----------------------------------------------------------------







\medskip
\FrameTitle{对称性下定积分的计算} % 解答

\StartFrame\bf
练习
\startitemize[n]
  \item 计算
\startformula
\int_0^{2n}x\cdot(x-1)\cdot(x-2)\cdots [x-(2n-1)]\cdot(x-2n)\cdot dx
\stopformula
\stopitemize


\StopFrame

%%----------------------------------
%%----------------------------------




%%----------------------------------------------------------------
%% 逐步显示

\usemodule[tikz]         


\starttikzpicture[      scale=3,line cap=round
                        axes/.style=,         
                        important line/.style={very thick},
                        information text/.style={rounded corners,fill=red!10,inner sep=1ex} ]
        \draw[xshift=0.0cm]
                node[right,text width=17cm,information text]
                {
                        \bf
%%%%----------------------------


对称中心 $(n,0)$, 作变量代换 $x-n=t$, 
%%%%----------------------------

};

\stoptikzpicture
%% 逐步显示
%%----------------------------------------------------------------







\medskip
\FrameTitle{对称性下定积分的计算} % 解答

\StartFrame\bf\index{定积分的几何意义}
练习
\startitemize[n]
  \item 计算
\startformula
\int_0^{2020}x\cdot(x-1)\cdot(x-2)\cdots (x-2019)\cdot(x-2020)\cdot dx
\stopformula
\stopitemize


\StopFrame

%%----------------------------------
%%----------------------------------




%%----------------------------------------------------------------
%% 逐步显示

\usemodule[tikz]         


\starttikzpicture[      scale=3,line cap=round
                        axes/.style=,         
                        important line/.style={very thick},
                        information text/.style={rounded corners,fill=red!10,inner sep=1ex} ]
        \draw[xshift=0.0cm]
                node[right,text width=17cm,information text]
                {
                        \bf
%%%%----------------------------
对称中心 $(1010,0)$, 作变量代换 $x-1010=t$, 


%%%%----------------------------

};

\stoptikzpicture
%% 逐步显示
%%----------------------------------------------------------------






\medskip
\FrameTitle{对称性下定积分的计算} % 解答

\StartFrame\bf\index{p158, 例9.25}\index{定积分的几何意义}
考点: 定积分的几何意义。 二次多项式配方。
\startitemize[n]
  \item 计算
\startformula
\int_0^4x\sqrt{4x-x^2}\cdot dx
\stopformula
\stopitemize


\StopFrame

%%----------------------------------
%%----------------------------------



%%----------------------------------------------------------------
%% 逐步显示

\usemodule[tikz]         


\starttikzpicture[      scale=3,line cap=round
                        axes/.style=,         
                        important line/.style={very thick},
                        information text/.style={rounded corners,fill=red!10,inner sep=1ex} ]
        \draw[xshift=0.0cm]
                node[right,text width=17cm,information text]
                {
                        \bf
%%%%----------------------------
解: 
由 $4x-x^2=2^2-(x-2)^2$, 令 $x-2=t$, 则有
\startformula
\startalign
  \NC I \NC = \int_{-2}^2 (t+2)\cdot\sqrt{2^2-t^2}\cdot dt  \NR
  \NC   \NC = \int_{-2}^2 {\darkgreen t\cdot\sqrt{2^2-t^2}}\cdot dt + 2 {\darkgreen{\int_{-2}^2 \sqrt{2^2-t^2}\cdot dt} }  
          = 4\pi \NR
\stopalign
\stopformula
 



%%%%----------------------------

};

\stoptikzpicture
%% 逐步显示
%%----------------------------------------------------------------







\medskip
\FrameTitle{对称性下定积分的计算} % 解答

\StartFrame\bf\index{p158, 例9.25}
考点: 定积分的几何意义。 二次多项式配方。
\startitemize[n]
  \item 计算
\startformula
\int_0^2 (2x+1)\cdot \sqrt{2x-x^2}\cdot dx
\stopformula
\stopitemize


\StopFrame

%%----------------------------------
%%----------------------------------





%%----------------------------------------------------------------
%% 逐步显示

\usemodule[tikz]         


\starttikzpicture[      scale=3,line cap=round
                        axes/.style=,         
                        important line/.style={very thick},
                        information text/.style={rounded corners,fill=red!10,inner sep=1ex} ]
        \draw[xshift=0.0cm]
                node[right,text width=17cm,information text]
                {
                        \bf
%%%%----------------------------
解: 
由 $2x-x^2=1-(x-1)^2$, 令 $x-1=t$, 则有
\startformula
\startalign
  \NC I \NC = \int_{-1}^1 (2t+3)\cdot\sqrt{1-t^2}\cdot dt  \NR
  \NC   \NC = 2\int_{-1}^1 {\darkgreen t\cdot\sqrt{1-t^2}}\cdot dt + 3 {\darkgreen{\int_{-1}^1 \sqrt{1-t^2}\cdot dt} }
      = \frac{3}{2}\pi \NR
\stopalign
\stopformula
 



%%%%----------------------------

};

\stoptikzpicture
%% 逐步显示
%%----------------------------------------------------------------






\medskip
\FrameTitle{定积分分部积分中升阶降阶} % 解答

\StartFrame\bf




概念。
\startitemize[n]
  \item 升阶: 
\startformula
\int_a^x f(t)dt  {\darkgreen \Rightarrow} f(x) {\darkgreen \Rightarrow} f'(x) {\darkgreen \Rightarrow} f''(x)
\stopformula
  \item 降阶: 
\startformula
\int_a^x f(t)dt  {\darkgreen \Leftarrow} f(x)  {\darkgreen \Leftarrow} f'(x)  {\darkgreen \Leftarrow} f''(x)
\stopformula
\stopitemize

沟通的桥梁: 分部积分公式。 变上限积分求导。
\StopFrame

%%----------------------------------
%%----------------------------------






\medskip
\FrameTitle{定积分分部积分中升阶降阶} % 解答

\StartFrame\bf\index{p158, 例9.26}\index{降阶}
\index{导数定义}
\index{第一种重要类型极限}
\index{原函数定义}
\index{分式极限}
\index{分部积分}
\index{牛莱公式}
\index{变量代换}
\index{换元换限}
考点: 降阶。  $\displaystyle \int_a^x f(t)dt {\darkgreen \Leftarrow} f(x)  {\darkgreen \Leftarrow} f'(x) $ 。 导数定义。 第一种重要类型极限。 原函数的定义。 求分式的极限。 分部积分。 牛莱公式。 变量代换。 换元必换限。

\startitemize[n]
  \item 设 $g(x)$ 的一个原函数为 $\ln (x+1)$, 求 $\displaystyle \int_0^1 f(x)\cdot dx$, 其中
\startformula
f(x)=\lim _{t\rightarrow \infty}t^2\cdot \left[g\left(2x+\frac{1}{t}\right)-g(2x)\right]\cdot \sin \frac{x}{t}
\stopformula
\stopitemize


\StopFrame

%%----------------------------------
%%----------------------------------




%%----------------------------------------------------------------
%% 逐步显示

\usemodule[tikz]         


\starttikzpicture[      scale=3,line cap=round
                        axes/.style=,         
                        important line/.style={very thick},
                        information text/.style={rounded corners,fill=red!10,inner sep=1ex} ]
        \draw[xshift=0.0cm]
                node[right,text width=17cm,information text]
                {
                        \bf
%%%%----------------------------
解: 
 
\startformula
\startalign
  \NC f(x) \NC = \lim _{t\rightarrow \infty} 
      {\darkgreen \frac{g\left(2x+\frac{1}{t}\right)-g(2x)}{\frac{1}{t}}}
      \cdot x\cdot
      {\darkgreen \frac{\sin \frac{x}{t}}{\frac{x}{t}}}   
            = 1\cdot x \cdot g'(2x)  
            = xg'(2x) \NR
\stopalign
\stopformula
 
故 $f(x)=xg'(2x)$, 所以令 $2x=t$, 有

\startformula
\startalign
  \NC I \NC = \int_0^1 f(x)dx 
            = \int_0^1 xg'(2x)dx  
        = \int_0^2 \frac{t}{2}\cdot g'(t)\cdot \frac{1}{2}\cdot dt 
            = \frac{1}{4}\int_0^2 x\cdot d\left[\,g(x)\,\right] \NR
  \NC  \NC = \frac{1}{4}\left[ xg(x)\big|_0^2 -\int_0^2 g(x)dx \right]
           = \frac{1}{4}\left.\left[ x\cdot \frac{1}{1+x} -\ln (x+1) \right]\right|_0^2 \NR
  \NC \NC  = \frac{1}{4}\left(\frac{2}{3}-\ln 3\right)
           = \frac{1}{6}-\frac{1}{4}\ln 3 \NR
\stopalign
\stopformula

%%%%----------------------------

};

\stoptikzpicture
%% 逐步显示
%%----------------------------------------------------------------








\medskip
\FrameTitle{定积分分部积分中升阶降阶} % 解答

\StartFrame\bf\index{p159, 例9.28}\index{变上限积分求导。}
考点: 升阶。  $\displaystyle \int_a^x f(t)dt \Rightarrow f(x) \Rightarrow f'(x) $ 。 变上限积分求导。 分部积分。 牛莱公式。 变量代换。 换元换限。 第一换元法。 第二换元法。
\startitemize[n]  \item 设
\startformula
f(x)=\int_0^x e^{-t^2+2t}\cdot dt
\stopformula
求
\startformula
I = \int_0^1 (x-1)^2\cdot f(x)\cdot dx
\stopformula
\stopitemize


\StopFrame

%%----------------------------------
%%----------------------------------





%%----------------------------------------------------------------
%% 逐步显示

\usemodule[tikz]         


\starttikzpicture[      scale=3,line cap=round
                        axes/.style=,         
                        important line/.style={very thick},
                        information text/.style={rounded corners,fill=red!10,inner sep=1ex} ]
        \draw[xshift=0.0cm]
                node[right,text width=17cm,information text]
                {
                        \bf
%%%%----------------------------
解:  
$f(0)=0$, $f'(x)=e^{-x^2+2x}$, 有


%%%%----------------------------

};

\stoptikzpicture
%% 逐步显示
%%----------------------------------------------------------------








%%----------------------------------------------------------------
%% 逐步显示

\usemodule[tikz]         


\starttikzpicture[      scale=3,line cap=round
                        axes/.style=,         
                        important line/.style={very thick},
                        information text/.style={rounded corners,fill=red!10,inner sep=1ex} ]
        \draw[xshift=0.0cm]
                node[right,text width=17cm,information text]
                {
                        \bf
%%%%----------------------------

\startformula
\startalign
  \NC I \NC = \left.\frac{1}{3}(x-1)^3\cdot f(x)\right|_0^1 -\frac{1}{3}\int_0^1 (x-1)^3\cdot f'(x)\cdot dx \NR
  \NC  \NC = -\frac{1}{3}\int_0^1 (x-1)^3\cdot e^{-x^2+2x}\cdot dx  \NR
  \NC \NC  = -\frac{1}{6}\int_0^1 (x-1)^2\cdot e^{-(x-1)^2+1}\cdot d\left[(x-1)^2\right] \NR
\stopalign
\stopformula



%%%%----------------------------

};

\stoptikzpicture
%% 逐步显示
%%----------------------------------------------------------------







%%----------------------------------------------------------------
%% 逐步显示

\usemodule[tikz]         


\starttikzpicture[      scale=3,line cap=round
                        axes/.style=,         
                        important line/.style={very thick},
                        information text/.style={rounded corners,fill=red!10,inner sep=1ex} ]
        \draw[xshift=0.0cm]
                node[right,text width=17cm,information text]
                {
                        \bf
%%%%----------------------------


令 $t=(x-1)^2$, 则

\startformula
I = -\frac{e}{6}\int_1^0 t e^{-t}dt =\frac{1}{6}(e-2)
\stopformula

%%%%----------------------------

};

\stoptikzpicture
%% 逐步显示
%%----------------------------------------------------------------






\medskip
\FrameTitle{分段函数的定积分} % 解答

\StartFrame\bf\index{p160, 例9.29}
\index{变量代换}
\index{换元换限}
\index{路径分段}
\index{积分性质}
\index{牛莱公式}
\index{分段函数}
\index{积分公式}
考点: 变量代换。 换元换限。 路径分段。  积分性质。 牛莱公式。 分段函数。
\startitemize[n]
  \item 计算
\startformula
I = \int_{-2}^2f(x-1)\cdot dx
\stopformula
其中
$
 f(x) = \startmathcases
   \NC e^{-x}, \NC  $x\geq 0$ \NR
   \NC 1+x^2 ,\NC  $x<0$ \NR
\stopmathcases
$
\stopitemize


\StopFrame

%%----------------------------------
%%----------------------------------







%%----------------------------------------------------------------
%% 逐步显示

\usemodule[tikz]         


\starttikzpicture[      scale=3,line cap=round
                        axes/.style=,         
                        important line/.style={very thick},
                        information text/.style={rounded corners,fill=red!10,inner sep=1ex} ]
        \draw[xshift=0.0cm]
                node[right,text width=17cm,information text]
                {
                        \bf
%%%%----------------------------
解: 
令 $t=x-1$, 则

\startformula
\startalign
  \NC I \NC = \int_{-2}^2 f(x-1)dx 
            = \int_{-3}^1 f(t)dt   
            = \int_{-3}^0 f(t)dt +  \int_0^1 f(t)dt  \NR
  \NC  \NC = \int_{-3}^0 (1+t^2)dt + \int_0^1 e^{-t}dt  
      = \left.\left.\left(t+\frac{t^3}{3}\right)\right|_{-3}^0 -e^{-t}\right|_0^1
          = 13 - e^{-1} \NR
\stopalign
\stopformula

%%%%----------------------------

};

\stoptikzpicture
%% 逐步显示
%%----------------------------------------------------------------







\medskip
\FrameTitle{变限积分的计算} % 解答

\StartFrame\bf
分段函数的变限积分。

\index{变限积分}
\index{变量代换}
\index{换元换限}
\index{路径分段}
\index{积分性质}
\index{牛莱公式}
\index{分段函数}
\index{积分公式}
考点: 变量代换。 换元换限。 路径分段。  积分性质。 牛莱公式。 分段函数。

\startitemize[n]
  \item 求
\startformula
F(x)=\int_{-1}^x f(t)\cdot dt
\stopformula
的表达式, 其中

\startformula
 f(x) = \startmathcases
   \NC 2x+\displaystyle\frac{3}{2}x^2, \NC  $-1\leq x < 0$ \NR
   \NC \displaystyle\frac{xe^x}{(e^x+1)^2} ,\NC  $0\leq x\leq 1$ \NR
\stopmathcases
\stopformula
\stopitemize


\StopFrame

%%----------------------------------
%%----------------------------------







%%----------------------------------------------------------------
%% 逐步显示

\usemodule[tikz]         


\starttikzpicture[      scale=3,line cap=round
                        axes/.style=,         
                        important line/.style={very thick},
                        information text/.style={rounded corners,fill=red!10,inner sep=1ex} ]
        \draw[xshift=0.0cm]
                node[right,text width=17cm,information text]
                {
                        \bf
%%%%----------------------------
解: 
当 $x\in [-1,0)$ 时, 
\startformula
\startalign
  \NC F(x) \NC = \int_{-1}^x f(t)dt 
               = \int_{-1}^x \left( 2t+\frac{3}{2}t^2 \right)dt 
               = \left.\left( t^2+ \frac{1}{2}t^3\right)\right|_{-1}^x
               = \frac{1}{2}x^3+x^2-\frac{1}{2} \NR
\stopalign
\stopformula


%%%%----------------------------

};

\stoptikzpicture
%% 逐步显示
%%----------------------------------------------------------------






%%----------------------------------------------------------------
%% 逐步显示

\usemodule[tikz]         


\starttikzpicture[      scale=3,line cap=round
                        axes/.style=,         
                        important line/.style={very thick},
                        information text/.style={rounded corners,fill=red!10,inner sep=1ex} ]
        \draw[xshift=0.0cm]
                node[right,text width=17cm,information text]
                {
                        \bf
%%%%----------------------------

当 $x\in [0,1]$ 时, 
\startformula
\startalign
  \NC F(x) \NC = \int_{-1}^x f(t)dt
               = \int_{-1}^0 f(t)dt + \int_0^x f(t)dt \NR
  \NC \NC = \left.\left( t^2+\frac{1}{2}t^3 \right)\right|_{-1}^0 + \int _0^x \frac{te^t}{(e^t+1)^2}dt  
       = -\frac{1}{2}+\int_0^x (-t)\cdot d\left( \frac{1}{e^t+1} \right) \NR
  \NC  \NC = -\frac{1}{2}-\left.\frac{t}{e^t+1}\right|_0^x +  \int _0^x \frac{1}{e^t+1}dt  
       = -\frac{1}{2}- \frac{x}{e^x+1} +  \int _0^x \frac{d(e^t)}{e^t(e^t+1)} \NR
\stopalign
\stopformula

%%%%----------------------------

};

\stoptikzpicture
%% 逐步显示
%%----------------------------------------------------------------





%%----------------------------------------------------------------
%% 逐步显示

\usemodule[tikz]         


\starttikzpicture[      scale=3,line cap=round
                        axes/.style=,         
                        important line/.style={very thick},
                        information text/.style={rounded corners,fill=red!10,inner sep=1ex} ]
        \draw[xshift=0.0cm]
                node[right,text width=17cm,information text]
                {
                        \bf
%%%%----------------------------

\startformula
\startalign
  \NC F(x) \NC =  -\frac{1}{2}- \frac{x}{e^x+1} +  \int _0^x \left( \frac{1}{e^t} - \frac{1}{e^t+1} \right) d(e^t) \NR
  \NC      \NC =  -\frac{1}{2}- \frac{x}{e^x+1} +  \left.\ln \frac{e^t}{e^t+1}\right|_0^x  
               =  -\frac{1}{2}- \frac{x}{e^x+1} +  \ln \frac{e^x}{e^x+1} - \ln \frac{1}{2} \NR
\stopalign
\stopformula

%%%%----------------------------

};

\stoptikzpicture
%% 逐步显示
%%----------------------------------------------------------------





%%----------------------------------------------------------------
%% 逐步显示

\usemodule[tikz]         


\starttikzpicture[      scale=3,line cap=round
                        axes/.style=,         
                        important line/.style={very thick},
                        information text/.style={rounded corners,fill=red!10,inner sep=1ex} ]
        \draw[xshift=0.0cm]
                node[right,text width=17cm,information text]
                {
                        \bf
%%%%----------------------------

所以, 有

\startformula
 F(x) = \startmathcases
   \NC \displaystyle\frac{1}{2}x^3+x^2-\frac{1}{2} , \NC  $-1\leq x < 0$ \NR
   \NC \displaystyle-\frac{1}{2}- \frac{x}{e^x+1} +  \ln \frac{e^x}{e^x+1} - \ln \frac{1}{2} ,\NC  $0\leq x\leq 1$ \NR
\stopmathcases
\stopformula

%%%%----------------------------

};

\stoptikzpicture
%% 逐步显示
%%----------------------------------------------------------------







%%----------------------------------------------------------------
%% 逐步显示

\usemodule[tikz]         


\starttikzpicture[      scale=3,line cap=round
                        axes/.style=,         
                        important line/.style={very thick},
                        information text/.style={rounded corners,fill=red!10,inner sep=1ex} ]
        \draw[xshift=0.0cm]
                node[right,text width=17cm,information text]
                {
                        \bf
%%%%----------------------------

{\darkgreen 练习}\index{练习}

\startformula
\int \frac{1}{e^x+1}dx
\stopformula

%%%%----------------------------

};

\stoptikzpicture
%% 逐步显示
%%----------------------------------------------------------------








\medskip
\FrameTitle{变限积分的计算} % 解答

\StartFrame\bf\index{变限积分}\index{p161, 例9.32}
直接求导型。 求导公式。
\startitemize[n]
  \item 
\startformula
\left[\int_a^{\phi(x)}f(t)dt\right]'_x = f[\phi(x)]\cdot\phi'(x)
\stopformula
  \item 
\startformula
\left[\int_{\phi_1(x)}^{\phi_2(x)}f(t)dt\right]'_x = f[\phi_2(x)]\cdot\phi_2'(x)-f[\phi_1(x)]\cdot\phi_1'(x)
\stopformula

\stopitemize


\StopFrame

%%----------------------------------
%%----------------------------------







\medskip
\FrameTitle{变限积分的计算} % 解答

\StartFrame\bf
\index{零点定理}
\index{单调性}
\index{导数}
\index{零点}
\index{辅助函数}
\index{不等式}
\index{变限积分}
\index{连续}

考点: 零点定理。 单调性。 导数。 零点。 辅助函数。 不等式。 变限积分。 连续。
\startitemize[n] 
  \item 设 $f(x)$ 在 $[a,b]$ 上连续, 且 $f(x)>0$,  则方程\crlf
\startformula
\int_a^xf(t)\cdot dt+\int_b^x \frac{1}{f(t)}\cdot dt=0
\stopformula
在 $(a,b)$ 内的根有 (\kern2em) 个。

$(A)$  0\kern2em $(B)$ $1$ \kern2em $(C)$  $2$ \kern2em $(D)$ $3$ \kern2em
\stopitemize


\StopFrame

%%----------------------------------
%%----------------------------------






%%----------------------------------------------------------------
%% 逐步显示

\usemodule[tikz]         


\starttikzpicture[      scale=3,line cap=round
                        axes/.style=,         
                        important line/.style={very thick},
                        information text/.style={rounded corners,fill=red!10,inner sep=1ex} ]
        \draw[xshift=0.0cm]
                node[right,text width=17cm,information text]
                {
                        \bf
%%%%----------------------------

解:令 

\startformula
F(x)=\int_a^xf(t)dt+\int_b^x \frac{1}{f(t)}dt
\stopformula
则 $F(x)$ 在 $[a,b]$ 上连续, 且
\startformula
F(a)=\int_b^a \frac{1}{f(t)}dt <0, \kern2em F(b)=\int_a^b \frac{1}{f(t)}dt > 0
\stopformula
由零点定理知 $F(x)$ 在 $(a,b)$ 内有零点。 
%%%%----------------------------

};

\stoptikzpicture
%% 逐步显示
%%----------------------------------------------------------------





%%----------------------------------------------------------------
%% 逐步显示

\usemodule[tikz]         


\starttikzpicture[      scale=3,line cap=round
                        axes/.style=,         
                        important line/.style={very thick},
                        information text/.style={rounded corners,fill=red!10,inner sep=1ex} ]
        \draw[xshift=0.0cm]
                node[right,text width=17cm,information text]
                {
                        \bf
%%%%----------------------------
 

再由
\startformula
F'(x)=f(x)+\frac{1}{f(x)}>0
\stopformula
 知 $F(x)$ 单调增加, 所以 $F(x)$ 在 $(a, b)$ 内最多只有一个零点。
%%%%----------------------------

};

\stoptikzpicture
%% 逐步显示
%%----------------------------------------------------------------







\medskip
\FrameTitle{变限积分的计算} % 解答

\StartFrame\bf\index{p162, 例9.33}
\index{反函数}
\index{反函数性质}
\index{可导}
\index{变限积分}
\index{连续}
\index{连续定义}
\index{积分方程}
\index{微分方程}
\index{初始条件}

考点: 反函数。 反函数性质。 可导。 变限积分。 连续。 连续定义。 积分方程。 微分方程。 初始条件。
\startitemize[n]  
  \item 设 $f(x)$ 在 $[0,+\infty)$ 上可导, $f(0)=0$, 且其反函数为 $g(x)$, 若 
\startformula
\int_0^{f(x)} g(t)dt=x^2e^x
\stopformula
 求 $f(x)$ 的表达式.

\stopitemize


\StopFrame

%%----------------------------------
%%----------------------------------




%%----------------------------------------------------------------
%% 逐步显示

\usemodule[tikz]         


\starttikzpicture[      scale=3,line cap=round
                        axes/.style=,         
                        important line/.style={very thick},
                        information text/.style={rounded corners,fill=red!10,inner sep=1ex} ]
        \draw[xshift=0.0cm]
                node[right,text width=17cm,information text]
                {
                        \bf
%%%%----------------------------
 

解: 等式两边对 $x$ 求导, 得 

\startformula
g[f(x)]\cdot f'(x)= 2 xe^x + x^2 e^x
\stopformula
由 $g[f(x)]=0$, 知
$xf'(x)=2xe^x+x^2e^x$

当 $x\neq 0$时, $f'(x)=2e^x+xe^x$, 积分得 $f(x)=(x+1)e^x+C$

由于 $f(x)$ 在 $x=0$ 处右连续, 故由
\startformula
0=f(0)=\lim_{x\rightarrow 0^+} f(x) = \lim_{x\rightarrow 0^+} [(x+1)e^x+C]
\stopformula
得 $C=-1$, 因此 $f(x)=(x+1)e^x-1$
%%%%----------------------------

};

\stoptikzpicture
%% 逐步显示
%%----------------------------------------------------------------









\medskip
\FrameTitle{变限积分的计算} % 解答

\StartFrame\bf
换元求导型: 先用换元法, 再用求导公式。 

\StopFrame

%%----------------------------------
%%----------------------------------

 



\medskip
\FrameTitle{变限积分的计算} % 解答

\StartFrame\bf\index{p162, 例9.34}
\index{变限积分}
\index{换元法}
\index{反函数}
\index{反函数性质}
\index{极限}
\index{可导}
\index{连续}
\index{积分}
\index{微分方程}
\index{积分方程}

考点: 变限积分。 换元法。 反函数。 极限。 可导。 连续。 积分。 微分方程。 积分方程。 反函数性质。
\startitemize[n] 
  \item 设 $f(x)$ 在 $[0,+\infty)$ 上可导, $f(0)=0$, 且其反函数为 $g(x)$, 若 
\startformula
\int_x^{x+f(x)} g(t-x)dt=x^2\ln (1+x)
\stopformula
求 $f(x)$. 
\stopitemize


\StopFrame

%%----------------------------------
%%----------------------------------



%%----------------------------------------------------------------
%% 逐步显示

\usemodule[tikz]         


\starttikzpicture[      scale=3,line cap=round
                        axes/.style=,         
                        important line/.style={very thick},
                        information text/.style={rounded corners,fill=red!10,inner sep=1ex} ]
        \draw[xshift=0.0cm]
                node[right,text width=17cm,information text]
                {
                        \bf
%%%%----------------------------
 

解: 令 $\darkgreen t-x=u$, 则 $dt=du$, 于是
\startformula
\int_x^{x+f(x)} g(t-x)\cdot dt
= {\darkgreen \int_0^{f(x)} g(u)\cdot du }
= x^2 \cdot \ln (1+x) 
\stopformula

 


%%%%----------------------------

};

\stoptikzpicture
%% 逐步显示
%%----------------------------------------------------------------






%%----------------------------------------------------------------
%% 逐步显示

\usemodule[tikz]         


\starttikzpicture[      scale=3,line cap=round
                        axes/.style=,         
                        important line/.style={very thick},
                        information text/.style={rounded corners,fill=red!10,inner sep=1ex} ]
        \draw[xshift=0.0cm]
                node[right,text width=17cm,information text]
                {
                        \bf
%%%%----------------------------
 

由于 $g[f(x)]=x$, 上式两边对 $x$ 求导, 有
\startformula
xf'(x) = 2x\cdot \ln (1+x) +\frac{x^2}{1+x}
\stopformula
 


%%%%----------------------------

};

\stoptikzpicture
%% 逐步显示
%%----------------------------------------------------------------





%%----------------------------------------------------------------
%% 逐步显示

\usemodule[tikz]         


\starttikzpicture[      scale=3,line cap=round
                        axes/.style=,         
                        important line/.style={very thick},
                        information text/.style={rounded corners,fill=red!10,inner sep=1ex} ]
        \draw[xshift=0.0cm]
                node[right,text width=17cm,information text]
                {
                        \bf
%%%%----------------------------
 
 

当 $x\neq 0$ 时, 有
\startformula
f'(x) = 2\ln (1+x) +\frac{x}{1+x}
\stopformula


%%%%----------------------------

};

\stoptikzpicture
%% 逐步显示
%%----------------------------------------------------------------






%%----------------------------------------------------------------
%% 逐步显示

\usemodule[tikz]         


\starttikzpicture[      scale=3,line cap=round
                        axes/.style=,         
                        important line/.style={very thick},
                        information text/.style={rounded corners,fill=red!10,inner sep=1ex} ]
        \draw[xshift=0.0cm]
                node[right,text width=17cm,information text]
                {
                        \bf
%%%%----------------------------
 
 

两边积分, 有
\startformula
\startalign
  \NC f(x) \NC  = \int \left [ 2\ln (1+x) +\frac{x}{1+x} \right]dx \NR
  \NC  \NC = 2\left[\ln (1+x)+x\ln (1+x) -x\right ] +x -\ln (1+x) + C \NR
  \NC  \NC = \ln (1+x) +2x \ln (1+x)-x+C \NR
\stopalign
\stopformula


%%%%----------------------------

};

\stoptikzpicture
%% 逐步显示
%%----------------------------------------------------------------




%%----------------------------------------------------------------
%% 逐步显示

\usemodule[tikz]         


\starttikzpicture[      scale=3,line cap=round
                        axes/.style=,         
                        important line/.style={very thick},
                        information text/.style={rounded corners,fill=red!10,inner sep=1ex} ]
        \draw[xshift=0.0cm]
                node[right,text width=17cm,information text]
                {
                        \bf
%%%%----------------------------
 
上式两边求极限,可知 
\startformula
\lim_{x\rightarrow 0^+} f(x)=C
\stopformula



%%%%----------------------------

};

\stoptikzpicture
%% 逐步显示
%%----------------------------------------------------------------






%%----------------------------------------------------------------
%% 逐步显示

\usemodule[tikz]         


\starttikzpicture[      scale=3,line cap=round
                        axes/.style=,         
                        important line/.style={very thick},
                        information text/.style={rounded corners,fill=red!10,inner sep=1ex} ]
        \draw[xshift=0.0cm]
                node[right,text width=17cm,information text]
                {
                        \bf
%%%%----------------------------
 
由于 $f(x)$ 在 $x=0$ 处右连续, 又 $f(0)=0$, 解得 $C=0$, 于是
\startformula
f(x)=\ln (1+x) +2x \ln (1+x)-x
\stopformula


%%%%----------------------------

};

\stoptikzpicture
%% 逐步显示
%%----------------------------------------------------------------







\medskip
\FrameTitle{变限积分的计算} % 解答

\StartFrame\bf\index{p163, 例9.35}
\index{平均值}
\index{变限积分}
\index{连续}
\index{积分}
\index{原函数}
\index{Wallis公式}

考点: 平均值。 变限积分。 连续。 积分。 原函数。 Wallis公式。

见到 $\darkgreen f(x)\displaystyle \int_0^x f(t)dt$, 一般令 $\darkgreen F(x)=\displaystyle \int_0^x f(t)dt$。
\startitemize[n]  
  \item 设 $f(x)$ 在 $(-\infty,+\infty)$ 上非负连续, 且\crlf
\startformula
f(x)\int_0^xf(x-t)dt=\sin ^4x
\stopformula
 求 $f(x)$ 在 $[0,\pi]$ 上的平均值。
\stopitemize


\StopFrame

%%----------------------------------
%%----------------------------------





%%----------------------------------------------------------------
%% 逐步显示

\usemodule[tikz]         


\starttikzpicture[      scale=3,line cap=round
                        axes/.style=,         
                        important line/.style={very thick},
                        information text/.style={rounded corners,fill=red!10,inner sep=1ex} ]
        \draw[xshift=0.0cm]
                node[right,text width=17cm,information text]
                {
                        \bf
%%%%----------------------------
 
解: 令 $x-t=u$, 则
\startformula
\int_0^x f(x-t)dt = \int_0^x f(u)du
\stopformula

令 $F(x)=\displaystyle \int_0^x f(u)du$, 于是
\startformula
F(x)\cdot F'(x)=\sin ^4x
\stopformula 

%%%%----------------------------

};

\stoptikzpicture
%% 逐步显示
%%----------------------------------------------------------------






%%----------------------------------------------------------------
%% 逐步显示

\usemodule[tikz]         


\starttikzpicture[      scale=3,line cap=round
                        axes/.style=,         
                        important line/.style={very thick},
                        information text/.style={rounded corners,fill=red!10,inner sep=1ex} ]
        \draw[xshift=0.0cm]
                node[right,text width=17cm,information text]
                {
                        \bf
%%%%----------------------------
 
 
上式在 $[0,\pi]$ 上积分, 有
\startformula
\int_0^{\pi} F(x)F'(x)dx 
= \int_0^{\pi} F(x)d[F(x)]
= \left.\frac{1}{2} F^2(x)\right|_0^x
= \int_0^{\pi} \sin^4x dx
= \frac{3}{8}\pi
\stopformula

%%%%----------------------------

};

\stoptikzpicture
%% 逐步显示
%%----------------------------------------------------------------





%%----------------------------------------------------------------
%% 逐步显示

\usemodule[tikz]         


\starttikzpicture[      scale=3,line cap=round
                        axes/.style=,         
                        important line/.style={very thick},
                        information text/.style={rounded corners,fill=red!10,inner sep=1ex} ]
        \draw[xshift=0.0cm]
                node[right,text width=17cm,information text]
                {
                        \bf
%%%%----------------------------
 
 
故 $F(\pi) = \displaystyle \frac{\sqrt{3\pi}}{2}$, 则 $f(x)$ 在 $[0,\pi]$ 上的平均值为
\startformula
\frac{1}{\pi} \int_0^{\pi} f(x)dx=\frac{1}{2}\sqrt{\frac{3}{\pi}}
\stopformula

%%%%----------------------------

};

\stoptikzpicture
%% 逐步显示
%%----------------------------------------------------------------








\medskip
\FrameTitle{变限积分的计算} % 解答

\StartFrame\bf
拆分求导型: 被积函数含有绝对值, 先拆分区间, 化成若干个积分, 再用变限积分的求导公式。
 


\StopFrame

%%----------------------------------
%%----------------------------------






\medskip
\FrameTitle{变限积分的计算} % 解答

\StartFrame\bf\index{p163, 例9.36}
\index{变限积分}
\index{最值}
\index{绝对值}
\index{积分区间分段}
\index{导数}
\index{驻点}

考点: 变限积分。 最值。 绝对值。 积分区间分段。 导数。 驻点。 
\startitemize[n] 
  \item 设 $|t|\leq 1$, 求积分
\startformula
I(t)=\int_{-1}^1|x-t|\cdot e^{2x}dx
\stopformula
的最大值。
\stopitemize


\StopFrame

%%----------------------------------
%%----------------------------------







%%----------------------------------------------------------------
%% 逐步显示

\usemodule[tikz]         


\starttikzpicture[      scale=3,line cap=round
                        axes/.style=,         
                        important line/.style={very thick},
                        information text/.style={rounded corners,fill=red!10,inner sep=1ex} ]
        \draw[xshift=0.0cm]
                node[right,text width=17cm,information text]
                {
                        \bf
%%%%----------------------------
 
解: 由积分性质,得
\startformula
\startalign
  \NC I(t) \NC  = \int_{-1}^t (t-x)\cdot e^{2x}dt +\int_t^1 (x-t)\cdot e^{2x}dt \NR
  \NC \NC = t\int_{-1}^t  e^{2x}dt - \int_{-1}^t x\cdot e^{2x}dt +\int_t^1 x\cdot e^{2x}dt -t\int_t^1  e^{2x}dt  \NR
\stopalign
\stopformula

%%%%----------------------------

};

\stoptikzpicture
%% 逐步显示
%%----------------------------------------------------------------





%%----------------------------------------------------------------
%% 逐步显示

\usemodule[tikz]         


\starttikzpicture[      scale=3,line cap=round
                        axes/.style=,         
                        important line/.style={very thick},
                        information text/.style={rounded corners,fill=red!10,inner sep=1ex} ]
        \draw[xshift=0.0cm]
                node[right,text width=17cm,information text]
                {
                        \bf
%%%%----------------------------
两边求导, 得 
\startformula
\startalign
  \NC I'(t) \NC  = \int_{-1}^t  e^{2x}dt + t\cdot e^{2t}- t\cdot e^{2t}- t\cdot e^{2t} -\int_t^1 e^{2x}dx + t e^{2t}  \NR
  \NC  \NC = \int_{-1}^t e^{2x}dx - \int_t^1 e^{2x}dx 
           = e^{2t}-\frac{1}{2}(e^2+e^{-2})\NR
\stopalign
\stopformula

%%%%----------------------------

};

\stoptikzpicture
%% 逐步显示
%%----------------------------------------------------------------






%%----------------------------------------------------------------
%% 逐步显示

\usemodule[tikz]         


\starttikzpicture[      scale=3,line cap=round
                        axes/.style=,         
                        important line/.style={very thick},
                        information text/.style={rounded corners,fill=red!10,inner sep=1ex} ]
        \draw[xshift=0.0cm]
                node[right,text width=17cm,information text]
                {
                        \bf
%%%%----------------------------
令 $I'(t)=0$, 解得 $\displaystyle t=\frac{1}{2}\ln \frac{e^2+e^{-2}}{2}$ 为唯一驻点, 
$I''(t)=2e^{2t}>0$, 故 
\startformula
t=\frac{1}{2}\ln \frac{e^2+e^{-2}}{2}
\stopformula
 为 $I(t)$ 在 $[-1,1]$ 上的最小值, 最大值只能在端点 $t=1$, $t=-1$ 取得。 

又由
\startformula
I(-1)=\frac{3}{4}e^2+\frac{1}{4}e^{-2}, \kern2em I(1)=\frac{1}{4}e^2-\frac{5}{4}e^{-2},
\stopformula
所以, $\displaystyle I_{max} = I(-1) = \frac{3}{4}e^2+\frac{1}{4}e^{-2}$
%%%%----------------------------

};

\stoptikzpicture
%% 逐步显示
%%----------------------------------------------------------------







\medskip
\FrameTitle{变限积分的计算} % 解答

\StartFrame\bf
换序型。
\startitemize[n] 
  \item 积分是一种累次积分, 即先算里面一层积分, 再算外面一层积分。 一般里面一层积分不易处理, 所以化为二重积分再交换积分次序, 称这种类型的积分为换序型积分。 换序型积分也可以用分部积分法来处理。
\stopitemize


\StopFrame

%%----------------------------------
%%----------------------------------








\medskip
\FrameTitle{变限积分的计算} % 解答

\StartFrame\bf\index{p164, 例9.38}
\index{交换积分次序}
\index{变限积分}
\index{分部积分}

换序型: 变限积分作为被积函数, 求定积分, 交换积分次序或者采用分部积分。 


\StopFrame

%%----------------------------------
%%----------------------------------




\medskip
\FrameTitle{变限积分的计算} % 解答

\StartFrame\bf\index{p164, 例9.38}
\index{交换积分次序}
\index{变限积分}
\index{分部积分}

考点: 交换积分次序。 分部积分。
\startitemize[n] 
  \item 设
\startformula
f(x)=\int_0^x\frac{\sin t}{\pi - t}dt
\stopformula
求 $\displaystyle \int_0^{\pi}f(x)dx$
\stopitemize


\StopFrame

%%----------------------------------
%%----------------------------------







%%----------------------------------------------------------------
%% 逐步显示

\usemodule[tikz]         


\starttikzpicture[      scale=3,line cap=round
                        axes/.style=,         
                        important line/.style={very thick},
                        information text/.style={rounded corners,fill=red!10,inner sep=1ex} ]
        \draw[xshift=0.0cm]
                node[right,text width=17cm,information text]
                {
                        \bf
%%%%----------------------------
 
解: 交换积分次序, 得
\startformula
\startalign
  \NC I \NC  = \int_0^{\pi} f(x)dx 
             = \int_0^{\pi} dx \int_0^x \frac{\sin t}{\pi -t}dt
             = \int_0^{\pi} dt \int_t ^{\pi} \frac{\sin t}{\pi -t}dx  
      = \int_0^{\pi} \sin tdt = 2  \NR
\stopalign
\stopformula

%%%%----------------------------

};

\stoptikzpicture
%% 逐步显示
%%----------------------------------------------------------------






%%----------------------------------------------------------------
%% 逐步显示

\usemodule[tikz]         


\starttikzpicture[      scale=3,line cap=round
                        axes/.style=,         
                        important line/.style={very thick},
                        information text/.style={rounded corners,fill=red!10,inner sep=1ex} ]
        \draw[xshift=0.0cm]
                node[right,text width=17cm,information text]
                {
                        \bf
%%%%----------------------------
 
解法二: 
\startformula
\startalign
  \NC I \NC  = \int_0^{\pi} f(x)dx  
             =  \left. xf(x)\right|_0^{\pi} - \int_0^{\pi} xf'(x)dx  
         = \pi \int_0^{\pi} \frac{\sin t}{\pi -t}dt -\int_0^{\pi} x\cdot \frac{\sin x }{\pi -x }\cdot dx \NR
  \NC \NC = \int_0^{\pi} \frac{\pi -x}{\pi -x }\cdot \sin xdx = \int_0^{\pi} \sin xdx = 2  \NR
\stopalign
\stopformula

%%%%----------------------------

};

\stoptikzpicture
%% 逐步显示
%%----------------------------------------------------------------







\medskip
\FrameTitle{2019数学二填空题第13题4分} % 解答

\StartFrame
\index{2019数学二填空题第13题4分}
\index{2019数学二第13题4分}

\startitemize[n] 
  \item 设 
\startformula
 f(x)=x\int_1^x \frac{\sin t^2}{t}dt
\stopformula
 则 $\displaystyle \int_0^1 f(x)dx$ = \text{______________}
\stopitemize


\StopFrame

%%----------------------------------
%%----------------------------------






%%----------------------------------------------------------------
%% 逐步显示

\usemodule[tikz]         


\starttikzpicture[      scale=3,line cap=round
                        axes/.style=,         
                        important line/.style={very thick},
                        information text/.style={rounded corners,fill=red!10,inner sep=1ex} ]
        \draw[xshift=0.0cm]
                node[right,text width=17cm,information text]
                {
                        \bf
%%%%----------------------------
 
解: 由 $f(x)$ 表达式可得
$f(0)=0$, $f(1)=0$, $\displaystyle f'(x)=\int_1^x \frac{\sin t^2}{t}dt + \sin x^2$
%%%%----------------------------

};

\stoptikzpicture
%% 逐步显示
%%----------------------------------------------------------------







%%----------------------------------------------------------------
%% 逐步显示

\usemodule[tikz]         


\starttikzpicture[      scale=3,line cap=round
                        axes/.style=,         
                        important line/.style={very thick},
                        information text/.style={rounded corners,fill=red!10,inner sep=1ex} ]
        \draw[xshift=0.0cm]
                node[right,text width=17cm,information text]
                {
                        \bf
%%%%----------------------------
考虑分部积分法, 得
\startformula
\startalign
  \NC I \NC  = \int_0^1 f(x)dx  
             =  \left. xf(x)\right|_0^1 - \int_0^1 xf'(x)dx \NR
  \NC  \NC   = - \int_0^1\left( x\int_1^x \frac{\sin t^2}{t}dt + x\sin x^2 \right)dx \NR
  \NC \NC =   - \int_0^1\left( x\int_1^x \frac{\sin t^2}{t}dt  \right)dx -   \int_0^1 x\sin x^2 dx   \NR
  \NC  \NC = -I - \frac{1}{2}  \int_0^1 \sin x^2 d(x^2)    
       = -I + \left.\frac{1}{2}\cos x^2 \right|_0^1 = -I+\frac{1}{2}(\cos 1 - 1) \NR
\stopalign
\stopformula
%%%%----------------------------

};

\stoptikzpicture
%% 逐步显示
%%----------------------------------------------------------------






%%----------------------------------------------------------------
%% 逐步显示

\usemodule[tikz]         


\starttikzpicture[      scale=3,line cap=round
                        axes/.style=,         
                        important line/.style={very thick},
                        information text/.style={rounded corners,fill=red!10,inner sep=1ex} ]
        \draw[xshift=0.0cm]
                node[right,text width=17cm,information text]
                {
                        \bf
%%%%----------------------------

即 $\displaystyle 2I= \frac{1}{2}(\cos 1 - 1)$, 所以 $\displaystyle I= \frac{\cos 1 - 1}{4}$
%%%%----------------------------

};

\stoptikzpicture
%% 逐步显示
%%----------------------------------------------------------------








\medskip
\FrameTitle{2020数学二选择题第1题4分} % 解答

\StartFrame
\index{2020数学二选择题第1题4分}
\index{无穷小量阶的比较}
\index{变限积分}
\index{洛必达法则}
\index{求导}
考点: 无穷小阶的比较。 变限积分求导方法。 洛必达法则。

当 $x\rightarrow 0^+$ 时,下列无穷小量中最高阶的是 (\kern2em)
\startformula
\startalign
  \NC (A) \NC  \int_0^x \left(e^{t^2} -1\right)dt \kern2em
      (B) \int_0^x \ln \left( 1+\sqrt{t^3} \,\right)dt \NR
  \NC (C) \NC  \int_0^{\sin x} \sin t^2 dt \kern2em
      (D) \int_0^{1-\cos x} \sqrt{\sin ^3t} \cdot dt \NR
\stopalign
\stopformula
\StopFrame

%%----------------------------------
%%----------------------------------





%%----------------------------------------------------------------
%% 逐步显示

\usemodule[tikz]         


\starttikzpicture[      scale=3,line cap=round
                        axes/.style=,         
                        important line/.style={very thick},
                        information text/.style={rounded corners,fill=red!10,inner sep=1ex} ]
        \draw[xshift=0.0cm]
                node[right,text width=17cm,information text]
                {
                        \bf
%%%%----------------------------

解: 

\startformula
\startalign
  \NC (A)\kern2em  \NC \left [ \int_0^x \left (e^{t^2}-1\right)dt \right ]'
              =e^{x^2}-1 
              \sim {\darkgreen x^2} \NR
  \NC (B)\kern2em  \NC \left [ \int_0^x \ln \left( 1+\sqrt{t^3} \right) dt \right ]'
              = \ln \left( 1+\sqrt{t^3} \right)
              \sim {\darkgreen x^{\frac{3}{2}}} \NR
\stopalign
\stopformula

%%%%----------------------------

};

\stoptikzpicture
%% 逐步显示
%%----------------------------------------------------------------




%%----------------------------------------------------------------
%% 逐步显示

\usemodule[tikz]         


\starttikzpicture[      scale=3,line cap=round
                        axes/.style=,         
                        important line/.style={very thick},
                        information text/.style={rounded corners,fill=red!10,inner sep=1ex} ]
        \draw[xshift=0.0cm]
                node[right,text width=17cm,information text]
                {
                        \bf
%%%%----------------------------



\startformula
\startalign
  \NC (C)\kern2em  \NC \left ( \int_0^{\sin x} \sin t^2 dt \right )'
              = \sin (\sin x )^2 \cdot \cos x
              \sim {\darkgreen x^2} \NR
\stopalign
\stopformula

%%%%----------------------------

};

\stoptikzpicture
%% 逐步显示
%%----------------------------------------------------------------







%%----------------------------------------------------------------
%% 逐步显示

\usemodule[tikz]         


\starttikzpicture[      scale=3,line cap=round
                        axes/.style=,         
                        important line/.style={very thick},
                        information text/.style={rounded corners,fill=red!10,inner sep=1ex} ]
        \draw[xshift=0.0cm]
                node[right,text width=17cm,information text]
                {
                        \bf
%%%%----------------------------

\startformula
\startalign
  \NC (D)\kern2em \NC \left(\int_0^{1-\cos x} \sqrt{\sin ^3 t}\cdot  dt\right)'
           = \sqrt{\sin ^3 (1-\cos x)} \cdot \sin x \NR
  \NC  \NC  \sim x\cdot \sqrt{\left( \frac{x^2}{2} \right)^3}
            \sim {\darkgreen \frac{\sqrt{2}}{4}\cdot x^3\cdot |x|} \NR
\stopalign
\stopformula
故选$(D)$
%%%%----------------------------

};

\stoptikzpicture
%% 逐步显示
%%----------------------------------------------------------------








\medskip
\FrameTitle{反常积分的计算} % 解答

\StartFrame\bf
在收敛的条件下:
\startitemize[n] 
  \item 
\startformula
\int _a^{+\infty} f(x)dx=F(+\infty)-F(a)
\stopformula
其中
\startformula
F(+\infty)=\lim_{x\rightarrow +\infty}F(x)
\stopformula
  \item 若 $a$ 为瑕点, 则 
\startformula
\int_a^b f(x)dx=F(b)-F(a)
\stopformula
其中 
\startformula
F(a)=\lim_{x\rightarrow a^+}F(x)
\stopformula
  \item 换元后可能化为定积分。
\stopitemize


\StopFrame

%%----------------------------------
%%----------------------------------








\medskip
\FrameTitle{2019数学二选择题第3题4分} % 解答

\StartFrame
\index{2019数学二选择题第3题4分}
\bf
下列反常积分发散的是(\kern2em)

$(A)$ $\displaystyle \int _0^{+\infty} xe^{-x}dx$ \kern2em
$(B)$ $\displaystyle \int _0^{+\infty} xe^{-x^2}dx$ 

$(C)$ $\displaystyle \int _0^{+\infty} \frac{\arctan x}{1+x^2}dx$ \kern2em
$(D)$ $\displaystyle \int _0^{+\infty} \frac{x}{1+x^2}dx$


\StopFrame

%%----------------------------------
%%----------------------------------





%%----------------------------------------------------------------
%% 逐步显示

\usemodule[tikz]         


\starttikzpicture[      scale=3,line cap=round
                        axes/.style=,         
                        important line/.style={very thick},
                        information text/.style={rounded corners,fill=red!10,inner sep=1ex} ]
        \draw[xshift=0.0cm]
                node[right,text width=17cm,information text]
                {
                        \bf
%%%%----------------------------
解: $(D)$ 发散, 因为
\startformula
\startalign
  \NC \NC \int_0^{+\infty} \frac{x}{1+x^2}dx 
          = \frac{1}{2}\int_0^{+\infty} \frac{1}{1+x^2}d\left ( 1+x^2\right )  
      = \left. \frac{1}{2} \ln \left(1+x^2\right) \right|_0^{+\infty}
          = +\infty \NR
\stopalign
\stopformula
%%%%----------------------------

};

\stoptikzpicture
%% 逐步显示
%%----------------------------------------------------------------









\medskip
\FrameTitle{反常积分的计算} % 解答

\StartFrame\bf\index{p165, 例9.39}
\index{反常积分}
\index{裂项法}

考点: 反常积分。
\startitemize[n] 
  \item 计算$\displaystyle \int_5^{+\infty}\frac{1}{x^2-4x+3}dx$
\stopitemize


\StopFrame

%%----------------------------------
%%----------------------------------






%%----------------------------------------------------------------
%% 逐步显示

\usemodule[tikz]         


\starttikzpicture[      scale=3,line cap=round
                        axes/.style=,         
                        important line/.style={very thick},
                        information text/.style={rounded corners,fill=red!10,inner sep=1ex} ]
        \draw[xshift=0.0cm]
                node[right,text width=17cm,information text]
                {
                        \bf
%%%%----------------------------
解: 
\startformula
\startalign
  \NC I \NC =  \int_5^{+\infty}\frac{1}{x^2-4x+3}dx
          =  \int_5^{+\infty}\frac{1}{(x-1)(x-3)}dx \NR
  \NC \NC = \frac{1}{2}\cdot \int_5^{+\infty}\left ( \frac{1}{x-3} - \frac{1}{x-1}\right ) dx
          = \left.\frac{1}{2}\cdot \ln \frac{x-3}{x-1}\right|_5^{+\infty} \NR
  \NC \NC = \frac{1}{2}\cdot \left(\ln 1 - \ln \frac{1}{2}\right)
          = \frac{1}{2}\cdot \ln 2 \NR
\stopalign
\stopformula
%%%%----------------------------

};

\stoptikzpicture
%% 逐步显示
%%----------------------------------------------------------------









\medskip
\FrameTitle{反常积分的计算} % 解答

\StartFrame\bf\index{p165, 例9.40}
\index{反常积分}
\index{变量代换}
\index{凑积分}
\index{换元换限}
\index{倍角公式}

考点: 变量代换。 反常积分。 凑积分。 换元换限。 倍角公式。 $2\sin^2x = 1-\cos 2x$
\startitemize[n]
  \item 计算 $\displaystyle \int _0^1\frac{x^2\arcsin x}{\sqrt{1-x^2}}dx$
\stopitemize


\StopFrame

%%----------------------------------
%%----------------------------------





%%----------------------------------------------------------------
%% 逐步显示

\usemodule[tikz]         


\starttikzpicture[      scale=3,line cap=round
                        axes/.style=,         
                        important line/.style={very thick},
                        information text/.style={rounded corners,fill=red!10,inner sep=1ex} ]
        \draw[xshift=0.0cm]
                node[right,text width=17cm,information text]
                {
                        \bf
%%%%----------------------------
解: 由于 $\displaystyle \lim_{x\rightarrow 1^-} \frac{x^2\arcsin x}{\sqrt{1-x^2}} = +\infty$, 故
$\displaystyle \int _0^1\frac{x^2\arcsin x}{\sqrt{1-x^2}}dx$ 是反常积分。
%%%%----------------------------

};

\stoptikzpicture
%% 逐步显示
%%----------------------------------------------------------------






%%----------------------------------------------------------------
%% 逐步显示

\usemodule[tikz]         


\starttikzpicture[      scale=3,line cap=round
                        axes/.style=,         
                        important line/.style={very thick},
                        information text/.style={rounded corners,fill=red!10,inner sep=1ex} ]
        \draw[xshift=0.0cm]
                node[right,text width=17cm,information text]
                {
                        \bf
%%%%----------------------------

令 $\arcsin x =t$, 有 $x=\sin t$, $t\in \left [ 0, \displaystyle \frac{\pi}{2} \right )$, 则
\startformula
\startalign
  \NC I \NC =  \int _0^1\frac{x^2\arcsin x}{\sqrt{1-x^2}}dx
          =  \int_0^{\frac{\pi}{2}} \frac{t\sin^2t}{\cos t}\cdot \cos t\cdot dt \NR
  \NC \NC = \int_0^{\frac{\pi}{2}} t\sin^2t \cdot dt
          = \int_0^{\frac{\pi}{2}} \left( \frac{t}{2} - \frac{t\cos 2t}{2} \right) \cdot dt  
      = \left.\frac{t^2}{4}\right|_0^{\frac{\pi}{2}} -\frac{1}{4}\int_0^{\frac{\pi}{2}} t \cdot d(\sin 2t) \NR
\stopalign
\stopformula
%%%%----------------------------

};

\stoptikzpicture
%% 逐步显示
%%----------------------------------------------------------------








%%----------------------------------------------------------------
%% 逐步显示

\usemodule[tikz]         


\starttikzpicture[      scale=3,line cap=round
                        axes/.style=,         
                        important line/.style={very thick},
                        information text/.style={rounded corners,fill=red!10,inner sep=1ex} ]
        \draw[xshift=0.0cm]
                node[right,text width=17cm,information text]
                {
                        \bf
%%%%----------------------------

\startformula
\startalign
  \NC \NC = \frac{\pi^2}{16} - \left.\frac{t\sin 2t}{4}\right|_0^{\frac{\pi}{2}} +\frac{1}{4}\int_0^{\frac{\pi}{2}} \sin 2t dt  
      = \frac{\pi^2}{16} - \left.\frac{1}{8}\cos 2t\right|_0^{\frac{\pi}{2}}  
          = \frac{\pi^2}{16} + \frac{1}{4}\NR
\stopalign
\stopformula
%%%%----------------------------

};

\stoptikzpicture
%% 逐步显示
%%----------------------------------------------------------------











\medskip
\FrameTitle{2020数学二填空题第13题4分} % 解答

\StartFrame
\index{2020数学二填空题第13题4分}
\index{2020数学二第13题4分}
\index{无穷限的广义积分}
\index{二阶齐次微分方程}
\index{特征方程}
\index{初始条件}
\index{齐次通解}

考点: 反常积分。 微分方程。 初始条件。 特征方程。 齐次通解。

设 $y=y(x)$ 满足 $y''+2y'+y=0$, 且 $y(0)=0$, $y'(0)=1$, 则 $\displaystyle \int_0^{+\infty} y(x)dx = $ \text{___________}

\StopFrame

%%----------------------------------
%%----------------------------------






%%----------------------------------------------------------------
%% 逐步显示

\usemodule[tikz]         


\starttikzpicture[      scale=3,line cap=round
                        axes/.style=,         
                        important line/.style={very thick},
                        information text/.style={rounded corners,fill=red!10,inner sep=1ex} ]
        \draw[xshift=0.0cm]
                node[right,text width=17cm,information text]
                {
                        \bf
%%%%----------------------------
解: 微分的特征方程为  $r^2+2r+1=0$, 得 $r_1=r_2=-1$, 所以

$y=(C_1+C_2x)e^{-x}$, 
%%%%----------------------------

};

\stoptikzpicture
%% 逐步显示
%%----------------------------------------------------------------





%%----------------------------------------------------------------
%% 逐步显示

\usemodule[tikz]         


\starttikzpicture[      scale=3,line cap=round
                        axes/.style=,         
                        important line/.style={very thick},
                        information text/.style={rounded corners,fill=red!10,inner sep=1ex} ]
        \draw[xshift=0.0cm]
                node[right,text width=17cm,information text]
                {
                        \bf
%%%%----------------------------
代入 $y(0)=0$, $y'(0)=1$, 得
$C_1=0$, $C_2=1$, 
%%%%----------------------------

};

\stoptikzpicture
%% 逐步显示
%%----------------------------------------------------------------





%%----------------------------------------------------------------
%% 逐步显示

\usemodule[tikz]         


\starttikzpicture[      scale=3,line cap=round
                        axes/.style=,         
                        important line/.style={very thick},
                        information text/.style={rounded corners,fill=red!10,inner sep=1ex} ]
        \draw[xshift=0.0cm]
                node[right,text width=17cm,information text]
                {
                        \bf
%%%%----------------------------

于是 $y=xe^{-x}$, 代入积分式, 有
\startformula
\startalign
  \NC I \NC =  \int_0^{+\infty} y(x) dx
          =  \int_0^{+\infty} xe^{-x} dx 
          =  -e^{-x}(x+1)|_0^{+\infty} \NR
  \NC \NC = \lim_{x\rightarrow +\infty} \left [ -e^{-x}(x+1) \right] - \lim_{x\rightarrow 0^+} \left [ -e^{-x}(x+1) \right] 
          = 1 \NR
\stopalign
\stopformula
%%%%----------------------------

};

\stoptikzpicture
%% 逐步显示
%%----------------------------------------------------------------










\medskip
\FrameTitle{2020数学二选择题第3题4分} % 解答

\StartFrame
\index{2020数学二选择题第3题4分}
\index{2020数学二第3题4分}
\index{反常积分}
\index{变量代换}
\index{换元换限}
\index{反正弦}

考点: 反常积分。 变量代换。 换元换限。 反正弦函数。

\startformula
\int_0^1 \frac{\arcsin \sqrt{x}}{\sqrt{x(1-x)}} \cdot dx\, =\kern2em (\kern2em)
\stopformula
$(A)$ $\displaystyle\frac{\pi^2}{4}$\kern2em
$(B)$ $\displaystyle\frac{\pi^2}{8}$\kern2em
$(C)$ $\displaystyle\frac{\pi}{4}$\kern2em
$(D)$ $\displaystyle\frac{\pi}{8}$


\StopFrame

%%----------------------------------
%%----------------------------------




%%----------------------------------------------------------------
%% 逐步显示

\usemodule[tikz]         


\starttikzpicture[      scale=3,line cap=round
                        axes/.style=,         
                        important line/.style={very thick},
                        information text/.style={rounded corners,fill=red!10,inner sep=1ex} ]
        \draw[xshift=0.0cm]
                node[right,text width=17cm,information text]
                {
                        \bf
%%%%----------------------------
解: 令 $t=\arcsin \sqrt{x}$, 则
\startformula
\startalign
  \NC I \NC = \int_0^1 \frac{\arcsin \sqrt{x}}{\sqrt{x(1-x)}} \cdot dx  
      = \int_0^{\frac{\pi}{2}} \frac{t}{\sin t \cos t}2\sin t \cos tdt 
          = \frac{\pi^2}{4}\NR
\stopalign
\stopformula
%%%%----------------------------

};

\stoptikzpicture
%% 逐步显示
%%----------------------------------------------------------------











\medskip
\FrameTitle{一元函数积分学的几何应用} % 解答

\StartFrame\bf
面积。
\startitemize[n] 
  \item 直角坐标系下的面积公式 $\displaystyle S=\int _a^b \left|f(x)-g(x)\right|\cdot dx$
  \item 极坐标系下的面积公式 $\displaystyle S=\int _{\alpha}^{\beta} \frac{1}{2}\cdot\left|r_2^2(\theta)-r_1^2(\theta)\right|\cdot dx$
\stopitemize

\StopFrame

%%----------------------------------
%%----------------------------------








\medskip
\FrameTitle{一元函数积分学的几何应用} % 解答

\StartFrame\bf\index{p174, 例10.1}
\index{直角坐标}
\index{平面图形面积}
\index{三角函数}

考点: 平面图形面积。 三角函数。 直角坐标。 
\startitemize[n] 
  \item 求介于直线 $x=0$, $x=2\pi$ 之间,且由曲线 $y=\sin x$ 和 $y=\cos x$ 所围成的平面图形的面积。
\stopitemize

\placefigure[force]{second}{\externalfigure[2020-08-30-185556.png][height=8cm]}

\StopFrame

%%----------------------------------
%%----------------------------------





%%----------------------------------------------------------------
%% 逐步显示

\usemodule[tikz]         


\starttikzpicture[      scale=3,line cap=round
                        axes/.style=,         
                        important line/.style={very thick},
                        information text/.style={rounded corners,fill=red!10,inner sep=1ex} ]
        \draw[xshift=0.0cm]
                node[right,text width=17cm,information text]
                {
                        \bf
%%%%----------------------------
解: 
\startformula
\startalign
  \NC S \NC = \int_0^{2\pi} |\sin x -\cos x| dx \NR
  \NC \NC = \int_0^{\frac{\pi}{4}}  (\cos x -\sin x)dx 
            + \int_{\frac{\pi}{4}}^{\frac{5\pi}{4}} (\sin x - \cos x)dx 
            + \int_{\frac{5\pi}{4}}^{2\pi} (\cos x - \sin x)dx \NR
  \NC \NC = (\sqrt{2}-1) + 2\sqrt{2} + (1+\sqrt{2}) 
          = 4\sqrt{2} \NR
\stopalign
\stopformula
%%%%----------------------------

};

\stoptikzpicture
%% 逐步显示
%%----------------------------------------------------------------





%%----------------------------------------------------------------
%% 逐步显示

\usemodule[tikz]         


\starttikzpicture[      scale=3,line cap=round
                        axes/.style=,         
                        important line/.style={very thick},
                        information text/.style={rounded corners,fill=red!10,inner sep=1ex} ]
        \draw[xshift=0.0cm]
                node[right,text width=17cm,information text]
                {
                        \bf
%%%%----------------------------
解法二: 
\startformula
\startalign
  \NC S \NC = \int_0^{2\pi} |\sin x -\cos x| dx  
      = 2\int_{\frac{\pi}{4}}^{\frac{5\pi}{4}} (\sin x - \cos x)dx   
      = 2(2\sqrt{2})
          = 4\sqrt{2} \NR
\stopalign
\stopformula
%%%%----------------------------

};

\stoptikzpicture
%% 逐步显示
%%----------------------------------------------------------------








\medskip
\FrameTitle{2019数学二解答题第19题10分} % 解答

\StartFrame
\index{2019数学二解答题第19题10分}
\index{2019数学二第19题10分}
\index{面积}
\index{数列极限}
\index{变量代换}
\index{绝对值}
\index{分部积分}
\index{指数函数}
\index{三角函数}
\index{平方差公式}
\index{立方差公式}
\index{直角坐标}

考点: 面积。 数列极限。 变量代换。 绝对值。 分部积分。 指数函数。 三角函数。 平方差公式。 立方差公式。 直角坐标。 

设 $n$ 是正整数, 记 $S_n$ 是 $y=e^{-x}\sin x$, $(0\leq x\leq n\pi)$ 的图形与 $x$ 轴所围图形的面积,
\startitemize[n] 
  \item 求 $S_n$,
  \item  并求 $\displaystyle \lim_{n\rightarrow \infty} S_n$.
\stopitemize

\StopFrame

%%----------------------------------
%%----------------------------------




%%----------------------------------------------------------------
%% 逐步显示

\usemodule[tikz]         


\starttikzpicture[      scale=3,line cap=round
                        axes/.style=,         
                        important line/.style={very thick},
                        information text/.style={rounded corners,fill=red!10,inner sep=1ex} ]
        \draw[xshift=0.0cm]
                node[right,text width=17cm,information text]
                {
                        \bf
%%%%----------------------------
解: 所求面积为
\startformula
\startalign
  \NC S_n \NC = \int_0^{n\pi} \left|e^{-x}\sin x \right| dx  
          = \sum_{k=0}^{n-1} \int_{k\pi}^{(k+1)\pi} e^{-x}\cdot \left|\sin x \right| dx  \NR
  \NC \NC = \sum_{k=0}^{n-1} \int_0^{\pi} e^{-(k\pi+t)}\cdot \left|\sin (k\pi+t) \right| dt  \NR
\stopalign
\stopformula
%%%%----------------------------

};

\stoptikzpicture
%% 逐步显示
%%----------------------------------------------------------------





%%----------------------------------------------------------------
%% 逐步显示

\usemodule[tikz]         


\starttikzpicture[      scale=3,line cap=round
                        axes/.style=,         
                        important line/.style={very thick},
                        information text/.style={rounded corners,fill=red!10,inner sep=1ex} ]
        \draw[xshift=0.0cm]
                node[right,text width=17cm,information text]
                {
                        \bf
%%%%----------------------------

\startformula
\startalign
  \NC \NC = \sum_{k=0}^{n-1} \int_0^{\pi} e^{-(k\pi+t)}\cdot \sin t dt  
          = \frac{1}{2}\left(1+e^{\pi}\right)\sum_{k=0}^{n-1}  e^{-(k\pi+t)}  \NR
  \NC \NC = \frac{1+e^{\pi}}{2 (e^{\pi}-1)}\cdot \left (1-e^{-n\pi}\right) \NR
\stopalign
\stopformula
%%%%----------------------------

};

\stoptikzpicture
%% 逐步显示
%%----------------------------------------------------------------




%%----------------------------------------------------------------
%% 逐步显示

\usemodule[tikz]         


\starttikzpicture[      scale=3,line cap=round
                        axes/.style=,         
                        important line/.style={very thick},
                        information text/.style={rounded corners,fill=red!10,inner sep=1ex} ]
        \draw[xshift=0.0cm]
                node[right,text width=17cm,information text]
                {
                        \bf
%%%%----------------------------
所以
\startformula
\startalign
  \NC \lim_{n\rightarrow +\infty}S_n \NC = \frac{1+e^{\pi}}{2 (e^{\pi}-1)} \NR
\stopalign
\stopformula
%%%%----------------------------

};

\stoptikzpicture
%% 逐步显示
%%----------------------------------------------------------------






\medskip
\FrameTitle{一元函数积分学的几何应用} % 解答

\StartFrame\bf\index{p177, 例10.6}
\index{面积}
\index{对称}

练习。 考点: 面积。 对称。
\startitemize[n] 
  \item 求曲线 $y^2=(1+x^2)^3$ 所围图形的面积。
%\placefigure[here]{second}{\externalfigure[2020-08-30-192258.png][height=4cm]}
\stopitemize

\setupexternalfigures[location={local,default}]

    \placefigure[force]{none}{\externalfigure[2020-08-30-192258.png][height=8cm]}

\StopFrame

%%----------------------------------
%%----------------------------------



%%----------------------------------------------------------------
%% 逐步显示

\usemodule[tikz]         


\starttikzpicture[      scale=3,line cap=round
                        axes/.style=,         
                        important line/.style={very thick},
                        information text/.style={rounded corners,fill=red!10,inner sep=1ex} ]
        \draw[xshift=0.0cm]
                node[right,text width=17cm,information text]
                {
                        \bf
%%%%----------------------------
解: 图形关于 $x$ 轴, $y$ 轴均对称, 令 $x=\sin t$ 则所求面积为
\startformula
\startalign
  \NC S \NC = 4\int_0^1 (1-x^2)^{\frac{3}{2}} dx  
        = 4\int_0^{\frac{\pi}{2}} \cos ^4t dt  
            = 4\cdot \frac{3}{2}\cdot \frac{1}{2}\cdot \frac{\pi}{2} 
            = \frac{3}{4}\pi \NR
\stopalign
\stopformula
%%%%----------------------------

};

\stoptikzpicture
%% 逐步显示
%%----------------------------------------------------------------





\medskip
\FrameTitle{一元函数积分学的几何应用} % 解答

\StartFrame\bf\index{p177, 例10.7}
练习。
\startitemize[n] 
  \item 求曲线 $\sqrt{x}+\sqrt{y}=1$ 与坐标轴所围图形的面积。
%\placefigure[here]{second}{\externalfigure[2020-08-30-192258.png][height=4cm]}
\stopitemize

\setupexternalfigures[location={local,default}]

    \placefigure[force]{none}{\externalfigure[2020-08-30-194502.png][height=8cm]}

\StopFrame

%%----------------------------------
%%----------------------------------




%%----------------------------------------------------------------
%% 逐步显示

\usemodule[tikz]         


\starttikzpicture[      scale=3,line cap=round
                        axes/.style=,         
                        important line/.style={very thick},
                        information text/.style={rounded corners,fill=red!10,inner sep=1ex} ]
        \draw[xshift=0.0cm]
                node[right,text width=17cm,information text]
                {
                        \bf
%%%%----------------------------
解: 
\startformula
\startalign
  \NC S \NC = \int_0^1 (1-\sqrt{x})^2 dx \NR
  \NC   \NC = \int_0^1 (1-2\sqrt{x}+x) dx  
            = \left.\left(x-\frac{4}{3}x^{\frac{3}{2}}+\frac{x^2}{2}\right) \right|_0^1
            = \frac{1}{6} \NR
\stopalign
\stopformula
%%%%----------------------------

};

\stoptikzpicture
%% 逐步显示
%%----------------------------------------------------------------




\medskip
\FrameTitle{一元函数积分学的几何应用} % 解答

\StartFrame\bf\index{p178, 例10.10}
练习。
\startitemize[n] 
  \item 求星形线 $x=\cos^3t$, $y=\sin ^3t$, $0\leq t\leq 2\pi$ 所围图形的面积。
\stopitemize

\setupexternalfigures[location={local,default}]

    \placefigure[force]{none}{\externalfigure[2020-08-30-195725.png][height=8cm]}

\StopFrame

%%----------------------------------





%%----------------------------------------------------------------
%% 逐步显示

\usemodule[tikz]         


\starttikzpicture[      scale=3,line cap=round
                        axes/.style=,         
                        important line/.style={very thick},
                        information text/.style={rounded corners,fill=red!10,inner sep=1ex} ]
        \draw[xshift=0.0cm]
                node[right,text width=17cm,information text]
                {
                        \bf
%%%%----------------------------
解: 因为 $dx= -3\cos ^2 t \sin t dt$, 图形关于 $x$ 轴, $y$ 轴均对称, 故所求面积为
\startformula
\startalign
  \NC S \NC = 4\int_0^1 y dx 
            = 4\int_{\frac{\pi}{2}}^0 \sin ^3t \cdot (-3 \cos ^2t\sin t) \cdot dt \NR
  \NC   \NC = 12\int_0^{\frac{\pi}{2}} \sin ^4t \cdot \cos ^2t\cdot dt
            = 12\int_0^{\frac{\pi}{2}} \left( \sin ^4t -\sin^6t \right) dt \NR
  \NC   \NC = 12\left(\frac{3}{4}\cdot \frac{1}{2}\cdot \frac{\pi}{2}\cdot - \frac{5}{6}\cdot \frac{3}{4}\cdot \frac{1}{2}\cdot \frac{\pi}{2} \right) \NR
  \NC  \NC  = \frac{3}{8}\pi \NR
\stopalign
\stopformula
%%%%----------------------------

};

\stoptikzpicture
%% 逐步显示
%%----------------------------------------------------------------





%%----------------------------------
% https : // community.wolfram.com/groups/-/m/t/1044807
% Plot a folium of Descartes with ParametericPlot?
% 参数方程: ParametricPlot[{3 t/(1 + t^3), 3 t^2/(1 + t^3)}, {t, -100, 100},  PlotRange -> 3]
% 极坐标方程: PolarPlot[
% 3 Cos[\[Theta]] Sin[\[Theta]]/(Cos[\[Theta]]^3 + 
%     Sin[\[Theta]]^3), {\[Theta], 0, Pi/2}, Frame -> True, 
% PlotLabel -> r == \[Theta], PlotStyle -> {Green, Thickness[0.01]}]

\medskip
\FrameTitle{一元函数积分学的几何应用} % 解答

\StartFrame\bf
\index{笛卡尔叶形线}
\index{极坐标}

练习。
\startitemize[n] 
  \item 求曲线 $x^3+y^3-3axy=0$, $a>0$ 所围成的平面图形的面积。 笛卡尔叶形线, 极坐标方程为
$r=\displaystyle \frac{3a\cos\theta\sin\theta}{\cos^3\theta+\sin^3\theta}$, $0\leq \theta\leq \displaystyle \frac{\pi}{2}$,

\stopitemize

\setupexternalfigures[location={local,default}]

    \placefigure[force]{}{\externalfigure[2020-08-30-201336.png][height=8cm]}

\StopFrame

%%----------------------------------
%%----------------------------------




%%----------------------------------------------------------------
%% 逐步显示

\usemodule[tikz]         


\starttikzpicture[      scale=3,line cap=round
                        axes/.style=,         
                        important line/.style={very thick},
                        information text/.style={rounded corners,fill=red!10,inner sep=1ex} ]
        \draw[xshift=0.0cm]
                node[right,text width=17cm,information text]
                {
                        \bf
%%%%----------------------------
解: 所求面积为
\startformula
\startalign
  \NC S \NC = \frac{1}{2}\int_0^{\frac{\pi}{2}} r^2 d\theta 
            = \frac{9}{2}a^2\int_0^{\frac{\pi}{2}} \frac{\cos^2\theta\sin^2\theta}{(\cos^3\theta+\sin^3\theta)^2} d\theta  \NR
  \NC   \NC = \frac{9}{2}a^2\int_0^{\frac{\pi}{2}} \frac{\tan^2\theta d(\tan\theta)}{(1+\tan^3\theta)^2} d\theta
            = \frac{9}{2}a^2\int_0^{\frac{\pi}{2}} d\left( -\frac{1}{3}\cdot \frac{1}{1+\tan^3\theta}\right) \NR
  \NC   \NC =  -\frac{3}{2}a^2\cdot \left.\frac{1}{1+\tan^3\theta}\right|_0^{\frac{\pi}{2}} 
           = \frac{3}{2}a^2 \NR
\stopalign
\stopformula
%%%%----------------------------

};

\stoptikzpicture
%% 逐步显示
%%----------------------------------------------------------------






\medskip
\FrameTitle{一元函数积分学的几何应用} % 解答

\StartFrame\bf
练习。
\startitemize[n] 
  \item 求阿基米德螺旋线 $r=a\theta$, $a>0$ 的第一圈与极轴所围成的图形的面积。
\stopitemize

\setupexternalfigures[location={local,default}]

    \placefigure[force]{}{\externalfigure[2020-08-30-203528.png][height=8cm]}

\StopFrame

%%----------------------------------
%%----------------------------------




%%----------------------------------------------------------------
%% 逐步显示

\usemodule[tikz]         


\starttikzpicture[      scale=3,line cap=round
                        axes/.style=,         
                        important line/.style={very thick},
                        information text/.style={rounded corners,fill=red!10,inner sep=1ex} ]
        \draw[xshift=0.0cm]
                node[right,text width=17cm,information text]
                {
                        \bf
%%%%----------------------------
解: 所求面积为
\startformula
\startalign
  \NC S \NC = \frac{1}{2}\int_0^{2\pi} r^2 d\theta 
            = \frac{a^2}{2}\int_0^{2\pi} \theta^2 d\theta  
           = \frac{4}{3}a^2\pi^3 \NR
\stopalign
\stopformula
%%%%----------------------------

};

\stoptikzpicture
%% 逐步显示
%%----------------------------------------------------------------





\medskip
\FrameTitle{一元函数积分学的几何应用} % 解答

\StartFrame\bf
旋转体的体积。
\startitemize[n] 
  \item 绕 $x$ 轴:$V_x=\displaystyle \int_a^b \pi y^2(x)dx$
  \item 绕 $y$ 轴:$V_y=\displaystyle \int_a^b 2\pi x |y(x)|dx$,圆柱壳法。
\stopitemize


\StopFrame

%%----------------------------------
%%----------------------------------

\medskip
\FrameTitle{2020数学二解答题第18题10分} % 解答

\StartFrame\bf\index{2020数学二解答题第18题10分}
\index{体积}

设 $y=f(x)$ 在 $(0,+\infty)$ 上有定义, 且满足 $2f(x)+x^2\cdot f\left(\displaystyle \frac{1}{x}\right) = \displaystyle \frac{x^2+2x}{\sqrt{1+x^2}}$,
\startitemize[n] 
  \item 求 $y=f(x)$ 的表达式,
  \item 求曲线 $y=f(x)$, $y=\displaystyle \frac{1}{2}$, $y=\displaystyle \frac{\sqrt{3}}{2}$ 及 $y$ 轴围成的图形绕 $x$ 轴旋转一周所形成的体积。
\stopitemize


\StopFrame

%%----------------------------------
%%----------------------------------



%%----------------------------------------------------------------
%% 逐步显示

\usemodule[tikz]         


\starttikzpicture[      scale=3,line cap=round
                        axes/.style=,         
                        important line/.style={very thick},
                        information text/.style={rounded corners,fill=red!10,inner sep=1ex} ]
        \draw[xshift=0.0cm]
                node[right,text width=17cm,information text]
                {
                        \bf
%%%%----------------------------
解: (1) 由已知,得
\startformula
f\left(\frac{1}{x}\right)+\frac{1}{x^2}\cdot f(x)=\frac{2x+1}{x\cdot \sqrt{1+x^2}}
\stopformula
两式消去 $f\left(\displaystyle \frac{1}{x}\right)$, 得
\startformula
f(x)=\frac{x}{\sqrt{1+x^2}},\kern2em x\in (0,+\infty)
\stopformula
%%%%----------------------------

};

\stoptikzpicture
%% 逐步显示
%%----------------------------------------------------------------




%%----------------------------------------------------------------
%% 逐步显示

\usemodule[tikz]         


\starttikzpicture[      scale=3,line cap=round
                        axes/.style=,         
                        important line/.style={very thick},
                        information text/.style={rounded corners,fill=red!10,inner sep=1ex} ]
        \draw[xshift=0.0cm]
                node[right,text width=17cm,information text]
                {
                        \bf
%%%%----------------------------
(2) 由 $f(x)=\displaystyle \frac{x}{\sqrt{1+x^2}}$, 得 $x=\displaystyle \frac{y}{\sqrt{1-y^2}}$, 所以体积为
\startformula
V =2\pi \int_{\frac{1}{2}}^{\frac{\sqrt{3}}{2}}yxdy
= 2\pi \int_{\frac{1}{2}}^{\frac{\sqrt{3}}{2}}\frac{y^2}{\sqrt{1-y^2}}dy
\stopformula
令 $y=\sin t$, 得
\startformula
V=2\pi \int_{\frac{\pi}{6}}^{\frac{\pi}{3}} \sin^2tdt 
=\frac{\pi^2}{6}
\stopformula
%%%%----------------------------

};

\stoptikzpicture
%% 逐步显示
%%----------------------------------------------------------------



\medskip
\FrameTitle{一元函数积分学的几何应用} % 解答

\StartFrame\bf\index{p181, 例10.16}
\index{旋转体}
\index{体积}


考点: 旋转体。 体积。
\startitemize[n] 
  \item 求曲线 $y=\sqrt{x(1-x)^9\, }$ 在 $[0,1]$ 上与 $x$ 轴所围图形绕 $x$ 轴旋转一周所得的旋转体的体积。
\stopitemize

%\setupexternalfigures[location={local,default}]

    \placefigure[force]{}{\externalfigure[2020-08-30-210333.png][height=8cm]}


\StopFrame

%%----------------------------------
%%---------------------------------- 





%%----------------------------------------------------------------
%% 逐步显示

\usemodule[tikz]         


\starttikzpicture[      scale=3,line cap=round
                        axes/.style=,         
                        important line/.style={very thick},
                        information text/.style={rounded corners,fill=red!10,inner sep=1ex} ]
        \draw[xshift=0.0cm]
                node[right,text width=17cm,information text]
                {
                        \bf
%%%%----------------------------
解: 令 $\darkgreen 1-x=t$, 有
\startformula
\startalign
  \NC  V_x \NC  = \int_0^1 \pi \left [y(x) \right ]^2 dx 
                = \int_0^1 \pi \left[ \sqrt{x(1-x)^9} \,\right]^2 dx \NR
  \NC      \NC  = \pi\int_0^1  x(1-x)^9 dx 
                {\darkgreen \,\, = \,\,} \pi \int_0^1 (1-t)t^9 dt  
            = \pi\left( \frac{1}{10} - \frac{1}{11} \right)
                = \frac{\pi}{110} \NR
\stopalign
\stopformula
%%%%----------------------------

};

\stoptikzpicture
%% 逐步显示
%%----------------------------------------------------------------






\medskip
\FrameTitle{一元函数积分学的几何应用} % 解答

\StartFrame\bf
\index{平均值}

平均值。
\startitemize[n] 
  \item 
\startformula
\bar{f}=\frac{1}{b-a}\int_a^b f(x)dx
\stopformula

\stopitemize




\StopFrame

%%----------------------------------
%%----------------------------------


\medskip
\FrameTitle{一元函数积分学的几何应用} % 解答

\StartFrame\bf\index{p183, 例10.19}
\index{平均值}
\index{零点定理}
\index{积分中值定理}
\index{零点}
\index{唯一性}
\index{交换积分次序}
\index{二重积分}

考点: 平均值。 零点定理。 积分中值定理。 交换积分次序。 二重积分。
\startitemize[n] 
  \item 已知 $f(x)$ 在 $\left[0,\displaystyle \frac{3\pi}{2}\right]$ 上连续, 在 $\left(0,\displaystyle \frac{3\pi}{2}\right)$ 内是函数 $\displaystyle \frac{\cos x}{2x-3\pi}$ 的一个原函数, 且 $f(0)=0$。\crlf
$(1)$ 求 $f(x)$ 在 $\left[0,\displaystyle \frac{3\pi}{2}\right]$ 上的平均值。\crlf
$(2)$ 证明 $f(x)$ 在区间 $\left(0,\displaystyle \frac{3\pi}{2}\right)$ 内存在唯一零点。
\stopitemize




\StopFrame

%%----------------------------------
%%----------------------------------





%%----------------------------------------------------------------
%% 逐步显示

\usemodule[tikz]         


\starttikzpicture[      scale=3,line cap=round
                        axes/.style=,         
                        important line/.style={very thick},
                        information text/.style={rounded corners,fill=red!10,inner sep=1ex} ]
        \draw[xshift=0.0cm]
                node[right,text width=17cm,information text]
                {
                        \bf
%%%%----------------------------
解: $(1)$ $f(x)$ 在区间 $\left[0,\displaystyle \frac{3\pi}{2}\right]$ 上的平均值为
\startformula
\startalign
  \NC  \bar{f} \NC  = \frac{2}{3\pi} \int_0^{\frac{3\pi}{2}} f(x)dx 
                = \frac{2}{3\pi} \int_0^{\frac{3\pi}{2}} \left( \int_0^x \frac{\cos t}{2t-3\pi}dt \right)dx \NR
  \NC      \NC  = \frac{2}{3\pi} \int_0^{\frac{3\pi}{2}} dt\int_t^{\frac{3\pi}{2}} \frac{\cos t}{2t-3\pi}dx
                = -\frac{1}{3\pi}\int_0^{\frac{3\pi}{2}} \cos t dt 
                = \frac{1}{3\pi} \NR
\stopalign
\stopformula
%%%%----------------------------

};

\stoptikzpicture
%% 逐步显示
%%----------------------------------------------------------------







%%----------------------------------------------------------------
%% 逐步显示

\usemodule[tikz]         


\starttikzpicture[      scale=3,line cap=round
                        axes/.style=,         
                        important line/.style={very thick},
                        information text/.style={rounded corners,fill=red!10,inner sep=1ex} ]
        \draw[xshift=0.0cm]
                node[right,text width=17cm,information text]
                {
                        \bf
%%%%----------------------------
$(2)$ 由题意, 得  
$\displaystyle f'(x)=\frac{\cos x }{2x-3\pi}$, $\displaystyle x\in \left (0, \frac{3\pi}{2}\right)$.

当 $0<x<\displaystyle \frac{\pi}{2}$ 时, 因为 $f'(x)<0$, 所以 $f(x)<f(0)=0$, 故 $f(x)$ 在 $\left (0, \displaystyle \frac{3\pi}{2}\right)$ 内无零点, 且 $\displaystyle f\left(\frac{\pi}{2}\right)<0$.



%%%%----------------------------

};

\stoptikzpicture
%% 逐步显示
%%----------------------------------------------------------------











%%----------------------------------------------------------------
%% 逐步显示

\usemodule[tikz]         


\starttikzpicture[      scale=3,line cap=round
                        axes/.style=,         
                        important line/.style={very thick},
                        information text/.style={rounded corners,fill=red!10,inner sep=1ex} ]
        \draw[xshift=0.0cm]
                node[right,text width=17cm,information text]
                {
                        \bf
%%%%----------------------------

由积分中值定理知, 存在 $x_0\in \left[0,\displaystyle \frac{3\pi}{2}\right]$, 使得 $f(x_0)=\bar{f}=\displaystyle \frac{1}{3\pi}>0$, 由于当 $x\in \left( 0, \displaystyle \frac{\pi}{2} \right]$ 时,  $f(x)<0$, 所以 $x_0\in \left ( \displaystyle \frac{\pi}{2},\frac{3\pi}{2} \right]$.


%%%%----------------------------

};

\stoptikzpicture
%% 逐步显示
%%----------------------------------------------------------------







%%----------------------------------------------------------------
%% 逐步显示

\usemodule[tikz]         


\starttikzpicture[      scale=3,line cap=round
                        axes/.style=,         
                        important line/.style={very thick},
                        information text/.style={rounded corners,fill=red!10,inner sep=1ex} ]
        \draw[xshift=0.0cm]
                node[right,text width=17cm,information text]
                {
                        \bf
%%%%----------------------------

根据连续函数零点定理知, 存在 $\xi \in \left( \displaystyle \frac{\pi}{2}, x_0 \right)\subset \left( \displaystyle \frac{\pi}{2}, \frac{3\pi}{2} \right)$, 使得 $f(\xi)=0$. 又因为当 $x\in \left( \displaystyle \frac{\pi}{2}, \frac{3\pi}{2} \right)$ 时, $f'(x)>0$, 所以 $f(x)$ 在 $x\in \left( \displaystyle \frac{\pi}{2}, \frac{3\pi}{2} \right)$ 内至多只有一个零点。

%%%%----------------------------

};

\stoptikzpicture
%% 逐步显示
%%----------------------------------------------------------------








%%----------------------------------------------------------------
%% 逐步显示

\usemodule[tikz]         


\starttikzpicture[      scale=3,line cap=round
                        axes/.style=,         
                        important line/.style={very thick},
                        information text/.style={rounded corners,fill=red!10,inner sep=1ex} ]
        \draw[xshift=0.0cm]
                node[right,text width=17cm,information text]
                {
                        \bf
%%%%----------------------------

综上所述, $f(x)$ 在 $\left( 0, \displaystyle \frac{3\pi}{2} \right)$ 内存在唯一的零点。

%%%%----------------------------

};

\stoptikzpicture
%% 逐步显示
%%----------------------------------------------------------------







\medskip
\FrameTitle{一元函数积分学的几何应用} % 解答

\StartFrame\bf
平面曲线的弧长(仅数一、数二)。
\index{弧长}

\startitemize[n] 
  \item 若平面光滑曲线由直角坐标方程 $y=y(x)$, $(a\leq x\leq b)$ 给出, 则
\startformula
s=\int_a^b \sqrt{1+[y'(x)]^2}\cdot dx
\stopformula

  \item 若平面光滑曲线由参数方程 $x=x(t)$,  $y=y(t)$, $(\alpha\leq t\leq \beta)$ 给出, 则
\startformula
s=\int_{\alpha}^{\beta} \sqrt{[x'(t)]^2+[y'(t)]^2}\cdot dt
\stopformula

  \item 若平面光滑曲线由极坐标方程 $r=r(\theta)$, $(\alpha\leq\theta\leq\beta)$ 给出,则
\startformula
s=\int_{\alpha}^{\beta} \sqrt{[r(\theta)]^2+[r'(\theta)]^2}\cdot d\theta
\stopformula

\stopitemize




\StopFrame

%%----------------------------------
%%----------------------------------


\medskip
\FrameTitle{一元函数积分学的几何应用} % 解答

\StartFrame\bf\index{p184, 例10.20}
\index{变限积分}
\index{弧长}

考点: 弧长。 变限积分。 
\startitemize[n] 
  \item 曲线 $\displaystyle \int_0^x\tan t \cdot dt$, $\left (0\leq x \leq \displaystyle \frac{\pi}{4}\right)$ 的弧长 $s$ =\text{____________ }
\stopitemize




\StopFrame

%%----------------------------------
%%----------------------------------






%%----------------------------------------------------------------
%% 逐步显示

\usemodule[tikz]         


\starttikzpicture[      scale=3,line cap=round
                        axes/.style=,         
                        important line/.style={very thick},
                        information text/.style={rounded corners,fill=red!10,inner sep=1ex} ]
        \draw[xshift=0.0cm]
                node[right,text width=17cm,information text]
                {
                        \bf
%%%%----------------------------
解: 令 $\darkgreen 1-x=t$, 有
\startformula
\startalign
  \NC  s \NC  = \int_0^{\frac{\pi}{4}} \sqrt{1+(y')^2}  dx 
                = \int_0^{\frac{\pi}{4}}  \sqrt{1+(\tan x)^2} dx\NR
  \NC      \NC  = \int_0^{\frac{\pi}{4}}  \sec x \cdot dx
                = \ln |\sec x + \tan x|_0^{\frac{\pi}{4}}  
            = \ln \left(\sqrt{2}+1\right) \NR
\stopalign
\stopformula
%%%%----------------------------

};

\stoptikzpicture
%% 逐步显示
%%----------------------------------------------------------------










\medskip
\FrameTitle{一元函数积分学的几何应用} % 解答

\StartFrame\bf\index{p184, 例10.21}
\index{倍角公式}
\index{弧长}
考点: 倍角公式。 弧长。

\startitemize[n] 
  \item 求曲线 $\displaystyle y=\int_0^x \sqrt{\cos t}\cdot dt$ 的全长。
\stopitemize




\StopFrame

%%----------------------------------
%%----------------------------------







%%----------------------------------------------------------------
%% 逐步显示

\usemodule[tikz]         


\starttikzpicture[      scale=3,line cap=round
                        axes/.style=,         
                        important line/.style={very thick},
                        information text/.style={rounded corners,fill=red!10,inner sep=1ex} ]
        \draw[xshift=0.0cm]
                node[right,text width=17cm,information text]
                {
                        \bf
%%%%----------------------------
解: $y(0)=0$, 故 $x=0$ 对应的点 $(0,0)$ 在曲线上, 且要求 $\cos t\geq 0$, 曲线 $y$ 在 $\displaystyle -\frac{\pi}{2}\leq x \leq \frac{\pi}{2}$ 上存在, 则有 $y'=\sqrt{\cos t}$, 
\startformula
ds=\sqrt{1+(y')^2} \cdot dx = \sqrt{1+\cos x} \cdot dx
\stopformula
故
\startformula
\startalign
  \NC  s \NC  = \int_{-\frac{\pi}{2}}^{\frac{\pi}{2}} \sqrt{1+(\cos x)^2}  \cdot dx 
                = 2\sqrt{2}\int_0^{\frac{\pi}{2}}  \cos \frac{x}{2} \cdot  dx 
            = 4 \NR
\stopalign
\stopformula
%%%%----------------------------

};

\stoptikzpicture
%% 逐步显示
%%----------------------------------------------------------------








\medskip
\FrameTitle{2019数学二填空题第12题4分} % 解答

\StartFrame
\index{2019数学二填空题第12题4分}
\index{2019数学二第12题4分}
\bf
\index{弧长}
\index{变量代换}

考点: 弧长。 变量代换。 

\startitemize[n] 
  \item 设函数 $y=\ln \cos x$,  $\left( 0\leq x\leq \displaystyle \frac{\pi}{6} \right)$ 的弧长为 \text{____________}
\stopitemize



\StopFrame

%%----------------------------------
%%----------------------------------







%%----------------------------------------------------------------
%% 逐步显示

\usemodule[tikz]         


\starttikzpicture[      scale=3,line cap=round
                        axes/.style=,         
                        important line/.style={very thick},
                        information text/.style={rounded corners,fill=red!10,inner sep=1ex} ]
        \draw[xshift=0.0cm]
                node[right,text width=17cm,information text]
                {
                        \bf
%%%%----------------------------
解: 由弧长计算公式, 有
\startformula
\startalign
  \NC  s \NC  = \int_0^{\frac{\pi}{6}} \sqrt{1+(y')^2}  \cdot dx 
                = \int_0^{\frac{\pi}{6}} \sec x  \cdot dx  
                = \int_0^{\frac{\pi}{6}} \frac{1}{\cos ^2x}  \cdot d(\sin x) \NR
  \NC      \NC  = \int_0^{\frac{\pi}{6}} \frac{1}{1-\sin ^2x}   \cdot d(\sin x) 
            = \int_0^{\frac{1}{2}} \frac{1}{1-u^2}  \cdot du
                = \frac{1}{2}\int_0^{\frac{1}{2}} \left( \frac{1}{1+u} +  \frac{1}{1-u} \right)  \cdot du   \NR
  \NC      \NC  = \left.\frac{1}{2}\left[\ln (1+u)-\ln (1-u) \right]\right|_0^{\frac{1}{2}}  
            = \frac{1}{2}\left( \ln\frac{3}{2} -  \ln\frac{1}{2} \right)   
                = \frac{\ln 3}{2}  \NR
\stopalign
\stopformula
%%%%----------------------------

};

\stoptikzpicture
%% 逐步显示
%%----------------------------------------------------------------






\medskip
\FrameTitle{一元函数积分学的几何应用} % 解答

\StartFrame\bf\index{p187, 例10.27}
\index{双曲余弦}
\index{旋转体}
\index{体积}
\index{侧面积}
\index{极限}

考点: 双曲余弦。 旋转体。 体积。 侧面积。 极限。
\startitemize[n] 
  \item 双曲余弦曲线 $y=ch x=\displaystyle \frac{e^x+e^{-x}}{2}$ 与直线 $x=0,x=t,(t>0)$ 及 $y=0$ 所围成一曲边梯形,该曲边梯形绕 $x$ 轴旋转一周得一旋转体, 其体积为 $V(t)$, 侧面积为 $S(t)$, 在 $x=t$ 处的底面积为 $F(t)$。\crlf
$(1)$ 求 $\displaystyle \frac{S(t)}{V(t)}$ 的值。 
$(2)$ 计算极限 $\displaystyle \lim_{t\rightarrow +\infty}\frac{S(t)}{F(t)}$
\stopitemize


%\setupexternalfigures[location={local,default}]

    \placefigure[force]{}{\externalfigure[2020-08-31-090331.png][height=8cm]}


\StopFrame

%%----------------------------------
%%----------------------------------







%%----------------------------------------------------------------
%% 逐步显示

\usemodule[tikz]         


\starttikzpicture[      scale=3,line cap=round
                        axes/.style=,         
                        important line/.style={very thick},
                        information text/.style={rounded corners,fill=red!10,inner sep=1ex} ]
        \draw[xshift=0.0cm]
                node[right,text width=17cm,information text]
                {
                        \bf
%%%%----------------------------
解: $(1)$
\startformula
\startalign
  \NC S(t) \NC  = \int_0^t 2\pi y\sqrt{1+(y')^2}\cdot dx \NR
  \NC      \NC  = 2\pi\int_0^t \frac{e^x+e^{-x}}{2}\cdot \sqrt{1+\frac{e^{2x}-2+e^{-2x}}{4}} dx    
            = 2\pi \int_0^t \left( \frac{e^x+e^{-x}}{2} \right)^2dx \NR
\stopalign
\stopformula
%%%%----------------------------

};

\stoptikzpicture
%% 逐步显示
%%----------------------------------------------------------------





%%----------------------------------------------------------------
%% 逐步显示

\usemodule[tikz]         


\starttikzpicture[      scale=3,line cap=round
                        axes/.style=,         
                        important line/.style={very thick},
                        information text/.style={rounded corners,fill=red!10,inner sep=1ex} ]
        \draw[xshift=0.0cm]
                node[right,text width=17cm,information text]
                {
                        \bf
%%%%----------------------------

\startformula
\startalign
  \NC V(t) \NC  = \pi \int_0^t \left( \frac{e^x+e^{-x}}{2} \right)^2dx \NR
\stopalign
\stopformula

所以, $\displaystyle \frac{S(t)}{V(t)} = 2$
%%%%----------------------------

};

\stoptikzpicture
%% 逐步显示
%%----------------------------------------------------------------





%%----------------------------------------------------------------
%% 逐步显示

\usemodule[tikz]         


\starttikzpicture[      scale=3,line cap=round
                        axes/.style=,         
                        important line/.style={very thick},
                        information text/.style={rounded corners,fill=red!10,inner sep=1ex} ]
        \draw[xshift=0.0cm]
                node[right,text width=17cm,information text]
                {
                        \bf
%%%%----------------------------
$(2)$
\startformula
\startalign
  \NC F(t) \NC  = \pi y^2 \big |_{x=t} = \pi \left( \frac{e^x+e^{-x}}{2} \right)^2 \NR
\stopalign
\stopformula

\startformula
\startalign
  \NC \lim_{t\rightarrow +\infty} \frac{S(t)}{F(t)} \NC  = \lim_{t\rightarrow +\infty} \frac{2\pi \displaystyle \int_0^t \left( \frac{e^x+e^{-x}}{2} \right)^2 dx}{\pi \left( \displaystyle \frac{e^t+e^{-t}}{2} \right)^2} \NR
  \NC  \NC = \lim_{t\rightarrow +\infty} \frac{2 \cdot \left( \displaystyle \frac{e^t+e^{-t}}{2} \right)^2}{2\cdot \displaystyle \frac{e^t+e^{-t}}{2} \cdot \frac{e^t-e^{-t}}{2} }  
       = \lim_{t\rightarrow +\infty} \frac{e^t+e^{-t}}{e^t-e^{-t}} =1 \NR
\stopalign
\stopformula


%%%%----------------------------

};

\stoptikzpicture
%% 逐步显示
%%----------------------------------------------------------------






\medskip
\FrameTitle{一元函数积分学的几何应用} % 解答

\StartFrame\bf\index{p188, 例10.28}
\index{旋转体}
\index{体积}
\index{表面积}
\index{参数方程}
\index{星形线}
\index{Wallis公式}
\index{微元法}

考点: 旋转体。 体积。 表面积。 参数方程。 星形线。 Wallis公式。 微元法。
\startitemize[n] 
  \item 设 $D$ 是由曲线 $y=\sqrt{1-x^2}$, $(0\leq x\leq 1)$ 与 $x(t)=\cos^3t$, $y(t)=\sin^3t$, $0\leq t \leq \displaystyle \frac{\pi}{2}$ 围成的平面区域, 求 $D$ 绕 $x$ 轴旋转一周所得旋转体的体积和表面积。
\stopitemize


%\setupexternalfigures[location={local,default}]

    \placefigure[force]{}{\externalfigure[2020-08-31-091305.png][height=8cm]}


\StopFrame

%%----------------------------------
%%----------------------------------





%%----------------------------------------------------------------
%% 逐步显示

\usemodule[tikz]         


\starttikzpicture[      scale=3,line cap=round
                        axes/.style=,         
                        important line/.style={very thick},
                        information text/.style={rounded corners,fill=red!10,inner sep=1ex} ]
        \draw[xshift=0.0cm]
                node[right,text width=17cm,information text]
                {
                        \bf
%%%%----------------------------
解: 设 $D$绕 $x$ 轴旋转一周所得旋转体的体积为 $V$, 表面积为 $S$, 则
\startformula
\startalign
  \NC V \NC  = \frac{2}{3}\pi - \int_0^1 \pi y^2(t)\cdot d(x(t))  
         = \frac{2}{3}\pi - \int_{\frac{\pi}{2}}^0 \pi \sin ^6t \cdot (\cos ^3t)'\cdot dt \NR
  \NC  \NC  = \frac{2}{3}\pi +3\pi \int_0^{\frac{\pi}{2}} (1-\cos ^2t)^3\cos ^2t \cdot d(\cos t)  
        = \frac{2}{3}\pi - \frac{16}{105}\pi = \frac{18}{35}\pi \NR
\stopalign
\stopformula



%%%%----------------------------

};

\stoptikzpicture
%% 逐步显示
%%----------------------------------------------------------------




%%----------------------------------------------------------------
%% 逐步显示

\usemodule[tikz]         


\starttikzpicture[      scale=3,line cap=round
                        axes/.style=,         
                        important line/.style={very thick},
                        information text/.style={rounded corners,fill=red!10,inner sep=1ex} ]
        \draw[xshift=0.0cm]
                node[right,text width=17cm,information text]
                {
                        \bf
%%%%----------------------------


\startformula
\startalign
  \NC S \NC  = 2\pi +\int_0^{\frac{\pi}{2}}2\pi y(t) \cdot \sqrt{x'^2+y'^2}\cdot dt \NR
  \NC  \NC = 2\pi + 2\pi \int_0^{\frac{\pi}{2}} \sin^3t \cdot \sqrt{9\cos ^4t\sin^2t+9\sin^4t\cos^2t}\cdot dt \NR
  \NC  \NC = 2\pi +6\pi  \int_0^{\frac{\pi}{2}} \sin^4t\cdot \cos t\cdot dt =\frac{16}{5}\pi \NR
\stopalign
\stopformula


%%%%----------------------------

};

\stoptikzpicture
%% 逐步显示
%%----------------------------------------------------------------





\medskip
\FrameTitle{一元函数积分学的几何应用} % 解答

\StartFrame\bf
平面上的曲边梯形的形心坐标公式。 (仅数一、数二)
\startitemize[n] 
  \item 设 $D=\{(x,y)|0\leq y \leq f(x), a\leq x\leq b\}$, $f(x)$ 在 $[a,b]$ 上连续, $D$ 的形心坐标 $\bar{x}$, $\bar{y}$ 的计算公式为\crlf

\startformula
\bar{x}
=\frac{\displaystyle \iint_D x d\sigma}{\displaystyle \iint_D  d\sigma}
=\frac{\displaystyle \int_a^b dx\int_0^{f(x)}xdy}{\displaystyle \int_a^b dx\int_0^{f(x)}dy}=\frac{\displaystyle \int_a^b xf(x)dx}{\displaystyle \int_a^b f(x) dx}
\stopformula

\startformula
\bar{y}=\frac{\displaystyle \iint_D y d\sigma}{\displaystyle \iint_D  d\sigma}
=\frac{\displaystyle \int_a^b dx\int_0^{f(x)}ydy}{\displaystyle \int_a^b dx\int_0^{f(x)}dy}
=\frac{\displaystyle \int_a^b yf(x)dx}{\displaystyle \int_a^b f(x) dx}
\stopformula

  \item 特别地, 质量均匀分布的平面薄片的质心, 即形心。
\stopitemize

\StopFrame

%%----------------------------------
%%----------------------------------


\medskip
\FrameTitle{一元函数积分学的几何应用} % 解答

\StartFrame\bf\index{p189, 例10.29}
\index{形心}
\index{对称性}
\index{三角代换}
\index{Wallis公式}

考点: 形心。 对称性。 三角代换。 Wallis公式。
\startitemize[n] 
  \item 求由曲线 $y^2=x^3-x^4$ 所围成的平面图形的形心。
\stopitemize

%\setupexternalfigures[location={local,default}]

    \placefigure[force]{}{\externalfigure[2020-08-31-094329.png][height=8cm]}

\StopFrame

%%----------------------------------
%%----------------------------------






%%----------------------------------------------------------------
%% 逐步显示

\usemodule[tikz]         


\starttikzpicture[      scale=3,line cap=round
                        axes/.style=,         
                        important line/.style={very thick},
                        information text/.style={rounded corners,fill=red!10,inner sep=1ex} ]
        \draw[xshift=0.0cm]
                node[right,text width=17cm,information text]
                {
                        \bf
%%%%----------------------------

解: 图形关于 $x$ 轴对称, 故 $\bar{y}=0$, 当 $0\leq x \leq 1$ 时, $y=\mp\sqrt{x^3-x^4}$, 平面图形的高 $h=2\sqrt{x^3-x^4}$, 令 $\darkgreen x=\sin^2 \theta$, 则

\startformula
\bar{x}
=\frac{\displaystyle\int_0^1 x\cdot 2 \sqrt{x^3-x^4} \cdot dx}{\displaystyle\int_0^1 2 \sqrt{x^3-x^4}\cdot  dx}
=\frac{\displaystyle\int_0^1 x^{\frac{5}{2}}\cdot  \sqrt{1-x} \cdot dx}{\displaystyle\int_0^1 x^{\frac{3}{2}}\cdot  \sqrt{1-x}\cdot  dx}
\stopformula



%%%%----------------------------

};

\stoptikzpicture
%% 逐步显示
%%----------------------------------------------------------------







%%----------------------------------------------------------------
%% 逐步显示

\usemodule[tikz]         


\starttikzpicture[      scale=3,line cap=round
                        axes/.style=,         
                        important line/.style={very thick},
                        information text/.style={rounded corners,fill=red!10,inner sep=1ex} ]
        \draw[xshift=0.0cm]
                node[right,text width=17cm,information text]
                {
                        \bf
%%%%----------------------------



\startformula
\startalign
  \NC  \NC   \int_0^1 x^{\frac{5}{2}}\cdot  \sqrt{1-x} \cdot dx 
        = {\darkgreen \int_0^{\frac{\pi}{2}} \sin^5\theta \cdot \cos \theta \cdot 2\sin\theta\cdot \cos \theta\cdot d\theta} \NR
  \NC  \NC  = 2\int_0^{\frac{\pi}{2}} \sin^6\theta\cdot \cos ^2\theta \cdot d\theta  
       = 2\left( \int_0^{\frac{\pi}{2}} \sin^6\theta \cdot d\theta -  \int_0^{\frac{\pi}{2}} \sin^8\theta \cdot d\theta \right)  \NR
  \NC \NC = 2\left( \frac{5}{6}\cdot\frac{3}{4}\cdot\frac{1}{2}- \frac{7}{8}\cdot\frac{5}{6}\cdot\frac{3}{4}\cdot\frac{1}{2}   \right)\cdot \frac{\pi}{2}
         = \frac{5}{128}\pi \NR
\stopalign
\stopformula


%%%%----------------------------

};

\stoptikzpicture
%% 逐步显示
%%----------------------------------------------------------------







%%----------------------------------------------------------------
%% 逐步显示

\usemodule[tikz]         


\starttikzpicture[      scale=3,line cap=round
                        axes/.style=,         
                        important line/.style={very thick},
                        information text/.style={rounded corners,fill=red!10,inner sep=1ex} ]
        \draw[xshift=0.0cm]
                node[right,text width=17cm,information text]
                {
                        \bf
%%%%----------------------------



\startformula
\startalign
  \NC  \NC \int_0^1 x^{\frac{3}{2}}\cdot  \sqrt{1-x} \cdot dx
        = {\darkgreen \int_0^{\frac{\pi}{2}} \sin^3\theta \cdot \cos \theta \cdot 2\sin\theta\cdot \cos \theta\cdot d\theta} \NR
  \NC  \NC  = 2\int_0^{\frac{\pi}{2}} \sin^4\theta\cdot \cos ^2\theta \cdot d\theta  
       = 2\left( \int_0^{\frac{\pi}{2}} \sin^4\theta \cdot d\theta -  \int_0^{\frac{\pi}{2}} \sin^6\theta \cdot d\theta \right)  \NR
  \NC \NC = 2\left( \frac{3}{4}\cdot\frac{1}{2}- \frac{5}{6}\cdot\frac{3}{4}\cdot\frac{1}{2}   \right)\cdot \frac{\pi}{2}
         = \frac{1}{16}\pi \NR
\stopalign
\stopformula


%%%%----------------------------

};

\stoptikzpicture
%% 逐步显示
%%----------------------------------------------------------------







%%----------------------------------------------------------------
%% 逐步显示

\usemodule[tikz]         


\starttikzpicture[      scale=3,line cap=round
                        axes/.style=,         
                        important line/.style={very thick},
                        information text/.style={rounded corners,fill=red!10,inner sep=1ex} ]
        \draw[xshift=0.0cm]
                node[right,text width=17cm,information text]
                {
                        \bf
%%%%----------------------------


所以, $\displaystyle\bar{x}= \frac{\frac{5\pi}{128}}{\frac{\pi}{16}}=\frac{5}{8}$, 故形心为
$\left( \displaystyle\frac{5}{8}, 0 \right)$


%%%%----------------------------

};

\stoptikzpicture
%% 逐步显示
%%----------------------------------------------------------------






\medskip
\FrameTitle{一元函数积分学的几何应用} % 解答

\StartFrame\bf
平行截面面积已知的立体体积。
\startitemize[n] 
  \item 在 $[a,b]$ 区间上, 垂直于 $x$ 轴的平面截立体 $\Omega$ 所得到的截面面积为 $x$ 的连续函数 $S(x)$, 则 $\Omega$ 的体积为 $V=\displaystyle \int_a^b S(x)dx$
\stopitemize



\StopFrame

%%----------------------------------
%%----------------------------------


\medskip
\FrameTitle{一元函数积分学的几何应用} % 解答

\StartFrame\bf\index{p191, 例10.31}
\index{楔形体}
\index{体积}
\index{平行截面面积已知的立体体积}

{\darkgreen 自己练习。}
\startitemize[n] 
  \item 设一个底面半径为 $3$ 的圆柱体, 被一个与圆柱的底面相交、 夹角为 $\displaystyle\frac{\pi}{4}$ 且过底面直径的平面所截, 求截下的楔形体的体积。
\stopitemize



\StopFrame

%%----------------------------------
%%----------------------------------






%%----------------------------------------------------------------
%% 逐步显示

\usemodule[tikz]         


\starttikzpicture[      scale=3,line cap=round
                        axes/.style=,         
                        important line/.style={very thick},
                        information text/.style={rounded corners,fill=red!10,inner sep=1ex} ]
        \draw[xshift=0.0cm]
                node[right,text width=17cm,information text]
                {
                        \bf
%%%%----------------------------
提示: 建立适当的坐标系, 垂直于 $x$ 轴的截面是直角三角形, 底边长 $\sqrt{3^2-x^2}$, 对边长  
$\sqrt{3^2-x^2}\cdot \tan \displaystyle\frac{\pi}{4} = \sqrt{3^2-x^2}$, 故截面面积 $S= \displaystyle\frac{1}{2}\cdot (3^2-x^2)$, 则
\startformula
V=\int_{-3}^3 \frac{1}{2}\cdot (3^2-x^2)\cdot dx =18
\stopformula

%%%%----------------------------

};

\stoptikzpicture
%% 逐步显示
%%----------------------------------------------------------------






%%----------------------------------
%%----------------------------------

\page

%% 

\usemodule[chart]
\setupFLOWcharts[
height=2.5\lineheight,
width=6\bodyfontsize,
dx=1\bodyfontsize,
dy=0.2\bodyfontsize,
]

\setupFLOWshapes
[framecolor=pragmacolor,
background=color,
backgroundcolor=white,
]

\setupFLOWlines[framecolor=pragmacolor]
\startFLOWchart[example]

\startFLOWcell 
  \name {01}
  \location {0,4}
  \text {一元函数积分学4}
  \connection [rl] {11}
  \connection [rl] {12}
\stopFLOWcell

\startFLOWcell
  \name {11}
  \location{2,1}
  \text {积分等式}
  \connection [rl] {31}
  \connection [rl] {33}
\stopFLOWcell

\startFLOWcell
  \name {12}
  \location{2,6}
  \text {积分不等式}
  \connection [rl] {38}
  \connection [rl] {32}
  \connection [rl] {34}
  \connection [rl] {35}
  \connection [rl] {36}
\stopFLOWcell

\startFLOWcell
  \name {31}
  \location{3,1}
  \text {常用积分等式}
\stopFLOWcell

\startFLOWcell
  \name {32}
  \location{3,3}
  \text {函数单调性}
\stopFLOWcell

\startFLOWcell
  \name {33}
  \location{3,2}
  \text {积分形式中值定理}
\stopFLOWcell

\startFLOWcell
  \name {34}
  \location{3,4}
  \text {被积函数}
  \connection [rl] {41}
  \connection [rl] {42}
  \connection [rl] {43}
  \connection [rl] {45}
  \connection [rl] {46}
  \connection [rl] {47}
  \connection [rl] {48}
\stopFLOWcell

\startFLOWcell
  \name {35}
  \location{3,5}
  \text {夹逼准则极限问题}
\stopFLOWcell

\startFLOWcell
  \name {36}
  \location{3,6}
  \text {曲边梯形的面积}
\stopFLOWcell

\startFLOWcell
  \name {42}
  \location{4,1}
  \text {积分的保号性}
\stopFLOWcell

\startFLOWcell
  \name {43}
  \location{4,2}
  \text {拉格朗日中值定理}
\stopFLOWcell

\startFLOWcell
  \name {45}
  \location{4,6}
  \text {泰勒公式}
\stopFLOWcell

\startFLOWcell
  \name {46}
  \location{4,3}
  \text {放缩法}
\stopFLOWcell

\startFLOWcell
  \name {47}
  \location{4,4}
  \text {分部积分法}
\stopFLOWcell

\startFLOWcell
  \name {48}
  \location{4,5}
  \text {换元法}
\stopFLOWcell
\stopFLOWchart
\FLOWchart[example]

%%------------------------
%%------------------------
%% 


\page
\FrameTitle{一元函数积分学的应用} % 解答

\StartFrame\bf
积分等式与积分不等式。
\startitemize[n] 
  \item 积分等式。
  \item 积分形式的中值定理。
\stopitemize



\StopFrame

%%----------------------------------
%%----------------------------------
\medskip
\FrameTitle{一元函数积分学的应用} % 解答

\StartFrame\bf\index{p195, 例11.1}
\index{偶函数}
\index{周期函数}
\index{奇偶性}
\index{周期性}
\index{积分等式}

考点: 偶函数。 周期函数。 积分等式。 奇偶性。 周期性。
\startitemize[n] 
  \item 设 $n = 0, 1, 2, 3, \cdots$, 且 $f(x)$ 是连续的偶函数, 且是以 $T$ 为周期的周期函数,  
证明:
\startformula
\int_0^{nT}xf(x)\cdot dx=\frac{n^2T}{2}\int_0^T f(x)\cdot dx, 
\stopformula
  \item 计算
\startformula
\int_0^{n\pi} x|\sin x|\cdot dx
\stopformula
\stopitemize



\StopFrame

%%----------------------------------
%%----------------------------------





%%----------------------------------------------------------------
%% 逐步显示

\usemodule[tikz]         


\starttikzpicture[      scale=3,line cap=round
                        axes/.style=,         
                        important line/.style={very thick},
                        information text/.style={rounded corners,fill=red!10,inner sep=1ex} ]
        \draw[xshift=0.0cm]
                node[right,text width=17cm,information text]
                {
                        \bf
%%%%----------------------------
解: $(1)$ 令 $\darkgreen x=nT-t$, 则

\startformula
L=
\int_0^{nT}xf(x)dx
=\darkgreen nT\int_0^{nT} f(t)dt - \int_0^{nT} tf(t)dt
\stopformula

于是

\startformula
\int_0^{nT}xf(x)dx
=\frac{nT}{2} \int_0^{nT} f(x)dx
\stopformula

%%%%----------------------------

};

\stoptikzpicture
%% 逐步显示
%%----------------------------------------------------------------






%%----------------------------------------------------------------
%% 逐步显示

\usemodule[tikz]         


\starttikzpicture[      scale=3,line cap=round
                        axes/.style=,         
                        important line/.style={very thick},
                        information text/.style={rounded corners,fill=red!10,inner sep=1ex} ]
        \draw[xshift=0.0cm]
                node[right,text width=17cm,information text]
                {
                        \bf
%%%%----------------------------

又 $f(x+T)=f(x)$, 则

\startformula
\int_0^{nT}f(x)dx
=n \int_0^T f(x)dx
\stopformula

所以, 
\startformula
L=
\int_0^{nT}xf(x)dx
=\frac{n^2T}{2} \int_0^{T} f(x)dx
=R
\stopformula

%%%%----------------------------

};

\stoptikzpicture
%% 逐步显示
%%----------------------------------------------------------------






%%----------------------------------------------------------------
%% 逐步显示

\usemodule[tikz]         


\starttikzpicture[      scale=3,line cap=round
                        axes/.style=,         
                        important line/.style={very thick},
                        information text/.style={rounded corners,fill=red!10,inner sep=1ex} ]
        \draw[xshift=0.0cm]
                node[right,text width=17cm,information text]
                {
                        \bf
%%%%----------------------------
 $(2)$ 因为 $|\sin x|$ 是连续的以 $\pi$ 为周期的偶函数, 故

\startformula
I=  \int_0^{nT} x|\sin x|dx
=\frac{n^2\pi}{2}\int_0^{\pi} |\sin x|dx 
=\frac{n^2\pi}{2}\int_0^{\pi} \sin x dx  
=n^2\pi
\stopformula

%%%%----------------------------

};

\stoptikzpicture
%% 逐步显示
%%----------------------------------------------------------------




\medskip
\FrameTitle{一元函数积分学的应用} % 解答

\StartFrame\bf\index{p195, 例11.2}
\index{牢记结论}
\index{恒等式}
\index{偶函数}
\index{定积分性质}
\index{变量代换}
\index{连续}
\index{换元换限}
\index{对称区间}

考点: 偶函数。 恒等式。 变量代换。 定积分性质。 连续。 换元换限。 对称区间。
\startitemize[n] 
  \item ({\darkgreen 牢记结论}) 设 $f(x)$, $g(x)$ 在 $[-a,a]$, $(a>0)$ 上连续, $g(x)$ 为偶函数, 且 $f(x)$ 满足条件 
$f(x)+f(-x)=A$, (常数),\crlf

$(1)$ 证明: 
\startformula
\int_{-a}^a f(x)\cdot g(x)\cdot dx =A\int_0^a g(x)\cdot dx
\stopformula

$(2)$ 计算 
\startformula
I = \int_{-\frac{\pi}{2}}^{\frac{\pi}{2}} |\sin x|\cdot  \arctan e^x\cdot  dx
\stopformula

\stopitemize



\StopFrame

%%----------------------------------
%%----------------------------------






%%----------------------------------------------------------------
%% 逐步显示

\usemodule[tikz]         


\starttikzpicture[      scale=3,line cap=round
                        axes/.style=,         
                        important line/.style={very thick},
                        information text/.style={rounded corners,fill=red!10,inner sep=1ex} ]
        \draw[xshift=0.0cm]
                node[right,text width=17cm,information text]
                {
                        \bf
%%%%----------------------------
解: $(1)$ 令 $\darkgreen x=-t$, 则


\startformula
\startalign
  \NC \int_{-a}^0 f(x)g(x)dx \NC = {\darkgreen -\int_{a}^0 f(-t)g(-t)dt }
            = \int_{0}^a f(-x)g(x)dx  \NR
\stopalign
\stopformula



%%%%----------------------------

};

\stoptikzpicture
%% 逐步显示
%%----------------------------------------------------------------










%%----------------------------------------------------------------
%% 逐步显示

\usemodule[tikz]         


\starttikzpicture[      scale=3,line cap=round
                        axes/.style=,         
                        important line/.style={very thick},
                        information text/.style={rounded corners,fill=red!10,inner sep=1ex} ]
        \draw[xshift=0.0cm]
                node[right,text width=17cm,information text]
                {
                        \bf
%%%%----------------------------


于是, 有
\startformula
\startalign
  \NC L \NC = \int_{-a}^a f(x)g(x)dx 
            = \int_{-a}^0 f(x)g(x)dx +\int_{0}^a f(x)g(x)dx  \NR
  \NC   \NC = \int_{0}^a f(-x)g(x)dx +\int_{0}^a f(x)g(x)dx  \NR
  \NC   \NC = \int_{0}^a \left[f(-x) + f(x)\right]g(x)\cdot dx 
            = A \int_{0}^a g(x)dx  
            = R \NR
\stopalign
\stopformula

%%%%----------------------------

};

\stoptikzpicture
%% 逐步显示
%%----------------------------------------------------------------





%%----------------------------------------------------------------
%% 逐步显示

\usemodule[tikz]         


\starttikzpicture[      scale=3,line cap=round
                        axes/.style=,         
                        important line/.style={very thick},
                        information text/.style={rounded corners,fill=red!10,inner sep=1ex} ]
        \draw[xshift=0.0cm]
                node[right,text width=17cm,information text]
                {
                        \bf
%%%%----------------------------
$(2)$ 令 $f(x)=\arctan e^x$, $g(x)=|\sin x|$,  则 $f(x)$, $g(x)$ 在 $\displaystyle \left[ -\frac{\pi}{2}, \frac{\pi}{2} \right]$ 上连续,  $g(x)$ 为偶函数。



%%%%----------------------------

};

\stoptikzpicture
%% 逐步显示
%%----------------------------------------------------------------






%%----------------------------------------------------------------
%% 逐步显示

\usemodule[tikz]         


\starttikzpicture[      scale=3,line cap=round
                        axes/.style=,         
                        important line/.style={very thick},
                        information text/.style={rounded corners,fill=red!10,inner sep=1ex} ]
        \draw[xshift=0.0cm]
                node[right,text width=17cm,information text]
                {
                        \bf
%%%%----------------------------
又 $(\arctan e^x + \arctan e^{-x})'=0$, 所以
$\arctan e^x + \arctan e^{-x}=A$.



%%%%----------------------------

};

\stoptikzpicture
%% 逐步显示
%%----------------------------------------------------------------





%%----------------------------------------------------------------
%% 逐步显示

\usemodule[tikz]         


\starttikzpicture[      scale=3,line cap=round
                        axes/.style=,         
                        important line/.style={very thick},
                        information text/.style={rounded corners,fill=red!10,inner sep=1ex} ]
        \draw[xshift=0.0cm]
                node[right,text width=17cm,information text]
                {
                        \bf
%%%%----------------------------


令 $x=0$, 得 $2\arctan 1 =A$, 故 $A=\displaystyle\frac{\pi}{2}$, 即 $f(x)+f(-x)=\displaystyle\frac{\pi}{2}$.

于是, 有


\startformula
\startalign
  \NC I \NC = \int_{-\frac{\pi}{2}}^{\frac{\pi}{2}} |\sin x|\cdot  \arctan e^x\cdot  dx
            = \frac{\pi}{2}\int_0^{\frac{\pi}{2}} |\sin x|\cdot dx   
        = \frac{\pi}{2}\int_0^{\frac{\pi}{2}} \sin x\cdot dx  
            = \frac{\pi}{2} \NR
\stopalign
\stopformula

%%%%----------------------------

};

\stoptikzpicture
%% 逐步显示
%%----------------------------------------------------------------






\medskip
\FrameTitle{一元函数积分学的应用} % 解答

\StartFrame\bf\index{p196, 例11.3}
\index{罗尔定理}
\index{分部积分}
\index{反证法}
\index{定积分性质}
\index{升阶}
\index{原函数}
\index{积分公式}
\index{变限积分的定义}
\index{变限积分的性质}
\index{辅助函数}
\index{可导}
\index{可积的充分条件}

考点: 罗尔定理。 分部积分。 反证法。 定积分性质。 变限积分的定义。 变限积分的性质。 辅助函数。 可积的充分条件。 可导。 原函数。
\startitemize[n] 
  \item 设 
\startformula
\int_{-1}^1 f(x)dx = \int_{-1}^1 f(x)\cdot \tan x\cdot dx = 0
\stopformula

其中 $f(x)$ 在区间 $[-1,1]$ 上连续, 

证明: 在区间 $(-1,1)$ 内至少存在互异的两点 $\xi_1$, $\xi_2$, 使得\crlf
 $f(\xi_1)=f(\xi_2)=0$

\stopitemize



\StopFrame

%%----------------------------------
%%----------------------------------





%%----------------------------------------------------------------
%% 逐步显示

\usemodule[tikz]         


\starttikzpicture[      scale=3,line cap=round
                        axes/.style=,         
                        important line/.style={very thick},
                        information text/.style={rounded corners,fill=red!10,inner sep=1ex} ]
        \draw[xshift=0.0cm]
                node[right,text width=17cm,information text]
                {
                        \bf
%%%%----------------------------
解: 
令 $\displaystyle F(x)=\int _{-1}^x f(t)dt$, 则 $F(x)$ 在 $[-1,1]$ 上可导, 且 $F(-1)=F(1)=0$。 



%%%%----------------------------

};

\stoptikzpicture
%% 逐步显示
%%----------------------------------------------------------------






%%----------------------------------------------------------------
%% 逐步显示

\usemodule[tikz]         


\starttikzpicture[      scale=3,line cap=round
                        axes/.style=,         
                        important line/.style={very thick},
                        information text/.style={rounded corners,fill=red!10,inner sep=1ex} ]
        \draw[xshift=0.0cm]
                node[right,text width=17cm,information text]
                {
                        \bf
%%%%----------------------------

({\darkgreen 反证法}) 若 $F(x)$ 在 $(-1,1)$ 内无零点, 不妨设 $F(x)>0$, $x\in (-1,1)$, 则

\startformula
\startalign
  \NC  \NC \int_{-1}^1 f(x)\cdot \tan x\cdot dx
       = \int_{-1}^1  \tan x\cdot d\left[ F(x) \right] \NR
  \NC   \NC = \left. F(x)\cdot \tan x \right|_{-1}^1 - \int_{-1}^1 F(x)\cdot d(\tan x)  
        = -\int_{-1}^1 F(x)\cdot \sec^2 x dx <0 \NR
\stopalign
\stopformula

与所给条件 $\displaystyle\int_{-1}^1 f(x)\cdot \tan x dx =0$ {\darkgreen 矛盾}, 故至少存在一点 $x_0 \in (-1,1)$, 使得 $F(x_0)=0$。


%%%%----------------------------

};

\stoptikzpicture
%% 逐步显示
%%----------------------------------------------------------------








%%----------------------------------------------------------------
%% 逐步显示

\usemodule[tikz]         


\starttikzpicture[      scale=3,line cap=round
                        axes/.style=,         
                        important line/.style={very thick},
                        information text/.style={rounded corners,fill=red!10,inner sep=1ex} ]
        \draw[xshift=0.0cm]
                node[right,text width=17cm,information text]
                {
                        \bf
%%%%----------------------------


再对 $F(x)$ 分别在区间 $[-1, x_0]$,  $[x_0,1]$ 上使用{\darkgreen 罗尔定理}, 得到

至少存在一点 $\xi_1\in (-1, x_0)$ 和 $\xi_2\in (x_0,1)$,  使得 $F'(\xi_1)=0$,  $F'(\xi_2)=0$, 即  $f(\xi_1)=f(\xi_2)=0$。
%%%%----------------------------

};

\stoptikzpicture
%% 逐步显示
%%----------------------------------------------------------------








\medskip
\FrameTitle{一元函数积分学的应用} % 解答

\StartFrame\bf\index{p197, 例11.4}
\index{积分中值定理}
\index{积分中值定理的推广}
\index{闭区间上连续函数的性质}
\index{最值性定理}
\index{介值性定理}
\index{不等式组}

考点: 积分中值定理。 积分中值定理的推广。 闭区间上连续函数的性质。 最值性定理。 介值性定理。 
\index{积分中值定理}
\startitemize[n] 
  \item 设 $f(x)$, $g(x)$ 在 $[a,b]$ 上连续且 $g(x)$ 在 $[a,b]$ 上不变号, 

证明: 至少存在一点 $\xi\in[a,b]$, 使得 
\startformula
\int_a^b f(x)g(x)dx=f(\xi)\int_a^b g(x)dx
\stopformula

\item 特别地, 当 $g(x)\equiv 1$ 时, 即得积分中值定理。

\stopitemize



\StopFrame

%%----------------------------------
%%----------------------------------







%%----------------------------------------------------------------
%% 逐步显示

\usemodule[tikz]         


\starttikzpicture[      scale=3,line cap=round
                        axes/.style=,         
                        important line/.style={very thick},
                        information text/.style={rounded corners,fill=red!10,inner sep=1ex} ]
        \draw[xshift=0.0cm]
                node[right,text width=17cm,information text]
                {
                        \bf
%%%%----------------------------
证明: 当 $g(x)\equiv 0$ 时, 有
$\displaystyle \int_a^b g(x)dx=0$,\quad $\displaystyle \int_a^b f(x)g(x)dx=\int_a^b f(x)\cdot 0 dx=0$, 

此时, $\xi$ 可以是 $[a,b]$ 上任何一点, 都会有下式 

$\displaystyle \int_a^b f(x)g(x)dx=f(\xi)\cdot \int_a^b g(x)dx$ 
成立。
%%%%----------------------------

};

\stoptikzpicture
%% 逐步显示
%%----------------------------------------------------------------








%%----------------------------------------------------------------
%% 逐步显示

\usemodule[tikz]         


\starttikzpicture[      scale=3,line cap=round
                        axes/.style=,         
                        important line/.style={very thick},
                        information text/.style={rounded corners,fill=red!10,inner sep=1ex} ]
        \draw[xshift=0.0cm]
                node[right,text width=17cm,information text]
                {
                        \bf
%%%%----------------------------
当 $g(x)\nequiv 0$ 时, 必存在 $x_0\in [a, b]$, $g(x_0)$ 或者大于零或者小于零。 不妨设 $g(x_0)>0$, 此时由 $g(x)$ 不变号且连续, 必有  $\displaystyle\int_a^b g(x)dx>0$. 
%%%%----------------------------

};

\stoptikzpicture
%% 逐步显示
%%----------------------------------------------------------------










%%----------------------------------------------------------------
%% 逐步显示

\usemodule[tikz]         


\starttikzpicture[      scale=3,line cap=round
                        axes/.style=,         
                        important line/.style={very thick},
                        information text/.style={rounded corners,fill=red!10,inner sep=1ex} ]
        \draw[xshift=0.0cm]
                node[right,text width=17cm,information text]
                {
                        \bf
%%%%----------------------------

又 $f(x)$ 在 $[a, b]$ 上连续, 根据闭区间上连续函数的{\darkgreen 最值性定理}, 必能取到最小值 $m$ 与最大值 $M$, 于是对于一切 $x\in [a,b]$, 都有
$m\leq f(x)\leq M$, 则

\startformula
mg(x)\leq f(x)g(x)\leq Mg(x)
\stopformula
%%%%----------------------------

};

\stoptikzpicture
%% 逐步显示
%%----------------------------------------------------------------










%%----------------------------------------------------------------
%% 逐步显示

\usemodule[tikz]         


\starttikzpicture[      scale=3,line cap=round
                        axes/.style=,         
                        important line/.style={very thick},
                        information text/.style={rounded corners,fill=red!10,inner sep=1ex} ]
        \draw[xshift=0.0cm]
                node[right,text width=17cm,information text]
                {
                        \bf
%%%%----------------------------


于是,有

\startformula
\startalign
  \NC \NC m\int_a^b g(x)dx = \int_a^b mg(x)dx \NR
  \NC \NC \leq \int_a^b f(x)g(x)dx  \leq \int_a^b Mg(x)dx = M\int_a^b g(x)dx \NR
\stopalign
\stopformula
%%%%----------------------------

};

\stoptikzpicture
%% 逐步显示
%%----------------------------------------------------------------








%%----------------------------------------------------------------
%% 逐步显示

\usemodule[tikz]         


\starttikzpicture[      scale=3,line cap=round
                        axes/.style=,         
                        important line/.style={very thick},
                        information text/.style={rounded corners,fill=red!10,inner sep=1ex} ]
        \draw[xshift=0.0cm]
                node[right,text width=17cm,information text]
                {
                        \bf
%%%%----------------------------
由于 $\displaystyle\int_a^b g(x)dx>0$, 得 
\startformula
m\leq \frac{\int_a^b f(x)g(x)dx}{\int_a^b g(x)dx}
\leq M
\stopformula
%%%%----------------------------

};

\stoptikzpicture
%% 逐步显示
%%----------------------------------------------------------------







%%----------------------------------------------------------------
%% 逐步显示

\usemodule[tikz]         


\starttikzpicture[      scale=3,line cap=round
                        axes/.style=,         
                        important line/.style={very thick},
                        information text/.style={rounded corners,fill=red!10,inner sep=1ex} ]
        \draw[xshift=0.0cm]
                node[right,text width=17cm,information text]
                {
                        \bf
%%%%----------------------------
根据{\darkgreen 介值定理}, 至少存在一点 $\xi\in [a,b]$, 使得 
\startformula
\frac{\int_a^b f(x)g(x)dx}{\int_a^b g(x)dx}
=f(\xi)
\stopformula
即得
\startformula
\int_a^b f(x)g(x)dx
=f(\xi)\int_a^b g(x)dx
\stopformula

%%%%----------------------------

};

\stoptikzpicture
%% 逐步显示
%%----------------------------------------------------------------









\medskip
\FrameTitle{一元函数积分学的应用} % 解答

\StartFrame\bf\index{p197, 例11.5}
\index{柯西中值定理}
\index{变限积分}

考点: 柯西中值定理。 变限积分。 
\startitemize[n] 
  \item 设 $f(x)$, $g(x)$ 在 $[a,b]$ 上连续, 且 $g(x)$ 在 $[a,b]$ 上不变号, 

证明:  至少 $\exists$ 一点 $\xi\in(a,b)$, 使得 
\startformula
\int_a^b f(x)g(x)dx=f(\xi)\int_a^b g(x)dx
\stopformula

\stopitemize



\StopFrame

%%----------------------------------
%%----------------------------------





%%----------------------------------------------------------------
%% 逐步显示

\usemodule[tikz]         


\starttikzpicture[      scale=3,line cap=round
                        axes/.style=,         
                        important line/.style={very thick},
                        information text/.style={rounded corners,fill=red!10,inner sep=1ex} ]
        \draw[xshift=0.0cm]
                node[right,text width=17cm,information text]
                {
                        \bf
%%%%----------------------------
解:  若 $g(x)\equiv 0$, 结论显然成立。

若 $g(x)\nequiv 0$, 由于不变号, 不防设 $g(x)>0$, 令 $\displaystyle F(x)=\int_a^x f(t)g(t)dt$, 
$\displaystyle  G(x)=\int_a^x g(t)dt$, 在 $[a,b]$ 上用{\darkgreen 柯西中值定理}, 有

\startformula
\frac{F(b)-F(a)}{G(b)-G(a)}
=\frac{F'(\xi)}{G'(\xi)}
\stopformula
%%%%----------------------------

};

\stoptikzpicture
%% 逐步显示
%%----------------------------------------------------------------







%%----------------------------------------------------------------
%% 逐步显示

\usemodule[tikz]         


\starttikzpicture[      scale=3,line cap=round
                        axes/.style=,         
                        important line/.style={very thick},
                        information text/.style={rounded corners,fill=red!10,inner sep=1ex} ]
        \draw[xshift=0.0cm]
                node[right,text width=17cm,information text]
                {
                        \bf
%%%%----------------------------

即得
\startformula
\frac{\int_a^b f(x)g(x)dx-0}{\int_a^b g(x)dx-0}
=\frac{f(\xi)g(\xi)}{g(\xi)}
\stopformula
%%%%----------------------------

};

\stoptikzpicture
%% 逐步显示
%%----------------------------------------------------------------







%%----------------------------------------------------------------
%% 逐步显示

\usemodule[tikz]         


\starttikzpicture[      scale=3,line cap=round
                        axes/.style=,         
                        important line/.style={very thick},
                        information text/.style={rounded corners,fill=red!10,inner sep=1ex} ]
        \draw[xshift=0.0cm]
                node[right,text width=17cm,information text]
                {
                        \bf
%%%%----------------------------

\startformula
\int_a^b f(x)g(x)dx 
=f(\xi)\int_a^b g(x)dx, \quad \xi \in (a, b)
\stopformula
其中 $\displaystyle \int_a^b g(x)dx>0$, 命题得证。
%%%%----------------------------

};

\stoptikzpicture
%% 逐步显示
%%----------------------------------------------------------------







\medskip
\FrameTitle{一元函数积分学的应用} % 解答

\StartFrame\bf
积分不等式。用函数的单调性。
\startitemize[n] 
  \item 首先将某一限(上限或下限)变量化, 然后移项构造辅助函数, 由辅助函数的单调性来证明不等式。

适用情况:所给条件为 “ $f(x)$ 在 $[a,b]$ 上连续”。

\stopitemize



\StopFrame

%%----------------------------------
%%----------------------------------


\medskip
\FrameTitle{一元函数积分学的应用} % 解答

\StartFrame\bf\index{p198, 例11.6}
\index{辅助函数}
\index{导数}
\index{可导}
\index{单调性}

考点: 辅助函数。 导数。 单调性。 
\startitemize[n] 
  \item 设 $f(x)$ 在 $[0,1]$ 上可导, $0\leq f'(x)\leq 1$ 且 $f(0)=0$, 证明:
\startformula
\left[\int_0^1 f(x)dx\right]^2\geq \int_0^1 f^3(x)dx
\stopformula

\stopitemize



\StopFrame

%%----------------------------------
%%-----------------------------------




%%----------------------------------------------------------------
%% 逐步显示

\usemodule[tikz]         


\starttikzpicture[      scale=3,line cap=round
                        axes/.style=,         
                        important line/.style={very thick},
                        information text/.style={rounded corners,fill=red!10,inner sep=1ex} ]
        \draw[xshift=0.0cm]
                node[right,text width=17cm,information text]
                {
                        \bf
%%%%----------------------------
解: 令
\startformula
F(x)
= \left[\int_0^1 f(x)dx\right]^2 - \int_0^1 f^3(x)dx,\quad 0\leq x \leq 1,
\stopformula
则有

\startformula
F'(x)
= 2f(x)\int_0^x f(t)dt - f^3(x)
=f(x)\left[ 2\int_0^x f(t)dt - f^2(x) \right]
\stopformula
%%%%----------------------------

};

\stoptikzpicture
%% 逐步显示
%%----------------------------------------------------------------







%%----------------------------------------------------------------
%% 逐步显示

\usemodule[tikz]         


\starttikzpicture[      scale=3,line cap=round
                        axes/.style=,         
                        important line/.style={very thick},
                        information text/.style={rounded corners,fill=red!10,inner sep=1ex} ]
        \draw[xshift=0.0cm]
                node[right,text width=17cm,information text]
                {
                        \bf
%%%%----------------------------
再令
\startformula
G(x)
= 2\int_0^x f(t)dt - f^2(x)
\stopformula 
则有

\startformula
G'(x)
= 2f(x)-2f(x)f'(x)=2f(x)[1-f'(x)]\geq 0 
\stopformula
%%%%----------------------------

};

\stoptikzpicture
%% 逐步显示
%%----------------------------------------------------------------






%%----------------------------------------------------------------
%% 逐步显示

\usemodule[tikz]         


\starttikzpicture[      scale=3,line cap=round
                        axes/.style=,         
                        important line/.style={very thick},
                        information text/.style={rounded corners,fill=red!10,inner sep=1ex} ]
        \draw[xshift=0.0cm]
                node[right,text width=17cm,information text]
                {
                        \bf
%%%%----------------------------
所以, $G(x)$单调不减, 故 $G(x)\geq G(0)=0$, 从而 $F'(x)\geq 0$, 于是 $F(x)$ 单调不减, $F(1)\geq F(0)=0$, 即

\startformula
\left[ \int_0^1 f(x)dx \right]^2 \geq \int_0^1 f^3(x)dx
\stopformula

%%%%----------------------------

};

\stoptikzpicture
%% 逐步显示
%%----------------------------------------------------------------







\medskip
\FrameTitle{一元函数积分学的应用} % 解答

\StartFrame\bf\index{p198, 例11.7}
\index{辅助函数}
\index{单调性}
\index{连续}
\index{可导}

考点: 辅助函数。 单调性。 连续。 可导。
\startitemize[n] 
  \item 设 $f(x)$ 在 $[0,1]$ 上单调增加且连续,证明:
\startformula
\int_a^b xf(x)\cdot dx\geq \frac{a+b}{2}\int_a^b f(x)\cdot dx
\stopformula

\stopitemize



\StopFrame

%%----------------------------------
%%----------------------------------




%%----------------------------------------------------------------
%% 逐步显示

\usemodule[tikz]         


\starttikzpicture[      scale=3,line cap=round
                        axes/.style=,         
                        important line/.style={very thick},
                        information text/.style={rounded corners,fill=red!10,inner sep=1ex} ]
        \draw[xshift=0.0cm]
                node[right,text width=17cm,information text]
                {
                        \bf
%%%%----------------------------
解: 不等式含有参数 $a$, $b$, 将其中的参数 $b$ “变易”为变量 $t$, 构造如下辅助函数: 

\startformula
F(t)= \int_a^t x f(x)\cdot dx - \frac{a+t}{2}\int_a^x f(x)\cdot dx
\stopformula
欲使相应的积分有意义, 则需 $a\leq t \leq b$, 易知 $F(a)=0$, $F(t)$ 在  $[a,b]$ 上可导, 且

\startformula
F'(t)
= tf(t)-\frac{1}{2}\int_a^t f(x)\cdot dx-\frac{a+t}{2}\cdot f(t)
=\frac{1}{2}\int_a^t [f(t)-f(x)]\cdot dx
\stopformula
%%%%----------------------------

};

\stoptikzpicture
%% 逐步显示
%%----------------------------------------------------------------






%%----------------------------------------------------------------
%% 逐步显示

\usemodule[tikz]         


\starttikzpicture[      scale=3,line cap=round
                        axes/.style=,         
                        important line/.style={very thick},
                        information text/.style={rounded corners,fill=red!10,inner sep=1ex} ]
        \draw[xshift=0.0cm]
                node[right,text width=17cm,information text]
                {
                        \bf
%%%%----------------------------
因为 $f(x)$ 在 $[a, b]$上单调增加, 所以当 $a\leq x\leq t\leq b$时, $f(x)\leq f(t)$, 从而 $F'(t)\geq 0$。 故 $F(t)$ 在$[a,b]$ 上单调不减, $F(b)\geq F(a)=0$, 即
\startformula
\int_a^b xf(x)\cdot dx\geq \frac{a+b}{2}\int_a^b f(x)\cdot dx
\stopformula

%%%%----------------------------

};

\stoptikzpicture
%% 逐步显示
%%----------------------------------------------------------------






\medskip
\FrameTitle{一元函数积分学的应用} % 解答

\StartFrame\bf
\index{积分保号性}
\index{定积分性质}

处理被积函数。\crlf
已知 $f(x)\leq g(x)$, 用积分保号性证得\crlf
\startformula
\int_a^b f(x)dx\leq \int_a^b g(x)dx,\quad a<b
\stopformula
\StopFrame

%%----------------------------------
%%----------------------------------




\medskip
\FrameTitle{一元函数积分学的应用} % 解答

\StartFrame\bf\index{p198, 例11.8}
自己练习。 不讲。
\startitemize[n] 
  \item 设 $f(x)$ 在 $[a,b]$ 上连续, 且对任意的 $t\in [0,1]$ 以及任意的 $x_1$, $x_2\in [a,b]$ 恒满足不等式 \crlf
\startformula
f[tx_1+(1-t)x_2] \leq tf(x_1)+(1-t)f(x_2)
\stopformula
证明:\crlf
\startformula
f\left(\frac{a+b}{2}\right)\leq \frac{1}{b-a}\int_a^b f(x)dx\leq \frac{f(a)+f(b)}{2}
\stopformula

\stopitemize



\StopFrame

%%----------------------------------
%%----------------------------------


\medskip
\FrameTitle{一元函数积分学的应用} % 解答

\StartFrame\bf
\index{拉格朗日中值定理}
\index{积分保号性}

处理被积函数。\crlf
用拉格朗日中值定理。
\startitemize[n] 
  \item 用拉格朗日中值定理处理被积函数 $f(x)$, 再作不等式, 进一步, 用积分保号性。 适用于所给条件为 "$f(x)$ 一阶可导" 且题中有较简单函数值 ( 甚至为 $0$ ) 的题目。
\stopitemize



\StopFrame

%%----------------------------------
%%----------------------------------


\medskip
\FrameTitle{一元函数积分学的应用} % 解答

\StartFrame\bf\index{p198, 例11.9}
\index{拉格朗日中值定理}

考点: 拉格朗日中值定理
\startitemize[n] 
  \item 设 $f(x)$ 在 $[0,2]$ 上连续, 在 $(0,2)$ 内可导, 且 $f(0)=f(2)=1$,  $|f'(x)|\leq 1$, 证明: $\displaystyle \int_0^2 f(x)dx<3$ 
\stopitemize



\StopFrame

%%----------------------------------
%%----------------------------------





%%----------------------------------------------------------------
%% 逐步显示

\usemodule[tikz]         


\starttikzpicture[      scale=3,line cap=round
                        axes/.style=,         
                        important line/.style={very thick},
                        information text/.style={rounded corners,fill=red!10,inner sep=1ex} ]
        \draw[xshift=0.0cm]
                node[right,text width=17cm,information text]
                {
                        \bf
%%%%----------------------------
解: 任取 $x\in (0,2)$, 对 $f(x)$ 分别在区间 $[0,x]$ 与  $[x,2]$ 上应用拉格朗日中值定理, 得
\startformula
f(x)=f(0)+xf'(\xi_1), 0<\xi_1<x
\stopformula

\startformula
f(x)=f(2)+(x-2)f'(\xi_2), x<\xi_2<2
\stopformula

又因为 $|f'(x)|\leq 1$, 即 $-1\leq f'(x)\leq 1$, 且 $f(0)=f(2)=1$, 

%%%%----------------------------

};

\stoptikzpicture
%% 逐步显示
%%----------------------------------------------------------------






%%----------------------------------------------------------------
%% 逐步显示

\usemodule[tikz]         


\starttikzpicture[      scale=3,line cap=round
                        axes/.style=,         
                        important line/.style={very thick},
                        information text/.style={rounded corners,fill=red!10,inner sep=1ex} ]
        \draw[xshift=0.0cm]
                node[right,text width=17cm,information text]
                {
                        \bf
%%%%----------------------------

故由以上两式分别得到
$xf'(\xi_1)\leq x$ 和 $(x-2)f'(\xi_2)\leq 2-x$, 

所以有 $f(x)\leq 1+x$, $f(x)\leq 3-x$。

若令
\startformula
 g(x) = \startmathcases
   \NC 1+x, \NC $0\leq x \leq 1$ \NR
   \NC 3-x ,\NC $1 < x\leq 2$ \NR
\stopmathcases
\stopformula 
则 $f(x)\leq g(x)$, $x\in [0,2]$。 故
%%%%----------------------------

};

\stoptikzpicture
%% 逐步显示
%%----------------------------------------------------------------









%%----------------------------------------------------------------
%% 逐步显示

\usemodule[tikz]         


\starttikzpicture[      scale=3,line cap=round
                        axes/.style=,         
                        important line/.style={very thick},
                        information text/.style={rounded corners,fill=red!10,inner sep=1ex} ]
        \draw[xshift=0.0cm]
                node[right,text width=17cm,information text]
                {
                        \bf
%%%%----------------------------
\startformula
\int_0^2 f(x)dx 
\leq \int_0^2 g(x)dx
=3
\stopformula
但等号不成立, 因为若不然, 则有 $f(x)\equiv g(x)$ 在 $x=1$ 处不可导, 与题设矛盾。 
因此, 有
\startformula
\int_0^2 f(x)dx 
<3
\stopformula
%%%%----------------------------

};

\stoptikzpicture
%% 逐步显示
%%----------------------------------------------------------------






\medskip
\FrameTitle{一元函数积分学的应用} % 解答

\StartFrame\bf
用泰勒公式。
\startitemize[n] 
  \item 将 $f(x)$ 展开成泰勒公式, 再作不等式, 进一步, 用积分保号性。 适用于所给条件为 “ $f(x)$ 二阶( 或更高阶 )可导”, 且题目中有较简单函数值( 甚至为 $0$ )的题目。
\stopitemize



\StopFrame

%%----------------------------------
%%----------------------------------



\medskip
\FrameTitle{一元函数积分学的应用} % 解答

\StartFrame\bf\index{p200, 例11.10}
练习。
\startitemize[n] 
  \item 设 $f(x)$ 二阶可导, 且 $f''(x)\geq 0$,  $u(t)$ 为任一连续函数,  $a>0$, 
证明:\crlf
\startformula
\frac{1}{a}\int_0^a f[u(t)]dt\geq f\left[\frac{1}{a}\int_0^a u(t)dt\right]
\stopformula

\stopitemize



\StopFrame

%%----------------------------------
%%----------------------------------





%%----------------------------------------------------------------
%% 逐步显示

\usemodule[tikz]         


\starttikzpicture[      scale=3,line cap=round
                        axes/.style=,         
                        important line/.style={very thick},
                        information text/.style={rounded corners,fill=red!10,inner sep=1ex} ]
        \draw[xshift=0.0cm]
                node[right,text width=17cm,information text]
                {
                        \bf
%%%%----------------------------
解: 由于 $f''(x)\geq 0$, 则由泰勒公式, 有
\startformula
\startalign
  \NC f(x) \NC =f(x_0)+f'(x_0)(x-x_0)+\frac{1}{2}f''(\xi)(x-x_0)^2 \NR
  \NC   \NC  \geq f(x_0)+f'(x_0)(x-x_0) \NR
\stopalign
\stopformula

%%%%----------------------------

};

\stoptikzpicture
%% 逐步显示
%%----------------------------------------------------------------





%%----------------------------------------------------------------
%% 逐步显示

\usemodule[tikz]         


\starttikzpicture[      scale=3,line cap=round
                        axes/.style=,         
                        important line/.style={very thick},
                        information text/.style={rounded corners,fill=red!10,inner sep=1ex} ]
        \draw[xshift=0.0cm]
                node[right,text width=17cm,information text]
                {
                        \bf
%%%%----------------------------


取 $\displaystyle x_0=\frac{1}{a}\int_0^a u(t)dt$, $x=u(t)$, 代入上式, 有

\startformula
f[u(t)]
\geq f\left[ \frac{1}{a}\int_0^a u(t)dt \right] +f'(x_0)[u(t)-x_0]
\stopformula

%%%%----------------------------

};

\stoptikzpicture
%% 逐步显示
%%----------------------------------------------------------------






%%----------------------------------------------------------------
%% 逐步显示

\usemodule[tikz]         


\starttikzpicture[      scale=3,line cap=round
                        axes/.style=,         
                        important line/.style={very thick},
                        information text/.style={rounded corners,fill=red!10,inner sep=1ex} ]
        \draw[xshift=0.0cm]
                node[right,text width=17cm,information text]
                {
                        \bf
%%%%----------------------------
对上式两端从$0$到$a$积分, 有
\startformula
\startalign
  \NC \int_0^a f[u(t)]dt \NC \geq af\left[ \frac{1}{a}\int_0^a u(t)dt \right] + f'(x_0)\left[ \int_0^a u(t)dt-ax_0 \right] \NR
  \NC \NC = af\left[ \frac{1}{a}\int_0^a u(t)dt \right] \NR
\stopalign
\stopformula
%%%%----------------------------

};

\stoptikzpicture
%% 逐步显示
%%----------------------------------------------------------------






%%----------------------------------------------------------------
%% 逐步显示

\usemodule[tikz]         


\starttikzpicture[      scale=3,line cap=round
                        axes/.style=,         
                        important line/.style={very thick},
                        information text/.style={rounded corners,fill=red!10,inner sep=1ex} ]
        \draw[xshift=0.0cm]
                node[right,text width=17cm,information text]
                {
                        \bf
%%%%----------------------------
亦即
\startformula
\startalign
  \NC \frac{1}{a}\int_0^a f[u(t)]dt \NC \geq f\left[ \frac{1}{a}\int_0^a u(t)dt \right] \NR
\stopalign
\stopformula
%%%%----------------------------

};

\stoptikzpicture
%% 逐步显示
%%----------------------------------------------------------------





\medskip
\FrameTitle{$2019$ 数学二解答题第 $21$ 题 $11$ 分} % 解答

\StartFrame
\index{2019数学二解答题第21题11分}
\index{2019数学二第21题11分}
\index{积分中值定理}
\index{罗尔中值定理}
\index{拉格朗日中值定理}

已知函数 $f(x)$ 在 $[0,1]$ 上具有二阶导数, 且 $f(0)=0$, $f(1)=1$, $\displaystyle \int_0^1 f(x)dx=1$, 证明:
\startitemize[n] 
  \item 存在 $\xi\in (0,1)$, 使得 $f'(\xi)=0$;
  \item 存在 $\eta\in (0,1)$, 使得 $f''(\eta)<-2$。
\stopitemize



\StopFrame

%%----------------------------------
%%----------------------------------



%%----------------------------------------------------------------
%% 逐步显示

\usemodule[tikz]         


\starttikzpicture[      scale=3,line cap=round
                        axes/.style=,         
                        important line/.style={very thick},
                        information text/.style={rounded corners,fill=red!10,inner sep=1ex} ]
        \draw[xshift=0.0cm]
                node[right,text width=17cm,information text]
                {
                        \bf
%%%%----------------------------
解: $(i)$ 因为 $\displaystyle \int_0^1 f(x)dx=1$, 由{\darkgreen 积分中值定理}, 存在 $x_0\in (0,1)$, 使得 
$f(x_0)(1-0)=1$, 即 $f(x_0)=1$。

又因为 $f(1)=1$, 所以在区间 $[x_0,1]$上应用{\darkgreen 罗尔中值定理}, 可知存在 $\xi \in (x_0,1)\subset (0,1)$, 使得 $f'(\xi)=0$
%%%%----------------------------

};

\stoptikzpicture
%% 逐步显示
%%----------------------------------------------------------------




%%----------------------------------------------------------------
%% 逐步显示

\usemodule[tikz]         


\starttikzpicture[      scale=3,line cap=round
                        axes/.style=,         
                        important line/.style={very thick},
                        information text/.style={rounded corners,fill=red!10,inner sep=1ex} ]
        \draw[xshift=0.0cm]
                node[right,text width=17cm,information text]
                {
                        \bf
%%%%----------------------------
$(ii)$ 令 $g(x)=f(x)+x^2$, 问题转化为证明存在 $\eta \in (0,1)$, 使得 $g''(\eta)=f''(\eta)+2<0$。 
容易得到 $g(0)=f(0)+0=0$, $g(1)=f(1)+1=2$, $g(x_0)=1+x_0^2$。

所以分别在 $[0,x_0]$, $[x_0,1]$ 上应用{\darkgreen 拉格朗日中值定理}, 有

\startformula
g'(\xi_1)=\frac{1+x_0^2-g(0)}{x_0-0}
=\frac{1+x_0^2}{x_0},\quad \xi_1\in  (0,x_0)
\stopformula

\startformula
g'(\xi_2)
=\frac{g(1)-g(x_0)}{1-x_0}
=\frac{2-(1+x_0^2)}{1-x_0}
=\frac{1-x_0^2}{1-x_0}
=1+x_0, \xi_2 \in (x_0,1)
\stopformula
%%%%----------------------------

};

\stoptikzpicture
%% 逐步显示
%%----------------------------------------------------------------





%%----------------------------------------------------------------
%% 逐步显示

\usemodule[tikz]         


\starttikzpicture[      scale=3,line cap=round
                        axes/.style=,         
                        important line/.style={very thick},
                        information text/.style={rounded corners,fill=red!10,inner sep=1ex} ]
        \draw[xshift=0.0cm]
                node[right,text width=17cm,information text]
                {
                        \bf
%%%%----------------------------
再在 $[\xi_1, \xi_2]$ 上应用{\darkgreen 拉格朗日中值定理}, 可知存在 $\eta\in (\xi_1, \xi_2)\subset (0,1)$,使得
\startformula
g''(\eta)
=\frac{g'(\xi_2)-g'(\xi_1)}{\xi_2-\xi_1}
=\frac{1+x_0-\frac{1+x_0^2}{x_0}}{\xi_2-\xi_1}\cdot 
\frac{\frac{x_0-1}{x_0}}{\xi_2-\xi_1}
<0
\stopformula

因为 $x_0\in (0,1)$, 所以结论成立。
%%%%----------------------------

};

\stoptikzpicture
%% 逐步显示
%%----------------------------------------------------------------






\medskip
\FrameTitle{一元函数积分学的应用} % 解答

\StartFrame\bf
用放缩法。利用常见不等式关系处理被积函数,进一步用积分保号性。常见不等关系有:
\startitemize[n] 
  \item $|\sin x| \leq 1$, $|\cos |\leq 1$ 
  \item $\sin x\leq x$ , ($x\geq 0$)
  \item 闭区间上连续函数 $f(x)$ 有, $|f(x)|\leq M, (\exists M>0)$
  \item $\displaystyle \sqrt{ab}\leq \frac{a+b}{2}\leq \sqrt{\frac{a^2+b^2}{2}}, (a,b>0)$
\stopitemize



\StopFrame

%%----------------------------------
%%----------------------------------



\medskip
\FrameTitle{一元函数积分学的应用} % 解答

\StartFrame\bf\index{p201, 例11.11}
\index{变量代换}
\index{分部积分}
\index{绝对值}
\index{不等式}
\index{放大缩小}
\index{变限积分}

考点: 变量代换。 分部积分。 绝对值。 不等式。 放大缩小。 变限积分。
\startitemize[n] 
  \item 设函数 $\displaystyle f(x)=\int_x^{x+1}\sin e^t\cdot dt$, 证明:$e^x \cdot |f(x)|\leq 2$
\stopitemize



\StopFrame

%%----------------------------------
%%----------------------------------


%%----------------------------------------------------------------
%% 逐步显示

\usemodule[tikz]         


\starttikzpicture[      scale=3,line cap=round
                        axes/.style=,         
                        important line/.style={very thick},
                        information text/.style={rounded corners,fill=red!10,inner sep=1ex} ]
        \draw[xshift=0.0cm]
                node[right,text width=17cm,information text]
                {
                        \bf
%%%%----------------------------
解: 被积函数比较复杂, 无法积分, 通过变量替换可将其变得简单些 (此时积分区间的上、下限必然会变得复杂些) 然后再做下去。
%%%%----------------------------

};

\stoptikzpicture
%% 逐步显示
%%----------------------------------------------------------------






%%----------------------------------------------------------------
%% 逐步显示

\usemodule[tikz]         


\starttikzpicture[      scale=3,line cap=round
                        axes/.style=,         
                        important line/.style={very thick},
                        information text/.style={rounded corners,fill=red!10,inner sep=1ex} ]
        \draw[xshift=0.0cm]
                node[right,text width=17cm,information text]
                {
                        \bf
%%%%----------------------------


令 $\darkgreen u=e^t$, 则有
\startformula
\startalign
  \NC f(x) \NC = {\darkgreen \int_{e^x}^{e^{x+1}} \frac{1}{u}\sin u \cdot du } 
               =\left.-\frac{1}{u}\cos u\right|_{e^x}^{e^{x+1}}- \int_{e^x}^{e^{x+1}} \frac{1}{u^2}\cos u du\NR
  \NC      \NC = \frac{\cos e^x}{e^x}-\frac{\cos e^{x+1}}{ e^{x+1}}- 
                     \int_{e^x}^{e^{x+1}} \frac{1}{u^2}\cos u du\NR
\stopalign
\stopformula
%%%%----------------------------

};

\stoptikzpicture
%% 逐步显示
%%----------------------------------------------------------------







%%----------------------------------------------------------------
%% 逐步显示

\usemodule[tikz]         


\starttikzpicture[      scale=3,line cap=round
                        axes/.style=,         
                        important line/.style={very thick},
                        information text/.style={rounded corners,fill=red!10,inner sep=1ex} ]
        \draw[xshift=0.0cm]
                node[right,text width=17cm,information text]
                {
                        \bf
%%%%----------------------------

取绝对值, 故有
\startformula
\startalign
  \NC |f(x)| \NC \leq \left|\frac{\cos e^x}{e^x}\right| + \left|\frac{\cos e^{x+1}}{ e^{x+1}}\right| + 
                     \int_{e^x}^{e^{x+1}} \left| \frac{1}{u^2}\cos u \right|\cdot du \NR
  \NC \NC \leq \frac{1}{e^x} + \frac{1}{e^{x+1}} + \int_{e^x}^{e^{x+1}} \frac{1}{u^2} du  
      \leq \frac{1}{e^x} + \frac{1}{e^{x+1}} - \frac{1}{e^{x+1}} + \frac{1}{e^x} 
          = \frac{2}{e^x}  \NR
\stopalign
\stopformula
命题得证。
%%%%----------------------------

};

\stoptikzpicture
%% 逐步显示
%%----------------------------------------------------------------






\medskip
\FrameTitle{一元函数积分学的应用} % 解答

\StartFrame\bf
用分部积分法。

\startitemize[n] 
  \item 利用分部积分法处理被积分函数,再利用已知条件进一步证明。 
\stopitemize



\StopFrame

%%----------------------------------
%%----------------------------------




\medskip
\FrameTitle{一元函数积分学的应用} % 解答

\StartFrame\bf\index{p201, 例11.12}
\index{分部积分法}
\index{变限积分}
\index{绝对值}
\index{不等式}
\index{放大缩小}
\index{复合函数}
\index{积分公式}

考点: 分部积分法。 变限积分。 绝对值。 不等式。 放大缩小。 复合函数。 积分公式。

\startitemize[n] 
  \item 设 $\displaystyle f(x)=\int_x^{x+1}\sin t^2 dt$, 证明: 当$x>0$时, $|f(x)|\leq \displaystyle \frac{1}{x}$
\stopitemize



\StopFrame

%%----------------------------------
%%----------------------------------



%%----------------------------------------------------------------
%% 逐步显示

\usemodule[tikz]         


\starttikzpicture[      scale=3,line cap=round
                        axes/.style=,         
                        important line/.style={very thick},
                        information text/.style={rounded corners,fill=red!10,inner sep=1ex} ]
        \draw[xshift=0.0cm]
                node[right,text width=17cm,information text]
                {
                        \bf
%%%%----------------------------
分析: 所证不等式右端分母含有 $x$,  而左端定积分的上、下限均含有 $x$,  若让被积函数 $\sin t^2$ 进入微分符号, 则有 $\displaystyle -\frac{1}{2t}d\left(\cos t^2\right)$, 从而可以获得 $\displaystyle \frac{1}{x}$。  这是一个重要的辅助信息, 启发我们考虑先对左端的定积分进行分部积分。  因此有如下解法。

%%%%----------------------------

};

\stoptikzpicture
%% 逐步显示
%%----------------------------------------------------------------







%%----------------------------------------------------------------
%% 逐步显示

\usemodule[tikz]         


\starttikzpicture[      scale=3,line cap=round
                        axes/.style=,         
                        important line/.style={very thick},
                        information text/.style={rounded corners,fill=red!10,inner sep=1ex} ]
        \draw[xshift=0.0cm]
                node[right,text width=17cm,information text]
                {
                        \bf
%%%%----------------------------
解:因为
\startformula
\startalign
  \NC f(x) \NC = \int_x^{x+1}\sin t^2 dt 
    =   -\left.\frac{\cos t^2}{2t}\right|_x^{x+1} + \frac{1}{2}\int_x^{x+1}\cos t^2 d\left(\frac{1}{t}\right) \NR
  \NC \NC = \frac{1}{2}\left[ \frac{\cos x^2}{x} - \frac{\cos (x+1)^2}{x+1} \right] 
           - \frac{1}{2} \int_x^{x+1}\frac{\cos t^2}{t^2}dt\NR
\stopalign
\stopformula


%%%%----------------------------

};

\stoptikzpicture
%% 逐步显示
%%----------------------------------------------------------------










%%----------------------------------------------------------------
%% 逐步显示

\usemodule[tikz]         


\starttikzpicture[      scale=3,line cap=round
                        axes/.style=,         
                        important line/.style={very thick},
                        information text/.style={rounded corners,fill=red!10,inner sep=1ex} ]
        \draw[xshift=0.0cm]
                node[right,text width=17cm,information text]
                {
                        \bf
%%%%----------------------------

所以

\startformula
\startalign
  \NC |f(x)| \NC \leq \frac{1}{2}\left( \frac{1}{x}+  \frac{1}{x+1}\right) + \frac{1}{2}\int_x^{x+1} \frac{1}{t^2}dt \NR
  \NC \NC = \frac{1}{2}\left( \frac{1}{x}+  \frac{1}{x+1}\right)+ \frac{1}{2}\left( \frac{1}{x} -  \frac{1}{x+1}\right)
         = \frac{1}{x} \NR
\stopalign
\stopformula
%%%%----------------------------

};

\stoptikzpicture
%% 逐步显示
%%----------------------------------------------------------------








\medskip
\FrameTitle{一元函数积分学的应用} % 解答

\StartFrame\bf
用换元法。

\startitemize[n] 
  \item 见到复合函数的积分,考虑换元法。
\stopitemize



\StopFrame

%%----------------------------------
%%----------------------------------




\medskip
\FrameTitle{一元函数积分学的应用} % 解答

\StartFrame\bf\index{p201, 例11.13}
\index{换元法}
\index{定积分性质}
\index{反函数}
\index{反函数求导}
\index{变量代换}
\index{绝对值}
\index{放大缩小}
\index{严格单调}
\index{不等式}
\index{被积函数}
\index{抽象函数}

考点: 换元法。 定积分性质。 反函数。 反函数求导。 变量代换。 绝对值。 放大缩小。 严格单调。 不等式。 被积函数。 抽象函数。

\startitemize[n] 
  \item 设 $|f(x)|\leq \pi$, $f'(x)\geq m >0$, $(a\leq x \leq b)$, \crlf
证明: 
$\left\| \displaystyle \int_a^b \sin f(x) \cdot dx \right\| \leq \displaystyle \frac{2}{m}$
\stopitemize



\StopFrame

%%----------------------------------
%%----------------------------------



%%----------------------------------------------------------------
%% 逐步显示

\usemodule[tikz]         


\starttikzpicture[      scale=3,line cap=round
                        axes/.style=,         
                        important line/.style={very thick},
                        information text/.style={rounded corners,fill=red!10,inner sep=1ex} ]
        \draw[xshift=0.0cm]
                node[right,text width=17cm,information text]
                {
                        \bf
%%%%----------------------------
证明: 当 $a\leq x \leq b$ 时,  $f'(x)>0$,  $f(x)$ 在 $[a, b]$ 上严格单调增加, 故其存在反函数。 记 $t=f(x)$, 反函数记 $x=g(t)$, 又记 $\alpha=f(a)$, $\beta = f(b)$, 由 $|f(x)|\leq \pi$, 则 $-\pi \leq \alpha \leq \beta \leq \pi$。


%%%%----------------------------

};

\stoptikzpicture
%% 逐步显示
%%----------------------------------------------------------------







%%----------------------------------------------------------------
%% 逐步显示

\usemodule[tikz]         


\starttikzpicture[      scale=3,line cap=round
                        axes/.style=,         
                        important line/.style={very thick},
                        information text/.style={rounded corners,fill=red!10,inner sep=1ex} ]
        \draw[xshift=0.0cm]
                node[right,text width=17cm,information text]
                {
                        \bf
%%%%----------------------------
故 
\startformula
\int_a^b \sin f(x)\cdot dx = \int_{\alpha}^{\beta} \sin t \cdot g'(t)\cdot dt
\stopformula

其中 $f'(x)\geq m >0$。



%%%%----------------------------

};

\stoptikzpicture
%% 逐步显示
%%----------------------------------------------------------------








%%----------------------------------------------------------------
%% 逐步显示

\usemodule[tikz]         


\starttikzpicture[      scale=3,line cap=round
                        axes/.style=,         
                        important line/.style={very thick},
                        information text/.style={rounded corners,fill=red!10,inner sep=1ex} ]
        \draw[xshift=0.0cm]
                node[right,text width=17cm,information text]
                {
                        \bf
%%%%----------------------------


故 $0< g'(t)=\displaystyle \frac{1}{f'(x)}\leq \frac{1}{m}$, 则

\startformula
\startalign
  \NC \left|\int_a^b \sin f(x) dx\right| \NC = \left|\int_{\alpha}^{\beta} \sin t \cdot g'(t)dt\right| \NR
  \NC  \NC \leq \left | \int_0^{\pi} \sin t \cdot g'(t)dt\right|
           \leq \frac{1}{m}\int_0^{\pi} \sin t dt =\frac{2}{m} \NR
\stopalign
\stopformula

%%%%----------------------------

};

\stoptikzpicture
%% 逐步显示
%%----------------------------------------------------------------








\medskip
\FrameTitle{一元函数积分学的应用} % 解答


\StartFrame\bf\index{p202, 例11.14}
\index{数列}
\index{积分型数列}
\index{单调性}
\index{极限}
\index{夹逼准则}
\index{放大缩小}
\index{三角恒等式}
\index{递推公式}
\index{积分公式}
\index{幂函数性质}
\index{有理分式函数的极限公式}

用夹逼准则求解积分极限。

考点: 数列。 积分型数列。 单调性。 极限。 夹逼准则。 放大缩小。 三角恒等式。 递推公式。 积分公式。 幂函数性质。 有理分式函数的极限公式。
\startitemize[n] 
  \item 计算 $I = \displaystyle \lim_{n\rightarrow\infty} \int_0^{\frac{\pi}{4}}\tan ^nxdx$
\stopitemize



\StopFrame

%%----------------------------------
%%----------------------------------




%%----------------------------------------------------------------
%% 逐步显示

\usemodule[tikz]         


\starttikzpicture[      scale=3,line cap=round
                        axes/.style=,         
                        important line/.style={very thick},
                        information text/.style={rounded corners,fill=red!10,inner sep=1ex} ]
        \draw[xshift=0.0cm]
                node[right,text width=17cm,information text]
                {
                        \bf
%%%%----------------------------
{\darkgreen 分析}: 此题的关键是建立 $f(n)=\displaystyle \int_0^{\frac{\pi}{4}}\tan ^nxdx$ 的递推公式, 利用此递推公式来解题。


%%%%----------------------------

};

\stoptikzpicture
%% 逐步显示
%%----------------------------------------------------------------





%%----------------------------------
%%----------------------------------


%%----------------------------------------------------------------
%% 逐步显示

\usemodule[tikz]         


\starttikzpicture[      scale=3,line cap=round
                        axes/.style=,         
                        important line/.style={very thick},
                        information text/.style={rounded corners,fill=red!10,inner sep=1ex} ]
        \draw[xshift=0.0cm]
                node[right,text width=17cm,information text]
                {
                        \bf
%%%%----------------------------

解: 令 $f(n)=\displaystyle \int_0^{\frac{\pi}{4}}\tan ^nx \cdot dx$, 则

\startformula
\startalign
  \NC f(n)+f(n+2) \NC = \int_0^{\frac{\pi}{4}}\tan ^nx \cdot (1+\tan ^2x) \cdot dx \NR
  \NC \NC = \left.\frac{1}{1+n} \cdot \tan ^{n+1}x\right|_0^{\frac{\pi}{4}}
         = \frac{1}{1+n} \NR
\stopalign
\stopformula
%%%%----------------------------

};

\stoptikzpicture
%% 逐步显示
%%----------------------------------------------------------------







%%----------------------------------------------------------------
%% 逐步显示

\usemodule[tikz]         


\starttikzpicture[      scale=3,line cap=round
                        axes/.style=,         
                        important line/.style={very thick},
                        information text/.style={rounded corners,fill=red!10,inner sep=1ex} ]
        \draw[xshift=0.0cm]
                node[right,text width=17cm,information text]
                {
                        \bf
%%%%----------------------------

因为当 $0\leq x \leq \displaystyle \frac{\pi}{4}$ 时, 有 $\tan^{n+2}x \leq \tan^{n}x \leq \tan^{n-2}x$, 所以

$f(n+2)\leq f(n)\leq f(n-2)$, 

于是

\startformula
\startalign
  \NC \frac{1}{1+n} \NC = f(n)+f(n+2) \leq 2f(n) \leq f(n-2)+f(n) =\frac{1}{n-1} \NR
\stopalign
\stopformula


%%%%----------------------------

};

\stoptikzpicture
%% 逐步显示
%%----------------------------------------------------------------








%%----------------------------------------------------------------
%% 逐步显示

\usemodule[tikz]         


\starttikzpicture[      scale=3,line cap=round
                        axes/.style=,         
                        important line/.style={very thick},
                        information text/.style={rounded corners,fill=red!10,inner sep=1ex} ]
        \draw[xshift=0.0cm]
                node[right,text width=17cm,information text]
                {
                        \bf
%%%%----------------------------

故有
$\displaystyle \frac{n}{2(1+n)} \leq nf(n) \leq \frac{n}{2(n-1)}$ , 

利用夹逼准则, 即得
$\displaystyle I =  \lim_{n\rightarrow\infty}f(n) =  \lim_{n\rightarrow\infty} \int_0^{\frac{\pi}{4}}\tan ^nxdx = \frac{1}{2}$。
%%%%----------------------------

};

\stoptikzpicture
%% 逐步显示
%%----------------------------------------------------------------







\medskip
\FrameTitle{一元函数积分学的应用} % 解答
\index{连续化}
\index{离散化}
\index{曲边梯形面积}
\StartFrame\bf
曲边梯形面积的连续化与离散化问题。


\StopFrame

%%----------------------------------
%%----------------------------------




\medskip
\FrameTitle{一元函数积分学的应用} % 解答
\index{连续化}
\index{离散化}
\index{曲边梯形面积}
\index{放大缩小}
\index{取整函数}

\StartFrame\bf
曲边梯形面积的连续化与离散化问题。

\startitemize[n] 
  \item 计算 $\displaystyle \left[\sum_{n=1}^{100}\frac{1}{\sqrt{n}}\right]$, 其中$[\cdot]$为取整函数。
\stopitemize



\StopFrame

%%----------------------------------
%%----------------------------------




\medskip
\FrameTitle{一元函数积分学的物理应用(微元法)(仅数一、数二)} % 解答
\index{总路程}
\index{物理应用}
\index{微元法} 

\StartFrame\bf
总路程。

\startitemize[n] 
  \item $\displaystyle S=\int_{t_1}^{t_2}v(t)dt$, 其中 $v(t)$ 为时间 $t_1$ 到 $t_2$ 上的速度函数,积分即得总位移(路程) $S$.
\stopitemize



\StopFrame

%%----------------------------------
%%----------------------------------




\medskip
\FrameTitle{一元函数积分学的物理应用(微元法)(仅数一、数二)} % 解答
\index{总路程}
\index{物理应用}
\index{微元法} 

\StartFrame\bf\index{p208, 例12.1}
\index{总路程}

考点: 总路程。

\startitemize[n] 
  \item 质点以速度 $t\cdot \sin t^2$ 米/秒作直线运动, 则从时刻 $t_1=\sqrt{\displaystyle \frac{\pi}{2}}$ 秒到 $t_2=\sqrt{\pi}$ 秒内质点所经过的路程等于 \text{__________}米。
\stopitemize



\StopFrame

%%----------------------------------
%%----------------------------------





%%----------------------------------------------------------------
%% 逐步显示

\usemodule[tikz]         


\starttikzpicture[      scale=3,line cap=round
                        axes/.style=,         
                        important line/.style={very thick},
                        information text/.style={rounded corners,fill=red!10,inner sep=1ex} ]
        \draw[xshift=0.0cm]
                node[right,text width=17cm,information text]
                {
                        \bf
%%%%----------------------------
解: 质点所经过的总路程是

\startformula
\startalign
  \NC S \NC = \int_{\sqrt{\frac{\pi}{2}}}^{\sqrt{\pi}} t\cdot \sin t^2 \cdot dt
            = \frac{1}{2} \int_{\sqrt{\frac{\pi}{2}}}^{\sqrt{\pi}}  \sin t^2 \cdot d(t^2) 
      = \left. -\frac{1}{2} \cos t^2 \right|_{\sqrt{\frac{\pi}{2}}}^{\sqrt{\pi}} 
            = \frac{1}{2}  \NR
\stopalign
\stopformula
%%%%----------------------------

};

\stoptikzpicture
%% 逐步显示
%%----------------------------------------------------------------







\medskip
\FrameTitle{一元函数积分学的物理应用(微元法)(仅数一、数二)} % 解答
\index{变力做功}
\index{物理应用}
\index{微元法} 

\StartFrame\bf
变力沿直线做功。

\startitemize[n] 
  \item 设方向沿 $x$ 轴正向的力函数为 $F(x)$, $(a\leq x \leq b)$, 则物体沿 $x$ 轴从点 $a$ 移动到点 $b$ 时, 变力
$F(x)$ 所做的功为
$W=\displaystyle \int_a^b F(x)dx$, 功的元素 $dW=F(x)dx$。
  \item 变力关于路程的定积分就是功。
\stopitemize



\StopFrame

%%----------------------------------
%%----------------------------------




\medskip
\FrameTitle{一元函数积分学的物理应用(微元法)(仅数一、数二)} % 解答
\index{抽水做功}
\index{物理应用}
\index{微元法} 

\StartFrame\bf
提取物体做功。

\startitemize[n] 
  \item 将容器中的水全部抽出所做的功为 $W=\rho g\displaystyle \int_a^b xA(x)dx$, 其中 $\rho$ 为水的密度, $g$ 为重力加速度。
  \item 功的元素 $dW=\rho g x A(x)dx$ 为位于 $x$ 处厚度为 $dx$,水平截面面积为 $A(x)$ 的一层水被抽出(路程为 $x$)所做的功。
  \item 抽水做功的特点: 力(重力)不变, 路程在变。
  \item 求解的关键是, 确定 $x$ 处的水平截面面积 $A(x)$, 其余的量都是固定的。
\stopitemize



\StopFrame

%%----------------------------------
%%----------------------------------




\medskip
\FrameTitle{一元函数积分学的物理应用(微元法)(仅数一、数二)} % 解答
\index{抽水做功}
\index{水中提取重物做功}
\index{物理应用}
\index{微元法} 

\StartFrame\bf\index{p212, 例12.5}
\index{做功}
\index{提取物体做功}
\index{球}

考点: 提取物体做功。 做功。 物理应用。 微元法。 球。

\startitemize[n] 
  \item 半径为 $1$ 的球沉入水中, 球的上顶与水平面齐平, 球与水的密度相同, 记为 $\rho$, 重力加速度记为 $g$, 现将球打捞出水, 至少需做多少功?
\stopitemize



\StopFrame

%%----------------------------------
%%----------------------------------




%%----------------------------------------------------------------
%% 逐步显示

\usemodule[tikz]         


\starttikzpicture[      scale=3,line cap=round
                        axes/.style=,         
                        important line/.style={very thick},
                        information text/.style={rounded corners,fill=red!10,inner sep=1ex} ]
        \draw[xshift=0.0cm]
                node[right,text width=17cm,information text]
                {
                        \bf
%%%%----------------------------
解: 水平向右 $x$ 轴正向, 竖直向上 $y$ 轴正向, 建立坐标系, 则边界方程为 $x^2+(y+1)^2=1$。


%%%%----------------------------

};

\stoptikzpicture
%% 逐步显示
%%----------------------------------------------------------------






%%----------------------------------------------------------------
%% 逐步显示

\usemodule[tikz]         


\starttikzpicture[      scale=3,line cap=round
                        axes/.style=,         
                        important line/.style={very thick},
                        information text/.style={rounded corners,fill=red!10,inner sep=1ex} ]
        \draw[xshift=0.0cm]
                node[right,text width=17cm,information text]
                {
                        \bf
%%%%----------------------------

由于球与水的密度相同, 打捞时, 球在水下的行程不做功, 球出水后, 阴影部分 $[y, y+dy]$ 的做功微元为
\startformula
dW= (y+2)\rho g \pi x^2 dy
= (y+2)\rho g \pi [1-(y+1)^2]dy
\stopformula


%%%%----------------------------

};

\stoptikzpicture
%% 逐步显示
%%----------------------------------------------------------------








%%----------------------------------------------------------------
%% 逐步显示

\usemodule[tikz]         


\starttikzpicture[      scale=3,line cap=round
                        axes/.style=,         
                        important line/.style={very thick},
                        information text/.style={rounded corners,fill=red!10,inner sep=1ex} ]
        \draw[xshift=0.0cm]
                node[right,text width=17cm,information text]
                {
                        \bf
%%%%----------------------------

故总功为
\startformula
\startalign
  \NC W \NC = \int_{-2}^0 (y+2)\rho g \pi [1-(y+1)^2]dy  
        = \rho g \pi \int_{-2}^0 (2+y)(-y^2-2y)dy \NR
  \NC   \NC = \rho g \pi \left.\left( -\frac{1}{4}y^4-\frac{4}{3}y^3 - 2y^2 \right)\right|_{-2}^0  
        = \rho g \pi \left( 12-\frac{32}{3} \right) 
            = \frac{4}{3}\rho g \pi\NR
\stopalign
\stopformula
%%%%----------------------------

};

\stoptikzpicture
%% 逐步显示
%%----------------------------------------------------------------







\medskip
\FrameTitle{一元函数积分学的物理应用(微元法)(仅数一、数二)} % 解答
\index{静水压力}
\index{侧压力}
\index{物理应用}
\index{微元法} 

\StartFrame\bf
静水压力。

\startitemize[n] 
  \item 垂直浸没在水中的平板的一侧受到的水压力为 \crlf
$P=\rho g \displaystyle \int_a^b x[f(x)-h(x)]dx$, 其中 $\rho$ 为水的密度, $g$ 为重力加速度。
  \item 压力元素 \crlf
$dP=\rho g\displaystyle  \int_a^b [f(x)-h(x)]dx$ 是平板中矩形条所受到的压力, $x$ 表示水深,  $f(x)-h(x)$ 是矩形条的宽度, $dx$ 是矩形条的高度。
  \item 水压力问题的特点:压强随水的深度而改变。
  \item 求解的关键是确定水深 $x$ 处的平板的宽度 $f(x)-h(x)$。 
\stopitemize



\StopFrame

%%----------------------------------
%%----------------------------------




\medskip
\FrameTitle{2020数学二填空题第12题4分} % 解答
\index{2020数学二填空题第12题4分}
\index{2020数学二第12题4分}
\index{静水压力}
\index{侧压力}
\index{物理应用}
\index{微元法} 
\index{等腰直角三角形}

\StartFrame\bf
静水压力。 侧压力。 物理应用。 微元法。 等腰直角三角形。

\startitemize[n] 
  \item 斜边长为 $2a$ 的等腰直角三角形平板铅直地沉没在水中, 且斜边与水面相齐, 记重力加速度为 $g$, 
水密度为 $\rho$, 则三角形平板的一侧受到的压力为 \text{______________}
\stopitemize



\StopFrame

%%----------------------------------
%%----------------------------------







%%----------------------------------------------------------------
%% 逐步显示

\usemodule[tikz]         


\starttikzpicture[      scale=3,line cap=round
                        axes/.style=,         
                        important line/.style={very thick},
                        information text/.style={rounded corners,fill=red!10,inner sep=1ex} ]
        \draw[xshift=0.0cm]
                node[right,text width=17cm,information text]
                {
                        \bf
%%%%----------------------------
解: 三角形平板的一侧受到的压力为

\startformula
\startalign
  \NC F \NC = \int_0^a 2\rho g (a-y)y dy 
      = 2 \rho g  \int_0^a (ay-y^2)dy \NR
  \NC   \NC = 2 \rho g  \left( \frac{1}{2}a^3-\frac{1}{3}a^3 \right) 
            = \frac{1}{3}\rho g a^3\NR
\stopalign
\stopformula
%%%%----------------------------

};

\stoptikzpicture
%% 逐步显示
%%----------------------------------------------------------------








\medskip
\FrameTitle{一元函数积分学的物理应用(微元法)(仅数一、数二)} % 解答
\index{细杆质心}
\index{物理应用}
\index{微元法} 

\StartFrame\bf
细杆质心。

\startitemize[n] 
  \item 设直线段上的线密度为 $\rho (x)$ 的细直杆,则其质心为
\startformula
\bar{x}= \frac{\displaystyle\int_a^b x\rho (x)dx}{\displaystyle\int_a^b \rho (x)dx}
\stopformula

\stopitemize



\StopFrame

%%----------------------------------
%%----------------------------------





\medskip
\FrameTitle{一元函数积分学的物理应用(微元法)(仅数一、数二)} % 解答
\index{细杆质心}
\index{物理应用}
\index{微元法} 

\StartFrame\bf\index{p214, 例12.8}
\index{细杆质心}

考点: 质心。
\startitemize[n] 
  \item 一根长度为1的细杆位于 x 轴的区间 $[0,1]$ 上, 若其线密度 $\rho(x)=-x^2+2x+1$, 则该细杆的质心坐标 $\bar{x} = $ \text{____________}
\stopitemize



\StopFrame

%%----------------------------------
%%----------------------------------





%%----------------------------------------------------------------
%% 逐步显示

\usemodule[tikz]         


\starttikzpicture[      scale=3,line cap=round
                        axes/.style=,         
                        important line/.style={very thick},
                        information text/.style={rounded corners,fill=red!10,inner sep=1ex} ]
        \draw[xshift=0.0cm]
                node[right,text width=17cm,information text]
                {
                        \bf
%%%%----------------------------
解: 质心坐标为
\startformula
\startalign
  \NC \bar{x} \NC = \frac{\displaystyle\int_0^1 x\rho (x)dx}{\displaystyle\int_0^1 \rho (x)dx}  
           = \frac{\displaystyle\int_0^1 (-x^3+2x^2+x)dx}{\displaystyle\int_0^1 (-x^2+2x+1)dx} 
           = \frac{11}{20} \NR
\stopalign
\stopformula
%%%%----------------------------

};

\stoptikzpicture
%% 逐步显示
%%----------------------------------------------------------------






%%------------------------
\usemodule[chart]
\setupFLOWcharts[
height=2.2\lineheight,
width=6\bodyfontsize,
dx=1\bodyfontsize,
dy=0.3\bodyfontsize,
]

\setupFLOWshapes
[framecolor=pragmacolor,
background=color,
backgroundcolor=white,
]

\setupFLOWlines[framecolor=pragmacolor]

\startFLOWchart[example]

\startFLOWcell 
  \name {01}
  \location {0,4}
  \text {二重积分}
  \connection [rl] {11}
  \connection [rl] {12}
  \connection [rl] {14}
\stopFLOWcell

\startFLOWcell
  \name {11}
  \location{2,1}
  \text {概念}
  \connection [rl] {31}
  \connection [rl] {32}
  \connection [rl] {33}
  \connection [rl] {34}
\stopFLOWcell

\startFLOWcell
  \name {12}
  \location{2,5}
  \text {计算}
  \connection [rl] {35}
  \connection [rl] {36}
\stopFLOWcell

\startFLOWcell
  \name {14}
  \location{2,7}
  \text {应用}
  \connection [rl] {37}
\stopFLOWcell

\startFLOWcell
  \name {31}
  \location{3,1}
  \text {和式极限}
\stopFLOWcell

\startFLOWcell
  \name {32}
  \location{3,2}
  \text {对称性}
  \connection [rl] {41}
  \connection [rl] {42}
\stopFLOWcell

\startFLOWcell
  \name {33}
  \location{3,3}
  \text {大小比较}
  \connection [rl] {43}
  \connection [rl] {44}
\stopFLOWcell

\startFLOWcell
  \name {34}
  \location{3,4}
  \text {周期性}
\stopFLOWcell

\startFLOWcell
  \name {35}
  \location{3,5}
  \text {交换积分次序}
  \connection [rl] {45}
  \connection [rl] {46}
\stopFLOWcell

\startFLOWcell
  \name {36}
  \location{3,6}
  \text {直极互化}
\stopFLOWcell

\startFLOWcell
  \name {37}
  \location{3,7}
  \text {面积}
\stopFLOWcell

\startFLOWcell
  \name {41}
  \location{4,1}
  \text {普通}
\stopFLOWcell

\startFLOWcell
  \name {42}
  \location{4,2}
  \text {轮换}
\stopFLOWcell

\startFLOWcell
  \name {43}
  \location{4,3}
  \text {对称性}
\stopFLOWcell

\startFLOWcell
  \name {44}
  \location{4,4}
  \text {保号性}
\stopFLOWcell

\startFLOWcell
  \name {45}
  \location{4,5}
  \text {直角坐标}
\stopFLOWcell

\startFLOWcell
  \name {46}
  \location{4,6}
  \text {极坐标}
\stopFLOWcell
\stopFLOWchart
\FLOWchart[example]

%%------------------------

\Topic{03 二重积分的概念}
\index{03 二重积分的概念}
1. 二重积分的几何背景就是曲顶柱体的体积。

2. 用“分割、近似、求和、取极限”的方法求出“曲顶柱体的体积”,这就是二重积分
\startformula
\iint_D f(x,y) d\sigma
\stopformula


\Topic{04 题型1. 二重积分的概念之和式极限}
\index{04 题型1. 二重积分的概念之和式极限}
\index{题型1. 二重积分的概念之和式极限}


\startformula
\iint_D f(x,y) d\sigma =\lim_{n\rightarrow \infty}\sum_{i=1}^n\sum_{j=1}^n
f\left(a+\frac{b-a}{n}i,c+\frac{d-c}{n}j\right)\cdot \frac{b-a}{n}\cdot \frac{d-c}{n}
\stopformula
{\bf 注意:} 这里的 $D$ 不是一般的平面有界闭区域, 而是一个 “长方形区域 $[a,b]\times [c,d]$ ”。

%%
%%%%%%%%%%% %%%%%%%%%%%%%%%%%%%%%%%%%%
% 内容分割线 2020.07.16
%%%%%%%%%%% %%%%%%%%%%%%%%%%%%%%%%%%%%
%%

\Topic{06 例 01}
\index{06 例 01}
\startformula
\lim_{n\rightarrow\infty}\sum_{i=1}^n\sum_{j=1}^n\frac{n}{(n+i)(n^2+j^2)}= ( \qquad)
\stopformula 

\medskip

\startformula
\startalign
 \NC (A) \int_0^1dx\int_0^x\frac{1}{(1+x)(1+y^2)}dy \NC \qquad (B) \int_0^1dx\int_0^x\frac{1}{(1+x)(1+y)}dy \NR
 \NC (C) \int_0^1dx\int_0^1\frac{1}{(1+x)(1+y)}dy \NC \qquad (D) \int_0^1dx\int_0^1\frac{1}{(1+x)(1+y^2)}dy \NR
\stopalign
\stopformula

{\bf 考点:}二重积分的概念与将和式转化为积分和的方法。 二重积分的和式。 二重积分的概念。
\page

解答过程:
设 $D=\{(x,y)|0\leq x\leq 1, 0\leq y \leq 1\}$, 记 $f(x,y)=\displaystyle\frac{1}{(1+x)(1+y^2)}$. 
用直线 $x=x_i=\displaystyle\frac{i}{n}$, $(i=0,1,2,\cdots,n)$ 与 $y=y_j=\displaystyle\frac{j}{n}$, $(j=0,1,2,\cdots,n)$ 将 $D$ 分成 $n^2$
等份,和式
\startformula
\startalign
 \NC \sum_{i=1}^n\sum_{j=1}^n\frac{1}{(1+x_i)(1+y_j^2)}\cdot\frac{1}{n^2} \NC = \sum_{i=1}^n\sum_{j=1}^n\frac{1}{\left(1+\frac{i}{n}\right)\left(1+\frac{j^2}{n^2}\right)}\cdot\frac{1}{n^2} \NR
 \NC    \NC = \sum_{i=1}^n\sum_{j=1}^n\frac{n}{(n+i)\left(n^2+j^2\right)} \NR
\stopalign
\stopformula
是函数 $f(x,y)$ 在 $D$ 上的一个二重积分的和式,所以

\startformula
 \NC \text{原式} \NC = \iint_D \frac{1}{(1+x)(1+y^2)}dxdy = \int_0^1dx\int_0^1\frac{1}{(1+x)(1+y^2)}dy
\stopformula
故选 $(D)$, 最后结果为 $\displaystyle\frac{\pi}{4}\ln 2$. 

\hfill 例01完毕。


%%----------------------------------------------------------------
%% 逐步显示
\page
\usemodule[tikz]         


\starttikzpicture[      scale=3,line cap=round
                        axes/.style=,         
                        important line/.style={very thick},
                        information text/.style={rounded corners,fill=red!10,inner sep=1ex} ]
        \draw[xshift=0.0cm]
                node[right,text width=17cm,information text]
                {
                        \bfd 练习题: {\darkgreen 计算}

$\displaystyle\int_0^1dx\int_0^1\frac{1}{(1+x)(1+y^2)}dy$
                };

\stoptikzpicture

%%----------------------------------------------------------------
\page

\Topic{07 题型2. 二重积分的概念之普通对称性}
\index{07 题型2. 二重积分的概念之普通对称性}
\index{题型2. 二重积分的概念之普通对称性}

(1). 若 $D$ 关于 $y$ 轴对称, 则
\startformula
 \iint_D f(x,y)d\sigma = \startmathcases
   \NC 2\displaystyle \iint_{D_1}f(x,y)d\sigma, \NC 当 $f(-x,y)=f(x,y)$ 时 \NR
   \NC 0 ,\NC 当 $f(-x,y)=-f(x,y)$ 时 \NR
\stopmathcases
\stopformula 
其中 $D_1$ 是 $D$ 在 $y$ 轴左或右侧的部分。

\blackrule[color=black,width=\textwidth,height=.01cm,depth=0cm]

(2). 若 $D$ 关于 $x$ 轴对称, 则
\startformula
 \iint_D f(x,y)d\sigma = \startmathcases
   \NC 2\displaystyle \iint_{D_1}f(x,y)d\sigma, \NC 当 $f(x,-y)=f(x,y)$ 时 \NR
   \NC 0 ,\NC 当 $f(x,-y)=-f(x,y)$ 时 \NR
\stopmathcases
\stopformula 
其中 $D_1$ 是 $D$ 在 $x$ 轴上或下侧的部分。

\page
 
(3). 若 $D$ 关于原点对称, 则
\startformula
 \iint_D f(x,y)d\sigma = \startmathcases
   \NC 2\displaystyle \iint_{D_1}f(x,y)d\sigma, \NC 当 $f(-x,-y)=f(x,y)$ 时 \NR
   \NC 0 ,\NC 当 $f(-x,-y)=-f(x,y)$ 时 \NR
\stopmathcases
\stopformula 
其中 $D_1$ 是 $D$ 关于原点对称的半个部分。

\blackrule[color=black,width=\textwidth,height=.01cm,depth=0cm]

(4). 若 $D$ 关于 $y=x$ 对称, 则
\startformula
 \iint_D f(x,y)d\sigma = \startmathcases
   \NC 2\displaystyle \iint_{D_1}f(x,y)d\sigma, \NC 当 $f(x,y)=f(y,x)$ 时 \NR
   \NC 0 ,\NC 当 $f(x,y)=-f(y,x)$ 时 \NR
\stopmathcases
\stopformula 
其中 $D_1$ 是 $D$ 关于 $y=x$ 对称的半个部分。

\page
 
(5). 若 $D$ 关于 $y=a(\neq 0)$ 对称, 则
\startformula
 \iint_D f(x,y)d\sigma = \startmathcases
   \NC 2\displaystyle \iint_{D_1}f(x,y)d\sigma, \NC 当 $f(x,2a-y)=f(x,y)$ 时 \NR
   \NC 0 ,\NC 当 $f(x,2a-y)=-f(x,y)$ 时 \NR
\stopmathcases
\stopformula 
其中 $D_1$ 是 $D$ 在 $y=a$ 上或下侧的部分。

\blackrule[color=black,width=\textwidth,height=.01cm,depth=0cm]

(6). 若 $D$ 关于 $x=a(\neq 0)$ 对称, 则
\startformula
 \iint_D f(x,y)d\sigma = \startmathcases
   \NC 2\displaystyle \iint_{D_1}f(x,y)d\sigma, \NC 当 $f(2a-x,y)=f(x,y)$ 时 \NR
   \NC 0 ,\NC 当 $f(2a-x,y)=-f(x,y)$ 时 \NR
\stopmathcases
\stopformula 
其中 $D_1$ 是 $D$ 在 $x=a$ 左或右侧的部分。

\Topic{08 题型3. 二重积分的概念之轮换对称性}
\index{08 题型3. 二重积分的概念之轮换对称性}
\index{题型3. 二重积分的概念之轮换对称性}

若将 $D$ 中的 $x$, $y$ 对调后, $D$ 不变, 则有
\startformula
I=\iint_Df(x,y)dx\,dy=\iint_Df(y,x)dx\,dy
\stopformula

{\bf 注意:}

1. 若 $f(x,y)+f(y,x) =a$, 则
\startformula
I=\frac{1}{2}\iint_D[f(x,y)+f(y,x)]dx\,dy=\frac{1}{2}\iint_Dadx\,dy=\frac{a}{2}\cdot S_D
\stopformula

2. 若 $f(x,y)+f(y,x) >a$, 则
\startformula
I=\frac{1}{2}\iint_D[f(x,y)+f(y,x)]dx\,dy > \frac{1}{2}\iint_Dadx\,dy=\frac{a}{2}\cdot S_D
\stopformula

\Topic{09 题型4. 二重积分的概念之二重积分比大小}
\index{09 题型4. 二重积分的概念之二重积分比大小}
\index{题型4. 二重积分的概念之二重积分比大小}
(1) 用对称性。

(2) 用保号性。

%%
%%%%%%%%%%% %%%%%%%%%%%%%%%%%%%%%%%%%%
% 内容分割线 2020.07.16
%%%%%%%%%%% %%%%%%%%%%%%%%%%%%%%%%%%%%
%%

\Topic{10 例 02}
\index{10 例 02}
设 $D_1 = \left\{ (x,y) | 0\leq x\leq 1, 0\leq y\leq 1\right\}$, \,
$D_2 = \left\{ (x,y) | 0\leq x\leq 1, 0\leq y\leq \sqrt{x}\right\}$, \,
$D_3 = \left\{ (x,y) | 0\leq x\leq 1, x^2\leq y\leq 1\right\}$,且
\startformula
J_i=\iint_{D_i}\sqrt[3]{x-y}\cdot dxdy\qquad (i=1,2,3)
\stopformula
则 ( \kern2em).

\medskip

\startformula
(A)\, J_1<J_2<J_3 \quad (B)\, J_3<J_1<J_2 \quad (C)\, J_2<J_3<J_1  \quad (D)\, J_2<J_1<J_3 
\stopformula

{\bf 考点:}二重积分的普通对称性。
\index{二重积分}
\index{对称性}
\index{函数保号性}


解答过程:
$D_1$ 被直线 $y=x$ 分成两部分 $D_{11}$ 和 $D_{12}$, 
( 普通对称性, $\sqrt[3]{x-y}=-\sqrt[3]{y-x} $\,), 
故
\startformula
\startalign
 \NC J_1\NC = \iint_{D_1}\sqrt[3]{x-y}\,dxdy = \iint_{D_{11}+D_{12}}\sqrt[3]{x-y}\,dxdy 
            = 0 \NR
\stopalign
\stopformula

\page
$D_2$ 被直线 $y=x^2$ 分成两部分 $D_{21}$ 和 $D_{22}$, 故
\startformula
\startalign
 \NC J_2\NC = \iint_{D_2}\,  = \iint_{D_{21}+D_{22}}\sqrt[3]{x-y}\,dxdy
            = 0 + \iint_{D_{22}}\sqrt[3]{x-y}\,dxdy > 0 \NR
\stopalign
\stopformula

($ D_{21}$ 关于 $y=x$ 对称, 故 $\iint_{D_{21}} \nolimits=0$; $D_{22}$ 上 $x>y$, 由保号性知 $\iint_{D_{21}} \nolimits>0$)

$D_3$ 被直线 $y=\sqrt{x}$ 分成两部分 $D_{31}$ 和 $D_{32}$, 故
\startformula
\startalign
 \NC J_3\NC = \iint_{D_3}\,  = \iint_{D_{31}+D_{32}}\sqrt[3]{x-y}\,dxdy 
            = 0 + \iint_{D_{31}}\sqrt[3]{x-y}\,dxdy < 0 \NR 
\stopalign
\stopformula

($D_{32}$ 关于 $y=x$ 对称, 故 $\iint_{D_{32}} \nolimits = 0$;  
             在 $D_{31}$ 上 $x>y$, 由保号性知 $\iint_{D_{32}} \nolimits < 0$ )
 
综上, 有 $J_3<J_1<J_2$, 选 
$(B)$.

二重积分的普通对称性。 \hfill 例02完毕。

%%
%%%%%%%%%%% %%%%%%%%%%%%%%%%%%%%%%%%%%
% 内容分割线 2020.07.16
%%%%%%%%%%% %%%%%%%%%%%%%%%%%%%%%%%%%%
%%

\Topic{11 例19 2016数学三填空题第12题4分}
\index{例 19}\index{11 例19 }
\index{2016数学三填空题第12题4分}

设 $D=\{(x,y) | |x|\leq y \leq 1, -1\leq x\leq 1\}$, 则
\startformula
I=\iint_D x^2 e^{-y^2}dxdy \, = \, \text{_____________________________}
\stopformula

\medskip

{\bf 考点:}二重积分的对称性。
\index{对称性}




解答过程: 由被积函数关于两个自变量都为偶函数, 并且积分区域关于 $y$ 轴对称, 设
$D_1 =\{ (x,y)| 0\leq x \leq 1, x\leq y \leq 1 \}$, 则

\startformula
\startalign
  \NC I \NC = 2\iint_{D_1}  x^2 e^{-y^2}dxdy = 2\int_0^1 dy \int_0^y x^2 e^{-y^2} dx  
        = \frac{1}{3} -  \frac{2}{3e} \NR
\stopalign
\stopformula
 
二重积分的对称性。 \hfill 例19毕。

%%
%%%%%%%%%%% %%%%%%%%%%%%%%%%%%%%%%%%%%
% 内容分割线 2020.07.16
%%%%%%%%%%% %%%%%%%%%%%%%%%%%%%%%%%%%%
%%

\Topic{12 例17 2016数学二解答题第18题10分}
\index{例 17}\index{12 例17 }
\index{2016数学二解答题第18题10分}

计算二重积分

\startformula
I = \iint_D \frac{x^2-xy-y^2}{x^2+y^2} dxdy
\stopformula
其中平面区域 $D$ 是由直线 $y=1$, $y=x$, $y=-x$ 围成的有界区域。

\medskip

{\bf 考点:} 二重积分的计算之积分区域为三角形区域。 极坐标。 二重积分的计算。 二重积分的对称性。
\index{对称性}
\index{积分公式}
\index{极坐标}

 

积分公式表(19)

\startformula
\int \frac{1}{a^2+x^2} dx  =  \frac{1}{a}\arctan \frac{x}{a} + C
\stopformula

{\darkgreen 拓展}: 积分公式表(21)

\startformula
\int \frac{1}{x^2 - a^2} dx  =  \frac{1}{2a}\ln \left|\frac{x-a}{x+a}\right| + C
\stopformula


\page

解答过程: {\red 法一}: 极坐标。 借助对称性。 计算繁琐。

{\red 法二}: 
\startformula
\startalign
  \NC I \NC  = \int_0^1 dy \int_{-y}^y \frac{x^2 - y^2}{x^2 + y^2}dx 
      = \int_0^1 (2y - y \pi )dy  = 1-\frac{\pi}{2} \NR
\stopalign
\stopformula

二重积分的计算之积分区域为三角形区域。 \hfill 例17毕。

\bigskip

{\darkgreen 拓展}: 计算
\startformula
\int_{\frac{\pi}{4}}^{\frac{3\pi}{4}} d\theta \int_0^{\frac{1}{\sin\theta}} r(\cos^2\theta - \sin^2\theta)dr
\stopformula


%%
%%%%%%%%%%% %%%%%%%%%%%%%%%%%%%%%%%%%%
% 内容分割线 2020.07.16
%%%%%%%%%%% %%%%%%%%%%%%%%%%%%%%%%%%%%
%%

\Topic{13 例20 2015数学一选择题第4题4分}
\index{例 20}
\index{13 例20}
\index{2015数学一选择题第4题4分}

设 $D$ 是第一象限中曲线 $2xy=1$, $4xy=1$ 与直线 $y=x$, $y=\sqrt{3}x$ 围成的平面区域, 函数 $f(x,y)$ 在 $D$ 上连续, 则 $I=\displaystyle\iint_D f(x,y) dxdy \, = \, (\qquad\qquad)$

$\displaystyle (A)    \int_{\frac{\pi}{4}}^{\frac{\pi}{3}} d\theta \int_{\frac{1}{2\sin 2\theta}}^{\frac{1}{\sin 2\theta}}f(r\cos\theta, r\sin\theta)rdr $

$\displaystyle  (B)  \int_{\frac{\pi}{4}}^{\frac{\pi}{3}} d\theta \int_{\frac{1}{\sqrt{2\sin 2\theta}}}^{\frac{1}{\sqrt{\sin 2\theta}}}f(r\cos\theta, r\sin\theta)rdr $

$\displaystyle  (C)   \int_{\frac{\pi}{4}}^{\frac{\pi}{3}} d\theta \int_{\frac{1}{2\sin 2\theta}}^{\frac{1}{\sin 2\theta}}f(r\cos\theta, r\sin\theta)dr $

$\displaystyle  (D)   \int_{\frac{\pi}{4}}^{\frac{\pi}{3}} d\theta \int_{\frac{1}{\sqrt{2\sin 2\theta}}}^{\frac{1}{\sqrt{\sin 2\theta}}}f(r\cos\theta, r\sin\theta)dr $

\medskip

{\bf 考点:}  二重积分的计算之极坐标。  极坐标。 
\index{极坐标}
\index{直极互换}

 

\page
 
解答过程: 画图绘制边界曲线, 并将边界曲线方程用 $x=r\cos\theta$,  $y=r\sin\theta$ 代入转换为极坐标方程, 于是积分区域用极坐标可以描述为

$ D: \displaystyle\frac{\pi}{4} \leq \theta \leq \frac{\pi}{3}, \quad
\frac{1}{\sqrt{2\sin 2\theta}} \leq r \leq \frac{1}{\sqrt{\sin 2\theta}}$

所以直角坐标与极坐标的变换关系, 得

\startformula
\iint_D f(x,y)dxdy = \int_{\frac{\pi}{4}}^{\frac{\pi}{3}} d\theta \int_{\frac{1}{\sqrt{2\sin 2\theta}}}^{\frac{1}{\sqrt{\sin 2\theta}}}f(r\cos\theta, r\sin\theta)rdr
\stopformula

故选 $(B)$.
 
二重积分的计算之极坐标。 \hfill 例20毕。

%%
%%%%%%%%%%% %%%%%%%%%%%%%%%%%%%%%%%%%%
% 内容分割线 2020.07.16
%%%%%%%%%%% %%%%%%%%%%%%%%%%%%%%%%%%%%
%%

\Topic{14 例3 2020数学三解答题第18题10分}
\index{例 03}\index{14 例3 }
\index{2020数学三解答题第18题10分}




%%----------------------------------------------------------------
%% 粉色背景,例题模板

\usemodule[tikz]         


\starttikzpicture[      scale=3,line cap=round
                        axes/.style=,         
                        important line/.style={very thick},
                        information text/.style={rounded corners,fill=red!10,inner sep=1ex} ]
        \draw[xshift=0.0cm]
                node[right,text width=17cm,information text]
                {\bf
%-----------------------例题开始


设区域 $D=\left\{(x,y)|x^2+y^2\leq 1,y\geq 0\right\}$, 
\startformula
f(x,y) = y\sqrt{1-x^2} +x\displaystyle\iint_D f(x,y)dxdy
\stopformula
计算
\startformula
I=\displaystyle \iint_D xf(x,y)dxdy
\stopformula

%-----------------------例题结束
};

\stoptikzpicture

%%----------------------------------------------------------------

%-----------------------考点,开始
{\bf 考点:} 二重积分的计算之对称性。 三角代换公式。 Wallis公式。 极坐标。
\index{三角代换}
\index{对称性}
\index{Wallis公式}
\index{极坐标}

\EnglishRule

%-----------------------考点,结束



%-----------------------解答,开始
\page

解答过程: 令 $\displaystyle\iint_D f(x,y)dxdy=A$, 则由已知条件,有

\startformula
f(x,y) = y\sqrt{1-x^2} +xA
\stopformula

两边积分,得
\startformula
\startalign
  \NC  A \NC  = \iint_D f(x,y) dxdy = \iint_D y\sqrt{1-x^2} dxdy + \iint_D Ax dxdy \NR
  \NC  \NC  = 2  \int_0^1\sqrt{1-x^2}dx \int_0^{\sqrt{1-x^2}}ydy  
            = \frac{3\pi}{16}  \NR
\stopalign
\stopformula

于是
\startformula
I =\displaystyle \iint_D x \left[y\sqrt{1-x^2} +x  \frac{3\pi}{16} \right]dxdy = \frac{3\pi^2}{128}
\stopformula

用极坐标 \hfill 例03毕。

%-----------------------解答,结束

{\darkgreen 练习题} 1: 
$\displaystyle\iint_D x \left[y\sqrt{1-x^2} +x  \frac{3\pi}{16} \right]dxdy$, 
$ D=\{(x,y)|x^2+y^2\leq 1,y\geq 0\}$


%%
%%%%%%%%%%% %%%%%%%%%%%%%%%%%%%%%%%%%%
% 内容分割线 2020.07.16
%%%%%%%%%%% %%%%%%%%%%%%%%%%%%%%%%%%%%
%%

\Topic{15 例 03}
\index{15 例 03}
设 $D=\left\{(x,y)|0\leq x\leq \pi,0\leq y\leq \pi\right\}$, 证明
\startformula
\iint_D\left(e^{\sin y}+e^{-\sin x}\right)dxdy\geq 2\pi^2
\stopformula


\medskip

{\bf 考点:} 二重积分的轮换对称性。 Cauchy-Schwarz不等式。
\index{轮换对称性}
\index{Cauchy-Schwarz不等式}




解答过程:
积分区域 $D$ 关于 $y=x$ 对称,
\startformula
\startalign
  \NC I \NC = \iint_D\left(e^{\sin y}+e^{-\sin x}\right)dxdy  
       =\iint_De^{\sin y}dxdy+\iint_De^{-\sin x}dxdy \NR
  \NC   \NC = \iint_De^{\sin x}dxdy+\iint_De^{-\sin x}dxdy \qquad(\text{轮换对称性}) \NR
  \NC   \NC = \iint_D\left(e^{\sin x}+e^{-\sin x}\right)dxdy  
        \geq \iint_D2dxdy = 2\pi^2 \, (\text{Cauchy-Schwarz不等式}) \NR
\stopalign
\stopformula
二重积分的轮换对称性。 Cauchy-Schwarz不等式。 \hfill 例03证毕。

\Topic{16 题型5. 二重积分的概念之周期性}
\index{16 题型5. 二重积分的概念之周期性}
\index{题型5. 二重积分的概念之周期性}

若化为累次积分后, 一元积分有用周期性的机会, 则可化简计算。

\Topic{17 例 04}
\index{17 例 04}
设 $D=\left\{(x,y)|0\leq x\leq \pi, 0\leq y\leq \pi\right\}$, 计算
\startformula
I=\iint_D\left|\cos(x+y)\right|d\sigma
\stopformula

\medskip

{\bf 考点:}二重积分之被积函数的周期性。
\index{周期性}
\index{平移变换}




\medskip

{\bf 解答过程:} 
因为 $|\cos (a+y)|$ 是 $|\cos y|$ 的水平平移, 且 $|\cos y|$ 周期为 $\pi$ ,所以

\startformula
\startalign
  \NC I \NC = \int_0^\pi dx\int_0^\pi |\cos(x+y)|dy  
       = \int_0^\pi dx\int_0^\pi |\cos y|dy \NR
  \NC   \NC = 2\int_0^\pi dx\int_0^{\frac{\pi}{2}}\cos ydy = 2\int_0^\pi dx = 2\pi \NR
\stopalign
\stopformula

二重积分之被积函数的周期性。   \hfill 例04完毕。





\Topic{18 练习题 05(仅数一)}
\index{18 练习题 05}
设 $\Omega=\left\{(x,y,z)|0\leq x\leq  \pi, 0\leq y\leq \pi, 0\leq z \leq \pi\right\}$, 计算
\startformula
I=\iiint_{\Omega}\left|\cos(x+y+z)\right|d v
\stopformula

\medskip

{\bf 解答提示:} $2\pi^2$, 因为 $|\cos (a+z)|$ 是 $|\cos z|$ 的水平平移。

%%
%%%%%%%%%%% %%%%%%%%%%%%%%%%%%%%%%%%%%
% 内容分割线 2020.07.16
%%%%%%%%%%% %%%%%%%%%%%%%%%%%%%%%%%%%%
%%

\Topic{19 二重积分的计算}\index{19 二重积分的计算}
1. 直角坐标系与交换积分顺序。

2. 极坐标系与交换积分顺序。

3. 直角坐标系与极坐标系互相转化。
%%
%%%%%%%%%%% %%%%%%%%%%%%%%%%%%%%%%%%%%
% 内容分割线 2020.07.16
%%%%%%%%%%% %%%%%%%%%%%%%%%%%%%%%%%%%%
%%

\Topic{20 题型1. 二重积分的计算之直角坐标系与交换积分顺序}
\index{20 题型1. 二重积分的计算之直角坐标系与交换积分顺序}
\index{题型1. 二重积分的计算之直角坐标系与交换积分顺序}

(1) 当积分区域 $D$ 为 $X$ 型区域: $\phi_1(x)\leq y\leq \phi_2(x),a\leq x\leq b$ 时,
\startformula
\iint_Df(x,y)d\sigma=\int_a^bdx\int_{\phi_1(x)}^{\phi_2(x)}f(x,y)dy
\stopformula

(2) 当积分区域 $D$ 为 $Y$ 型区域: $\psi_1(y)\leq x\leq \psi_2(y),c\leq y\leq d$ 时,
\startformula
\iint_Df(x,y)d\sigma=\int_c^d dy\int_{\psi_1(y)}^{\psi_2(y)}f(x,y)dx
\stopformula

{\bf 注意:}上限大于等于下限。

%%
%%%%%%%%%%% %%%%%%%%%%%%%%%%%%%%%%%%%%
% 内容分割线 2020.07.16
%%%%%%%%%%% %%%%%%%%%%%%%%%%%%%%%%%%%%
%%

\Topic{21 例18 2016数学三选择题第3题4分}
\index{21 例 18}\index{18 例18 }
\index{2016数学三选择题第3题4分}

设
\startformula
J_i = \iint_{D_i} \sqrt[3]{x-y} dxdy\quad (i=1,2,3)
\stopformula
其中
$ D_1  =\left\{ (x,y)| 0\leq x \leq 1, 0\leq y \leq 1 \right\}$,  
$ D_2  =\left\{ (x,y)| 0\leq x \leq 1, 0\leq y \leq \sqrt{x} \right\}$,  
$ D_3  =\left\{ (x,y)| 0\leq x \leq 1, x^2 \leq y \leq 1 \right\}$, 则 ( \kern2em ) 

$(A)$ $J_1<J_2<J_3$  \quad $(B)$ $J_3<J_1<J_2$  \quad $(C)$ $J_2<J_3<J_1$  \quad $(D)$ $J_2<J_1<J_3$  
\medskip

{\bf 考点:}二重积分的积分性质。
\index{积分性质}
\index{二重积分的性质}



%\page

解答过程: 画图, 并由被积函数在积分区域内的取值的正负性, 并借助积分性质, 知选 $(B)$。


\smallskip
%%%%-------------------输入代码,开始
\defineframedtext
  [framedcode]
  [strut=yes,
   offset=2mm,
   width=18cm,
   align=right]

\definetyping[code][numbering=line,
                    bodyfont=,
                    before={\startframedcode},
                    after={\stopframedcode}]

\startcode
(* Wolfram Mathematica 11.3.0.0 *)
In[1]:= Integrate [ CubeRoot [ x - y ], { x, 0, 1 }, { y, 0, 1 } ] Shift+Enter
\stopcode

\index{Wolfram Mathematica}

%%%%-------------------输入代码,结束



 
二重积分的积分性质。 \hfill 例18毕。
\blackrule[width=0.5em]\,\blackrule[width=0.5em]\,\blackrule[width=0.5em]
%证明完,解答完

%%
%%%%%%%%%%% %%%%%%%%%%%%%%%%%%%%%%%%%%
% 内容分割线 2020.07.16
%%%%%%%%%%% %%%%%%%%%%%%%%%%%%%%%%%%%%
%%

\Topic{22 例1 2020数学二填空题第10题4分}
\index{22 例 01}
\index{2020数学二填空题第10题4分}

\startformula
I = \int_0^1dy\int_{\sqrt{y}}^1\sqrt{x^3+1}dx=\text{_____________________________}.
\stopformula

\medskip

{\bf 考点:}二重积分的计算之交换积分次序。 定积分换元法之凑微分。 根号下是 $3$ 次多项式, 根号外面是 $2$ 次多项式。
\index{交换积分次序}
\index{凑微分}


 

解答过程:
$I = \displaystyle\int_0^1 dx\int_0^{x^2}\sqrt{x^3+1}dy = \int_0^1\sqrt{x^3+1}x^2dx = \frac{4\sqrt{2}}{9}-\frac{2}{9}$


\medskip

\hfill 例01毕。
\blackrule[width=0.5em]\,\blackrule[width=0.5em]\,\blackrule[width=0.5em]

%%
%%%%%%%%%%% %%%%%%%%%%%%%%%%%%%%%%%%%%
% 内容分割线 2020.07.16
%%%%%%%%%%% %%%%%%%%%%%%%%%%%%%%%%%%%%
%%

\Topic{22 例1 2018数学二选择题第6题4分}
\index{22 例 01}
\index{2018数学二选择题第6题4分}
\index{2018数学二第6题4分}

\startformula
 \int_{-1}^0dx\int_{-x}^{2-x^2}(1-xy)dy + \int_0^1dx\int_x^{2-x^2}(1-xy)\cdot dy = (\kern2em)
\stopformula

\startformula
(A)\quad \frac{5}{3} \qquad (B)\quad \frac{5}{6} \qquad (C) \quad \frac{7}{3} \qquad (D) \quad \frac{7}{6}
\stopformula
\medskip

{\bf 考点:} 二重积分的计算之交换积分次序。
\index{交换积分次序}
\index{偶倍奇零}
\index{积分性质}
\index{定积分}
\index{区间的可加性}


 

解答过程: 法一: 直接计算。 原积分转为定积分, 借助积分对区间的可加性和 “偶倍奇零” 的计算性质, 有
\startformula
\startalign
  \NC I \NC = \int_{-1}^0dx\int_{-x}^{2-x^2}(1-xy)dy + \int_0^1dx\int_x^{2-x^2}(1-xy)\cdot dy \NR
  \NC  \NC = 2\int_0^1 \left(2-x^2-x\right)dx 
           = 2\left.\left(2x-\frac{x^3}{3}-\frac{x^2}{2}\right)\right|_0^1
           = 2\left(2-\frac{1}{3}-\frac{1}{2}\right)
          = \frac{7}{3} \NR
\stopalign
\stopformula

\page
法二: 交换积分次序, 或者通过绘制积分区域, 考察积分区域特点, 转换、简化积分计算。 图形关于 $y$ 轴对称, 而被积函数为 $1-xy$, 第二部分为关于变量 $x$ 的奇函数, 所以最終的积分等于


\startformula
\startalign
  \NC I \NC = \int_{-1}^0dx\int_{-x}^{2-x^2}(1-xy)dy + \int_0^1dx\int_x^{2-x^2}(1-xy)\cdot dy  
       = \iint_D dxdy \NR
\stopalign
\stopformula

即为积分区域 $D$ 的面积。 从而有

\startformula
\startalign
  \NC I \NC = \int_{-1}^0dx\int_{-x}^{2-x^2}(1-xy)dy + \int_0^1dx\int_x^{2-x^2}(1-xy)\cdot dy  \NR
  \NC   \NC = \int_{-1}^0\left(2-x^2+x \right)dx  + \int_0^1\left(2-x^2-x \right)dx  
        = 2\int_0^1\left(2-x^2-x \right)dx
            = \frac{7}{3} \NR
\stopalign
\stopformula

\medskip

二重积分的计算之交换积分次序 \hfill 例01毕。
\blackrule[width=0.5em]\,\blackrule[width=0.5em]\,\blackrule[width=0.5em]

%%
%%%%%%%%%%% %%%%%%%%%%%%%%%%%%%%%%%%%%
% 内容分割线 2020.07.16
%%%%%%%%%%% %%%%%%%%%%%%%%%%%%%%%%%%%%
%%

\Topic{23 题型2. 二重积分的计算之极坐标系与交换积分顺序}
\index{23 题型2. 二重积分的计算之极坐标系与交换积分顺序}
\index{题型2. 二重积分的计算之极坐标系与交换积分顺序}

(1) 若极点 $O$ 在积分区域 $D$ 的外部,则
\startformula
\iint_D f(x,y)d\sigma = \int_\alpha^\beta d\theta \int_{r_1(\theta)}^{r_2(\theta)}f(r\cos\theta,r\sin \theta) r \,dr
\stopformula

(2) 若极点 $O$ 在积分区域 $D$ 的边界上,则
\startformula
\iint_D f(x,y)d\sigma = \int_\alpha^\beta d\theta \int_0^{r(\theta)}f(r\cos\theta,r\sin \theta) r \,dr
\stopformula

(3) 若极点 $O$ 在积分区域 $D$的内部,则
\startformula
\iint_D f(x,y)d\sigma = \int_0^{2\pi} d\theta \int_0^{r(\theta)}f(r\cos\theta,r\sin \theta) r \,dr
\stopformula

{\bf 注意:}正确写出极坐标系下边界曲线的方程。 确定极角的范围。
%%
%%%%%%%%%%% %%%%%%%%%%%%%%%%%%%%%%%%%%
% 内容分割线 2020.07.16
%%%%%%%%%%% %%%%%%%%%%%%%%%%%%%%%%%%%%
%%

\Topic{24 题型3. 二重积分的计算之直极互换}
\index{24 题型3. 二重积分的计算之直极互换}
\index{题型3. 二重积分的计算之直极互换}
\index{直极互换}

{\bf 注意:}

1. 关于积分区域 $D$:

(1) 图形变换(平移变换、对称变换、伸缩变换)。

(2) 直角坐标系下的方程给出(已知直线、未知直线)。

(3) 极坐标系下的方程给出(已知曲线、未知曲线)。

(4) 参数方程给出(已知曲线、未知直线)。

(5) 动区域(含有其他参数)。 

2 关于被积函数 $f(x,y)$:

(1) 分段函数(含绝对值)。\qquad(2) 最大最小值函数。

(3) 取整函数。\qquad(4) 符号函数。\qquad(5) 抽象函数。 

(6) 复合函数 $f(u),u=u(x,y)$。\qquad(7) 偏导函数 $f''_{xy}(x,y)$。


3 换元法:

(1) 一元函数换元法。\qquad(2) 二重积分换元法。\qquad(3) 三重积分换元法(仅数一)。

\page

3-1-1-1. 平移变换。

(a)将函数 $y=f(x)$ 的图像沿 $x$ 轴向左平移 $x_0(x_0>0)$ 个单位长度, 得到函数 $y=f(x+x_0)$ 的图像。

将函数 $y=f(x)$ 的图像沿 $x$ 轴向右平移 $x_0(x_0>0)$ 个单位, 得到函数 $y=f(x-x_0)$ 的图像。

(b)将函数 $y=f(x)$ 的图像沿 $y$ 轴向上平移 $y_0(y_0>0)$ 个单位, 得到函数 $y=f(x)+y_0$ 的图像。

将函数 $y=f(x)$ 的图像沿 $y$ 轴向下平移 $y_0(y_0>0)$ 个单位, 得到函数 $y=f(x)-y_0$ 的图像。

\blackrule[color=black,width=\textwidth,height=.01cm,depth=0cm]

3-1-1-2. 对称变换。

(a)将函数 $y=f(x)$ 的图像关于 $x$ 轴对称, 得到函数 $y=-f(x)$ 的图像。

(b)将函数 $y=f(x)$ 的图像关于 $y$ 轴对称, 得到函数 $y=f(-x)$ 的图像。

(c)将函数 $y=f(x)$ 的图像关于原点对称, 得到函数 $y=-f(-x)$ 的图像。

(d)将函数 $y=f(x)$ 的图像关于直线 $y=x$ 对称, 得到函数 $y=f^{-1}(x)$ 的图像。

(e)保留函数 $y=f(x)$ 的图像在 $x$ 轴及 $x$ 轴上方的部分, 把 $x$ 轴下方的部分关于 $x$ 轴对称到 $x$ 轴上并去掉原来下方的部分, 得到函数 $y=|f(x)|$ 的图像。

(f)保留函数 $y=f(x)$ 的图像在 $y$ 轴及 $y$ 轴右侧的部分, 把 $y$ 轴右侧的部分关于 $y$ 轴对称到 $y$ 轴左侧并去掉原来 $y$ 轴左侧的部分, 得到函数 $y=f(|x|)$ 的图像。

\page

3-1-1-3. 伸缩变换。

\medskip

(a)水平伸缩。

$y=f(kx)(k>1)$ 的图像, 可由 $y=f(x)$ 的图像上每点的横坐标缩短到原来的 $\displaystyle \frac{1}{k}$ 倍且纵坐标不变得到。

$y=f(kx)(0<k<1)$ 的图像, 可由 $y=f(x)$ 的图像上每点的横坐标伸长到原来的 $\displaystyle \frac{1}{k}$ 倍且纵坐标不变得到。

\blackrule[color=black,width=\textwidth,height=.01cm,depth=0cm]

(b)垂直伸缩。

$y=kf(x)(k>1)$ 的图像, 可由 $y=f(x)$ 的图像上每点的纵坐标伸长到原来的 $k$ 倍且横坐标不变得到。

$y=kf(x)(0<k<1)$ 的图像, 可由 $y=f(x)$ 的图像上每点的纵坐标缩短到原来的 $k$ 倍且横坐标不变得到。

\page











%%----------------------------------
%%----------------------------------









\FrameTitle{3-1-2. 直角坐标系下的方程给出。} % 解答

\StartFrame\bf
(a)已知曲线。 可以直接画出。
\startitemize[n] 
  \item 比如:$x+y=1$, $x^2+y^2=1$, $\sqrt{x}+\sqrt{y}=1$。
\stopitemize

\placefigure[force]{second}{\externalfigure[2020-09-01-141630.png][height=8cm]}

\StopFrame

%%----------------------------------
%%----------------------------------









\FrameTitle{3-1-2. 直角坐标系下的方程给出。} % 解答

\StartFrame\bf
(a)已知曲线。可以直接画出。
\startitemize[n] 
  \item 1. 三次拋物线\index{三次拋物线} \ $y=ax^3$  \kern2em
2. 半立方拋物线\index{半立方拋物线}\ $y^2=ax^3$   \kern2em
3. 概率曲线\index{概率曲线}\ $y=e^{-x^2}$
\stopitemize

\placefigure[force]{second}{\externalfigure[2020-09-01-142833.png][height=8cm]}

\StopFrame

%%----------------------------------
%%----------------------------------









\FrameTitle{3-1-2. 直角坐标系下的方程给出。} % 解答

\StartFrame\bf
(a)已知曲线。可以直接画出。
\startitemize[n] 
  \item 箕舌线\index{箕舌线}
$y=\displaystyle \frac{8a^3}{x^2+4a^2}$, \kern2em  
蔓叶线\index{蔓叶线}
\ $y^2(2a-x)=x^3$
\stopitemize

\placefigure[force]{second}{\externalfigure[2020-09-01-144331.png][height=8cm]}

\StopFrame

%%----------------------------------
%%----------------------------------









\FrameTitle{3-1-2. 直角坐标系下的方程给出。} % 解答

\StartFrame\bf
(b)未知曲线。\kern1em(i)描绘特殊点(定义域、值域)。 \kern1em
(ii)用图形变换。\kern1em
(iii)用导数工具 (一阶导数确定单调性、驻点。 二阶导数确定凹凸性、拐点等)。
\startitemize[n] 
  \item 比如:$y^2=(1-x+x^2)^3$。
\stopitemize

\placefigure[force]{second}{\externalfigure[2020-09-01-145745.png][height=8cm]}

\StopFrame

%%----------------------------------
%%----------------------------------









\FrameTitle{ 3-1-3. 极坐标方程给出。} % 解答

\StartFrame\bf
(a)已知曲线。
可以直接画图。
比如:
\startitemize[n] 
  \item 心形线(外摆线的一种)\index{心形线}\ (1) $\rho=a(1-\cos \theta)$,
\stopitemize

\placefigure[force]{second}{\externalfigure[2020-09-01-171950.png][height=8cm]}

\StopFrame

%%----------------------------------
%%----------------------------------









\FrameTitle{ 3-1-3. 极坐标方程给出。} % 解答

\StartFrame\bf
(a)已知曲线。
可以直接画图。
比如:
\startitemize[n] 
  \item 心形线(外摆线的一种)
\index{心形线}
(2) $\rho=a(1+\cos \theta)$,
\stopitemize

\placefigure[force]{second}{\externalfigure[2020-09-01-172151.png][height=8cm]}

\StopFrame

%%----------------------------------
%%----------------------------------









\FrameTitle{ 3-1-3. 极坐标方程给出。} % 解答

\StartFrame\bf
(a)已知曲线。
可以直接画图。
比如:
\startitemize[n] 
  \item 阿基米德螺旋线
\index{阿基米德螺旋线}
\, $\rho=a\theta$, 
\stopitemize

\placefigure[force]{second}{\externalfigure[2020-09-01-173014.png][height=8cm]}

\StopFrame

%%----------------------------------
%%----------------------------------









\FrameTitle{ 3-1-3. 极坐标方程给出。} % 解答

\StartFrame\bf
(a)已知曲线。
可以直接画图。
比如:
\startitemize[n] 
  \item 对数螺旋线\index{对数螺旋线}\ $\rho=e^{a\theta}$, 
\stopitemize

\placefigure[force]{}{\externalfigure[2020-09-01-173558.png][height=5cm]}

\StopFrame

%%----------------------------------
%%----------------------------------









\FrameTitle{ 3-1-3. 极坐标方程给出。} % 解答

\StartFrame\bf
(a)已知曲线。
可以直接画图。
比如:
\startitemize[n] 
  \item 双曲螺旋线\index{双曲螺旋线}\ $\rho\theta=a$,
\stopitemize

\placefigure[force]{}{\externalfigure[2020-09-01-173849.png][height=5cm]}

\StopFrame

%%----------------------------------
%%----------------------------------









\FrameTitle{ 3-1-3. 极坐标方程给出。} % 解答

\StartFrame\bf
(a)已知曲线。
可以直接画图。
比如:
\startitemize[n] 
  \item 伯努力双纽线\index{伯努力双纽线} (1) $\rho^2=a^2\sin 2\theta$, 
  \item 关于{\darkgreen 伯努利双纽线} 的描述首见于 $1694$ 年, {\darkgreen 雅各布·伯努利}将其作为椭圆的一种类比来处理。 椭圆是由到两个定点距离之和为定值的点的轨迹。 而{\darkgreen 卡西尼卵形线}则是由到两定点距离之乘积为定值的点的轨迹。 当此定值使得轨迹经过两定点的中点时, 轨迹便为伯努利双纽线。
\stopitemize
\index{雅各布·伯努利}
\index{1694年}
\index{卡西尼卵形线}

\placefigure[force]{}{\externalfigure[2020-09-01-175136.png][height=8cm]}

\StopFrame

%%----------------------------------
%%----------------------------------









\FrameTitle{ 3-1-3. 极坐标方程给出。} % 解答

\StartFrame\bf
(a)已知曲线。
可以直接画图。
比如:
\startitemize[n] 
  \item 伯努力双纽线\index{伯努力双纽线} (2) $\rho^2=a^2\cos 2\theta$, 
\stopitemize

\placefigure[force]{}{\externalfigure[2020-09-01-175426.png][height=6cm]}

\StopFrame

%%----------------------------------
%%----------------------------------









\FrameTitle{ 3-1-3. 极坐标方程给出。} % 解答

\StartFrame\bf
(a)已知曲线。
可以直接画图。
比如:
\startitemize[n] 
  \item 三叶玫瑰线\index{三叶玫瑰线} (1) $\rho=a\cos 3\theta$
\stopitemize

\placefigure[force]{}{\externalfigure[2020-09-01-175749.png][height=8cm]}

\StopFrame

%%----------------------------------
%%----------------------------------









\FrameTitle{ 3-1-3. 极坐标方程给出。} % 解答

\StartFrame\bf
(a)已知曲线。
可以直接画图。
比如:
\startitemize[n] 
  \item 三叶玫瑰线\index{三叶玫瑰线} (2) $\rho=a\sin 3\theta$,
\stopitemize

\placefigure[force]{}{\externalfigure[2020-09-01-175952.png][height=8cm]}

\StopFrame

%%----------------------------------
%%----------------------------------









\FrameTitle{ 3-1-3. 极坐标方程给出。} % 解答

\StartFrame\bf
(a)已知曲线。
可以直接画图。
比如:
\startitemize[n] 
  \item 四叶玫瑰线\index{四叶玫瑰线} (1) $\rho=a\sin 2\theta$,
\stopitemize

\placefigure[force]{}{\externalfigure[2020-09-01-180158.png][height=8cm]}

\StopFrame

%%----------------------------------
%%----------------------------------









\FrameTitle{ 3-1-3. 极坐标方程给出。} % 解答

\StartFrame\bf
(a)已知曲线。
可以直接画图。
比如:
\startitemize[n] 
  \item 四叶玫瑰线\index{四叶玫瑰线} (2) $\rho=a\cos 2\theta$,
\stopitemize

\placefigure[force]{}{\externalfigure[2020-09-01-180318.png][height=8cm]}

\StopFrame

%%----------------------------------
%%----------------------------------



\FrameTitle{ 3-1-3. 极坐标方程给出。} % 解答

\StartFrame\bf
(a)已知曲线。
可以直接画图。

{\bf 注意:} 见到平方和 $x^2+y^2$ 化为极坐标方程。比如:

伯努力双纽线\index{伯努力双纽线} (2) $\rho^2=a^2\cos 2\theta$, 
化为
$\rho^2=a^2\cos 2\theta$

\StopFrame

%%----------------------------------
%%----------------------------------

\medskip

\FrameTitle{ 3-1-3. 极坐标方程给出。} % 解答

\StartFrame\bf

(b)未知曲线。

1. 描绘特殊点。\quad 2. 用图形变换。\quad 3. 极坐标与直角坐标相互转化。
\index{直极互换}




\StopFrame

%%----------------------------------
%%----------------------------------
 

\FrameTitle{3-1-4. 参数方程给出。} % 解答

\StartFrame\bf
(a)已知曲线。
直接画图。比如:
\startitemize[n] 
  \item 笛卡尔叶形线\index{笛卡尔叶形线} $x=\frac{3at}{1+t^3},\qquad y=\frac{3at^2}{1+t^3}$
  \item 著名科学家{\darkgreen 笛卡儿}, 根据他所研究的一簇花瓣和叶形曲线特征, 列出了 $x^3+y^3-3axy=0$ 的方程式, 这就是现代数学中有名的 “{\darkgreen 笛卡儿叶线}” (茉莉花瓣曲线)。 

\stopitemize

\index{笛卡儿}
\index{花瓣}
\index{叶形曲线特征}
\index{笛卡儿叶线}
\index{茉莉花瓣曲线}

\placefigure[force]{}{\externalfigure[2020-09-01-181755.png][height=8cm]}

\StopFrame

%%----------------------------------
%%----------------------------------
 

\FrameTitle{3-1-4. 参数方程给出。} % 解答

\StartFrame\bf
(a)已知曲线。
直接画图。比如:
\startitemize[n] 
  \item 星形线(内摆线的一种)\index{星形线}
$x=a\cos^3\theta,\qquad y=a\sin^3\theta$

\stopitemize

\placefigure[force]{}{\externalfigure[2020-09-01-183903.png][height=8cm]}

\StopFrame

%%----------------------------------
%%----------------------------------
 

\FrameTitle{3-1-4. 参数方程给出。} % 解答

\StartFrame\bf
(a)已知曲线。
直接画图。比如:
\startitemize[n] 
  \item 摆线\index{摆线} $x=a(\theta-\sin\theta),\qquad y=a(1-\cos\theta)$

\stopitemize

\placefigure[force]{}{\externalfigure[2020-09-01-184604.png][height=6cm]}

\StopFrame

%%----------------------------------
%%----------------------------------
 \medskip 

\FrameTitle{3-1-4. 参数方程给出。} % 解答

\StartFrame\bf
(b)未知曲线。

(i)描点法。

(ii)化为直角坐标系下的方程或者极坐标系下的方程。

\StopFrame

%%----------------------------------
%%----------------------------------
\medskip 

\FrameTitle{3-1-4. 参数方程给出。} % 解答

\StartFrame\bf
3-1-5. 动区域(含其他参数)。

\StopFrame

%%----------------------------------
%%----------------------------------



\medskip
 

\FrameTitle{3-2. 关于被积函数$f(x,y)$:} % 解答

\StartFrame\bf
 
\startitemize[n] 
  \item 分段函数(含绝对值)。
  \item 最大最小值函数。
  \item 取整函数。
  \item 符号函数。
  \item 抽象函数。 
  \item 复合函数 $f(u),u=u(x,y)$。
  \item 偏导函数 $f''_{xy}(x,y)$。

\stopitemize

\StopFrame

%%----------------------------------
%%----------------------------------




\page


3-3. 一元函数积分换元法 (拼凑系数法、三角代换法、Wallis公式等)。  二重积分换元法。  三重积分换元法(仅数一)。

(i) 若 $x=\phi(t)$ 单调、 导数存在且连续, $x=\phi(t)$, $dx=\phi'(t)dt$, $a=\phi(\alpha)$, $b=\phi(\beta)$, 则
\startformula
\int_a^b\nolimits f(x)dx=\int_\alpha^\beta \nolimits f[\phi(t)]\phi'(t)dt
\stopformula

(ii) 若 $x=x(u,v)$, $y=y(u,v)$ 是 $(x,y)$ 面到 $(u,v)$ 面的一对一映射, 且 $x=x(u,v)$, $y=y(u,v)$ 存在一阶连续偏导数,
\medskip
%%
%%%%%%%%%%% %%%%%%%%%%%%%%%%%%%%%%%%%%
% 内容分割线 2020.07.16
%%%%%%%%%%% %%%%%%%%%%%%%%%%%%%%%%%%%%
%% 矩阵
%%%%%%%%%%%%%%%%%%%%%%%%%%%%%%%%%%%%%%%%
\definemathmatrix[rightmatrix]
                 [left={\left |\,},
                  right={\,\right |},
                  align=right]

\startformula
J = \frac{\partial (x,y)}{\partial (u,v)} = 
\startrightmatrix
   \NC \frac{\partial x}{\partial u} \NC \frac{\partial x}{\partial v} \NR
   \NC \frac{\partial y}{\partial u} \NC \frac{\partial y}{\partial v} \NR
\stoprightmatrix
\neq 0
\stopformula
%%%%%%%%%%%%%%%%%%%%%%%%%%%%%%%%%%%%%%%%
则
\startformula
\iint_{D_{xy}}f(x,y)dxdy = \iint_{D_{uv}}f[x(u,v),y(u,v)] \cdot |J|\cdot dudv
\stopformula

\medskip

{\bf 注意:} 若 $x=\rho\cos\theta$, $y=\rho\sin\theta$, 则
\startformula
\iint_{D_{xy}}f(x,y)dxdy = \iint_{D_{\rho\theta}}f[\rho\cos\theta,\rho\sin\theta] \cdot\rho d\rho d\theta
\stopformula
即直角坐标化为极坐标的换元过程。
\index{直极互换}
%%
%%%%%%%%%%% %%%%%%%%%%%%%%%%%%%%%%%%%%
% 内容分割线 2020.07.16
%%%%%%%%%%% %%%%%%%%%%%%%%%%%%%%%%%%%%
%%

\Topic{25 例 01}
\index{25 例 01}
计算
\startformula
I=\int_0^1 \nolimits dy\int_y^1 \nolimits \frac{y}{1+x^2+y^2}dx.
\stopformula

\medskip

{\bf 考点:} 二重积分的计算之交换积分次序。
\index{交换积分次序}

 

\page

解答过程:
\startformula
\startalign
 \NC I  \NC = \int_0^1dy\int_y^1\frac{y}{1+x^2+y^2}dx 
       = \int_0^1dx\int_0^x\frac{y}{1+x^2+y^2}dy \NR
 \NC    \NC = \frac{1}{2}\int_0^1\ln(1+x^2+y^2)\Big|_0^x dx 
         = \frac{1}{2}\int_0^1\left[\ln(1+2x^2)-\ln (1+x^2)\right]dx \NR
 \NC    \NC = \frac{1}{2}\int_0^1\ln \frac{1+2x^2}{1+x^2}dx  
       = \left.\frac{x}{2}\ln\frac{1+2x^2}{1+x^2}\right|_0^1-\int_0^1\frac{x^2}{(1+2x^2)(1+x^2)}dx \NR
 \NC    \NC = \frac{1}{2}\cdot\ln\frac{3}{2}+\int_0^1\frac{dx}{1+2x^2}-\int_0^1\frac{dx}{1+x^2} \NR
  \NC  \NC = \frac{1}{2}\cdot\ln\frac{3}{2}+\frac{1}{\sqrt{2}}\cdot\arctan \sqrt{2}-\frac{\pi}{4} \NR
\stopalign
\stopformula


二重积分的计算之交换积分次序。  \hfill 例01毕。

%%
%%%%%%%%%%% %%%%%%%%%%%%%%%%%%%%%%%%%%
% 内容分割线 2020.07.16
%%%%%%%%%%% %%%%%%%%%%%%%%%%%%%%%%%%%%
%%

\Topic{26 例11 2018数学二解答题第17题10分}
\index{26 例 11}\index{11 例11 }
\index{2018数学二解答题第17题10分}
\index{2018数学二第17题10分}

设平面区域D由曲线 $x=t-\sin t$, $y=1-\cos t$, $0\leq t \leq 2\pi$ 与 $x$ 轴围成, 计算二重积分 

\startformula
\iint_D (x+2y)dxdy
\stopformula

\medskip

{\bf 考点:} 二重积分的计算。 二重积分。 外摆线的一拱。 参数方程。
 
\index{外摆线}
\index{参数方程}
\index{积分公式}

积分公式表(95) $\displaystyle\int \sin^3 x dx$, 

积分公式表(96) $\displaystyle\int \cos^3 x dx$

\page

解答过程:

\startcode
(* Wolfram Mathematica 11.3.0.0 *)
In[1]:=ParametricPlot[ {t - Sin [t], 1-Cos [t] }, {t, 0, 2Pi }, 
PlotStyle -> Thickness [0.01]] Shift+Enter
\stopcode

\index{Wolfram Mathematica}

设曲线为 $y=y(x)$, 则有
\startformula
\startalign
  \NC I \NC = \iint_D (x+2y)dxdy = \int_0^{2\pi}dx \int_0^{y(x)} (x+2y)dy  
       = \int_0^{2\pi} \left(xy+y^2\right) dx  \NR
  \NC \NC = \int_0^{2\pi} \left[(t-\sin t)(1-\cos t)^2 + (1-\cos t)^3\right] dt \NR
  \NC  \NC  = 5\pi + 3\pi ^2\NR
\stopalign
\stopformula

二重积分的计算之外摆线方程。 \hfill 例11毕。

%%
%%%%%%%%%%% %%%%%%%%%%%%%%%%%%%%%%%%%%
% 内容分割线 2020.07.16
%%%%%%%%%%% %%%%%%%%%%%%%%%%%%%%%%%%%%
%%

\Topic{27 例15 2017数学三解答题第16题10分}
\index{27 例 15}\index{15 例15 }
\index{2017数学三解答题第16题10分}
\index{2017数学三第16题10分}

计算二重积分

\startformula
I = \iint_D \frac{y^3}{(1+x^2+y^4)^2}dxdy
\stopformula
其中 $D$ 是由第一象限中曲线 $y=\sqrt{x}$ 与 $x$ 轴为边界的无界区域。
\medskip

{\bf 考点:} 二重积分的计算之积分区域为无界区域。 二重积分的计算。 积分区域为无界区域。 裂项法。
\index{积分区域为无界区域}
\index{裂项法}
\index{积分公式}

 

积分公式表(19)

\startformula
\int \frac{1}{x^2+a^2}dx  = \frac{1}{a}\arctan \frac{x}{a} +C
\stopformula

\page

解答过程:

\startformula
\startalign
  \NC  I  \NC = \iint_D \frac{y^3}{(1+x^2+y^4)^2}dxdy \NR
  \NC  \NC = \int_0^{+\infty}dx \int_0^{\sqrt{x}}\frac{y^3}{(1+x^2+y^4)^2} dy \NR
  \NC \NC = \frac{1}{4}\int_0^{+\infty} \left( \frac{1}{1+x^2} - \frac{1}{1+2x^2} \right) dx = \frac{2-\sqrt{2}}{16}\pi \NR
\stopalign
\stopformula

二重积分的计算之无界区域。 \hfill 例15毕。


%%
%%%%%%%%%%% %%%%%%%%%%%%%%%%%%%%%%%%%%
% 内容分割线 2020.07.16
%%%%%%%%%%% %%%%%%%%%%%%%%%%%%%%%%%%%%
%%

\Topic{28 例12 2018数学三解答题第16题10分}
\index{28 例 12}\index{12 例12 }
\index{2018数学三解答题第16题10分}
\index{2018数学三第16题10分}

设平面区域 $D$ 由曲线 $y=\sqrt{3(1-x^2)}$ 与 $y=\sqrt{3}x$ 和 $y$ 轴围成, 计算二重积分 

\startformula
I = \iint_D x^2dxdy
\stopformula

\medskip

{\bf 考点:} 二重积分的计算。 三角代换。 倍角公式。
\index{三角代换}
\index{倍角公式}

 

(1) $\sin 2\alpha = 2\sin \alpha\cos\alpha$, 

(2) $\cos 2\alpha =\cos ^2\alpha -\sin ^2\alpha = 1-2\sin ^2\alpha = 2\cos ^2\alpha -1$,

(3) $\sin^2 \alpha = \displaystyle \frac{1-\cos 2\alpha}{2}$, 

(4) $\cos^2 \alpha = \displaystyle \frac{1+\cos 2\alpha}{2}$, 

\page

解答过程:


\startcode
(* Wolfram Mathematica 11.3.0.0 *)
In[1]:=Plot[ {Sqrt[3-3x^2], Sqrt[3]x, 0 }, {x, -2, 2 } ] Shift+Enter
\stopcode
\index{Wolfram Mathematica} 

画图, 交点坐标 $\left(\displaystyle \frac{\sqrt{2}}{2}, \frac{\sqrt{6}}{2}\right)$, 于是, 有

\startformula
\startalign
  \NC I \NC  = \int_0^{\frac{\sqrt{2}}{2}} x^2dx \int _{\sqrt{3}x}^{\sqrt{3(1-x^2)}} dy  
       = \sqrt{3} \int_0^{\frac{\sqrt{2}}{2}} x^2\sqrt{1-x^2} dx-\sqrt{3} \int_0^{\frac{\sqrt{2}}{2}}x^3 dx \NR
  \NC \NC (x=\sin t, dx = \cos t dt ) \NR
  \NC \NC = \sqrt{3}\left(\frac{\pi}{32} - \frac{1}{16}\right) \NR
\stopalign
\stopformula

二重积分的计算之三角代换。 \hfill 例12毕。
%%
%%%%%%%%%%% %%%%%%%%%%%%%%%%%%%%%%%%%%
% 内容分割线 2020.07.16
%%%%%%%%%%% %%%%%%%%%%%%%%%%%%%%%%%%%%
%%

\Topic{29 例 02}
\index{29 例 02}
计算
\startformula
I=\iint_D(1+x)\sqrt{1-\cos^2y}\,dxdy
\stopformula
其中 $D$ 是由直线 $y=x+3$, $y=\displaystyle \frac{x}{2}-\frac{5}{2}$, $y=\displaystyle \frac{\pi}{2}$ 及 $y=-\displaystyle \frac{\pi}{2}$ 所围成的区域。
\medskip

{\bf 考点:} 二重积分的计算之 $Y$ 型区域。
\index{Y型区域}


\page


解答过程: $Y$ 型区域, 先 $x$ 后 $y$ 的积分次序。

\startformula
\startalign
  \NC I \NC = \int_{-\frac{\pi}{2}}^{\frac{\pi}{2}} dy\int_{y-3}^{2y+5} (1+x)\sqrt{1-\cos ^2y}\cdot dx \NR
  \NC  \NC = \int_{-\frac{\pi}{2}}^{\frac{\pi}{2}}\sqrt{1-\cos ^2y}\cdot \left.\left(x+\frac{x^2}{2}\right)\right|_{y-3}^{2y+5}dy \NR
  \NC   \NC = \int_{-\frac{\pi}{2}}^{\frac{\pi}{2}}\left(\frac{3}{2}y^2+14y+16\right)\sqrt{1-\cos ^2y}\cdot dy \NR
 \NC  \NC  = \int_0^{\frac{\pi}{2}}(3y^2+32)\sin y\cdot dy = 3\pi+26 \NR
\stopalign
\stopformula

二重积分的计算之Y型区域。   \hfill 例02毕。
%%
%%%%%%%%%%% %%%%%%%%%%%%%%%%%%%%%%%%%%
% 内容分割线 2020.07.16
%%%%%%%%%%% %%%%%%%%%%%%%%%%%%%%%%%%%%
%%

\Topic{30 例 03}
\index{30 例 03}
设区域 $D$: $ay\leq x^2+y^2\leq 2ay (a>0)$,计算
\startformula
I=\iint_D(x+y)^2\,dxdy
\stopformula

\medskip

{\bf 考点:} 二重积分的计算之 $Y$ 型区域。 Wallis公式。
\index{Wallis公式}
\index{Y型区域}


 

解答过程: 积分区域 $D$ 关于 $y$ 轴对称, 被积函数 $f(x,y)=(x+y)^2=(x^2+y^2)+2xy$ 的第一项关于 $x$ 是偶函数, 第二项关于 $x$ 是奇函数。 记 $D_1$ 为 $D$ 在第一象限的部分, 故根据对称性有

\startformula
\startalign
  \NC I \NC = \iint_D(x^2+y^2)dxdy + 2\iint_D xydxdy  
       =2\iint_{D_1}(x^2+y^2)dxdy \NR
  \NC  \NC  = 2\int_0^{\frac{\pi}{2}}d\theta\int_{a\sin\theta}^{2a\sin\theta}r^3dr 
            = 2\int_0^{\frac{\pi}{2}}\left.\frac{r^4}{4}\right|_{a\sin\theta}^{2a\sin\theta}d\theta \NR
  \NC  \NC        = \frac{15a^4}{2}\int_0^{\frac{\pi}{2}}\sin^4\theta d\theta  
              = \frac{15a^4}{2}\cdot\frac{3}{4}\cdot\frac{1}{2}\cdot\frac{\pi}{2}=\frac{45\pi}{32}a^4 \NR
\stopalign
\stopformula
Wallis公式 \hfill 例03毕。
%%
%%%%%%%%%%% %%%%%%%%%%%%%%%%%%%%%%%%%%
% 内容分割线 2020.07.16
%%%%%%%%%%% %%%%%%%%%%%%%%%%%%%%%%%%%%
%%

\Topic{31 例 04}
\index{31 例 04}
计算
\startformula
I=\int_{-\sqrt{2}}^0 dx\int_{-x}^{\sqrt{4-x^2}}(x^2+y^2)dy+\int_0^2dx\int_{\sqrt{2x-x^2}}^{\sqrt{4-x^2}}(x^2+y^2)\,dy
\stopformula

\medskip

{\bf 考点:} 二重积分的计算之极坐标系。  Wallis公式。 直角坐标转化为极坐标。
\index{Wallis公式}
\index{直极互换}
\index{极坐标}




解答过程:由于被积函数 $f(x,y)=x^2+y^2$, 且由积分限所确定的积分区域的边界曲线为直线与 $y=-x$ 与
圆弧 $y=\sqrt{2x-x^2}$ 及 $y=\sqrt{4-x^2}$, 故采用极坐标计算比较方便。

\startformula
\startalign
  \NC I \NC = \int_0^{\frac{\pi}{2}}d\theta\int_{2\cos\theta}^2r^2\cdot r dr+\int_{\frac{\pi}{2}}^{\frac{3\pi}{4}}d\theta\int_0^2r^2\cdot r dr \NR
  \NC  \NC    = 4\int_0^{\frac{\pi}{2}}(1-\cos^4\theta)d\theta + \frac{\pi}{4}\cdot \frac{1}{4}\cdot2^4  
        = 4\left(\frac{\pi}{2}-\frac{3}{4}\cdot\frac{1}{2}\cdot\frac{\pi}{2}\right)+\pi=\frac{9\pi}{4} \NR
\stopalign
\stopformula

二重积分的计算之极坐标系。     \hfill 例04毕。
%%
%%%%%%%%%%% %%%%%%%%%%%%%%%%%%%%%%%%%%
% 内容分割线 2020.07.16
%%%%%%%%%%% %%%%%%%%%%%%%%%%%%%%%%%%%%
%%

\Topic{32 例16 2016数学一解答题第15题10分}
\index{32 例 16}\index{16 例16 }
\index{2016数学一解答题第15题10分}
\index{2016数学一第15题10分}

计算二重积分

\startformula
I = \iint_D xdxdy
\stopformula
其中平面区域
\startformula
D=\left\{(r,\theta)\Big|2\leq r \leq 2(1+\cos\theta), -\frac{\pi}{2}\leq \theta\leq \frac{\pi}{2}\right\}
\stopformula

\medskip

{\bf 考点:} 二重积分的计算之积分区域为心形线和圆围成的区域。 极坐标。 Wallis公式 $(n=2,3,4)$。 二重积分的对称性。 Wallis公式。 心形线 $r=1-\cos\theta$, $r=1+\cos\theta$。
\index{心形线}
\index{圆型区域}
\index{极坐标}
\index{Wallis公式}
\index{对称性}


积分公式表(94)

\startformula
\int \cos^2x dx  = \frac{x}{2} + \frac{1}{4} \sin 2x +C
\stopformula

积分公式表(96)

\startformula
\int \cos^nx dx  = \frac{1}{n}\cos ^{n-1}x \sin x + \frac{n-1}{n} \int \cos^{n-2} x dx
\stopformula

\page

解答过程:


\startcode
(* Wolfram Mathematica 11.3.0.0 *)
In[1:= PolarPlot[ { 2, 2(1+Cos[x]) },  {x, -Pi/2,Pi/2 }] Shift+Enter
\stopcode
\index{Wolfram Mathematica}


\startformula
\startalign
  \NC I \NC = \int_{-\pi/2}^{\pi/2} d\theta\int_2^{2(1+\cos\theta)}r^2\cos\theta dr  
         =  \int_{-\pi/2}^{\pi/2} \cos\theta \left[\frac{r^3}{3}\right]_2^{2(1+\cos\theta)} d\theta \NR
  \NC \NC = 16\int_0^{\pi/2} \cos^2\theta d\theta +  16\int_0^{\pi/2} \cos^3\theta d\theta +  \frac{16}{3}\int_0^{\pi/2} \cos^4\theta d\theta  \NR
  \NC \NC =\frac{32}{3}+5\pi \NR
\stopalign
\stopformula

\text{Polar Double Integral: Calculate double Integral in Polar Coordinates?}



\startcode
(* Wolfram Mathematica 11.3.0.0 *)
In[1]:= Integrate[9 Cos [ \[Theta] ], { \[Theta] ,  \[Theta], \[Pi]/2 } ] Shift+Enter
In[2]:= Integrate[r*r*Cos[ \[Theta]], {\[Theta], -(Pi/2), Pi/2}, 
{r, 2, 2*(1+Cos[ \[Theta]]) } ] Shift+Enter 
\stopcode
Out[2]:= $\displaystyle \frac{32}{3} + 5\pi$
\index{Wolfram Mathematica}


\medskip




二重积分的计算之积分区域为心形线和圆围成的区域。 \hfill 例16毕。
%%
%%%%%%%%%%% %%%%%%%%%%%%%%%%%%%%%%%%%%
% 内容分割线 2020.07.16
%%%%%%%%%%% %%%%%%%%%%%%%%%%%%%%%%%%%%
%%

\Topic{33 例14 2017数学二解答题第20题11分}
\index{33 例 14}\index{14 例14 }
\index{2017数学二填空题第20题11分}
\index{2017数学二第20题11分}

已知平面区域 $D=\{(x,y)|x^2+y^2\leq 2y\}$, 计算二重积分

\startformula
I = \iint_D (x+1)^2dxdy
\stopformula

\medskip

{\bf 考点:} 二重积分的计算之极坐标。 二重积分的计算。  圆域。 倍角公式。
\index{极坐标}
\index{倍角公式}
\index{圆域}

(1) $\sin 2\alpha = 2\sin \alpha\cos\alpha$, 

(2) $\cos 2\alpha =\cos ^2\alpha -\sin ^2\alpha = 1-2\sin ^2\alpha = 2\cos ^2\alpha -1$,

(3) $\sin^2 x = \displaystyle \frac{1-\cos 2\alpha}{2}$, 

(4) $\cos^2 x = \displaystyle \frac{1+\cos 2\alpha}{2}$, 

\page

解答过程: 极坐标系中, $D$: $0\leq \theta\leq \pi, 0\leq \rho\leq 2\sin\theta$, 所以有

\startformula
\startalign
  \NC I \NC = \iint_D (x+1)^2dxdy  
       = \iint_D x^2dxdy +  \iint_D 2xdxdy + \iint_D dxdy \NR
  \NC \NC = \int_0^{\pi} d\theta \int_0^{2\sin\theta}\rho^3 \cos^2\theta d\rho + \pi  
        = 4\int_0^{\pi} \sin ^4\theta \cos ^2\theta d\theta + \pi \NR
  \NC \NC = \int_0^{\pi} \sin^22\theta \cdot\frac{1-\cos2\theta}{2} \cdot d\theta + \pi = \frac{5\pi}{4} \NR
\stopalign
\stopformula

二重积分的计算之极坐标。 \hfill 例14毕。

%%
%%%%%%%%%%% %%%%%%%%%%%%%%%%%%%%%%%%%%
% 内容分割线 2020.07.16
%%%%%%%%%%% %%%%%%%%%%%%%%%%%%%%%%%%%%
%%

\Topic{34 例07 2019数学二解答题第18题10分}
\index{34 例 07}\index{34 例7}
\index{2019数学二解答题第18题10分}
\index{2019数学二第18题10分}

已知平面区域 $D$ 满足 $|x|\leq y$, $(x^2+y^2)^3\leq y^4$, 求
\startformula
\iint_D \frac{x+y}{\sqrt{x^2+y^2}}dxdy
\stopformula

\medskip

{\bf 考点:}二重积分的计算之极坐标。
\index{极坐标}
\index{对称性}



\page

解答过程: 图形关于 $y$ 轴对称,所以
\startformula
I=\iint_D \frac{x+y}{\sqrt{x^2+y^2}}=2\iint_{D_1} \frac{y}{\sqrt{x^2+y^2}}=2I_1
\stopformula

在极坐标系下, $D$ 为 $0 \leq \rho\leq \sin^2\theta, |\cos \theta|=\sin\theta$, 所以

\startformula
I_1=\iint_{D_1}\frac{y}{\sqrt{x^2+y^2}} = \int_{\frac{\pi}{4}}^{\frac{\pi}{2}}d\theta\int_0^{\sin^2\theta}\rho\sin\theta d\rho
=\frac{43}{120\sqrt{2}}
\stopformula

所以 $I=2I_1=\displaystyle \frac{43}{60\sqrt{2}}$.

\startcode
(* Wolfram Mathematica 11.3.0.0 *)
In[1]:= ContourPlot[(x^2+y^2)^3==y^4, \{x,-1,1\},\{y, 0, 1\}],
ImageSize -> \{500, 500\} Shift+Enter
\stopcode
 
\index{Wolfram Mathematica}

用极坐标计算二重积分。 \hfill 例07毕。
%%
%%%%%%%%%%% %%%%%%%%%%%%%%%%%%%%%%%%%%
% 内容分割线 2020.07.16
%%%%%%%%%%% %%%%%%%%%%%%%%%%%%%%%%%%%%
%%

\Topic{35 例 05}
\index{35 例 05}
设 $\displaystyle D=\left\{(r,\theta)\Big| 0\leq r \leq \sec \theta, 0\leq \theta \leq \frac{\pi}{4}\right\}$, 计算
\startformula
I=\iint_D\sqrt{1-r^2\cos2\theta}\cdot r^2\sin\theta \cdot drd\theta
\stopformula

\medskip

{\bf 考点:} 二重积分的计算之极坐标系转化为直角坐标系。 倍角公式。 Wallis公式。 极坐标转化为直角坐标。
\index{倍角公式}
\index{直极互换}
\index{Wallis公式}

\page

解答过程: 令 $\darkgreen x=\sin t$, 则
\startformula
\startalign
  \NC I \NC = \iint_D r^2 \cdot\sqrt{1-r^2\cos^2\theta+r^2\sin^2\theta}\cdot \sin\theta\cdot drd\theta \NR
  \NC \NC = \iint_Dy\sqrt{1-x^2+y^2} dxdy  
        = \frac{1}{2}\int_0^1dx\int_0^x\sqrt{1-x^2+y^2}\cdot d(1-x^2+y^2) \NR
  \NC  \NC =\frac{1}{3}\int_0^1\left.\left(1-x^2+y^2\right)^{\frac{3}{2}}\right|_0^xdx  
        = \darkgreen\frac{1}{3}\int_0^1\left[1-\left(1-x^2\right)^{\frac{3}{2}}\right]dx \NR
  \NC   \NC = \frac{1}{3}-\frac{1}{3}\int_0^{\frac{\pi}{2}}\cos^4t \,dt 
            =\frac{1}{3}-\frac{1}{3}\times \frac{3}{4}\times \frac{1}{2}\times \frac{\pi}{2}
            = \frac{1}{3}- \frac{\pi}{16} \NR
\stopalign
\stopformula
\hfill 例05毕。
%%
%%%%%%%%%%% %%%%%%%%%%%%%%%%%%%%%%%%%%
% 内容分割线 2020.07.16
%%%%%%%%%%% %%%%%%%%%%%%%%%%%%%%%%%%%%
%% 

\Topic{36 例2 2020数学二解答题第19题10分}
\index{36 例 02}\index{02 例2 }
\index{2020数学二解答题第19题10分}
\index{2020数学二第19题10分}

计算二重积分
\startformula
I=\iint_D \frac{\sqrt{x^2+y^2}}{x}d\sigma
\stopformula
其中区域 $D$ 由 $x=1$, $x=2$, $y=x$ 及 $x$ 轴上围成。
\medskip

{\bf 考点:} 二重积分的计算之交换积分次序(直角坐标系换成极坐标系)。 被积函数含有 $\sqrt{x^2+y^2}$, 用极坐标。 {\darkgreen \bf 极坐标}。 交换积分次序。
积分公式表之含有三角函数的积分公式(98)
$\displaystyle \int \sec^3x dx$
\index{交换积分次序}
\index{极坐标}
\index{积分公式}


\page

解答过程: 先 $x$ 后 $y$, 解不出来。
\startformula
\startalign
  \NC I \NC = \iint_{D_1} +  \iint_{D_2} = \int_0^1dy\int_1^2 \frac{\sqrt{x^2+y^2}}{x} + \int_1^2dy\int_y^2 \frac{\sqrt{x^2+y^2}}{x} \NR
\stopalign
\stopformula

先 $y$ 后 $x$, 有

\startformula
\startalign
  \NC I \NC = \int_1^2dx\int_0^x \frac{\sqrt{x^2+y^2}}{x} dy  
      = \int_0^{\frac{\pi}{4}}d\theta\int_{\frac{1}{\cos\theta}}^{\frac{2}{\cos\theta}} \frac{r}{\cos\theta} d\theta = \frac{3}{4}\left[\sqrt{2}+\ln\left(1+\sqrt{2}\right)\right] \NR
\stopalign
\stopformula

直角坐标系换成极坐标系。 \hfill 例02毕。

\page

{\darkgreen \bf 例2拓展}: \index{拓展}
\startformula
\startalign
(1)  \NC I \NC = \int \frac{1}{\sin x}dx = \int\frac{1}{2\sin \frac{x}{2}\cdot \cos \frac{x}{2}} dx \NR
  \NC  \NC = \int \frac{\frac{1}{2\cos^2 \frac{x}{2}} }{\tan \frac{x}{2}} dx = \int \frac{1}{\tan \frac{x}{2}} d\left(\tan \frac{x}{2}\right)   
      = \ln \left|\tan \frac{x}{2}\right| +C\NR
\stopalign
\stopformula

\startformula
(2)  I=\int \frac{1}{\cos x}dx =\int \frac{d\left(x+\frac{\pi}{2}\right)}{\sin\left(x+\frac{\pi}{2}\right)}dx 
=  \ln \left|\tan \left(\frac{x}{2} + \frac{\pi}{4}\right)\right| +C
\stopformula

\startformula
\startalign
(3)  \NC I \NC  =  \int \frac{1}{\sin^3 x}dx = -\int \frac{1}{\sin x }d (\cot x) 
      = -\frac{\cot x}{\sin x} - \int \cot x \frac{\cos x}{\sin^2 x} dx \NR
  \NC \NC =  -\frac{\cos x}{\sin^2 x} - \int \frac{1- \sin^2 x}{\sin^3 x} dx 
     =  -\frac{\cos x}{\sin^2 x} - \int \frac{1}{\sin^3 x} dx + \ln \left|\tan \frac{x}{2}\right| \NR
\stopalign
\stopformula
于是, 
\startformula
\startalign
  \NC I \NC  =  \int \frac{1}{\sin^3 x}dx = -\frac{\cos x}{2\sin^2 x} + \frac{1}{2}\ln \left|\tan \frac{x}{2}\right| + C\NR
\stopalign
\stopformula

\startformula
\startalign
(4) \NC  \NC \int \frac{1}{\cos ^3 x}dx 
             = \int \frac{d\left(x+\frac{\pi}{2}\right)}{\sin ^3 \left(x+\frac{\pi}{2}\right)}  
         = -\frac{\cos \left(x+\frac{\pi}{2}\right)}{2\sin^2 \left(x+\frac{\pi}{2}\right)} + \frac{1}{2}\ln \left|\tan \frac{x+\frac{\pi}{2}}{2}\right| + C
\stopalign
\stopformula

\bigskip\bigskip 

{\darkgreen \bf 练习题}1: \index{练习题}

\startformula
\int \frac{1}{\sin^nx}dx  \qquad n=1,2,3,4,5, \ldots
\stopformula

{\darkgreen \bf 练习题}2:\index{练习题}

\startformula
\int \frac{1}{\cos^nx}dx   \qquad n=1,2,3,4,5, \ldots
\stopformula

{\darkgreen \bf 练习题}3:\index{练习题}
写出Wallis公式的内容并给予证明。

%%
%%%%%%%%%%% %%%%%%%%%%%%%%%%%%%%%%%%%%
% 内容分割线 2020.07.16
%%%%%%%%%%% %%%%%%%%%%%%%%%%%%%%%%%%%%
%%

\Topic{37 例13 2017数学二填空题第13题4分}
\index{37 例 13}\index{13 例13 }
\index{2017数学二填空题第13题4分}
\index{2017数学二第13题4分}

\startformula
I = \int_0^1dy \int _ y ^1 \frac{\tan x }{x} dx\, = \, (\qquad\qquad)
\stopformula

\medskip

{\bf 考点:} 二重积分的计算之交换积分次序。 二重积分的计算。 {\darkgreen \bf 交换积分次序}。
\index{交换积分次序}
\index{积分公式}
 

积分公式表(85)$\displaystyle \int \tan x dx = -\ln |\cos x| +C$

拓展

积分公式表(86)$\displaystyle \int \cot x dx = \ln |\sin x| +C$

\page

解答过程: 交换积分次序, 有
\startformula
\startalign
  \NC I \NC  = \int_0^1 \frac{\tan x}{x} dx \int_0^x dy = \int_0^1 \tan x dx  
       = \int _0^1\frac{\sin x}{\cos x}dx \NR
  \NC \NC  = -\int_0^1 \frac{1}{\cos x}d(\cos x) = -\left.\ln|\cos x | \right|_0^1  
       = -\ln \cos 1 \NR
\stopalign
\stopformula

二重积分的计算之交换积分次序。 \hfill 例13毕。

\bigskip

{\darkgreen \bf 练习题}1: 填空\index{练习题}
\startformula
I = \int_?^?dy \int _ ? ^? \frac{\cot x }{x} dx\, = \, (\qquad\qquad)
\stopformula

%%
%%%%%%%%%%% %%%%%%%%%%%%%%%%%%%%%%%%%%
% 内容分割线 2020.07.16
%%%%%%%%%%% %%%%%%%%%%%%%%%%%%%%%%%%%%
%%

\Topic{38 例 06}
\index{38 例 06}
设 $D$ 由曲线 $x=t-\sin t$, $y=1-\cos t$, $(0\leq t \leq 2\pi)$ 与 $x$ 轴围成, 计算

\startformula
I=\iint_D(x+2y)dxdy
\stopformula


\medskip

{\bf 考点:}二重积分的计算之 $D$ 的参数方程转化为直角坐标系下的方程。 二重积分的{\darkgreen \bf 对称性}。 参数方程转化为直角坐标。 {\darkgreen \bf 摆线方程}。
\index{参数方程}
\index{对称性}
\index{摆线}

\page

解答过程:
\startformula
\startalign
  \NC I \NC = \iint_D(x+2y)dxdy= \iint_D(x-\pi)dxdy + \iint_D(2y+\pi)dxdy \NR
  \NC   \NC = \iint_D(2y+\pi)dxdy \, \left(D\text{关于直线}x=\pi\text{{\darkgreen \bf 对称}}, \iint_D(x-\pi)dxdy=0\right) \NR
  \NC   \NC = \int_0^{2\pi}dx\int_0^{y(x)}(2y+\pi)dy = \int_0^{2\pi}\left[y^2(x)+\pi y(x)\right]dx \NR
  \NC   \NC = \int_0^{2\pi}\left[(1-\cos t)^3 + \pi (1-\cos t)^2\right ]dt \NR
  \NC   \NC = \int_0^{2\pi}\left[(1-3\cos t + 3\cos ^2t-\cos ^3 t)+\pi(1-2\cos t + \cos ^2t) \right ]dt \NR
  \NC   \NC = \pi(5+3\pi)
\stopalign
\stopformula
\hfill 例06毕。
%%
%%%%%%%%%%% %%%%%%%%%%%%%%%%%%%%%%%%%%
% 内容分割线 2020.07.16
%%%%%%%%%%% %%%%%%%%%%%%%%%%%%%%%%%%%%
%%

\Topic{39 例 07}
\index{39 例 07}
设 $\left\{D=(x,y)|x^2+y^2\leq \sqrt{2},x\geq 0,y\geq 0\right\}$, $\left\[1+x^2+y^2\right\]$ 表示不超过 $1+x^2+y^2$ 的最大整数, 计算二重积分
\startformula
I = \iint_Dxy \left[1+x^2+y^2\right]dxdy
\stopformula


\medskip

{\bf 考点:} 二重积分的计算之被积函数含有{\darkgreen \bf 取整函数}。 极坐标。 二重积分。
\index{取整函数}
\index{极坐标}



\page

解答过程:
\startformula
\startalign
  \NC I \NC = \iint_D xy\left[1+x^2+y^2\right]dxdy \NR
  \NC  \NC = \int_0^{\frac{\pi}{2}}d\theta\int_0^{\sqrt[4]{2}}r^3\sin\theta\cos\theta\cdot \left[1+r^2\right]dr\NR
  \NC   \NC = \int_0^{\frac{\pi}{2}}\sin\theta\cos\theta d\theta \int_0^{\sqrt[4]{2}}r^3\left[1+r^2\right]dr \NR
  \NC   \NC  = \left.\frac{1}{2}\sin ^2\theta \right|_0^{\frac{\pi}{2}} \cdot \left(\int_0^1 r^3dr + \int_1^{\sqrt[4]{2}}2r^3dr\right)  
         = \frac{3}{8}  \NR
\stopalign
\stopformula
二重积分的计算之被积函数含有取整函数。 \hfill 例07毕。

%%
%%%%%%%%%%% %%%%%%%%%%%%%%%%%%%%%%%%%%
% 内容分割线 2020.07.16
%%%%%%%%%%% %%%%%%%%%%%%%%%%%%%%%%%%%%
%%

\Topic{40 例5 2019数学二选择题第5题4分}
\index{40 例 05}\index{40 例5}
\index{2019数学二选择题第5题4分}
\index{2019数学二第5题4分}
设
\startformula
\startalign
  \NC D \NC = \left\{(x,y)\left||x|+|y|\leq \frac{\pi}{2}\right.\right\},  \kern2em
        I_1  = \iint_D \sqrt{x^2+y^2} dxdy, \NR
  \NC I_2 \NC = \iint_D \sin \sqrt{x^2+y^2} dxdy,   \kern2em
        I_3  = \iint_D \left(1- \cos \sqrt{x^2+y^2}\,\right) dxdy, \NR
\stopalign
\stopformula
比较 $I_1$, $I_2$, $I_3$ 的大小 ( \kern2em    )

$(A) \, I_3 \< I_2 <I_1\quad (B) \, I_1 \< I_2 <I_3\quad (C) \, I_2 \< I_1 <I_3\quad (D) \, I_2 \< I_1 <I_3$
\medskip

{\bf 考点:} 二重积分的计算之积分区域带绝对值的表达式。 三角不等式 $\sin x \leq x $。 连续函数凹凸性判别法一、二。
\index{绝对值}
\index{凹凸性判别法}
\index{三角不等式}


\page

解答过程: 由于 $|x|+|y|\leq \displaystyle\frac{\pi}{2}$, 所以 $x^2+y^2\leq \displaystyle\frac{\pi}{2}$,   令 $u=\sqrt{x^2+y^2}$,  则
\startformula
0\leq u \leq \sqrt{x^2+y^2},
\stopformula
并且有 $\sin u \leq u$, 即
\startformula
\sin \sqrt{x^2+y^2}\leq \sqrt{x^2+y^2},
\stopformula
所以 $I_2<I_1$, 令
\startformula
\startalign
  \NC g(u) \NC =\sin u -(1-\cos u)=\sin u +\cos u-1,\NR
  \NC g'(u) \NC = \cos u - \sin u,\NR
  \NC g''(u) \NC =-\cos u -\sin u <0,\, g(0)=g\left(\frac{\pi}{2}\right) =0 \NR
\stopalign
\stopformula
所以 $g(u)$ 为单峰函数, $g(u)\geq \min \left\{g(0), g\left(\displaystyle \frac{\pi}{2}\right)\right\}=0$, 即 $\sin u \geq 1-\cos u\geq 0$, 所以 $I_2>I_3$。 综上有   $I_3 \< I_2 <I_1$, 选 $(A)$。

积分区域带绝对值。 \hfill 例05毕。
%%
%%%%%%%%%%% %%%%%%%%%%%%%%%%%%%%%%%%%%
% 内容分割线 2020.07.16
%%%%%%%%%%% %%%%%%%%%%%%%%%%%%%%%%%%%%
%%

\Topic{41 例 08}
\index{41 例 08}
计算二重积分
\startformula
I = \iint_D  e^ {x^2+y^2}\cdot sgn(x+y)dxdy
\stopformula
其中 $D=\left\{(x,y)|x^2 \leq y\leq \sqrt{1-x^2}\,\right\}$\,, $sgn(x)$ 为符号函数。

\medskip

{\bf 考点:}二重积分的计算之被积函数含有符号函数。 符号函数。 极坐标。 二重积分的普通对称性。 积分区域 $D$ 分成 $3$ 块。
\index{符号函数}
\index{对称性}
\index{积分区域分块}


\page

解答过程: 根据 {\darkgreen 符号函数} 的定义, 有
\startformula
 sgn(x+y) = \startmathcases
   \NC -1, \NC 当 $y < -x$ 时 \NR
   \NC 0 ,\NC 当 $y = -x$ 时 \NR
   \NC 1 ,\NC 当 $y > -x$ 时 \NR
\stopmathcases
\stopformula
直线 $y=-x$ 与拋物线 $y=x^2$, 圆弧 $y=\sqrt{1-x^2}$ 围成的区域记为 $D_1$,
 
直线 $y=-x$ 与 $y=x$, $y=\sqrt{1-x^2}$ 围成的区域记为 $D_2$,

直线 $y=x$ 与 $y=x^2$, $y=\sqrt{1-x^2}$ 围成的区域记为 $D_3$, 则

\startformula
\startalign
  \NC I \NC = \iint_{D_1} + \iint_{D_2} + \iint_{D_3}  e^ {x^2+y^2} \cdot sgn(x+y) dxdy \NR
  \NC   \NC = - \iint_{D_1} e^ {x^2+y^2} dxdy + \iint_{D_2} e^ {x^2+y^2} dxdy + \iint_{D_3} e^ {x^2+y^2} dxdy \NR
  \NC   \NC = \iint_{D_2} e^ {x^2+y^2} dxdy 
            = 2\int_{\frac{\pi}{4}}^{\frac{\pi}{2}} d\theta \int_0^1 re^{r^2}dr  
        = \frac{\pi}{4}e^{r^2}\Big|_0^1 = \frac{\pi}{4}(e-1) \NR
\stopalign
\stopformula
二重积分的计算之被积函数含有符号函数 \hfill 例08毕。
%%
%%%%%%%%%%% %%%%%%%%%%%%%%%%%%%%%%%%%%
% 内容分割线 2020.07.16
%%%%%%%%%%% %%%%%%%%%%%%%%%%%%%%%%%%%%
%%

\Topic{42 例 09}
\index{42 例 09}
计算累次积分
\startformula
I = \int_0^2dx\int_0^x\sqrt{(2-x)(2-y)} \cdot f'''(y)\cdot dy
\stopformula
其中 $f(x)$ 具有三阶连续的导数, 且 $f(0)=f'(0)=f''(0)=-1$, $\displaystyle f(2)=-\frac{1}{2}$。

\medskip

{\bf 考点:}二重积分的计算之{\darkgreen \bf 交换积分次序}。
\index{交换积分次序}


\page

解答过程:
\startformula
\startalign
  \NC I \NC = \int_0^2(2-y)^{\frac{1}{2}}\cdot f'''(y)dy\int_y^2 (2-x)^{\frac{1}{2}}dx \NR 
  \NC  \NC  = \frac{2}{3}\int_0^2(2-y)^2f'''(y)dy 
       = \frac{2}{3}\int_0^2(2-y)^2df''(y)  \NR
  \NC  \NC  = \frac{2}{3}(2-y)^2f''(y)\Big|_0^2+\frac{4}{3}\int_0^2(2-y)f''(y)dy \NR
  \NC  \NC = \frac{8}{3} + \frac{4}{3}(2-y)f'(y)\Big |_0^2 + \frac{4}{3}\int_0^2f'(y) dy  \NR
  \NC  \NC = \frac{8}{3} + \frac{8}{3} + \frac{4}{3}\left(-\frac{1}{2} + 1\right) = 6 \NR
\stopalign
\stopformula

直角坐标系下二重积分交换积分次序 \hfill 例09毕。
%%
%%%%%%%%%%% %%%%%%%%%%%%%%%%%%%%%%%%%%
% 内容分割线 2020.07.16
%%%%%%%%%%% %%%%%%%%%%%%%%%%%%%%%%%%%%
%%

\Topic{43 例 10}
\index{43 例 10}
设 $f(t)$ 连续, 证明
\startformula
\iint_D f(x-y)dxdy = \int_{-A}^A f(t) (A-|t|)dt
\stopformula
其中常数 $A>0$, 区域 $\displaystyle D: |x| \leq \frac{A}{2}$, $\displaystyle |y| \leq \frac{A}{2}$。

\medskip

{\bf 考点:} 二重积分的计算之交换积分次序。 被积函数中含有{\darkgreen \bf 绝对值函数}。
\index{交换积分次序}
\index{绝对值函数}

\page

解答过程:
\startformula
\startalign
  \NC \text{左边} \NC = \int_{-\frac{A}{2}}^{\frac{A}{2}}dy\int_{-\frac{A}{2}}^{\frac{A}{2}}f(x-y)dx   
       = \int_{-\frac{A}{2}}^{\frac{A}{2}}dy\int_{-\frac{A}{2}-y}^{\frac{A}{2}-y} f(t)dt \NR
  \NC  \NC= \int_{-A}^0 dt \int_{-t-\frac{A}{2}}^{\frac{A}{2}} f(t)dy + \int_0^A dt \int_{-\frac{A}{2}}^{\frac{A}{2}-t} f(t)dy\NR
  \NC   \NC = \int_{-A}^0 f(t)\cdot (A+t)dt +\int_0^A f(t)\cdot (A-t)dt \NR
  \NC   \NC = \int_{-A}^0 f(t) (A- |t|)dt + \int_0^A f(t)(A-|t|)dt= \text{右边} \NR
\stopalign
\stopformula

二重积分在计算过程中交换积分次序 \hfill 例10毕。
%%
%%%%%%%%%%% %%%%%%%%%%%%%%%%%%%%%%%%%%
% 内容分割线 2020.07.16
%%%%%%%%%%% %%%%%%%%%%%%%%%%%%%%%%%%%%
%%

\Topic{44 例 11}
\index{44 例 11}
设 $f(t)$ 连续, 且
\startformula
f(x)=1+\frac{1}{2}\int_x^1 f(y)\cdot f(y-x)dy
\stopformula
计算
\startformula
I=\int_0^1 f(x)dx
\stopformula
\medskip

{\bf 考点:}  二重积分的计算之交换积分次序。 将累次积分化成二重积分后交换积分次序。
\index{交换积分次序}
\index{变量代换}


\page

解答过程: 令 $\darkgreen t=y-x$, $\darkgreen dt=-dx$, 则
\startformula
\startalign
  \NC I \NC = 1+\frac{1}{2}\int_0^1dx\int_x^1 f(y)\cdot f(y-x)dy \NR
  \NC   \NC = 1+\frac{1}{2}\int_0^1 f(y)dy\int_0^y f(y-x)dx \,\text{(交换积分次序)} \NR
  \NC   \NC = \darkgreen 1+\frac{1}{2}\int_0^1 f(y)dy\int_0^yf(t)dt    
        = 1+\frac{1}{2}\int_0^1\left(\int_0^yf(t)dt\right)\cdot f(y)dy  \NR
  \NC   \NC = 1+\frac{1}{2}\int_0^1\left(\int_0^yf(t)dt\right)d\left(\int_0^yf(t)dt\right) \,\text{(变上限积分函数求导数)}  \NR
  \NC   \NC = 1+\frac{1}{4}\left.\left(\int_0^yf(t)dt\right)^2\right|_0^1 = 1+\frac{1}{4}I^2  \NR
\stopalign
\stopformula


即 $\displaystyle I=1+\frac{1}{4}I^2$, 解得 $I=2$。

二重积分的内层积分作变量代换, 再凑微分。  \hfill 例11毕。
%%
%%%%%%%%%%% %%%%%%%%%%%%%%%%%%%%%%%%%%
% 内容分割线 2020.07.16
%%%%%%%%%%% %%%%%%%%%%%%%%%%%%%%%%%%%%
%%

\Topic{45 例 12}
\index{45 例 12}
设 $f(x)$ 在 $[0,1]$ 上具有连续导数, $f(0)=1$,  且满足
\startformula
\iint_{D_t}f'(x+y)dxdy=\iint_{D_t}f(t)dxdy
\stopformula
其中 $D_t=\{(x,y)| 0\leq y\leq t-x, 0\leq x\leq t\}$, $(0<t\leq 1)$, 求 $f(x)$ 的表达式。

\medskip

{\bf 考点:} 二重积分的计算之被积函数中含有一元抽象函数的导数。  可分离变量的微分方程。
\index{抽象函数}
\index{可分离变量的微分方程}
 

\page

解答过程:
\startformula
\startalign
  \NC L \NC =\iint_{D_t}f'(x+y)\cdot dxdy  
       = \int_0^tdx\int_0^{t-x} f'(x+y)dy \NR
  \NC  \NC = \int_0^t\left[f(t)-f(x)\right] dx  
       = tf(t)-\int_0^t f(x)dx \NR
\stopalign
\stopformula

\startformula
R =\iint_{D_t}f(t)dxdy = \frac{t^2}{2}\cdot f(t) \qquad \text{所以,有}
\stopformula

\startformula
tf(t)-\int_0^t f(x)dx = \frac{t^2}{2}f(t) \qquad \text{两边求导,有}
\stopformula

\startformula
(2-t)f'(t)=2f(t) \qquad \text{分离变量,有}
\stopformula

\startformula
f(t)=\frac{C}{(2-t)^2} \qquad \text{代入}f(0)=1, \text{得} C=4, \text{故} 
\stopformula

\startformula
f(x)=\frac{4}{(2-x)^2}\qquad (0\leq x\leq 1)
\stopformula

二重积分的被积函数中含有一元抽象函数的导数。 \hfill 例12毕。
%%
%%%%%%%%%%% %%%%%%%%%%%%%%%%%%%%%%%%%%
% 内容分割线 2020.07.16
%%%%%%%%%%% %%%%%%%%%%%%%%%%%%%%%%%%%%
%%

\Topic{46 例 13}\index{p269,例14.21}\index{46 例 13}
设 $f(x,y)$ 具有二阶连续偏导数, 且 $f(1,y)=0$, $f(x,1)=0$, 且满足
\startformula
\iint_D f(x,y)dxdy = a
\stopformula
其中 $D=\{(x,y)| 0\leq x \leq 1, 0\leq y \leq 1\}$, 计算二重积分
\startformula
I=\iint_D xy\cdot f''_{xy}(x,y)\cdot dxdy
\stopformula

\medskip

{\bf 考点:} 二重积分的计算之被积函数中含有抽象函数的二阶偏导数。
\index{抽象函数}
\index{二阶偏导数}


\page

解答过程:
因为 $f(1,y)=0$, $f(x,1)=0$, 所以$f'_y(1,y)=0$, $f'_x(x,1)=0$.  从而, 有
\startformula
\startalign
  \NC I \NC = \iint_D xy f''_{xy}(x,y)dxdy  
       = \int_0^1 x dx \int_0^1 y f''_{xy}(x,y)dy  \NR
  \NC   \NC = \int_0^1 x \left[\left.yf'_x(x,y)\right|_{y=0}^{y=1}-\int_0^1f'_x(x,y)dy\right]dx  
       =-\int_0^1dy\int_0^1xf'_x(x,y)dx \NR
  \NC  \NC = -\int_0^1 \left[\left. xf(x,y)\right|_{x=0}^{x=1}-\int_0^1f(x,y)dx\right]dy  
       = \int_0^1dy\int_0^1f(x,y)dx \NR
  \NC  \NC = \iint_Df(x,y)dxdy =a \NR
\stopalign
\stopformula

 
二重积分的被积函数中含有二元抽象函数的二阶偏导数。 \hfill 例13毕。
%%
%%%%%%%%%%% %%%%%%%%%%%%%%%%%%%%%%%%%%
% 内容分割线 2020.07.16
%%%%%%%%%%% %%%%%%%%%%%%%%%%%%%%%%%%%%
%%


\Topic{47 例 14}\index{p269,例14.22}\index{47 例 14}
设 $D=\{(x,y)|  x + y \leq 1, x\geq 0, y\geq 0\}$, 计算二重积分
\startformula
I = \iint_D \cos \frac{x-y}{x+y}d\sigma
\stopformula

\medskip

{\bf 考点:}二重积分的计算之被积函数非常复杂。\medskip

解法一(直线的极坐标方程)较难。\medskip

解法二(雅可比行列式)容易。 二重积分的被积函数非常复杂, 直角坐标系下求不出来原函数。 积分区域为三角形。 转化为极坐标系。\medskip

积分公式 $\displaystyle \int\frac{1}{\cos^2x}dx$。\quad
三角恒等式 $\displaystyle \tan (\alpha\pm\beta) = \frac{\tan\alpha\pm\tan\beta}{1\mp\tan\alpha\tan\beta}$,  

$\cos(\alpha\pm\beta)=\cos\alpha\cos\beta\mp\sin\alpha\sin\beta$,

$\displaystyle 2\cos^2\left(\theta-\frac{\pi}{4}\right)
= 2\left(\cos\theta\cos\frac{\pi}{4}+\sin\theta\sin\frac{\pi}{4}\right)^2
= 2\left(\frac{\sqrt{2}}{2}\cos\theta+\frac{\sqrt{2}}{2}\sin\theta\right)^2 
= \left(\cos\theta+\sin\theta\right)^2$.

$\displaystyle \tan\left(\theta-\frac{\pi}{4}\right)=\frac{\tan\theta - \tan\frac{\pi}{4}}{1+\tan\theta\tan\frac{\pi}{4}}=\frac{\tan\theta-1}{1+\tan\theta}$,  

\index{极坐标}
\index{雅可比行列式}
\index{三角形区域}
\index{三角恒等式}
\index{积分公式}
\index{和差化积}
\index{雅可比行列式}

\page

解答过程:(解法一)
直线 $x+y=1$ 的极坐标方程为 $\displaystyle r=\frac{1}{\cos \theta+\sin\theta}$, 则

\startformula
D=\left\{(r,\theta)\left|0\leq r\leq \frac{1}{\cos\theta +\sin\theta},0\leq \theta\leq\frac{\pi}{2}\right.\right\}
\stopformula

\startformula
\startalign
  \NC I \NC = \int_0^{\frac{\pi}{2}}d\theta\int_0^{\frac{1}{\cos\theta +\sin\theta}}\cos\left(\frac{\cos\theta - \sin\theta}{\cos\theta +\sin\theta}\right) rdr\NR
  \NC \NC = \frac{1}{2}\int_0^{\frac{\pi}{2}}\cos\left(\frac{1 - \tan\theta}{1 +\tan\theta}\right)\frac{1}{(\cos\theta +\sin\theta)^2} d\theta \NR
  \NC \NC = \frac{1}{2}\int_0^{\frac{\pi}{2}}\cos\left[\tan\left(\theta-\frac{\pi}{4}\right)\right]\frac{1}{2\cos^2\left(\theta-\frac{\pi}{4}\right)}d\left(\theta-\frac{\pi}{4}\right) \NR
\stopalign
\stopformula



\startformula
\startalign
  \NC \NC = \frac{1}{2}\int_0^{\frac{\pi}{2}}\cos\left[\tan\left(\theta-\frac{\pi}{4}\right)\right]\frac{1}{2\cos^2\left(\theta-\frac{\pi}{4}\right)}d\left(\theta-\frac{\pi}{4}\right) \NR
  \NC \NC = \frac{1}{4}\int_0^{\frac{\pi}{2}}\cos\left[\tan\left(\theta-\frac{\pi}{4}\right)\right] d\left(\tan\left(\theta-\frac{\pi}{4}\right)\right) \NR
  \NC  \NC = \frac{1}{4}\sin\left[\left.\tan\left(\theta-\frac{\pi}{4}\right)\right]\right|_0^{\frac{\pi}{2}}=\frac{1}{2}\sin 1 \NR
\stopalign
\stopformula
二重积分的被积函数非常复杂。 \hfill 例14解法一(难)毕。

\page

解答过程:(解法二)
令 $u=x-y$, $v=x+y$, 则 $x=\displaystyle \frac{u+v}{2}$, $y=\displaystyle \frac{v-u}{2}$, 

在该变换下, 直线 $x=0$, $y=0$, $x+y=1$ 分别变为:$u=-v$, $u=v$, $v=1$. 

记变换后的区域为 $D'$, 则有 $D': -v\leq u \leq v, 0\leq v\leq 1$.
又因为
\startformula
J=\frac{\partial(x,y)}{\partial{(u,v)}}=
\left |
\startmatrix
\NC \displaystyle \frac{1}{2} \NC \displaystyle \frac{1}{2} \NR
\NC \displaystyle -\frac{1}{2} \NC \displaystyle \frac{1}{2} \NR
\stopmatrix
\right | 
= \frac{1}{2}
\stopformula
所以
\startformula
\startalign
  \NC I  \NC =\iint_{D'}\cos\frac{u}{v}\cdot | J |\cdot dudv = \frac{1}{2}\int_0^1dv\int_{-v}^v\cos \frac{u}{v}\cdot du \NR
  \NC \NC = \frac{1}{2}\int_0^12\sin 1\cdot v dv = \frac{1}{2}\sin 1 \NR
\stopalign
\stopformula

二重积分的被积函数非常复杂。 \hfill 例14解法二(容易)毕。

%%
%%%%%%%%%%% %%%%%%%%%%%%%%%%%%%%%%%%%%
% 内容分割线 2020.07.16
%%%%%%%%%%% %%%%%%%%%%%%%%%%%%%%%%%%%%
%%

\Topic{48 二重积分的应用}\index{48 二重积分的应用}
1. 面积。
\startformula
S=\iint_D d\sigma
\stopformula
2. 柱体体积({\bf \darkgreen 仅数一})。

曲顶为 $z=z(x,y)$, $(x,y)\in D_{xy}$ 的柱体体积
\startformula
V=\iint_{D_{xy}} |z(x,y)| d\sigma
\stopformula
3. 总质量({\bf \darkgreen 仅数一})。
\startformula
m=\iint_D \rho (x,y)d\sigma
\stopformula

\page 

4. 重心坐标({\bf \darkgreen 仅数一})。\index{重心坐标}

对于平面薄片,面密度为 $\rho (x,y)$, $D$ 是薄片所占的平面区域, 则计算重心 $(\bar{x},\bar{y})$ 的公式为
\startformula
\bar{x}=\frac{\iint_D x\rho (x,y)d\sigma}{\iint_D \rho (x,y)d\sigma},\qquad \bar{y}=\frac{\iint_D y\rho (x,y)d\sigma}{\iint_D \rho (x,y)d\sigma}
\stopformula
注意:

(1) 在考研的范围内, 重心就是质心。

(2) 当密度 $\rho(x,y)$ 或者 $\rho(x,y,z)$ 为常数时, 重心就成了形心。 \index{形心}

\page

5. 转动惯量({\bf \darkgreen 仅数一})。\index{转动惯量}

对于平面薄片, 面密度为 $\rho(x,y)$, $D$ 是薄片所占的平面区域, 则计算该薄片对 $x$ 轴、 $y$ 轴和原点 $O$ 的转动惯量 $I_x$, $I_y$, $I_z$, 公式分别为
\startformula
I_x=\iint_D y^2\rho(x,y)d\sigma,
\stopformula

\startformula
I_y=\iint_D x^2\rho(x,y)d\sigma,
\stopformula

\startformula
I_O=\iint_D (x^2+y^2)\rho(x,y)d\sigma,
\stopformula
%%
%%%%%%%%%%% %%%%%%%%%%%%%%%%%%%%%%%%%%
% 内容分割线 2020.07.16
%%%%%%%%%%% %%%%%%%%%%%%%%%%%%%%%%%%%%
%%


\Topic{49 例 01 ({\bf \darkgreen 仅数一})}
\index{49 例 01}\index{P271, 例14.23}
计算双纽线
\startformula
(x^2+y^2)^2=2a^2(x^2-y^2)\quad (a>0)
\stopformula
所围图形的面积。
\medskip

{\bf 考点:} 二重积分的应用之面积。
\index{伯努力双纽线}
\index{极坐标}


%\page

解答过程:
由直角坐标与极坐标的关系知, 双纽线的极坐标方程为
\startformula
r^2=2a^2\cos 2\theta
\stopformula
所围图形的面积为
\startformula
S=\iint_D d\sigma = 4\int_0^{\frac{\pi}{4}}d\theta \int_0^{a\sqrt{2\cos 2\theta}}r dr =2a^2
\stopformula

\hfill 例01毕。
%%
%%%%%%%%%%% %%%%%%%%%%%%%%%%%%%%%%%%%%
% 内容分割线 2020.07.16
%%%%%%%%%%% %%%%%%%%%%%%%%%%%%%%%%%%%%
%%

\Topic{50 例 02 ({\bf \darkgreen 仅数一})}
\index{50 例 02}\index{p271, 例14.24(仅数一)}
(仅数一)求圆柱面 $x^2+y^2=ay\,(a>0)$, 锥面 $z=\sqrt{x^2+y^2}$ 与平面 $z=0$ 所围成的立体体积 $V$。

\medskip

{\bf 考点:}二重积分的应用之体积。 极坐标。 Wallis公式。
\index{锥面}
\index{极坐标}
\index{Wallis公式}
\index{曲顶柱体}
 
%\page

解答过程:

在圆域 $D: x^2+y^2\leq ay$ 上,
以锥面 $z=\sqrt{x^2+y^2}$ 为曲顶的曲顶柱体,
积分域
$D: 0\leq \theta \leq \pi$ , $0\leq r\leq a\sin\theta$, 故

\startformula
\startalign
  \NC V \NC =\iint_D r^2drd\theta = 2\int_0^{\frac{\pi}{2}}d\theta\int_0^{a\sin\theta}r^2dr \NR
  \NC  \NC  = \frac{2a^3}{3}\int_0^{\frac{\pi}{2}}\sin^3\theta d\theta =\frac{4}{9}a^3 \NR
\stopalign
\stopformula

\hfill 例02毕。
%%
%%%%%%%%%%% %%%%%%%%%%%%%%%%%%%%%%%%%%
% 内容分割线 2020.07.16
%%%%%%%%%%% %%%%%%%%%%%%%%%%%%%%%%%%%%
%%

\Topic{51 例 03 ({\bf \darkgreen 仅数一})}
\index{51 例 03}
\index{p271, 例14.25}
求由两曲面 $x^2+y^2=az$ 与 $z=2a-\sqrt{x^2+y^2}$ 所围立体的体积  $(a>0)$。
\medskip

{\bf 考点:}二重积分的应用之体积。
\index{体积}

 

%\page

解答过程:
联立 $x^2+y^2=az$ 与 $z=2a-\sqrt{x^2+y^2}$, 得 $z=a$。
\startformula
\startalign
  \NC V  \NC = \iint_D (2a-\sqrt{x^2+y^2}-\frac{x^2+y^2}{a})dxdy \NR
  \NC    \NC = \int_0^{2\pi}d\theta\int_0^a(2a-r-\frac{r^2}{a})r dr=\frac{5\pi a^3}{6} \NR
\stopalign
\stopformula

体积 \hfill 例03毕。
%%
%%%%%%%%%%% %%%%%%%%%%%%%%%%%%%%%%%%%%
% 内容分割线 2020.07.16
%%%%%%%%%%% %%%%%%%%%%%%%%%%%%%%%%%%%%
%%

\Topic{52 例 04 ({\bf \darkgreen 仅数一})}
\index{52 例 04}\index{p272, 例14.26}
证明曲面
\startformula
(z-a)\phi(x)+(z-b)\phi(y)=0
\stopformula

以及 $x^2+y^2=c^2$ 和 $z=0$ 所围成的立体的体积为
\startformula
V=\frac{1}{2}\pi (a+b)c^2
\stopformula
其中 $\phi$ 为任意正的连续函数, $a$, $b$, $c$ 为正常数。
\medskip

{\bf 考点:}二重积分的应用之体积。
\index{体积}

 

\page

解答过程:
由 $(z-a)\phi(x)+(z-b)\phi(y)=0$, 解得
\startformula
z=\frac{a\phi(x)+b\phi(y)}{\phi(x)+\phi(y)}
\stopformula

记 $D$: $x^2+y^2\leq c^2$, 有
\startformula
\startalign
  \NC V \NC = \iint_D\frac{a\phi(x)+b\phi(y)}{\phi(x)+\phi(y)}dxdy  \NR
  \NC  \NC = \iint_D\frac{a\phi(y)+b\phi(x)}{\phi(y)+\phi(x)}dxdy \NR
  \NC   \NC = \frac{1}{2}\iint_D (\frac{a\phi(x)+b\phi(y)}{\phi(x)+\phi(y)} + \frac{a\phi(y)+b\phi(x)}{\phi(y)+\phi(x)})dxdy \NR
  \NC   \NC = \frac{1}{2}(a+b)\iint _Ddxdy = \frac{1}{2}\pi(a+b)c^2 \NR
\stopalign
\stopformula

\hfill 例04毕。
%%
%%%%%%%%%%% %%%%%%%%%%%%%%%%%%%%%%%%%%
% 内容分割线 2020.07.16
%%%%%%%%%%% %%%%%%%%%%%%%%%%%%%%%%%%%%
%%

\Topic{53 例 05 ({\bf \darkgreen 仅数一})}
\index{53 例 05}
\index{p272, 例14.27}
求由曲线 $x^2+3y-5=0$, $x=\sqrt{y+1}$ 以及 $x=0$ 所围成的均匀薄片(密度 $\mu$ 为常数) 对 $y$ 轴的转动惯量。

\medskip

{\bf 考点:}二重积分的应用之转动惯量。
\index{转动惯量}
\index{薄片}

%\page

解答过程:
\startformula
\startalign
  \NC I_y  \NC = \iint_D \mu x^2d\sigma = \mu\int_0^{\sqrt{2}}x^2dx \int_{x^2-1}^{\frac{1}{3}(5-x^2)} dy \NR
  \NC    \NC = \frac{4}{3}\mu \int_0^{\sqrt{2}}(2x^2-x^4)dx =\frac{32\sqrt{2}}{45}\mu \NR
\stopalign
\stopformula
转动惯量 \hfill 例05毕。
%%
%%%%%%%%%%% %%%%%%%%%%%%%%%%%%%%%%%%%%
% 内容分割线 2020.07.16
%%%%%%%%%%% %%%%%%%%%%%%%%%%%%%%%%%%%%
%%

\Topic{54 三重积分的概念 ({\bf \darkgreen 仅数一})}
\index{54 三重积分的概念
}\index{p361, 三重积分的概念}

1. 三重积分的物理背景:以 $f(x,y,z)$ 为点密度的空间物体的质量。
\index{质量}

当 $f(x,y,z)\equiv 1$ 时, 空间立体 $\Omega$ 的体积
\startformula
V=\iiint_\Omega dv
\stopformula

2. 对称性。\index{对称性}

(i) 普通对称性。 假设 $\Omega$ 关于 $yoz$ 面对称, 则
\startformula
\iiint_\Omega f(x,y,z)dv = 
 \startcases[align={right,left},distance=3pt]
   \NC 2\iiint_{\Omega_1} f(x,y,z) dv,  \NC \, $f(x,y,z) = f(-x,y,z)$ \NR
   \NC 0, \NC \, $f(x,y,z) = -f(-x,y,z)$ \NR
 \stopcases
\stopformula
其中 $\Omega_1$ 是 $\Omega$ 在 $yoz$ 面前面的部分。

\page 

3. 轮换对称性。
\index{轮换对称性}

若把 $x$ 与 $y$ 对调后, $\Omega$ 不变,有
\startformula
\iiint_\Omega f(x,y,z)dv =\iint_\Omega f(y,x,z) dv
\stopformula
这就是轮换对称性。

比如, 设 $\Omega = \{(x,y,z)| x^2+y^2+z^2\leq R^2\}$, 则

\startformula
\iiint_\Omega f(x)dv=\iiint_\Omega f(y)dv=\iiint_\Omega f(z)dv
\stopformula
可以简化计算。

%%
%%%%%%%%%%% %%%%%%%%%%%%%%%%%%%%%%%%%%
% 内容分割线 2020.07.16
%%%%%%%%%%% %%%%%%%%%%%%%%%%%%%%%%%%%%
%%
\Topic{55 三重积分的计算 ({\bf \darkgreen 仅数一})}
\index{55 三重积分的计算}
\index{p362}

1. 直角坐标系。\index{直角坐标系}

(i) 先一后二法(先 $z$ 后 $xy$ 法, 也叫投影穿线法)。

当 $\Omega$ 有下曲面 $z=z_1(x,y)$、 上曲面 $z=z_2(x,y)$, 无侧面或侧面为柱面时,有

\startformula
\iiint_\Omega f(x,y,z) dv =\iint_{D_{xy}}  d\sigma\int_{z_1(x,y)}^{z_2(x,y)}f(x,y,z) dz
\stopformula

(ii) 先二后一法(先 $xy$ 后 $z$ 法, 也叫定限截面法)。\index{先二后一}

当 $\Omega$ 是旋转体时, 其旋转曲面方程为 $\Sigma: z=z(x,y)$, 有

\startformula
\iiint_\Omega f(x,y,z) dv =\int_a^bdz \iint_{D_z} f(x,y,z) d\sigma 
\stopformula
%%
%%%%%%%%%%% %%%%%%%%%%%%%%%%%%%%%%%%%%
% 内容分割线 2020.07.16
%%%%%%%%%%% %%%%%%%%%%%%%%%%%%%%%%%%%%
%%

\page 
2. 柱面坐标系=极坐标系下二重积分与定积分。

在直角坐标系的先一后二法中, 若 $\iint_{D_{xy}}d\sigma$ 适用于极坐标系,则令
\startformula
\startcases
  \NC x  = r\cos\theta \NR
  \NC y  = r\sin\theta \NR
\stopcases
\stopformula
有

\startformula
\iiint_\Omega f(x,y,z)dxdydz=\iiint_\Omega f(r\cos\theta,r\sin\theta,z) rdrd\theta dz
\stopformula

这种方法称为柱面坐标系下三重积分的计算。
\index{柱面坐标系下三重积分的计算}
%%
%%%%%%%%%%% %%%%%%%%%%%%%%%%%%%%%%%%%%
% 内容分割线 2020.07.16
%%%%%%%%%%% %%%%%%%%%%%%%%%%%%%%%%%%%%
%%

\page

3. 球面坐标系。
\index{球面坐标}

(1) 被积函数中含有 $f(x^2+y^2+z^2)$ 或 $f(x^2+y^2)$, 积分区域为球或球的一部分、锥或锥的一部分时,用球面坐标系。\medskip

(2) 用三族面将空间 $\Omega$ 切分成一个个的微元体,其中 
$\theta=\theta_0$ (常数)为过 $z$ 轴的半平面, 其与 $xoz$ 面正向夹角为 $\theta_0$,  $0\leq \theta_0 \leq 2\pi$。\medskip

$\phi=\phi_0$ (常数)为以 $z$ 轴为中心轴的圆锥面, 其顶点为原点, 半顶角为 $\phi_0$, $0\leq \phi_0\leq \pi$。
\index{圆锥面}
\medskip

$r=r_0$(常数)为球心在原点的球面, 其半径为 $r_0$, $0\leq r_0 < +\infty$。
\index{球面}

\medskip

此微元体近似为长方体, 其三组边界面分别是:\medskip

过 $z$ 轴且与 $xoz$ 面正向夹角为 $\theta$ 与 $\theta+d\theta$ 的半平面。 \medskip

以 $z$ 轴为中心轴, 半顶角为 $\phi$ 与 $\phi+d\phi$ 的圆锥面。\medskip

以原点为圆心, 半径为 $r$ 与 $r+dr$ 的球面。\medskip

它的体积元素即为三个边长 $dr$, $rd\phi$ 与 $r\sin\phi d\theta$ 的乘积, 即
$dv=r^2\sin\phi d\theta d\phi dr$。

\page 

(3) 从原点出发画一条半射线 (取值范围 $(0,+\infty)$ ), 先碰到 $\Omega$, 记 $r_1(\phi, \theta)$, 后离开 $\Omega$, 记 $r_2(\phi, \theta)$。\medskip

顶点在原点, 以 $z$ 轴为中心轴的圆锥面半顶角 (取值范围 $[0,\pi]$ ), 先碰到 $\Omega$, 记 $\phi_1( \theta)$, 
后离开 $\Omega$, 记 $\phi_2( \theta)$。\medskip

过 $z$ 轴的半平面与 $xoz$ 面正向夹角(取值范围 $[0,2\pi]$ ), 先碰到 $\Omega$, 记 $\theta_1$,
 后离开 $\Omega$, 记 $\theta_2$。\medskip

于是有

\startformula
\startalign
  \NC   \NC \iiint_\Omega f(x,y,z)dv \NR
  \NC = \NC  \iiint_\Omega f(r\sin\phi\cos\theta, r\sin\phi\sin\theta,r\cos \phi) r^2 \sin\phi dr d\phi d\theta \NR
  \NC = \NC \int_{\theta_1}^{\theta_2} d\theta \int_{\phi_1(\theta)}^{\phi_2(\theta)} d\phi \int_{r_1(\phi,\theta)}^{r_2(\phi,\theta)} f(r\sin\phi\cos\theta, r\sin\phi\sin\theta,r\cos \phi) r^2 \sin\phi dr  \NR
\stopalign
\stopformula

%%
%%%%%%%%%%% %%%%%%%%%%%%%%%%%%%%%%%%%%
% 内容分割线 2020.07.16
%%%%%%%%%%% %%%%%%%%%%%%%%%%%%%%%%%%%%
%%

\page 

注意:

1. 关于积分区域 $\Omega$。 常见的空间图形有(16个)\index{空间图形}

(1) $z=\sqrt{a^2-x^2-y^2} \,,\qquad a>0$

(2) $\frac{x}{a}+\frac{y}{b}+\frac{z}{c}=1,\quad a,b,c>0$

(3) $z=\sqrt{x^2+y^2}$

(4) $x^2+y^2=z^2$

(5) $z=x^2+y^2$


\blackrule[color=black,width=\textwidth,height=.01cm,depth=0cm] % 分隔线

(6) $x^2+y^2=a^2$, $z\geq 0$, $a>0$

(7) $\sqrt{x}+\sqrt{y}+\sqrt{z}=\sqrt{a}$, $a>0$

(8) $z=xy$

(9) $z=xy$, $y=x$, $x=1$, $z=0$

(10) $z=xy$, $x+y=1$ , $z=0$


\blackrule[color=black,width=\textwidth,height=.01cm,depth=0cm] % 分隔线

\page

\blackrule[color=black,width=\textwidth,height=.01cm,depth=0cm] % 分隔线

(11) $z=xy$, $x^2+y^2=a^2$ $(a>0)$

(12) $z=x^2+y^2$, $z=1-x^2$

(13) $x^2+y^2=1$, $z=1-x^2$

(14) $x^2+(y-z)^2 = (1-z)^2$, $0\leq z\leq 1$

(15) $z=x^2+y^2$, $x^2+(y-1)^2=1$

(16) $z=2(x^2+y^2)$, $x^2+y^2=x$, $x^2+y^2=2x$, $z=0$

%%
%%%%%%%%%%% %%%%%%%%%%%%%%%%%%%%%%%%%%
% 内容分割线 2020.07.16
%%%%%%%%%%% %%%%%%%%%%%%%%%%%%%%%%%%%%
%%
\page 

2. 关于被积函数 $f(x,y,z)$。

\index{2. 关于被积函数 $f(x,y,z)$}

积分区域 $\Omega$ 复杂, 则被积函数简单。

积分区域 $\Omega$ 简单, 则被积函数复杂。

\medskip

3. 换元法。
\index{换元法}
\index{换元法}

若 $x=x(u,v,w)$, $y=y(u,v,w)$, $z=z(u,v,w)$ 是空间 $(x,y,z)$ 到空间 $(u,v,w)$ 的一一映射, 有一阶连续偏导数, 且
\startformula
\frac{\partial(x,y,z)}{\partial(u,v,w)} =
\left |
\startmatrix
  \NC \frac{\partial x}{\partial u} \NC \frac{\partial x}{\partial v} \NC \frac{\partial x}{\partial w} \NR
  \NC \frac{\partial y}{\partial u} \NC \frac{\partial y}{\partial v} \NC \frac{\partial y}{\partial w} \NR
  \NC \frac{\partial z}{\partial u} \NC \frac{\partial z}{\partial v} \NC \frac{\partial z}{\partial w}  \NR
\stopmatrix
\right |
\neq 0
\stopformula
则有

\startformula
\startalign
  \NC \NC \iiint_{\Omega_{xyz}} f(x,y,z) dxdydz \NR
  \NC = \NC \iiint_{\Omega_{uvw}} f[x(u,v,w),y(u,v,w),z(u,v,w)]\cdot \Big | \frac{\partial (x,y,z)}{\partial (u,v,w)} \Big | dudvdw \NR
\stopalign
\stopformula
%%
%%%%%%%%%%% %%%%%%%%%%%%%%%%%%%%%%%%%%
% 内容分割线 2020.07.16
%%%%%%%%%%% %%%%%%%%%%%%%%%%%%%%%%%%%%
%%
%%
\page 

特别地, (1) 若 $x=r\cos\theta$, $y=r\sin\theta$, $z=z$ ,
\startformula
J = 
\frac{\partial(x,y,z)}{\partial(r,\theta,z)} =
\left |
\startmatrix
  \NC \frac{\partial x}{\partial r} \NC \frac{\partial x}{\partial \theta} \NC \frac{\partial x}{\partial z} \NR
  \NC \frac{\partial y}{\partial r} \NC \frac{\partial y}{\partial \theta} \NC \frac{\partial y}{\partial z} \NR
  \NC \frac{\partial z}{\partial r} \NC \frac{\partial z}{\partial \theta} \NC \frac{\partial z}{\partial z}  \NR
\stopmatrix
\right |
=
\left |
\startmatrix
  \NC \cos\theta \NC -r\sin\theta \NC 0 \NR
  \NC \sin\theta \NC r\cos\theta \NC 0 \NR
  \NC 0 \NC 0 \NC 1 \NR
\stopmatrix
\right |
=r\neq 0
\stopformula
则有直角坐标系到柱面坐标系的换元, 即
\index{直柱换元}

\startformula
\startalign
  \NC \iiint_{\Omega_{xyz}} f(x,y,z) dxdydz
      = \NC \iiint_{\Omega_{r\theta z}} f(r\cos\theta,r\sin\theta,z)\cdot |J| drd\theta dz \NR
  \NC = \NC \iiint_{\Omega_{r\theta z}} f(r\cos\theta,r\sin\theta,z)\cdot r drd\theta dz \NR
\stopalign
\stopformula
%%
%%%%%%%%%%% %%%%%%%%%%%%%%%%%%%%%%%%%%
% 内容分割线 2020.07.16
%%%%%%%%%%% %%%%%%%%%%%%%%%%%%%%%%%%%%
%%
%%
\page 

特别地, (2) 若 $x=r\sin\phi\cos\theta$, $y=r\sin\phi\sin\theta$, $z=r\cos\phi$ ,
\startformula
J = 
\frac{\partial(x,y,z)}{\partial(r,\theta,\phi)} =
\left |
\startmatrix
  \NC \frac{\partial x}{\partial r} \NC \frac{\partial x}{\partial \theta} \NC \frac{\partial x}{\partial \phi} \NR
  \NC \frac{\partial y}{\partial r} \NC \frac{\partial y}{\partial \theta} \NC \frac{\partial y}{\partial \phi} \NR
  \NC \frac{\partial z}{\partial r} \NC \frac{\partial z}{\partial \theta} \NC \frac{\partial z}{\partial \phi}  \NR
\stopmatrix
\right |
=
\left |
\startmatrix
  \NC \cos\theta \NC -r\sin\theta \NC 0 \NR
  \NC \sin\theta \NC r\cos\theta \NC 0 \NR
  \NC 0 \NC 0 \NC 1 \NR
\stopmatrix
\right |
=r^2\sin\phi\neq 0
\stopformula
则有直角坐标系到球面坐标系的换元, 即
\index{直球换元}

\startformula
\startalign
  \NC \NC \iiint_{\Omega_{xyz}} f(x,y,z) dxdydz \NR
  \NC = \NC \iiint_{\Omega_{r\phi\theta }} f(r\sin\phi\cos\theta,r\sin\phi\sin\theta,r\cos\phi)\cdot |J| drd\phi d\theta  \NR
  \NC = \NC \iiint_{\Omega_{r\phi\theta }} f(r\sin\phi\cos\theta,r\sin\phi\sin\theta,r\cos\phi)\cdot r^2\sin\phi drd\phi d\theta  \NR
\stopalign
\stopformula
%%
%%%%%%%%%%% %%%%%%%%%%%%%%%%%%%%%%%%%%
% 内容分割线 2020.07.16
%%%%%%%%%%% %%%%%%%%%%%%%%%%%%%%%%%%%%
%%


\Topic{56 三重积分的应用 ({\bf \darkgreen 仅数一})}
\index{56 三重积分的应用 }
\index{P367}

1. 体积。
\index{体积}

对于空间物体 $\Omega$, 其体积计算公式为
\startformula
V=\iiint_\Omega dv
\stopformula

2. 总质量。
\index{质量}

对空间物体 $\Omega$, 其体积密度为 $\rho (x,y,z)$, 则其总质量计算公式为
\startformula
m=\iiint_\Omega \rho(x,y,z) dv
\stopformula

%\page 

3. 重心。
\index{重心}

对空间物体 $\Omega$, 其体积密度为 $\rho(x,y,z)$, 则重心 $(\bar{x},\bar{y},\bar{z})$ 的计算公式为
\startformula
\bar{x}=\frac{\iiint_\Omega x\rho(x,y,z)dv}{\iiint_\Omega \rho(x,y,z)dv},\,
\bar{y}=\frac{\iiint_\Omega y\rho(x,y,z)dv}{\iiint_\Omega \rho(x,y,z)dv},
\stopformula

\startformula
\bar{z}=\frac{\iiint_\Omega z\rho(x,y,z)dv}{\iiint_\Omega \rho(x,y,z)dv}
\stopformula

注意: 当 $\rho(x,y,z)$ 为常数时, 即为形心。
\index{形心}

%\page 

4. 转动惯量。
\index{转动惯量}

对于空间物体 $\Omega$, 其体积密度为 $\rho(x,y,z)$, 则该物体对 $x$ 轴、 $y$ 轴、 $z$ 轴和原点 $O$ 的转动惯量的计算公式分别为

\startformula
I_x=\iiint_\Omega (y^2+z^2) \rho(x,y,z)dv
\stopformula 

\startformula
I_y=\iiint_\Omega (z^2+x^2) \rho(x,y,z)dv
\stopformula

\startformula
I_z=\iiint_\Omega (x^2+y^2) \rho(x,y,z)dv,
\stopformula

\startformula
I_O=\iiint_\Omega (x^2+y^2+z^2) \rho(x,y,z)dv
\stopformula

\page 
%%
%%%%%%%%%%% %%%%%%%%%%%%%%%%%%%%%%%%%%
% 内容分割线 2020.07.16
%%%%%%%%%%% %%%%%%%%%%%%%%%%%%%%%%%%%%
%%

5. 引力。
\index{引力}

对于空间物体 $\Omega$, 其体积密度为 $\rho(x,y,z)$, 则该物体对点 $(x_0,y_0,z_0)$ 处的质量为 $m$ 的质点的引力 $(F_x,F_y,F_z)$ 的计算公式为

\startformula
F_x=Gm\iiint_\Omega \frac{\rho(x,y,z)(x-x_0)}{[(x-x_0)^2+(y-y_0)^2+(z-z_0)^2]^{\frac{3}{2}}}dv
\stopformula

\startformula
F_y=Gm\iiint_\Omega \frac{\rho(x,y,z)(y-y_0)}{[(x-x_0)^2+(y-y_0)^2+(z-z_0)^2]^{\frac{3}{2}}}dv
\stopformula

\startformula
F_z=Gm\iiint_\Omega \frac{\rho(x,y,z)(z-z_0)}{[(x-x_0)^2+(y-y_0)^2+(z-z_0)^2]^{\frac{3}{2}}}dv
\stopformula

\page 

%%
%%%%%%%%%%% %%%%%%%%%%%%%%%%%%%%%%%%%%
% 内容分割线 2020.07.16
%%%%%%%%%%% %%%%%%%%%%%%%%%%%%%%%%%%%%
%%

\Topic{57 例8 2019数学二解答题第17题10分, 2019 数学三 解答题 第17题 10分}
\index{57 例8 }
\index{2019数学三解答题第17题10分}
\index{2019数学三第17题10分}
\index{2019数学二解答题第17题10分}
\index{2019数学二第17题10分}

已知 $y(x)$ 满足微分方程
\startformula
y'-xy=\frac{1}{2\sqrt{x}}e^{\frac{x^2}{2}},\quad y(1)=\sqrt{e},
\stopformula

(1) 求$y(x)$,

(2) 求平面区域 $D$ 绕 $x$ 轴旋转所成旋转体的体积, 其中
\startformula
D=\{(x,y)|1\leq x\leq 2, 0\leq y \leq y(x)\}
\stopformula

\medskip

{\bf 考点:}三重积分的应用之旋转体的体积。 一阶线性微分方程的通解公式。 截面法计算三重积分。
\index{截面法}
\index{体积}
\index{旋转体}

%%
%%%%%%%%%%% %%%%%%%%%%%%%%%%%%%%%%%%%%
% 内容分割线 2020.07.16
%%%%%%%%%%% %%%%%%%%%%%%%%%%%%%%%%%%%%
%%

\Topic{58 例 01}
\index{58 例 01}\index{p367, 例18.1}
计算三重积分 $\iiint_\Omega (x+z)dv$, 其中 $\Omega$ 是由曲面 $z=\sqrt{x^2+y^2}$ 与 $z=\sqrt{1-x^2-y^2}$ 所围成的区域。

\medskip

{\bf 考点:}三重积分的应用之球面坐标。对称性。
\index{球面坐标}
\index{对称性}

 
%\page

解答过程:法一:
由 $\Omega$ 关于 $yoz$ 坐标面对称, 知 $\iiint_\Omega x dv=0$, 所以

\startformula
\startalign
  \NC  \NC \iiint_{\Omega} (x+z)dv \NR
  \NC  \NC  = \int_0^{2\pi}d\theta \int_0^{\frac{\pi}{4}}d \phi \int_0^1 r^3\cos\phi \sin\phi dr \NR
  \NC  \NC  = 2\pi\cdot \frac{1}{2}\sin^2\phi |_0^{\frac{\pi}{4}}\cdot \frac{1}{4} \NR
  \NC  \NC  = \frac{\pi}{8} \NR
\stopalign
\stopformula

\medskip

法二:直角坐标系下先二后一法(定限截面法)。
\index{先二后一}

\hfill 例01毕。
%%
%%%%%%%%%%% %%%%%%%%%%%%%%%%%%%%%%%%%%
% 内容分割线 2020.07.16
%%%%%%%%%%% %%%%%%%%%%%%%%%%%%%%%%%%%%
%%

\Topic{59 例4 2019数学一解答题第19题10分}
\index{59 例 04}
\index{59 例4 }
\index{2019数学一解答题第19题10分}

设 $\Omega$ 是由锥面 $x^2+(y-z)^2=(1-z)^2$, $0\leq z \leq 1$, 与平面 $z=0$ 所围成的锥体, 求$\Omega$ 的形心坐标。

\medskip

{\bf 考点:}三重积分的应用之形心坐标。  三重积分的换元法。
\index{换元法}
\index{形心坐标}


\page

解答过程: 由于图形关于 $yoz$ 坐标面左右对称, 所以 $\bar{x}=0$,
\medskip 

\startcode
(* Wolfram Mathematica 11.3.0.0 *)
In[1]: Plot3D[ (x^2+y^2-1)/(2y-2), {x,-4,4}, {y,-4,4}] Shift+ Enter
\stopcode
 
\index{Wolfram Mathematica}



\medskip

考虑三重积分换元公式, 令 $x=u$, $y-z=v$, $1-z=w$, 可得 $\Omega:\rightarrow \Omega_{uvw}:$  $u^2+v^2=w^2$, $0\leq w \leq 1$, $x=u$, $y=v+1-w$, $z=1-w$, 所以有

\startformula
| \, J\, |=\left|\frac{\partial (x,y,z)}{\partial (u,v,w)} \right |= 1
\stopformula

\startformula
\startalign
  \NC \iiint_{\Omega}dv \NC =\iiint_{\Omega_{uvw}}dudvdw = \int_0^1dw\iint_{u^2+v^2\leq w^2} dudv \NR
  \NC  \NC = \int_0^1 \pi w^2dw = \frac{\pi}{3}\NR
\stopalign
\stopformula

\startformula
\startalign
  \NC \iiint_{\Omega}ydv \NC =\iiint_{\Omega_{uvw}}(v+1-w)dudvdw  \NR
  \NC  \NC =\iiint_{\Omega_{uvw}}vdw + \iiint_{\Omega_{uvw}}(1-w)dudvdw\NR
  \NC  \NC  = \frac{\pi}{12} \NR
\stopalign
\stopformula

\startformula
\startalign
  \NC \iiint_{\Omega}zdv \NC =\iiint_{\Omega_{uvw}}(1-w)dudvdw \NR
  \NC  \NC = \int_0^1 (1-w) dw\iint_{u^2+v^2\leq w^2} dudv \NR
  \NC  \NC = \frac{\pi}{12} \NR
\stopalign
\stopformula

由三维立体的形心坐标计算公式, 有 $\bar{y}=\bar{z}=\displaystyle\frac{1}{4}$,  即形心坐标为 $\left(0,\displaystyle\frac{1}{4},\frac{1}{4}\right)$.

形心坐标 \hfill 例04毕。

%%
%%%%%%%%%%% %%%%%%%%%%%%%%%%%%%%%%%%%%
% 内容分割线 2020.07.16
%%%%%%%%%%% %%%%%%%%%%%%%%%%%%%%%%%%%%
%%

\Topic{60 例6 2019数学二解答题第17题10分}
\index{60 例 06}
\index{60 例6 }
\index{2019数学二解答题第17题10分}

{\bf\red 数二} 已知 $y(x)$ 满足微分方程
\startformula
y'-xy=\frac{1}{2\sqrt{x}}e^{\frac{x^2}{2}},\quad y(1)=\sqrt{e},
\stopformula

(1) 求 $y(x)$,

(2) 求平面区域 $D$ 绕 $x$ 轴旋转所成旋转体的体积, 其中
\startformula
D=\{(x,y)|1\leq x\leq 2, 0\leq y \leq y(x)\}
\stopformula

\medskip

{\bf 考点:} 三重积分的应用之旋转体的体积。 一阶线性微分方程的通解公式。 截面法计算三重积分。
\index{截面法}
\index{体积}
\index{旋转体}
\index{通解公式}

\page

解答过程: 由一阶线性微分方程的通解公式, 有
\startformula
\startalign
  \NC  y(x) \NC  =e^{-\int -x dx}\left[\int \frac{1}{2\sqrt{x}} e^{\frac{x^2}{2}} e^{\int -x dx}dx+C\right] \NR
  \NC  \NC = e^{\frac{x^2}{2}}(\sqrt{x}+C) \NR
\stopalign
\stopformula
代入初值 $y(1)=\sqrt{e}$, 有 $C=0$, 即 $y(x)=\sqrt{x}e^{\frac{x^2}{2}}$

旋转曲面方程为 $y^2+z^2=xe^{x^2}$, $0\leq x \leq 2$, 用截面法计算三重积分, 有
\medskip


\startcode
(* Wolfram Mathematica 11.3.0.0 *)
In[1]:= Plot[ Sqrt[x] Exp[x^2/2], {x,0,1 }] Shift+Enter
\stopcode
 
\index{Wolfram Mathematica}


\medskip
\startformula
\startalign
  \NC  V_x  \NC = \iiint_{\Omega}dv=\int_1^2 dx\iint_{y^2+z^2\leq xe^{x^2}}dydz \NR
  \NC  \NC  = \int_0^1\pi xe^{x^2}dx
=\frac{\pi}{2}(e^4-e) \NR
\stopalign
\stopformula


旋转体的体积。 \hfill 例06毕。
%%
%%%%%%%%%%% %%%%%%%%%%%%%%%%%%%%%%%%%%
% 内容分割线 2020.07.16
%%%%%%%%%%% %%%%%%%%%%%%%%%%%%%%%%%%%%
%%

\Topic{61 例 02}
\index{61 例 02}
\index{p368, 例18.2}

若 $\Omega: x^2+y^2+z^2\leq R^2$, $x\geq 0, y\geq 0, z\geq 0$,
计算
\startformula
I=\iiint_\Omega (x+y+z)dxdydz
\stopformula

\medskip

{\bf 考点:} 三重积分的应用之轮换对称性。
\index{轮换对称性}



%\page

解答过程:
由轮换对称性,有
\startformula
\startalign
  \NC I \NC = 3\iiint_{\Omega} z dxdydz \NR
  \NC  \NC = 3\int_0^{\frac{\pi}{2}}d\theta \int_0^{\frac{\pi}{2}}d\phi \int_0^R r^3\cos\phi \sin\phi dr \NR
  \NC  \NC = \frac{3\pi}{2}\int_0^{\frac{\pi}{2}}\sin\phi \cos\phi d\phi\int_0^Rr^3dr \NR
  \NC    \NC  =\frac{3}{16}\pi R^4
\stopalign
\stopformula

三重积分的应用之轮换对称性。 \hfill 例02毕。
%%
%%%%%%%%%%% %%%%%%%%%%%%%%%%%%%%%%%%%%
% 内容分割线 2020.07.16
%%%%%%%%%%% %%%%%%%%%%%%%%%%%%%%%%%%%%
%%

\Topic{62 例 03}
\index{62 例 03}\index{p368, 例18.3}
设 $\Omega$ 由拋物面 $y=\sqrt{x}$, $x+z=\frac{\pi}{2}$, $y=0$, $z=0$ 围成,计算

\startformula
I=\iiint_{\Omega} \frac{y\sin x}{x}dv
\stopformula

\medskip

{\bf 考点:} 三重积分的应用之先一后二法(投影穿线法)。
\index{先一后二}
\index{投影穿线法}



\page

解答过程:

\startformula
\startalign
  \NC I \NC = \iint_{D_{xy}} d\sigma \int_0^{\frac{\pi}{2}-x}\frac{y\sin x }{x} dz \NR
  \NC  \NC  = \int_0^{\frac{\pi}{2}} dx \int_0 ^ {\sqrt{x}} dy \int_0^{\frac{\pi}{2}-x} \frac{y\sin x }{x} dz   \NR
  \NC  \NC = \int_0^{\frac{\pi}{2}} (\frac{\pi}{2}-x)\frac{\sin x }{x} dx \int_0^{\sqrt{x}} y dy  
            \NR
  \NC  \NC = \frac{1}{2}\int_0^{\frac{\pi}{2}}(\frac{\pi}{2}-x)\sin x dx 
           = \frac{\pi}{4}-\frac{1}{2} \NR
\stopalign
\stopformula

三重积分的应用之先一后二法(投影穿线法)。  \hfill 例03毕。
%%
%%%%%%%%%%% %%%%%%%%%%%%%%%%%%%%%%%%%%
% 内容分割线 2020.07.16
%%%%%%%%%%% %%%%%%%%%%%%%%%%%%%%%%%%%%
%%

\Topic{63 例 04}
\index{63 例 04}
\index{p368, 例18.4}

设 $\Omega$ 由 $x^2+(y-1)^2=1$, $8z=x^2+y^2$, $z=0$ 所围成的区域, 计算
\startformula
I=\iiint_{\Omega}\sqrt{x^2+y^2} dv
\stopformula

\medskip

{\bf 考点:} 二重积分的应用之先一后二法(投影穿线法)。 极坐标。 Wallis公式。
\index{Wallis公式}
\index{极坐标}
\index{先一后二}


\page

解答过程:
\startformula
\startalign
  \NC I  \NC = \iint_{D_{xy}}d\sigma \int_0^{\frac{x^2+y^2}{8}} \sqrt{x^2+y^2} dz \qquad (\text{极坐标})\NR
  \NC    \NC = \int_0^{\pi} d\theta \int_0^{2\sin\theta}r dr \int_0^{\frac{r^2}{8}} r dz \NR
  \NC    \NC = \frac{1}{8}\int_0^{\pi} d\theta\int_0^{2\sin\theta}r^4dr \NR
  \NC    \NC = \frac{4}{5}\int_0^{\pi} \sin^5\theta d\theta \NR
  \NC    \NC = \frac{4}{5}(2\times \frac{4}{5}\times \frac{2}{3}) 
             =  \frac{64}{75} \qquad (Wallis\text{公式})\NR
\stopalign
\stopformula
先一后二法(投影穿线法) \hfill 例05毕。



%%
%%%%%%%%%%% %%%%%%%%%%%%%%%%%%%%%%%%%%
% 内容分割线 2020.07.16
%%%%%%%%%%% %%%%%%%%%%%%%%%%%%%%%%%%%%
%%

\Topic{64 例09 2018数学一解答题第17题10分}
\index{64 例 09}\index{09 例9 }
\index{2018数学一解答题第17题10分}

设曲面 $\Sigma:x=\sqrt{1-3y^2-3z^2}$ 取前侧, 求 
\startformula
I=\iint_{\Sigma} xdydz+(y^3+z)dxdz+z^3dxdy
\stopformula

\medskip

{\bf 考点:} 第二类曲面积分、 高斯公式、 三重积分的计算之先二后一法。 极坐标。 偏导数。 椭球体。
\index{第二类曲面积分}
\index{高斯公式}
\index{先二后一}
\index{极坐标}
\index{偏导数}
\index{椭球体}
 

\page

解答过程:


\startcode
(* Wolfram Mathematica 11.3.0.0 *)
In[1]:=Plot3D[Sqrt[1-3*y^2-3*z^2],  {y, -1,1 },  {z, -1, 1 }] Shift+Enter
\stopcode
 
\index{Wolfram Mathematica}


添加辅助面 $\Sigma_1: x=0$, 方向指向x轴的负半轴(负向), 记 $\Sigma +\Sigma_1$ 围成的封闭区域为 $\Omega$, 则 $\Sigma +\Sigma_1$ 的总体的方向取为封闭区域的外侧, 即满足高斯公式条件封闭性和方向性。
 由高斯公式、 截面法计算三重积分, 有
\startformula
\startalign
  \NC I \NC = \oiint_{\Sigma +\Sigma_1} - \iint_{\Sigma_1} \NR
  \NC  \NC = \int_0^1 dx\iint_{y^2+z^2\leq \frac{1-x^2}{3}} (1+3y^2+3z^2)dydz \NR
  \NC  \NC = \int_0^1dx\int_0^{2\pi} d\theta \int_0^{\sqrt{\frac{1-x^2}{3}}} (1+3r^2)rdr \NR
  \NC  \NC = \frac{14\pi}{45} \NR
\stopalign
\stopformula


第二类曲面积分、 高斯公式、 三重积分的计算之截面法。 \hfill 例09毕。

%%
%%%%%%%%%%% %%%%%%%%%%%%%%%%%%%%%%%%%%
% 内容分割线 2020.07.16
%%%%%%%%%%% %%%%%%%%%%%%%%%%%%%%%%%%%%
%%

\Topic{65 例 05}
\index{65例 05}\index{p370, 例18.6}
设 $\Omega$ 是 $x^2+y^2+z^2\leq R^2$, $x^2+y^2+z^2\leq 2Rz (R>0)$ 的公共部分, 计算
\startformula
I=\iiint_{\Omega}z^2 dv
\stopformula

\medskip

{\bf 考点:} 三重积分的应用之球面坐标。
\index{球面坐标}

 

\page

解答过程:
以 $z=\frac{\sqrt{3}}{3}\sqrt{x^2+y^2}$ 为分界面, 将区域 $\Omega$ 分为 $\Omega_1$, $\Omega_2$ 上下两部分, 则

\startformula
\startalign
  \NC I  \NC = \iiint_{\Omega_1}z^2 dv + \iiint_{\Omega_2}z^2 dv\NR
  \NC    \NC = \int_0^{2\pi} d\theta \int_0^{\frac{\pi}{3}} d\phi \int_0^R r^4\cos^2\phi\sin\phi dr \NR
  \NC  \NC \kern2em + \int_0^{2\pi} d\theta \int_{\frac{\pi}{3}}^{\frac{\pi}{2}} d\phi \int_0^{R\cos\phi} r^4\cos^2\phi\sin\phi dr  \NR
  \NC    \NC = ... = \frac{59}{480}\pi R^5 \NR
\stopalign
\stopformula
球面坐标 \hfill 例05毕。

%%
%%%%%%%%%%% %%%%%%%%%%%%%%%%%%%%%%%%%%
% 内容分割线 2020.07.16
%%%%%%%%%%% %%%%%%%%%%%%%%%%%%%%%%%%%%
%%

\Topic{66 例 06}
\index{66 例 06}\index{p370, 例18.7}
设 $f(x)$ 是定义在 $[0,+\infty)$ 上的连续函数, 且满足
\startformula
f(t)=\iiint_{\Omega} f\left(\sqrt{x^2+y^2+z^2}\,\right)dv + t^3
\stopformula
其中 $\Omega: x^2+y^2+z^2\leq t^2$, 求 $f(1)$。
\medskip

{\bf 考点:} 三重积分的应用之积分域空间立体 $\Omega$ 可变。 一阶线性微分方程通解公式。
\index{通解公式}
 

\page

解答过程:

\startformula
\startalign
  \NC f(t) \NC = \int_0^{2\pi} d\theta \int_0^{\pi} d\phi \int_0^t f(r)r^2\sin\phi dr + t^3 \NR
  \NC    \NC = 4\pi\int_0^t r^2f(r)dr +t^3  \NR
\stopalign
\stopformula
由上式知 $f(0)=0$, 上式两边对 $t$ 同时求导得
\startformula
f'(t)=4\pi t^2f(t)+3t^2
\stopformula
由一阶线性微分方程通解公式, 有
\startformula
\startalign
  \NC f(t) \NC = e^{\int 4\pi t^2dt}\left(\int 3t^2 e^{-\int 4\pi t^2 dt} dt +C\right) \NR
\stopalign
\stopformula
%%
%%%%%%%%%%% %%%%%%%%%%%%%%%%%%%%%%%%%%
% 内容分割线 2020.07.16
%%%%%%%%%%% %%%%%%%%%%%%%%%%%%%%%%%%%%
%%

\page

由一阶线性微分方程通解公式, 有
\startformula
\startalign
  \NC f(t) \NC = e^{\int 4\pi t^2dt}\left(\int 3t^2 e^{-\int 4\pi t^2 dt} dt +C\right) \NR
  \NC    \NC = e^{\frac{4}{3}\pi t^3}\left(\int 3t^2 e^{-\frac{4}{3}\pi t^3} dt +C\right) \NR
  \NC \NC = Ce^{\frac{4}{3}\pi t^3}-\frac{3}{4\pi} \NR
\stopalign
\stopformula

由 $f(0)=0$, $C=\frac{3}{4\pi}$, 故
\startformula
f(t) = \frac{3}{4\pi}\left(e^{\frac{4}{3}\pi t^3}-1\right),\, f(1)= \frac{3}{4\pi}\left(e^{\frac{4}{3}\pi }-1\right)
\stopformula

一阶线性微分方程通解公式 \hfill 例06毕。
%%
%%%%%%%%%%% %%%%%%%%%%%%%%%%%%%%%%%%%%
% 内容分割线 2020.07.16
%%%%%%%%%%% %%%%%%%%%%%%%%%%%%%%%%%%%%
%%
\Topic{67 例 07}
\index{67 例 07}\index{p371, 例18.8}
设 $\Omega: \sqrt{x^2+y^2}\leq z \leq 1$, 计算
\startformula
I=\iiint_{\Omega} \left|\sqrt{x^2+y^2+z^2} -1\right|dv
\stopformula

\medskip

{\bf 考点:}三重积分的应用之球面坐标。
\index{球面坐标}

\page

解答过程:
积分区域 $\Omega$ 被球面 $x^2+y^2+z^2=1$ 分成上下两部分, 依次为 $\Omega_1$, $\Omega_2$, 故
\startformula
\startalign
  \NC I \NC = \iiint_{\Omega_1} \left(\sqrt{x^2+y^2+z^2} -1\right) dv +  \iiint_{\Omega_2} \left(1 - \sqrt{x^2+y^2+z^2} \right)dv \NR
  \NC \NC = \int_0^{2\pi}d\theta\int _0^{\frac{\pi}{4}}d\phi \int_1^{\frac{1}{\cos\phi}} (r-1)r^2\sin\phi dr \NR
  \NC  \NC \kern1em + \int_0^{2\pi}d\theta\int _0^{\frac{\pi}{4}}d\phi \int_0^1 (1-r)r^2\sin\phi dr \NR
  \NC \NC = 2\pi\int _0^{\frac{\pi}{4}}\sin\phi d\phi \int_1^{\frac{1}{\cos\phi}} (r-1)r^2 dr +  2\pi\int _0^{\frac{\pi}{4}}\sin\phi d\phi \int_0^1 (1-r)r^2 dr \NR
  \NC \NC = \frac{\pi}{12}(2-\sqrt{2}) + \frac{\pi}{12}(3\sqrt{2}-4) = \frac{\pi}{12}(2-\sqrt{2}) = \frac{\pi}{6}(\sqrt{2} -1) \NR
\stopalign
\stopformula

三重积分的应用之球面坐标。 \hfill 例07毕。
%%
%%%%%%%%%%% %%%%%%%%%%%%%%%%%%%%%%%%%%
% 内容分割线 2020.07.16
%%%%%%%%%%% %%%%%%%%%%%%%%%%%%%%%%%%%%
%%


\Topic{68 例 08}
\index{68 例 08} 
\index{p371, 例18.9}

设 $\Sigma$ 为任意闭曲面,
\startformula
I=\oiint_{\Sigma_{\text{外侧}}} \left(x-\frac{1}{3}x^3\right)dydz - \frac{4}{3}y^3dzdx+\left(3y-\frac{1}{3}z^3\right)dxdy
\stopformula

(1) 证明: $\Sigma$ 为椭球面 $x^2+4y^2+z^2=1$ 时, $I$ 达到最大值。

(2) 求 $I$ 的最大值。
\medskip

{\bf 考点:} 三重积分的应用之 (第二类曲面积分 -> 高斯公式 -> 三重积分)。  椭球面方程。 高斯公式。 第二类曲面积分。
\index{第二类曲面积分}
\index{高斯公式}
\index{广义球面坐标}

 
\page

解答过程:(1)由高斯公式,知
\startformula
I=\iiint_{\Omega}\left(1-x^2-4y^2-z^2\right)dxdydz
\stopformula
其中 $\Omega$ 为 $\Sigma$ 所围的空间区域, 为使 $I$ 最大, 只要 $1-x^2-4y^2-z^2\geq 0$, 即
$\Omega = \{\, (x,y,z)| 1-x^2-4y^2-z^2\geq 0\}$, $\Sigma$ 为 $\Omega$ 的表面, 即为椭球面
$x^2+4y^2+z^2=1$时, $I$最大。

(2)由广义球面坐标系 $x=r\sin\phi\cos\theta$, $y=\frac{1}{2}r\sin\phi\sin\theta$, 
$z=r\cos\phi$, 有
\startformula
\startalign
  \NC I_{max} \NC = \iiint_{x^2+4y^2+z^2\leq 1} \left(1-x^2-4y^2-z^2\right)dv \NR
  \NC \NC = \int_0^{2\pi} d\theta\int_0^{\pi}d\phi\int_0^1 \left(1-r^2\right) \frac{1}{2}r^2\sin\phi dr \NR
  \NC  \NC  = \frac{4\pi}{15} \NR
\stopalign
\stopformula

三重积分的应用之高斯公式。 \hfill 例08毕。
%%
%%%%%%%%%%% %%%%%%%%%%%%%%%%%%%%%%%%%%
% 内容分割线 2020.07.16
%%%%%%%%%%% %%%%%%%%%%%%%%%%%%%%%%%%%%
%%

\Topic{69 例 09}
\index{69 例 09}
\index{p372, 例18.10}
求曲面 $z=2(x^2+y^2)$, $x^2+y^2=x$, $x^2+y^2=2x$ 和 $z=0$ 所围几何体的体积。
\index{体积}

\medskip

{\bf 考点:} 三重积分的应用之几何体的体积。 积分次序先一后二。
\index{先一后二}
 
\page

解答过程: $\Omega = \{(x,y,z)| 0\leq z \leq 2(x^2+y^2), x\leq x^2+y^2 \leq 2x\}$,  所以
\startformula
\startalign
  \NC V \NC = \iiint_\Omega dv =  \iint_{x\leq x^2+y^2 \leq 2x} dxdy \int_0^{2(x^2+y^2)}dz \NR
  \NC \NC = \iint_{x\leq x^2+y^2\leq 2x}2(x^2+y^2)dxdy \NR
  \NC  \NC = \int_{-\frac{\pi}{2}}^{\frac{\pi}{2}}d\theta\int_{\cos\theta}^{2\cos\theta}2r^2\cdot r dr \NR
  \NC \NC =\frac{1}{2}\int_{-\frac{\pi}{2}}^{\frac{\pi}{2}} 15 \cos ^4\theta d\theta \NR
  \NC  \NC= 15\cdot \frac{3}{4} \cdot \frac{1}{2}\cdot  \frac{\pi}{2} = \frac{45\pi}{16} \NR
\stopalign
\stopformula

三重积分的应用之几何体的体积。 \hfill 例09毕。
%%
%%%%%%%%%%% %%%%%%%%%%%%%%%%%%%%%%%%%%
% 内容分割线 2020.07.16
%%%%%%%%%%% %%%%%%%%%%%%%%%%%%%%%%%%%%
%%

\Topic{70 例 10}\index{70 例 10}\index{p372, 例18.11}
设直线 $L$ 过 $A(1,0,0)$, $B(0,1,1)$ 两点, 将 $L$ 绕 $z$ 轴旋转一周得到曲面 $\Sigma$, $\Sigma$ 与平面 $z=0$,  $z=2$ 所围成的立体为 $\Omega$ 。

(1)求曲面 $\Sigma$ 的方程。

(2)求 $\Omega$ 的形心坐标。

\medskip

{\bf 考点:} 三重积分的应用之形心坐标。 积分次序先二后一。 旋转曲面。
\index{形心坐标}
\index{先二后一}
\index{对称性} 

\page

解答过程:(1)直线 $L$ 的方程为 $\frac{x-1}{1}=\frac{y}{-1}=\frac{z}{-1}$, 写成参数式为
\startformula
\startmathcases
   \NC x = 1+t,  \NR
   \NC y  = -t, \qquad (t\text{为参数})\NR
   \NC z  = -t, \NR
\stopmathcases
\stopformula
设 $(x,y,z)$ 为曲面 $\Sigma$ 上的任一点, 则
\startformula
\startmathcases
   \NC x^2+y^2 = (1+t)^2+t^2,  \NR
   \NC z  = -t, \NR
\stopmathcases
\stopformula
所以曲面 $\Sigma$ 的方程为
\startformula
x^2+y^2-2z^2+2z=1
\stopformula
即
\startformula
x^2+y^2=2\left(z-\frac{1}{2}\right)^2+\frac{1}{2}
\stopformula
旋转拋物体面。

\page

(2)设 $\Omega$ 的形心坐标为 $(\bar{x},\bar{y},\bar{z})$, 根据对称性, 得 $\bar{x}=\bar{y}=0$。

设 $D_z=\{(x,y)|x^2+y^2\leq 2z^2-2z+1\}$, 则
\startformula
\Omega:
\startmathcases
   \NC (x,y)\in D_z,  \NR
   \NC 0\leq z\leq 2, \NR
\stopmathcases
\stopformula
所以

\startformula
\startalign
  \NC   \iiint_{\Omega}dxdydz \NC  = \int_0^2dz \iint_{D_z}dxdy \NR
  \NC  \NC = \pi \int_0^2 (2z^2-2z+1)dz \NR
  \NC \NC = \pi\left. \left(\frac{2}{3}z^3-z^2+z\right)\right|_0^2 \NR
  \NC  \NC =\frac{10\pi}{3} \NR
\stopalign
\stopformula
\startformula
\startalign
  \NC \iiint_{\Omega}zdxdydz \NC = \int_0^2zdz \iint_{D_z}dxdy \NR
  \NC  \NC = \pi \int_0^2 z(2z^2-2z+1)dz \NR
  \NC \NC = \pi \left.\left(\frac{1}{2}z^4-\frac{2}{3}z^3+\frac{1}{2}z^2\right)\right|_0^2 \NR
  \NC  \NC =\frac{14\pi}{3} \NR
\stopalign
\stopformula

\startformula
\bar{z}= \frac{\iiint_{\Omega}zdxdydz}{\iiint_{\Omega}dxdydz}=\frac{7}{5}
\stopformula
故 $\Omega$ 的形心坐标为 $\left(0,0,\displaystyle\frac{7}{5}\right)$.

三重积分的应用之形心坐标。 \hfill 例10毕。
%%
%%%%%%%%%%% %%%%%%%%%%%%%%%%%%%%%%%%%%
% 内容分割线 2020.07.16
%%%%%%%%%%% %%%%%%%%%%%%%%%%%%%%%%%%%%
%%

\Topic{71 例 11}
\index{71 例 11}
\index{p373, 例18.12}
设 $\Omega$ 是由锥面 $x^2+(y-z)^2=(1-z)^2 (0\leq z \leq 1)$ 与平面 $z=0$ 围成的锥体,
求 $\Omega$ 的形心坐标。

\medskip

{\bf 考点:}三重积分的应用之形心坐标。积分次序先二后一。
\index{锥面}
\index{形心坐标}
 \index{先二后一}

\page

解答过程:设 $\Omega$ 的形心坐标为 $(\bar{x},\bar{y},\bar{z})$, 因为 $\Omega$ 关于 $yoz$ 平面对称, 所以 $\bar{x}=0$。

对于 $0\leq z \leq 1$, 记 $D_z=\{(x,y)|x^2+(y-z)^2\leq (1-z)^2\}$, 因为

\startformula
\startalign
  \NC  V \NC = \iiint_{\Omega}dxdydz = \int_0^1 dz \iint_{D_z}dxdy \NR
  \NC  \NC = \int_0^1\pi(1-z)^2 dz=\frac{\pi}{3} \NR
\stopalign
\stopformula

令 $x=r\cos\theta$, $y=z+r\sin\theta$, 则
\startformula
\startalign
  \NC \iiint_{\Omega}ydxdydz \NC =  \int_0^1 dz \iint_{D_z}ydxdy \NR
  \NC  \NC  = \int_0^1dz\int_0^{2\pi}d\theta\int_0^{1-z}(z+r\sin\theta)rdr \NR
  \NC \NC = \int_0^1 \pi z (1-z)^2dz = \frac{\pi}{12} \NR 
\stopalign
\stopformula

\startformula
\startalign
   \NC  \iiint_{\Omega}zdxdydz \NC =  \int_0^1 dz \iint_{D_z}zdxdy  \NR
   \NC  \NC = \int_0^1 \pi z (1-z)^2dz = \frac{\pi}{12} \NR
\stopalign
\stopformula


所以
\startformula
\bar{y}=\frac{\iiint_{\Omega}ydxdydz}{V}=\frac{1}{4}, \quad 
\bar{z}=\frac{\iiint_{\Omega}zdxdydz}{V}=\frac{1}{4},
\stopformula
故 $\Omega$ 的形心坐标为 $(0,\displaystyle\frac{1}{4},\frac{1}{4})$

三重积分的应用之形心坐标。 \hfill 例11毕。
%%
%%%%%%%%%%% %%%%%%%%%%%%%%%%%%%%%%%%%%
% 内容分割线 2020.07.16
%%%%%%%%%%% %%%%%%%%%%%%%%%%%%%%%%%%%%
%%

\Topic{72 例 12}\index{72 例 12}\index{p374, 例18.13}
设物体由曲面 $z=x^2+y^2$ 和 $z=2x$ 所围成, 其上各点的密度 $\mu$ 等于该点到 $xoz$ 平面的距离的平方, 求该物体对 $z$ 轴的转动惯量。

\medskip

{\bf 考点:} 三重积分的应用之转动惯量。 积分次序先一后二。 柱面坐标。 Wallis公式。
\index{Wallis公式}
\index{先一后二}
\index{转动惯量}

\page

解答过程: 这里 $\mu = y^2$, 并注意到物体占有空间区域 $\Omega$ 在 $xoy$ 平面上的投影域为
$D: (x-1)^2+y^2\leq 1$, 从而所求转动惯量为
\startformula
\startalign
  \NC I_z \NC = \iiint_{\Omega} y^2(x^2+y^2)dv \NR
  \NC  \NC = 2\int_0^{\frac{\pi}{2}}d\theta\int_0^{2\cos\theta}r^5\sin^2\theta dr \int_{r^2}^{2r\cos\theta}dz \NR
  \NC \NC = \frac{64}{7}\int_0^{\frac{\pi}{2}} (\cos^8\theta - \cos^{10}\theta) d\theta \NR
  \NC \NC = \frac{64}{7}\left(\frac{7}{8}\cdot \frac{5}{6}\cdot \frac{3}{4}\cdot \frac{1}{2}\cdot \frac{\pi}{2}-\frac{9}{10}\cdot \frac{7}{8}\cdot \frac{5}{6}\cdot \frac{3}{4}\cdot \frac{1}{2}\cdot \frac{\pi}{2}\right) = \frac{\pi}{8} \NR
\stopalign
\stopformula

三重积分的应用之形心坐标。 Wallis公式。 \hfill 例12毕。
%%
%%%%%%%%%%% %%%%%%%%%%%%%%%%%%%%%%%%%%
% 内容分割线 2020.07.16
%%%%%%%%%%% %%%%%%%%%%%%%%%%%%%%%%%%%%
%%

\Topic{73 例 13}
\index{73 例 13}
\index{p374, 例18.14}

求由球面 $z=\sqrt{4-x^2-y^2}$ 与平面 $z=1$ 所围成的体积密度为 $1$ 的均匀体 $\Omega$ 对原点处单位质点的引力。

\medskip

{\bf 考点:} 三重积分的应用之引力。 积分次序先二后一。
\index{引力}
\index{先二后一}
\index{对称性}
\index{形心坐标}
\index{Wallis公式}
\index{球面}

\page

解答过程: 引力常数 $G$, $z_0=0$, $r=\sqrt{x^2+y^2+z^2}$, 由对称性,有 $F_x=F_y=0$,

\startformula
\startalign
  \NC F_z \NC = G\iiint_{\Omega} \frac{z-z_0}{r^3} dxdydz \NR
  \NC  \NC = G\iiint_{\Omega} z(x^2+y^2+z^2)^{-\frac{3}{2}}dxdydz \NR
\stopalign
\stopformula

因为积分区域 $\Omega$ 在 $z$ 轴上的投影区间为 $[1,2]$, 垂直于 $z$ 轴且竖标为 $z$ 的平面截 $\Omega$ 所得到的平面区域为 $D_z: x^2+y^2 \leq 4-z^2$, 故
\startformula
\startalign
  \NC F_z \NC  = G\int_1^2 dz \iint_{D_z}z (x^2+y^2+z^2)^{-\frac{3}{2}}dxdy \NR
  \NC \NC = G\int_1^2dz \int_0^{2\pi}d\theta \int_0^{\sqrt{4-z^2}} z(r^2+z^2)^{-\frac{3}{2}} r dr \NR
  \NC  \NC = G\pi \int_1^2 (2-z)dz = \frac{\pi}{2}G \NR
\stopalign
\stopformula
因此, 所求引力为 $F=\left(F_x, F_y, F_z\right) = \left(0,0,\frac{\pi G}{2}\right)$

三重积分的应用之形心坐标。 Wallis公式。 \hfill 例13毕。

\page

%%----------------------------------
%%----------------------------------
\startuseMPgraphic{FunnyFrame}
  picture p ; numeric o ; path a, b ; pair c ;
  p := textext.rt(\MPstring{FunnyFrame}) ;
  a := unitsquare xyscaled(1.0OverlayWidth,OverlayHeight) ; %指定宽度1.15倍
  o := BodyFontSize ;
  p := p shifted (2o,OverlayHeight-ypart center p) ; %标题右移的距离
  drawoptions (withpen pencircle scaled 1pt withcolor .625red) ;
  b := a randomized (o/2) ;
  fill b withcolor .98white ; draw b ; %方框内部的填充色
  b := (boundingbox p) randomized (o/8) ;
  fill b withcolor .95white ; 
  draw b ;draw p withcolor black;
  setbounds currentpicture to a ;
\stopuseMPgraphic

\def\StartFrame{\startFunnyText}
\def\StopFrame {\stopFunnyText }
\defineoverlay[FunnyFrame][\useMPgraphic{FunnyFrame}]

\defineframedtext[FunnyText][frame=off,background=FunnyFrame,width=fit] %fit, \textwidth

\def\FrameTitle#1%
  {\setMPtext{FunnyFrame}{\hbox spread 1em{\hss\strut#1\hss}}}
%%----------------------------------
%%----------------------------------
\FrameTitle{\bf 73 例 13}


\StartFrame\bfd 
求由球面 $z=\sqrt{4-x^2-y^2}$ 与平面 $z=1$ 所围成的体积密度为 $1$ 的均匀体 $\Omega$ 对原点处单位质点的引力。

\StopFrame

%%----------------------------------
%%----------------------------------
%%----------------------------------
%%----------------------------------
\FrameTitle{\bf 73 例 13,解答 }


\StartFrame\bf 
 引力常数 $G$, $z_0=0$, $r=\sqrt{x^2+y^2+z^2}$, 由对称性,有 $F_x=F_y=0$,
\startformula
F_z = G\iiint_{\Omega} \frac{z-z_0}{r^3} dxdydz = G\iiint_{\Omega} z(x^2+y^2+z^2)^{-\frac{3}{2}}dxdydz
\stopformula

因为积分区域 $\Omega$ 在 $z$ 轴上的投影区间为 $[1,2]$, 垂直于 $z$ 轴且竖标为 $z$ 的平面截 $\Omega$ 所得到的平面区域为 $D_z: x^2+y^2 \leq 4-z^2$, 故
\startformula
\startalign
  \NC F_z \NC  = G\int_1^2 dz \iint_{D_z}z (x^2+y^2+z^2)^{-\frac{3}{2}}dxdy \NR
  \NC \NC = G\int_1^2dz \int_0^{2\pi}d\theta \int_0^{\sqrt{4-z^2}} z(r^2+z^2)^{-\frac{3}{2}} r dr  = G\pi \int_1^2 (2-z)dz = \frac{\pi}{2}G
\stopalign
\stopformula
因此, 所求引力为 $F=(F_x, F_y, F_z)=(0,0,\frac{\pi G}{2})$

三重积分的应用之形心坐标。 Wallis公式。 \hfill 例13毕。

\StopFrame

%%----------------------------------
%%----------------------------------

%%----------------------------------------------------------------
\page
% https://wiki.contextgarden.net/Transparency

\definecolor [transparentred]  [r=1,t=.5,a=1]
\definecolor [transparentblue] [b=1,t=.5,a=1]
\definecolor [solidyellow]  [y=1,t=1,a=1]

\startTEXpage
This is some sample text that goes behind the rectangles\hskip-8cm
\blackrule[width=2cm,height=1cm,depth=1cm,color=solidyellow]\hskip-0.67cm
\blackrule[width=2cm,height=2cm,color=transparentred]\hskip-0.67cm
\blackrule[width=2cm,height=1cm,depth=1cm,color=transparentblue]\hskip-0.67cm
\blackrule[width=2cm,height=2cm,color=solidyellow]\hskip2cm.
\stopTEXpage


\startTEXpage

\blackrule[width=\textwidth,height=1cm,depth=1cm,color=solidyellow]\hskip-0.67cm
\blackrule[width=2cm,height=2cm,color=transparentred]\hskip-0.67cm
\blackrule[width=2cm,height=1cm,depth=1cm,color=transparentblue]\hskip-0.67cm
\blackrule[width=2cm,height=2cm,color=solidyellow]\hskip2cm.
\stopTEXpage

\page
\myProgressBar
test

test


\page
\myProgressBar
test

test


\page
\myProgressBar
test

test


\page
\myProgressBar
test

test


\page
\myProgressBar
test
\reference[NL]{牛顿-莱布尼茨公式}牛顿-莱布尼茨公式 reference

\at{page}[积分中值定理] at page 积分中值定理
test


\page
\myProgressBar
test
\textreference[NL]{牛顿-莱布尼茨公式}牛顿-莱布尼茨公式 text reference
test


\page
\myProgressBar
test
\at{page}[NL] at page NL
test

\page
\framed[align=right,width=\textwidth]{Some framed text, with \type{align=right}.}

\myProgressBar
test

\page
\startitemize

\SlideWithSteps{8}{
\item Consider the following nonlinear equation:
$$\Step{8}{{\partial u \over \partial t}} \Step{1}{-\Delta u + |u|^{p-1}u} \Step{2}{=} \Step{3}{f} \Step{4}{+{\rm div}(g)} \Step{5}{+|\nabla u|{\Step{6}{^2}}}$$
\StepBetween[3,5]{\item This line appears only between steps 3 and 5}
\Step{8}{\item The equation may be parabolic.}
\StepBefore{4}{\item This line appears only before step 4.}
\OnlyStep{4}{\item This line appears only at step 4.}
\vfill (Here you see step number \the\StepsCounter)
} % end of \SlideWithSteps

\stopitemize

\page
test
%%%%%%%%%%-----------
%%%%%%%%%%-----------
%%%%%%%%%%-----------
%%%%%%%%%%-----------
%%%%%%%%%%-----------
%%%%%%%%%%-----------
%%%%%%%%%%-----------
%%%%%%%%%%-----------
%%%%%%%%%%-----------
%% 2020.07.23
%%%%%%%%%%% %%%%%%%%%%%%%%%%%%%%%%%%%%
% 内容分割线 2020.07.16
%%%%%%%%%%% %%%%%%%%%%%%%%%%%%%%%%%%%%
%%
%%%%%%%%%%% %%%%%%%%%%%%%%%%%%%%%%%%%%
% 能修改的部分结束 2019.6.7
%%%%%%%%%%% %%%%%%%%%%%%%%%%%%%%%%%%%%
%%%%%%%%%%%%%%%%%%%%%%%%%%
%\input Calculus-1-2-Ex00 % Calculus-1-2-main.tex BackUP
%\input Calculus-1-2-Ex01 % Multiline Formula
%\input Calculus-1-2-Ex02 % Math and Chinese Font
%\input Calculus-1-2-Ex03 % Table
%\input Calculus-1-2-Ex04 % Long Math Formula
%\input Calculus-1-2-Ex05 % Font: em, underbar, overbar, overstrike,Frame
%\input Calculus-1-2-Ex06 % SetupBodyFont
%\input Calculus-1-2-Ex07 % Theorem
%\input Calculus-1-2-Ex08 % Two Columns
%\input Calculus-1-2-Ex09 % png格式插图
%\input Calculus-1-2-Ex10 % 插入代码段

%%%%%%%%%%% %%%%%%%%%%%%%%%%%%%%%%%%%%
% 能修改的部分结束 2019.6.7
%%%%%%%%%%% %%%%%%%%%%%%%%%%%%%%%%%%%%
%%
%%
%%


%%
%%
%%
% 索引

\switchtobodyfont[14pt] % 改变字体大小
\completeindex % 生成索引链接
%%%%%%%%%
\stoptext % 正文部分结束

